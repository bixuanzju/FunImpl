% SIGPLAN template
\documentclass[preprint]{sigplanconf}

%% -- Packages Imports --

% AMS stuff
\usepackage{amsmath}
\usepackage{amssymb}
\usepackage{amsthm}
\usepackage{xspace}

% Language
\usepackage{csquotes}
\usepackage[british]{babel}
\MakeOuterQuote{"}

\newcommand{\name}{{\bf $\lambda C_{\beta}$}\xspace}
\newcommand{\coc}{{\bf $\lambda C$}\xspace}
\newcommand{\ecore}{$\lambda C_{\mathsf{exp}}$\xspace}
\newcommand{\cc}{$\lambda C$\xspace}

\newcommand{\fold}[2]{\mathsf{cast}^\uparrow[#1]\,#2}
\newcommand{\unfold}[2][dummy]{\mathsf{cast}_\downarrow\,#2}
\newcommand{\cast}{$\kw{cast}$\xspace}

\newcommand{\lam}[3]{\lambda #1:#2.\,#3}
\newcommand{\pai}[3]{\Pi #1:#2.\,#3}
\newcommand{\miu}[3]{\mu #1:#2.\,#3}

\newcommand{\authornote}[3]{{\color{#2} {\sc #1}: #3}}
\newcommand\bruno[1]{\authornote{bruno}{red}{#1}}
\newcommand\jeremy[1]{\authornote{jeremy}{blue}{#1}}
\newcommand\linus[1]{\authornote{linus}{green}{#1}}
\newcommand{\fixme}[1]{{\color{red} #1}}

% Hyper links
\usepackage{hyperref}
\hypersetup{
   colorlinks,
   citecolor=black,
   filecolor=black,
   linkcolor=black,
   urlcolor=black
}

% Compact list
\usepackage{paralist}

% Figure import
\usepackage{graphicx}
\usepackage{float}

% Code highlighting
\usepackage{listings}

% Theorem
\usepackage{amsthm}
\newtheorem{thm}{Theorem}[section]
\newtheorem{lem}[thm]{Lemma}
\newtheorem{dfn}[thm]{Definition}

%% Typesetting inference rules
\usepackage{styles/mathpartir}  % by Didier Rémy (http://gallium.inria.fr/~remy/latex/mathpartir.html

% Ott includes
\usepackage{supertabular}
% generated by Ott 0.25 from: expcore.ott srclang.ott cocpts.ott
\newcommand{\ottdrule}[4][]{{\displaystyle\frac{\begin{array}{l}#2\end{array}}{#3}\quad\ottdrulename{#4}}}
\newcommand{\ottusedrule}[1]{\[#1\]}
\newcommand{\ottpremise}[1]{ #1 \\}
\newenvironment{ottdefnblock}[3][]{ \framebox{\mbox{#2}} \quad #3 \\[0pt]}{}
\newenvironment{ottfundefnblock}[3][]{ \framebox{\mbox{#2}} \quad #3 \\[0pt]\begin{displaymath}\begin{array}{l}}{\end{array}\end{displaymath}}
\newcommand{\ottfunclause}[2]{ #1 \equiv #2 \\}
\newcommand{\ottnt}[1]{\mathit{#1}}
\newcommand{\ottmv}[1]{\mathit{#1}}
\newcommand{\ottkw}[1]{\mathbf{#1}}
\newcommand{\ottsym}[1]{#1}
\newcommand{\ottcom}[1]{\text{#1}}
\newcommand{\ottdrulename}[1]{\textsc{#1}}
\newcommand{\ottcomplu}[5]{\overline{#1}^{\,#2\in #3 #4 #5}}
\newcommand{\ottcompu}[3]{\overline{#1}^{\,#2<#3}}
\newcommand{\ottcomp}[2]{\overline{#1}^{\,#2}}
\newcommand{\ottgrammartabular}[1]{\begin{supertabular}{llcllllll}#1\end{supertabular}}
\newcommand{\ottmetavartabular}[1]{\begin{supertabular}{ll}#1\end{supertabular}}
\newcommand{\ottrulehead}[3]{$#1$ & & $#2$ & & & \multicolumn{2}{l}{#3}}
\newcommand{\ottprodline}[6]{& & $#1$ & $#2$ & $#3 #4$ & $#5$ & $#6$}
\newcommand{\ottfirstprodline}[6]{\ottprodline{#1}{#2}{#3}{#4}{#5}{#6}}
\newcommand{\ottlongprodline}[2]{& & $#1$ & \multicolumn{4}{l}{$#2$}}
\newcommand{\ottfirstlongprodline}[2]{\ottlongprodline{#1}{#2}}
\newcommand{\ottbindspecprodline}[6]{\ottprodline{#1}{#2}{#3}{#4}{#5}{#6}}
\newcommand{\ottprodnewline}{\\}
\newcommand{\ottinterrule}{\\[5.0mm]}
\newcommand{\ottafterlastrule}{\\}
  \newcommand{\labeledjudge}[1]{\vdash_{\!\!\mathsf{#1} } }
  \newcommand{\kw}[1]{\mathbf{#1} }

\newcommand{\ottmetavars}{
\ottmetavartabular{
 $ \ottmv{x} ,\, \ottmv{y} ,\, \ottmv{z} ,\, \ottmv{d} ,\, \ottmv{X} ,\, \ottmv{D} ,\, \ottmv{K} ,\, \ottmv{N} ,\, \ottmv{R} ,\, \theta ,\, c ,\, b ,\, n ,\, u $ & \ottcom{Variable names} \\
 $ \ottmv{i} ,\, \ottmv{j} ,\, \ottmv{n} ,\, \ottmv{m} $ & \ottcom{Index variables} \\
}}

\newcommand{\otte}{
\ottrulehead{\ottnt{e}  ,\ \tau  ,\ \sigma  ,\ \rho}{::=}{\ottcom{Expressions}}\ottprodnewline
\ottfirstprodline{|}{\ottmv{x}}{}{}{}{\ottcom{Variable}}\ottprodnewline
\ottprodline{|}{\star}{}{}{}{\ottcom{Type of Types}}\ottprodnewline
\ottprodline{|}{\ottnt{e_{{\mathrm{1}}}} \, \ottnt{e_{{\mathrm{2}}}}}{}{}{}{\ottcom{Application}}\ottprodnewline
\ottprodline{|}{\lambda  \ottmv{x}  \ottsym{:}  \tau  \ottsym{.}  \ottnt{e}}{}{}{}{\ottcom{Abstraction}}\ottprodnewline
\ottprodline{|}{\Pi \, \ottmv{x}  \ottsym{:}  \tau  \ottsym{.}  \sigma}{}{}{}{\ottcom{Dependent Product}}\ottprodnewline
\ottprodline{|}{\mathsf{cast}^{\uparrow} \, \ottsym{[}  \tau  \ottsym{]} \,  \ottnt{e}}{}{}{}{\ottcom{Cast Up by Expansion}}\ottprodnewline
\ottprodline{|}{\mathsf{cast}_{\downarrow} \, \ottnt{e}}{}{}{}{\ottcom{Cast Down by Reduction}}\ottprodnewline
\ottprodline{|}{\mu \, \ottmv{x}  \ottsym{:}  \tau  \ottsym{.}  \ottnt{e}}{}{}{}{\ottcom{Polymorphic Recursion}}}

\newcommand{\ottlit}{
\ottrulehead{\ottnt{lit}}{::=}{}}

\newcommand{\otts}{
\ottrulehead{\star  ,\ \star}{::=}{\ottcom{Sorts}}\ottprodnewline
\ottfirstprodline{|}{\star}{}{}{}{\ottcom{Star}}}

\newcommand{\ottG}{
\ottrulehead{\Gamma}{::=}{\ottcom{Contexts}}\ottprodnewline
\ottfirstprodline{|}{\varnothing}{}{}{}{\ottcom{Empty}}\ottprodnewline
\ottprodline{|}{\Gamma  \ottsym{,}  \ottmv{x}  \ottsym{:}  \tau}{}{}{}{\ottcom{Variable Binding}}}

\newcommand{\ottv}{
\ottrulehead{\ottnt{v}}{::=}{\ottcom{Values}}\ottprodnewline
\ottfirstprodline{|}{\star}{}{}{}{\ottcom{Type of Types}}\ottprodnewline
\ottprodline{|}{\lambda  \ottmv{x}  \ottsym{:}  \tau  \ottsym{.}  \ottnt{e}}{}{}{}{\ottcom{Abstraction}}\ottprodnewline
\ottprodline{|}{\Pi \, \ottmv{x}  \ottsym{:}  \tau_{{\mathrm{1}}}  \ottsym{.}  \tau_{{\mathrm{2}}}}{}{}{}{\ottcom{Dependent Product}}\ottprodnewline
\ottprodline{|}{\mathsf{cast}^{\uparrow} \, \ottsym{[}  \tau  \ottsym{]} \,  \ottnt{e}}{}{}{}{\ottcom{Cast Up by Expansion}}}

\newcommand{\ottec}{
\ottrulehead{e  ,\ \tau  ,\ \sigma}{::=}{\ottcom{Expressions}}\ottprodnewline
\ottfirstprodline{|}{\ottmv{x}}{}{}{}{\ottcom{Variable}}\ottprodnewline
\ottprodline{|}{\star}{}{}{}{\ottcom{Type of Types}}\ottprodnewline
\ottprodline{|}{e_{{\mathrm{1}}} \, e_{{\mathrm{2}}}}{}{}{}{\ottcom{Application}}\ottprodnewline
\ottprodline{|}{\lambda  \ottmv{x}  \ottsym{:}  \tau  \ottsym{.}  e}{}{}{}{\ottcom{Abstraction}}\ottprodnewline
\ottprodline{|}{\Pi \, \ottmv{x}  \ottsym{:}  \tau  \ottsym{.}  \sigma}{}{}{}{\ottcom{Dependent Product}}\ottprodnewline
\ottprodline{|}{\mathsf{cast}^{\uparrow} \, \ottsym{[}  \tau  \ottsym{]} \,  e}{}{}{}{\ottcom{Cast Up by Expansion}}\ottprodnewline
\ottprodline{|}{\mathsf{cast}_{\downarrow} \, e}{}{}{}{\ottcom{Cast Down by Reduction}}}

\newcommand{\ottpgm}{
\ottrulehead{\ottnt{pgm}}{::=}{\ottcom{Program}}\ottprodnewline
\ottfirstprodline{|}{\,\overline{  \ottnt{decl}  }\,  \ottsym{;}  \ottnt{E}}{}{}{}{\ottcom{Declarations}}}

\newcommand{\ottdecl}{
\ottrulehead{\ottnt{decl}}{::=}{\ottcom{Declarations}}\ottprodnewline
\ottfirstprodline{|}{\kw{data} \, \ottmv{D}  \,\overline{  \ottnt{u}  \ottsym{:}  \kappa  }\,  \ottsym{=}  \,\overline{  \ottmv{K}  \,\overline{  \ottmv{x}  \ottsym{:}  T  }\,  }\,}{}{}{}{\ottcom{Datatype Declaration}}}

\newcommand{\ottp}{
\ottrulehead{\ottnt{p}}{::=}{\ottcom{Patterns}}\ottprodnewline
\ottfirstprodline{|}{\ottmv{K}  \,\overline{  \ottmv{x}  \ottsym{:}  T  }\,}{}{}{}{\ottcom{Pattern}}}

\newcommand{\ottu}{
\ottrulehead{\ottnt{u}  ,\ \upsilon}{::=}{\ottcom{Atoms}}\ottprodnewline
\ottfirstprodline{|}{\ottmv{x}}{}{}{}{\ottcom{Variable}}\ottprodnewline
\ottprodline{|}{\ottmv{K}}{}{}{}{\ottcom{Data Constructor}}}

\newcommand{\ottE}{
\ottrulehead{\ottnt{E}  ,\ T  ,\ S  ,\ \kappa}{::=}{\ottcom{Expressions}}\ottprodnewline
\ottfirstprodline{|}{\ottnt{u}}{}{}{}{\ottcom{Atom}}\ottprodnewline
\ottprodline{|}{\star}{}{}{}{\ottcom{Type of Types}}\ottprodnewline
\ottprodline{|}{\ottnt{E_{{\mathrm{1}}}} \, \ottnt{E_{{\mathrm{2}}}}}{}{}{}{\ottcom{Application}}\ottprodnewline
\ottprodline{|}{\lambda  \ottmv{x}  \ottsym{:}  T  \ottsym{.}  \ottnt{E}}{}{}{}{\ottcom{Abstraction}}\ottprodnewline
\ottprodline{|}{\Pi \, \ottmv{x}  \ottsym{:}  T  \ottsym{.}  S}{}{}{}{\ottcom{Dependent Product}}\ottprodnewline
\ottprodline{|}{\mu \, \ottmv{x}  \ottsym{:}  T  \ottsym{.}  \ottnt{E}}{}{}{}{\ottcom{Polymorphic Recursion}}\ottprodnewline
\ottprodline{|}{\kw{case} \, \ottnt{E_{{\mathrm{1}}}} \, \kw{of} \, \,\overline{  \ottnt{p}  \Rightarrow  \ottnt{E_{{\mathrm{2}}}}  }\,}{}{}{}{\ottcom{Case Analysis}}}

\newcommand{\ottGs}{
\ottrulehead{\Sigma}{::=}{\ottcom{Contexts}}\ottprodnewline
\ottfirstprodline{|}{\varnothing}{}{}{}{\ottcom{Empty}}\ottprodnewline
\ottprodline{|}{\Sigma  \ottsym{,}  \ottnt{u}  \ottsym{:}  T}{}{}{}{\ottcom{Atom Binding}}}

\newcommand{\ottee}{
\ottrulehead{e  ,\ \tau}{::=}{\ottcom{Expressions}}\ottprodnewline
\ottfirstprodline{|}{\ottmv{x}}{}{}{}{\ottcom{Variable}}\ottprodnewline
\ottprodline{|}{s}{}{}{}{\ottcom{Sort}}\ottprodnewline
\ottprodline{|}{e_{{\mathrm{1}}} \, e_{{\mathrm{2}}}}{}{}{}{\ottcom{Application}}\ottprodnewline
\ottprodline{|}{\lambda  \ottmv{x}  \ottsym{:}  \tau  \ottsym{.}  e}{}{}{}{\ottcom{Abstraction}}\ottprodnewline
\ottprodline{|}{\Pi \, \ottmv{x}  \ottsym{:}  \tau_{{\mathrm{1}}}  \ottsym{.}  \tau_{{\mathrm{2}}}}{}{}{}{\ottcom{Product}}}

\newcommand{\ottss}{
\ottrulehead{s  ,\ t}{::=}{\ottcom{Sorts}}\ottprodnewline
\ottfirstprodline{|}{\star}{}{}{}{\ottcom{Star}}\ottprodnewline
\ottprodline{|}{\Box}{}{}{}{\ottcom{Square}}}

\newcommand{\ottgrammar}{\ottgrammartabular{
\otte\ottinterrule
\ottlit\ottinterrule
\otts\ottinterrule
\ottG\ottinterrule
\ottv\ottinterrule
\ottec\ottinterrule
\ottpgm\ottinterrule
\ottdecl\ottinterrule
\ottp\ottinterrule
\ottu\ottinterrule
\ottE\ottinterrule
\ottGs\ottinterrule
\ottee\ottinterrule
\ottss\ottafterlastrule
}}

% defnss
% defns LintCore
%% defn ctx
\newcommand{\ottdruleEnvXXEmpty}[1]{\ottdrule[#1]{%
}{
\vdash  \varnothing}{%
{\ottdrulename{Env\_Empty}}{}%
}}


\newcommand{\ottdruleEnvXXVar}[1]{\ottdrule[#1]{%
{\vdash  \Gamma}%
\\ {\Gamma  \vdash  \tau  \ottsym{:}  \star}%
}{
\vdash  \Gamma  \ottsym{,}  \ottmv{x}  \ottsym{:}  \tau}{%
{\ottdrulename{Env\_Var}}{}%
}}

\newcommand{\ottdefnctx}[1]{\begin{ottdefnblock}[#1]{$\vdash  \Gamma$}{\ottcom{Well-formed context}}
\ottusedrule{\ottdruleEnvXXEmpty{}}
\ottusedrule{\ottdruleEnvXXVar{}}
\end{ottdefnblock}}

%% defn expr_rec
\newcommand{\ottdruleTXXMu}[1]{\ottdrule[#1]{%
{\Gamma  \ottsym{,}  \ottmv{x}  \ottsym{:}  \tau  \vdash  \ottnt{e}  \ottsym{:}  \tau}%
\\ {\Gamma  \vdash  \tau  \ottsym{:}  \star}%
}{
\Gamma  \vdash  \ottsym{(}  \mu \, \ottmv{x}  \ottsym{:}  \tau  \ottsym{.}  \ottnt{e}  \ottsym{)}  \ottsym{:}  \tau}{%
{\ottdrulename{T\_Mu}}{}%
}}

\newcommand{\ottdefnexprXXrec}[1]{\begin{ottdefnblock}[#1]{$\Gamma  \vdash  \ottnt{e}  \ottsym{:}  \tau$}{\ottcom{Expression typing}}
\ottusedrule{\ottdruleTXXMu{}}
\end{ottdefnblock}}

%% defn expr
\newcommand{\ottdruleTXXAx}[1]{\ottdrule[#1]{%
{\vdash  \Gamma}%
}{
\Gamma  \vdash  \star  \ottsym{:}  \star}{%
{\ottdrulename{T\_Ax}}{}%
}}


\newcommand{\ottdruleTXXVar}[1]{\ottdrule[#1]{%
{\vdash  \Gamma}%
\\ {\ottmv{x}  \ottsym{:}  \tau \, \in \, \Gamma}%
}{
\Gamma  \vdash  \ottmv{x}  \ottsym{:}  \tau}{%
{\ottdrulename{T\_Var}}{}%
}}


\newcommand{\ottdruleTXXApp}[1]{\ottdrule[#1]{%
{\Gamma  \vdash  \ottnt{e_{{\mathrm{1}}}}  \ottsym{:}  \ottsym{(}  \Pi \, \ottmv{x}  \ottsym{:}  \tau_{{\mathrm{2}}}  \ottsym{.}  \tau_{{\mathrm{1}}}  \ottsym{)}}%
\\ {\Gamma  \vdash  \ottnt{e_{{\mathrm{2}}}}  \ottsym{:}  \tau_{{\mathrm{2}}}}%
}{
\Gamma  \vdash  \ottnt{e_{{\mathrm{1}}}} \, \ottnt{e_{{\mathrm{2}}}}  \ottsym{:}  \tau_{{\mathrm{1}}}  \ottsym{[}  \ottmv{x}  \mapsto  \ottnt{e_{{\mathrm{2}}}}  \ottsym{]} \,}{%
{\ottdrulename{T\_App}}{}%
}}


\newcommand{\ottdruleTXXLam}[1]{\ottdrule[#1]{%
{\Gamma  \ottsym{,}  \ottmv{x}  \ottsym{:}  \tau_{{\mathrm{1}}}  \vdash  \ottnt{e}  \ottsym{:}  \tau_{{\mathrm{2}}}}%
\\ {\Gamma  \vdash  \ottsym{(}  \Pi \, \ottmv{x}  \ottsym{:}  \tau_{{\mathrm{1}}}  \ottsym{.}  \tau_{{\mathrm{2}}}  \ottsym{)}  \ottsym{:}  \star}%
}{
\Gamma  \vdash  \ottsym{(}  \lambda  \ottmv{x}  \ottsym{:}  \tau_{{\mathrm{1}}}  \ottsym{.}  \ottnt{e}  \ottsym{)}  \ottsym{:}  \ottsym{(}  \Pi \, \ottmv{x}  \ottsym{:}  \tau_{{\mathrm{1}}}  \ottsym{.}  \tau_{{\mathrm{2}}}  \ottsym{)}}{%
{\ottdrulename{T\_Lam}}{}%
}}


\newcommand{\ottdruleTXXPi}[1]{\ottdrule[#1]{%
{\Gamma  \vdash  \tau_{{\mathrm{1}}}  \ottsym{:}  \star}%
\\ {\Gamma  \ottsym{,}  \ottmv{x}  \ottsym{:}  \tau_{{\mathrm{1}}}  \vdash  \tau_{{\mathrm{2}}}  \ottsym{:}  \star}%
}{
\Gamma  \vdash  \ottsym{(}  \Pi \, \ottmv{x}  \ottsym{:}  \tau_{{\mathrm{1}}}  \ottsym{.}  \tau_{{\mathrm{2}}}  \ottsym{)}  \ottsym{:}  \star}{%
{\ottdrulename{T\_Pi}}{}%
}}


\newcommand{\ottdruleTXXCastUp}[1]{\ottdrule[#1]{%
{\Gamma  \vdash  \ottnt{e}  \ottsym{:}  \tau_{{\mathrm{2}}}}%
\\ {\Gamma  \vdash  \tau_{{\mathrm{1}}}  \ottsym{:}  \star}%
\\ {\tau_{{\mathrm{1}}}  \longrightarrow  \tau_{{\mathrm{2}}}}%
}{
\Gamma  \vdash  \ottsym{(}  \mathsf{cast}^{\uparrow} \, \ottsym{[}  \tau_{{\mathrm{1}}}  \ottsym{]} \,  \ottnt{e}  \ottsym{)}  \ottsym{:}  \tau_{{\mathrm{1}}}}{%
{\ottdrulename{T\_CastUp}}{}%
}}


\newcommand{\ottdruleTXXCastDown}[1]{\ottdrule[#1]{%
{\Gamma  \vdash  \ottnt{e}  \ottsym{:}  \tau_{{\mathrm{1}}}}%
\\ {\Gamma  \vdash  \tau_{{\mathrm{2}}}  \ottsym{:}  \star}%
\\ {\tau_{{\mathrm{1}}}  \longrightarrow  \tau_{{\mathrm{2}}}}%
}{
\Gamma  \vdash  \ottsym{(}  \mathsf{cast}_{\downarrow} \, \ottnt{e}  \ottsym{)}  \ottsym{:}  \tau_{{\mathrm{2}}}}{%
{\ottdrulename{T\_CastDown}}{}%
}}

\newcommand{\ottdefnexpr}[1]{\begin{ottdefnblock}[#1]{$\Gamma  \vdash  \ottnt{e}  \ottsym{:}  \tau$}{\ottcom{Expression typing}}
\ottusedrule{\ottdruleTXXAx{}}
\ottusedrule{\ottdruleTXXVar{}}
\ottusedrule{\ottdruleTXXApp{}}
\ottusedrule{\ottdruleTXXLam{}}
\ottusedrule{\ottdruleTXXPi{}}
\ottusedrule{\ottdruleTXXCastUp{}}
\ottusedrule{\ottdruleTXXCastDown{}}
\end{ottdefnblock}}


\newcommand{\ottdefnsLintCore}{
\ottdefnctx{}\ottdefnexprXXrec{}\ottdefnexpr{}}

% defns OpSem
%% defn step_rec
\newcommand{\ottdruleSXXMu}[1]{\ottdrule[#1]{%
}{
\mu \, \ottmv{x}  \ottsym{:}  \tau  \ottsym{.}  \ottnt{e}  \longrightarrow  \ottnt{e}  \ottsym{[}  \ottmv{x}  \mapsto  \mu \, \ottmv{x}  \ottsym{:}  \tau  \ottsym{.}  \ottnt{e}  \ottsym{]} \,}{%
{\ottdrulename{S\_Mu}}{}%
}}

\newcommand{\ottdefnstepXXrec}[1]{\begin{ottdefnblock}[#1]{$\ottnt{e}  \longrightarrow  \ottnt{e'}$}{\ottcom{One-step reduction}}
\ottusedrule{\ottdruleSXXMu{}}
\end{ottdefnblock}}

%% defn step
\newcommand{\ottdruleSXXBeta}[1]{\ottdrule[#1]{%
}{
\ottsym{(}  \lambda  \ottmv{x}  \ottsym{:}  \tau  \ottsym{.}  \ottnt{e_{{\mathrm{1}}}}  \ottsym{)} \, \ottnt{e_{{\mathrm{2}}}}  \longrightarrow  \ottnt{e_{{\mathrm{1}}}}  \ottsym{[}  \ottmv{x}  \mapsto  \ottnt{e_{{\mathrm{2}}}}  \ottsym{]} \,}{%
{\ottdrulename{S\_Beta}}{}%
}}


\newcommand{\ottdruleSXXCastDownUp}[1]{\ottdrule[#1]{%
}{
\mathsf{cast}_{\downarrow} \, \ottsym{(}  \mathsf{cast}^{\uparrow} \, \ottsym{[}  \tau  \ottsym{]} \,  \ottnt{e}  \ottsym{)}  \longrightarrow  \ottnt{e}}{%
{\ottdrulename{S\_CastDownUp}}{}%
}}


\newcommand{\ottdruleSXXApp}[1]{\ottdrule[#1]{%
{\ottnt{e_{{\mathrm{1}}}}  \longrightarrow  \ottnt{e'_{{\mathrm{1}}}}}%
}{
\ottnt{e_{{\mathrm{1}}}} \, \ottnt{e_{{\mathrm{2}}}}  \longrightarrow  \ottnt{e'_{{\mathrm{1}}}} \, \ottnt{e_{{\mathrm{2}}}}}{%
{\ottdrulename{S\_App}}{}%
}}


\newcommand{\ottdruleSXXCastDown}[1]{\ottdrule[#1]{%
{\ottnt{e}  \longrightarrow  \ottnt{e'}}%
}{
\mathsf{cast}_{\downarrow} \, \ottnt{e}  \longrightarrow  \mathsf{cast}_{\downarrow} \, \ottnt{e'}}{%
{\ottdrulename{S\_CastDown}}{}%
}}

\newcommand{\ottdefnstep}[1]{\begin{ottdefnblock}[#1]{$\ottnt{e}  \longrightarrow  \ottnt{e'}$}{\ottcom{One-step reduction}}
\ottusedrule{\ottdruleSXXBeta{}}
\ottusedrule{\ottdruleSXXCastDownUp{}}
\ottusedrule{\ottdruleSXXApp{}}
\ottusedrule{\ottdruleSXXCastDown{}}
\end{ottdefnblock}}

%% defn step_cast
\newcommand{\ottdruleSXXCastUpE}[1]{\ottdrule[#1]{%
}{
\mathsf{cast}^{\uparrow} \, \ottsym{[}  \tau  \ottsym{]} \,  \ottnt{e}  \longrightarrow  \ottnt{e}}{%
{\ottdrulename{S\_CastUpE}}{}%
}}


\newcommand{\ottdruleSXXCastDownE}[1]{\ottdrule[#1]{%
}{
\mathsf{cast}_{\downarrow} \, \ottnt{e}  \longrightarrow  \ottnt{e}}{%
{\ottdrulename{S\_CastDownE}}{}%
}}

\newcommand{\ottdefnstepXXcast}[1]{\begin{ottdefnblock}[#1]{$\ottnt{e}  \longrightarrow  \ottnt{e'}$}{\ottcom{One-step reduction}}
\ottusedrule{\ottdruleSXXCastUpE{}}
\ottusedrule{\ottdruleSXXCastDownE{}}
\end{ottdefnblock}}


\newcommand{\ottdefnsOpSem}{
\ottdefnstepXXrec{}\ottdefnstep{}\ottdefnstepXXcast{}}

% defns OpSemSrc

\newcommand{\ottdefnsOpSemSrc}{
}

% defns LintSrc
%% defn ctxtrans
\newcommand{\ottdruleTRenvXXEmpty}[1]{\ottdrule[#1]{%
}{
 \labeledjudge{wf}  \varnothing  \hlmath{ \rightsquigarrow   \varnothing } }{%
{\ottdrulename{TRenv\_Empty}}{}%
}}


\newcommand{\ottdruleTRenvXXVar}[1]{\ottdrule[#1]{%
{ \labeledjudge{wf}  \Sigma  \hlmath{ \rightsquigarrow   \Gamma } }%
\\ { \Sigma  \labeledjudge{s}  T  :  \star  \hlmath{ \rightsquigarrow   \tau } }%
}{
 \labeledjudge{wf}  \Sigma  \ottsym{,}  \ottmv{x}  \ottsym{:}  T  \hlmath{ \rightsquigarrow   \Gamma  \ottsym{,}  \ottmv{x}  \ottsym{:}  \tau } }{%
{\ottdrulename{TRenv\_Var}}{}%
}}

\newcommand{\ottdefnctxtrans}[1]{\begin{ottdefnblock}[#1]{$ \labeledjudge{wf}  \Sigma  \hlmath{ \rightsquigarrow   \Gamma } $}{\ottcom{Context well-formedness}}
\ottusedrule{\ottdruleTRenvXXEmpty{}}
\ottusedrule{\ottdruleTRenvXXVar{}}
\end{ottdefnblock}}

%% defn pgmtrans
\newcommand{\ottdruleTRpgmXXPgm}[1]{\ottdrule[#1]{%
{\,\overline{   \Sigma_{{\mathrm{0}}}  \labeledjudge{d}  \ottnt{decl}  :  \Sigma'  \hlmath{ \rightsquigarrow   \ottnt{e_{{\mathrm{1}}}} }   }\,}%
\\ {\Sigma  \ottsym{=}  \Sigma_{{\mathrm{0}}}  \ottsym{,}  \,\overline{  \Sigma'  }\,}%
\\ { \Sigma  \labeledjudge{s}  \ottnt{E}  :  T  \hlmath{ \rightsquigarrow   \ottnt{e} } }%
}{
 \Sigma_{{\mathrm{0}}}  \labeledjudge{pg}  \ottsym{(}  \,\overline{  \ottnt{decl}  }\,  \ottsym{;}  \ottnt{E}  \ottsym{)}  :  T  \hlmath{ \rightsquigarrow   \,\overline{  \ottnt{e_{{\mathrm{1}}}}  }\,  \uplus  \ottnt{e} } }{%
{\ottdrulename{TRpgm\_Pgm}}{}%
}}

\newcommand{\ottdefnpgmtrans}[1]{\begin{ottdefnblock}[#1]{$ \Sigma  \labeledjudge{pg}  \ottnt{pgm}  :  T  \hlmath{ \rightsquigarrow   \ottnt{e} } $}{\ottcom{Program translation}}
\ottusedrule{\ottdruleTRpgmXXPgm{}}
\end{ottdefnblock}}

%% defn decltrans
\newcommand{\ottdruleTRdeclXXData}[1]{\ottdrule[#1]{%
{ \Sigma  \labeledjudge{s}  \ottsym{(}  \,\overline{  \ottnt{u}  \ottsym{:}  \kappa  }^{n}\,  \ottsym{)}  \rightarrow  \star  :  \star  \hlmath{ \rightsquigarrow   \ottsym{(}  \,\overline{  \ottnt{u}  \ottsym{:}  \rho  }^{n}\,  \ottsym{)}  \rightarrow  \star } }%
\\ {\,\overline{   \Sigma  \ottsym{,}  \ottmv{D}  \ottsym{:}  \ottsym{(}  \,\overline{  \ottnt{u}  \ottsym{:}  \kappa  }^{n}\,  \ottsym{)}  \rightarrow  \star  \ottsym{,}  \,\overline{  \ottnt{u}  \ottsym{:}  \kappa  }^{n}\,  \labeledjudge{s}  \ottsym{(}  \,\overline{  \ottmv{x}  \ottsym{:}  T  }\,  \ottsym{)}  \rightarrow  \ottmv{D}    \,\overline{  \ottnt{u}  }^{n}\,  :  \star  \hlmath{ \rightsquigarrow   \ottsym{(}  \,\overline{  \ottmv{x}  \ottsym{:}  \tau  }\,  \ottsym{)}  \rightarrow  \ottmv{D}    \,\overline{  \ottnt{u}  }^{n}\, }   }\,}%
}{
 \Sigma  \labeledjudge{d}  \ottsym{(}  \kw{data} \, \ottmv{D}  \,\overline{  \ottnt{u}  \ottsym{:}  \kappa  }^{n}\,  \ottsym{=}  \,\overline{  \ottmv{K}  \,\overline{  \ottmv{x}  \ottsym{:}  T  }\,  }\,  \ottsym{)}  :  \ottsym{(}  \ottmv{D}  \ottsym{:}  \ottsym{(}  \,\overline{  \ottnt{u}  \ottsym{:}  \kappa  }^{n}\,  \ottsym{)}  \rightarrow  \star  \ottsym{,}  \,\overline{  \ottmv{K}  \ottsym{:}  \ottsym{(}  \,\overline{  \ottnt{u}  \ottsym{:}  \kappa  }^{n}\,  \ottsym{)}  \rightarrow  \ottsym{(}  \,\overline{  \ottmv{x}  \ottsym{:}  T  }\,  \ottsym{)}  \rightarrow  \ottmv{D}    \,\overline{  \ottnt{u}  }^{n}\,  }\,  \ottsym{)}  \hlmath{ \rightsquigarrow   \ottnt{e} } }{%
{\ottdrulename{TRdecl\_Data}}{}%
}}

\newcommand{\ottdefndecltrans}[1]{\begin{ottdefnblock}[#1]{$ \Sigma  \labeledjudge{d}  \ottnt{decl}  :  \Sigma'  \hlmath{ \rightsquigarrow   \ottnt{e} } $}{\ottcom{Datatype translation}}
\ottusedrule{\ottdruleTRdeclXXData{}}
\end{ottdefnblock}}

%% defn pattrans
\newcommand{\ottdruleTRpatXXAlt}[1]{\ottdrule[#1]{%
{\ottmv{K}  \ottsym{:}  \ottsym{(}  \,\overline{  \ottnt{u}  \ottsym{:}  \kappa  }^{n}\,  \ottsym{)}  \rightarrow  \ottsym{(}  \,\overline{  \ottmv{x}  \ottsym{:}  T  }\,  \ottsym{)}  \rightarrow  \ottmv{D}    \,\overline{  \ottnt{u}  }^{n}\, \, \in \, \Sigma}%
\\ { \Sigma  \ottsym{,}  \,\overline{  \ottmv{x}  \ottsym{:}  T  \ottsym{[}  \,\overline{  \ottnt{u}  \mapsto  \upsilon  }\,  \ottsym{]} \,  }\,  \labeledjudge{s}  \ottnt{E}  :  S  \hlmath{ \rightsquigarrow   \ottnt{e} } }%
\\ { \Sigma  \labeledjudge{s}  T  \ottsym{[}  \,\overline{  \ottnt{u}  \mapsto  \upsilon  }\,  \ottsym{]} \,  :  \star  \hlmath{ \rightsquigarrow   \tau } }%
}{
 \Sigma  \labeledjudge{p}  \ottmv{K}  \,\overline{  \ottmv{x}  \ottsym{:}  T  \ottsym{[}  \,\overline{  \ottnt{u}  \mapsto  \upsilon  }\,  \ottsym{]} \,  }\,   \Rightarrow   \ottnt{E}  :  \ottmv{D}    \,\overline{  \upsilon  }^{n}\,   \rightarrow   S  \hlmath{ \rightsquigarrow   \lambda  \,\overline{  \ottmv{x}  \ottsym{:}  \tau  }\,  \ottsym{.}  \ottnt{e} } }{%
{\ottdrulename{TRpat\_Alt}}{}%
}}

\newcommand{\ottdefnpattrans}[1]{\begin{ottdefnblock}[#1]{$ \Sigma  \labeledjudge{p}  \ottnt{p}   \Rightarrow   \ottnt{E}  :  T   \rightarrow   S  \hlmath{ \rightsquigarrow   \ottnt{e} } $}{\ottcom{Pattern translation}}
\ottusedrule{\ottdruleTRpatXXAlt{}}
\end{ottdefnblock}}

%% defn exprtrans
\newcommand{\ottdruleTRXXAx}[1]{\ottdrule[#1]{%
{ \labeledjudge{wf}  \Sigma }%
}{
 \Sigma  \labeledjudge{s}  \star  :  \star  \hlmath{ \rightsquigarrow   \star } }{%
{\ottdrulename{TR\_Ax}}{}%
}}


\newcommand{\ottdruleTRXXVar}[1]{\ottdrule[#1]{%
{ \labeledjudge{wf}  \Sigma  \hlmath{ \rightsquigarrow   \Gamma } }%
\\ {\ottmv{x}  \ottsym{:}  T \, \in \, \Sigma}%
}{
 \Sigma  \labeledjudge{s}  \ottmv{x}  :  T  \hlmath{ \rightsquigarrow   \ottmv{x} } }{%
{\ottdrulename{TR\_Var}}{}%
}}


\newcommand{\ottdruleTRXXApp}[1]{\ottdrule[#1]{%
{ \Sigma  \labeledjudge{s}  \ottnt{E_{{\mathrm{1}}}}  :  \ottsym{(}  \Pi \, \ottmv{x}  \ottsym{:}  T_{{\mathrm{2}}}  \ottsym{.}  T_{{\mathrm{1}}}  \ottsym{)}  \hlmath{ \rightsquigarrow   \ottnt{e_{{\mathrm{1}}}} } }%
\\ { \Sigma  \labeledjudge{s}  \ottnt{E_{{\mathrm{2}}}}  :  T_{{\mathrm{2}}}  \hlmath{ \rightsquigarrow   \ottnt{e_{{\mathrm{2}}}} } }%
}{
 \Sigma  \labeledjudge{s}  \ottnt{E_{{\mathrm{1}}}} \, \ottnt{E_{{\mathrm{2}}}}  :  T_{{\mathrm{1}}}  \ottsym{[}  \ottmv{x}  \mapsto  \ottnt{E_{{\mathrm{2}}}}  \ottsym{]} \,  \hlmath{ \rightsquigarrow   \ottnt{e_{{\mathrm{1}}}} \, \ottnt{e_{{\mathrm{2}}}} } }{%
{\ottdrulename{TR\_App}}{}%
}}


\newcommand{\ottdruleTRXXLam}[1]{\ottdrule[#1]{%
{ \Sigma  \ottsym{,}  \ottmv{x}  \ottsym{:}  T_{{\mathrm{1}}}  \labeledjudge{s}  \ottnt{E}  :  T_{{\mathrm{2}}}  \hlmath{ \rightsquigarrow   \ottnt{e} } }%
\\ { \Sigma  \labeledjudge{s}  \ottsym{(}  \Pi \, \ottmv{x}  \ottsym{:}  T_{{\mathrm{1}}}  \ottsym{.}  T_{{\mathrm{2}}}  \ottsym{)}  :  \star  \hlmath{ \rightsquigarrow   \Pi \, \ottmv{x}  \ottsym{:}  \tau_{{\mathrm{1}}}  \ottsym{.}  \tau_{{\mathrm{2}}} } }%
}{
 \Sigma  \labeledjudge{s}  \ottsym{(}  \lambda  \ottmv{x}  \ottsym{:}  T_{{\mathrm{1}}}  \ottsym{.}  \ottnt{E}  \ottsym{)}  :  \ottsym{(}  \Pi \, \ottmv{x}  \ottsym{:}  T_{{\mathrm{1}}}  \ottsym{.}  T_{{\mathrm{2}}}  \ottsym{)}  \hlmath{ \rightsquigarrow   \lambda  \ottmv{x}  \ottsym{:}  \tau_{{\mathrm{1}}}  \ottsym{.}  \ottnt{e} } }{%
{\ottdrulename{TR\_Lam}}{}%
}}


\newcommand{\ottdruleTRXXPi}[1]{\ottdrule[#1]{%
{ \Sigma  \labeledjudge{s}  T_{{\mathrm{1}}}  :  \star  \hlmath{ \rightsquigarrow   \tau_{{\mathrm{1}}} } }%
\\ { \Sigma  \ottsym{,}  \ottmv{x}  \ottsym{:}  T_{{\mathrm{1}}}  \labeledjudge{s}  T_{{\mathrm{2}}}  :  \star  \hlmath{ \rightsquigarrow   \tau_{{\mathrm{2}}} } }%
}{
 \Sigma  \labeledjudge{s}  \ottsym{(}  \Pi \, \ottmv{x}  \ottsym{:}  T_{{\mathrm{1}}}  \ottsym{.}  T_{{\mathrm{2}}}  \ottsym{)}  :  \star  \hlmath{ \rightsquigarrow   \Pi \, \ottmv{x}  \ottsym{:}  \tau_{{\mathrm{1}}}  \ottsym{.}  \tau_{{\mathrm{2}}} } }{%
{\ottdrulename{TR\_Pi}}{}%
}}


\newcommand{\ottdruleTRXXMu}[1]{\ottdrule[#1]{%
{ \Sigma  \ottsym{,}  \ottmv{x}  \ottsym{:}  T  \labeledjudge{s}  \ottnt{E}  :  T  \hlmath{ \rightsquigarrow   \ottnt{e} } }%
\\ { \Sigma  \labeledjudge{s}  T  :  \star  \hlmath{ \rightsquigarrow   \tau } }%
}{
 \Sigma  \labeledjudge{s}  \ottsym{(}  \mu \, \ottmv{x}  \ottsym{:}  T  \ottsym{.}  \ottnt{E}  \ottsym{)}  :  T  \hlmath{ \rightsquigarrow   \mu \, \ottmv{x}  \ottsym{:}  \tau  \ottsym{.}  \ottnt{e} } }{%
{\ottdrulename{TR\_Mu}}{}%
}}


\newcommand{\ottdruleTRXXCase}[1]{\ottdrule[#1]{%
{ \Sigma  \labeledjudge{s}  \ottnt{E_{{\mathrm{1}}}}  :  T  \hlmath{ \rightsquigarrow   \ottnt{e_{{\mathrm{1}}}} } }%
\\ {\,\overline{   \Sigma  \labeledjudge{p}  \ottnt{p}   \Rightarrow   \ottnt{E_{{\mathrm{2}}}}  :  T   \rightarrow   S  \hlmath{ \rightsquigarrow   \ottnt{e_{{\mathrm{2}}}} }   }\,}%
\\ { \Sigma  \labeledjudge{s}  S  :  \star  \hlmath{ \rightsquigarrow   \sigma } }%
}{
 \Sigma  \labeledjudge{s}  \kw{case} \, \ottnt{E_{{\mathrm{1}}}} \, \kw{of} \, \,\overline{  \ottnt{p}  \Rightarrow  \ottnt{E_{{\mathrm{2}}}}  }\,  :  S  \hlmath{ \rightsquigarrow   \ottsym{(}  \mathsf{cast}_{\downarrow}^{n+1} \, \ottnt{e_{{\mathrm{1}}}}  \ottsym{)} \, \sigma \, \,\overline{  \ottnt{e_{{\mathrm{2}}}}  }\, } }{%
{\ottdrulename{TR\_Case}}{}%
}}

\newcommand{\ottdefnexprtrans}[1]{\begin{ottdefnblock}[#1]{$ \Sigma  \labeledjudge{s}  \ottnt{E}  :  T  \hlmath{ \rightsquigarrow   \ottnt{e} } $}{\ottcom{Expression translation}}
\ottusedrule{\ottdruleTRXXAx{}}
\ottusedrule{\ottdruleTRXXVar{}}
\ottusedrule{\ottdruleTRXXApp{}}
\ottusedrule{\ottdruleTRXXLam{}}
\ottusedrule{\ottdruleTRXXPi{}}
\ottusedrule{\ottdruleTRXXMu{}}
\ottusedrule{\ottdruleTRXXCase{}}
\end{ottdefnblock}}


\newcommand{\ottdefnsLintSrc}{
\ottdefnctxtrans{}\ottdefnpgmtrans{}\ottdefndecltrans{}\ottdefnpattrans{}\ottdefnexprtrans{}}

% defns LintCC
%% defn exprcoc
\newcommand{\ottdruleTccXXAx}[1]{\ottdrule[#1]{%
}{
\varnothing  \vdash  \star  \ottsym{:}  \Box}{%
{\ottdrulename{Tcc\_Ax}}{}%
}}


\newcommand{\ottdruleTccXXVar}[1]{\ottdrule[#1]{%
{\Gamma  \vdash  \tau  \ottsym{:}  s}%
\\ { \ottmv{x}  \not \in \kw{dom}( \Gamma ) }%
}{
\Gamma  \ottsym{,}  \ottmv{x}  \ottsym{:}  \tau  \vdash  \ottmv{x}  \ottsym{:}  \tau}{%
{\ottdrulename{Tcc\_Var}}{}%
}}


\newcommand{\ottdruleTccXXWeak}[1]{\ottdrule[#1]{%
{\Gamma  \vdash  e  \ottsym{:}  \tau_{{\mathrm{2}}}}%
\\ {\Gamma  \vdash  \tau_{{\mathrm{1}}}  \ottsym{:}  s}%
\\ { \ottmv{x}  \not \in \kw{dom}( \Gamma ) }%
}{
\Gamma  \ottsym{,}  \ottmv{x}  \ottsym{:}  \tau_{{\mathrm{1}}}  \vdash  e  \ottsym{:}  \tau_{{\mathrm{2}}}}{%
{\ottdrulename{Tcc\_Weak}}{}%
}}


\newcommand{\ottdruleTccXXApp}[1]{\ottdrule[#1]{%
{\Gamma  \vdash  e_{{\mathrm{1}}}  \ottsym{:}  \ottsym{(}  \Pi \, \ottmv{x}  \ottsym{:}  \tau_{{\mathrm{2}}}  \ottsym{.}  \tau_{{\mathrm{1}}}  \ottsym{)}}%
\\ {\Gamma  \vdash  e_{{\mathrm{2}}}  \ottsym{:}  \tau_{{\mathrm{2}}}}%
}{
\Gamma  \vdash  e_{{\mathrm{1}}} \, e_{{\mathrm{2}}}  \ottsym{:}  \tau_{{\mathrm{1}}}  \ottsym{[}  \ottmv{x}  \mapsto  e_{{\mathrm{2}}}  \ottsym{]} \,}{%
{\ottdrulename{Tcc\_App}}{}%
}}


\newcommand{\ottdruleTccXXLam}[1]{\ottdrule[#1]{%
{\Gamma  \ottsym{,}  \ottmv{x}  \ottsym{:}  \tau_{{\mathrm{1}}}  \vdash  e  \ottsym{:}  \tau_{{\mathrm{2}}}}%
\\ {\Gamma  \vdash  \ottsym{(}  \Pi \, \ottmv{x}  \ottsym{:}  \tau_{{\mathrm{1}}}  \ottsym{.}  \tau_{{\mathrm{2}}}  \ottsym{)}  \ottsym{:}  s}%
}{
\Gamma  \vdash  \ottsym{(}  \lambda  \ottmv{x}  \ottsym{:}  \tau_{{\mathrm{1}}}  \ottsym{.}  e  \ottsym{)}  \ottsym{:}  \ottsym{(}  \Pi \, \ottmv{x}  \ottsym{:}  \tau_{{\mathrm{1}}}  \ottsym{.}  \tau_{{\mathrm{2}}}  \ottsym{)}}{%
{\ottdrulename{Tcc\_Lam}}{}%
}}


\newcommand{\ottdruleTccXXPi}[1]{\ottdrule[#1]{%
{\Gamma  \vdash  \tau_{{\mathrm{1}}}  \ottsym{:}  s}%
\\ {\Gamma  \ottsym{,}  \ottmv{x}  \ottsym{:}  \tau_{{\mathrm{1}}}  \vdash  \tau_{{\mathrm{2}}}  \ottsym{:}  t}%
}{
\Gamma  \vdash  \ottsym{(}  \Pi \, \ottmv{x}  \ottsym{:}  \tau_{{\mathrm{1}}}  \ottsym{.}  \tau_{{\mathrm{2}}}  \ottsym{)}  \ottsym{:}  t}{%
{\ottdrulename{Tcc\_Pi}}{}%
}}


\newcommand{\ottdruleTccXXConv}[1]{\ottdrule[#1]{%
{\Gamma  \vdash  e  \ottsym{:}  \tau_{{\mathrm{1}}}}%
\\ {\Gamma  \vdash  \tau_{{\mathrm{2}}}  \ottsym{:}  s}%
\\ {\tau_{{\mathrm{1}}}  =_{\beta}  \tau_{{\mathrm{2}}}}%
}{
\Gamma  \vdash  e  \ottsym{:}  \tau_{{\mathrm{2}}}}{%
{\ottdrulename{Tcc\_Conv}}{}%
}}

\newcommand{\ottdefnexprcoc}[1]{\begin{ottdefnblock}[#1]{$\Gamma  \vdash  e  \ottsym{:}  \tau$}{\ottcom{Typing rules of \cc}}
\ottusedrule{\ottdruleTccXXAx{}}
\ottusedrule{\ottdruleTccXXVar{}}
\ottusedrule{\ottdruleTccXXWeak{}}
\ottusedrule{\ottdruleTccXXApp{}}
\ottusedrule{\ottdruleTccXXLam{}}
\ottusedrule{\ottdruleTccXXPi{}}
\ottusedrule{\ottdruleTccXXConv{}}
\end{ottdefnblock}}


\newcommand{\ottdefnsLintCC}{
\ottdefnexprcoc{}}

\newcommand{\ottdefnss}{
\ottdefnsLintCore
\ottdefnsOpSem
\ottdefnsOpSemSrc
\ottdefnsLintSrc
\ottdefnsLintCC
}

\newcommand{\ottall}{\ottmetavars\\[0pt]
\ottgrammar\\[5.0mm]
\ottdefnss}


% Hack to use mathpartir for ott
\newcommand{\ottlinebreak}{}
\renewcommand{\ottdrule}[4][]{{\inferrule{#2 }{#3}\quad\ottdrulename{#4}}}
\newcommand{\gram}[1]{\ottgrammartabular{#1\ottafterlastrule}}
\newcommand{\ruleref}[1]{\ottdrulename{#1}}

% lhs2tex
\usepackage{mylhs2tex}

% Subsection style
\usepackage{titlesec}
\titleformat{\subsubsection}[runin]{\bfseries\itshape\normalsize}{}{0em}{}[$\;$]

%% -- Packages Imports --

% Main
\begin{document}

% Page size - US Letter
%\special{papersize=8.5in,11in}
%\setlength{\pdfpageheight}{\paperheight}
%\setlength{\pdfpagewidth}{\paperwidth}

% Conference info
\conferenceinfo{CONF 'yy}{Month d--d, 20yy, City, ST, Country}
\copyrightyear{20yy}
\copyrightdata{978-1-nnnn-nnnn-n/yy/mm}
\doi{nnnnnnn.nnnnnnn}

% Title
\titlebanner{DRAFT} % Only for preprint
\preprintfooter{} % Only for preprint

\title{Type-Level Computation One Step at a Time}
%\title{A Dependently-typed Intermediate Language with General Recursion}
%\subtitle{Or: Decidable Type-Checking in the presence of
%  Type-Level General Recursion}

\authorinfo{Foo \and Bar \and Baz}
           {The University of Foo}
           {\{foo,bar,baz\}@foo.edu}

\maketitle

% Abstract
\begin{abstract}
Many type systems support a conversion rule that allows type-level
computation. In such type systems ensuring the \emph{decidability} of
type checking requires type-level computation to terminate.
For calculi where the syntax of types and terms is the same, the
decidability of type-checking is usually dependent on the strong normalization
of the calculus, which ensures termination. An unfortunate
consequence of this coupling between decidability and strong
normalization is that adding (unrestricted) general recursion to such
calculi is not possible.

This paper proposes an alternative to the conversion rule that allows
the same syntax for types and terms, type-level computation, and
preserves decidability of type-checking under the presence of general
recursion. The key idea, which is inspired by the traditional
treatment of \emph{iso-recursive types}, is to make each type-level
computation step explicit. Each beta reduction or expansion at the
type-level is introduced by a language construct. This allows control
over the type-level computation and ensures decidability of
type-checking even in the presence of non-terminating programs at the
type-level.  We realize this idea by presenting a variant of the
calculus of constructions with general recursion and recursive types.
Furthermore we show how many advanced programming language features of
state-of-the-art functional languages (such as Haskell) can be encoded
in our minimalistic core calculus.
\end{abstract}

% Category, terms & keywords
\category{D.3.1}{Programming Languages}{Formal Definitions and Theory}
\terms Languages, Design
\keywords Dependent types, Intermediate langauge

%% -- Starting Point -- 

\section{Introduction}

Modern statically typed functional languages (such as ML, Haskell,
Scala or OCaml) have increasingly expressive type systems. Often these
large source languages are translated into a much smaller typed core
language. The choice of the core language is essential to ensure that
all the features of the source language can be encoded. For a simple
polymorphic functional language it is possible, for example, to pick a
variant of System $F$ as a core language. However, the desire for more
expressive type system features puts pressure on the core languages,
often requiring them to be extended to support new features. For
example, if the source language supports \emph{higher-kinded types} or
\emph{type-level functions} then System $F$ is not expressive enough
and can no longer be used as the core language. Instead another core
language that does provide support for higher-kinded types, such as
System~$F_{\omega}$, needs to be used.  However System~$F_{\omega}$ is
significantly more complex than System F and thus harder to
maintain. If later a new feature, such as \emph{kind polymorphism}, is
desired the core language may need to be changed again to account for
the new feature, introducing at the same time new sources of
complexity.

The more expressive type systems become, the more types become similar
to the terms. Therefore a natural idea is to unify terms and
types. There are obvious benefits in this approach: only one syntactic
level (terms) is needed; and there are much less language constructs,
making the core language easier to implement and maintain. At the same
time the core language becomes more expressive, giving us for free
many useful language features. \emph{Pure type systems}~\cite{} build
on this observation and they show how a whole family of type systems
(including System $F$ and System $F_{\omega}$) can be implemented
using just a single syntactic form. With the added expressiveness it
is even possible to have type-level programs expressed using the same
syntax as terms as well as dependently typed programs~\cite{}.

However having the same syntax for types and terms can also 
be problematic. If arbitrary type-level computation is allowed 
then type-level programs can use the same language constructs 
as terms. Usually type systems have a conversion rule to support type-level
computation. In such type systems ensuring the \emph{decidability} of
type checking requires type-level computation to terminate.  For
calculi where the syntax of types and terms is the same, the
decidability of type-checking is usually dependent on the strong
normalization of the calculus, which ensures termination. An
unfortunate consequence of this coupling between decidability and
strong normalization is that adding (unrestricted) general recursion
to such calculi is not possible.

Tension between decidability of type-checking, logical consistency and 
general recursion.

Brief summary of related work

This paper proposes an alternative to the conversion rule that allows
the same syntax for types and terms, type-level computation, and
preserves decidability of type-checking under the presence of general
recursion. The key idea, which is inspired by the traditional
treatment of \emph{iso-recursive types}, is to make each type-level
computation step explicit. 

Point about the treatment of type-level computation in Haskell. Haskell's
core language has type applications, but no type-level lambda. Equality 
is syntactic modulo alpha-conversion. This design choice was rooted in the 
desire to support Hindley-Milner type-inference... 

Each beta reduction or expansion at the
type-level is introduced by a language construct. This allows control
over the type-level computation and ensures decidability of
type-checking even in the presence of non-terminating programs at the
type-level.  We realize this idea by presenting a variant of the
calculus of constructions with general recursion and recursive types.
Furthermore we show how many programming language features of
state-of-the-art functional languages (such as Haskell) can be encoded
in our minimalistic core calculus.


\begin{enumerate}[a)]
\item Motivations:

\begin{itemize}

\item Because of the reluctance to introduce dependent
  types\footnote{This might be changed in the near future. See
    \url{https://ghc.haskell.org/trac/ghc/wiki/DependentHaskell/Phase1}.},
  the current intermediate language of Haskell, namely System $F_C$
  \cite{fc}, separates expressions as terms, types and kinds, which
  brings complexity to the implementation as well as further
  extensions \cite{fc:pro,fc:kind}.

\item Popular full-spectrum dependently typed languages, like Agda,
  Coq, Idris, have to ensure the termination of functions for the
  decidability of proofs. No general recursion and the limitation of
  enforcing termination checking make such languages impractical for
  general-purpose programming.

\item We would like to introduce a simple and compiler-friendly
  dependently typed core language with only one hierarchy, which
  supports general recursion at the same time.

\end{itemize}

\item Contribution:

\begin{itemize}

\item A core language based on Calculus of Constructions (CoC) that
  collapses terms, types and kinds into the same hierarchy.

\item General recursion by introducing recursive types for both terms
  and types by the same $\mu$ primitive.

\item Decidable type checking and managed type-level computation by
  replacing implicit conversion rule of CoC with generalized
  \textsf{fold}/\textsf{unfold} semantics.

\item First-class equality by coercion, which is used for encoding
  GADTs or newtypes without runtime overhead.

\item Surface language that supports datatypes, pattern matching and
  other language extensions for Haskell, and can be encoded into the
  core language.

\end{itemize}

\item Related work:

\begin{itemize}
\item Henk \cite{pts:henk} and one of its implementation \cite{pts:fp} show the simplicity of the Pure Type System (PTS). \cite{pts:rec} also tries to combine recursion with PTS.

\item \textsf{Zombie} \cite{zombie:popl14, zombie:thesis} is a language with two fragments supporting logics with non-termination. It limits the $\beta$-reduction for congruence closure \cite{zombie:popl15}.

\item $\Pi\Sigma$ \cite{dep:pisigma} is a simple, dependently-typed core language for expressing high-level constructions\footnote{But the paper didn't give any meta-theories about the langauge.}. UHC compiler \cite{fc:uhc} tries to use a simplified core language with coercion to encode GADTs.

\item System $F_C$ \cite{fc} has been extended with type promotion \cite{fc:pro} and kind equality \cite{fc:kind}. The latter one introduces a limited form of dependent types into the system\footnote{Richard A. Eisenberg is going to implement kind equality \cite{fc:kind} into GHC. The implementation is proposed at \url{https://phabricator.haskell.org/D808} and related paper is at \url{http://www.cis.upenn.edu/~eir/papers/2015/equalities/equalities-extended.pdf}.}, which mixes up types and kinds.
\end{itemize}

\end{enumerate}


%%% !!! WARNING: AUTO GENERATED. DO NOT MODIFY !!! %%%
%% ODER: format ==         = "\mathrel{==}"
%% ODER: format /=         = "\neq "
%
%
\makeatletter
\@ifundefined{lhs2tex.lhs2tex.sty.read}%
  {\@namedef{lhs2tex.lhs2tex.sty.read}{}%
   \newcommand\SkipToFmtEnd{}%
   \newcommand\EndFmtInput{}%
   \long\def\SkipToFmtEnd#1\EndFmtInput{}%
  }\SkipToFmtEnd

\newcommand\ReadOnlyOnce[1]{\@ifundefined{#1}{\@namedef{#1}{}}\SkipToFmtEnd}
\usepackage{amstext}
\usepackage{amssymb}
\usepackage{stmaryrd}
\DeclareFontFamily{OT1}{cmtex}{}
\DeclareFontShape{OT1}{cmtex}{m}{n}
  {<5><6><7><8>cmtex8
   <9>cmtex9
   <10><10.95><12><14.4><17.28><20.74><24.88>cmtex10}{}
\DeclareFontShape{OT1}{cmtex}{m}{it}
  {<-> ssub * cmtt/m/it}{}
\newcommand{\texfamily}{\fontfamily{cmtex}\selectfont}
\DeclareFontShape{OT1}{cmtt}{bx}{n}
  {<5><6><7><8>cmtt8
   <9>cmbtt9
   <10><10.95><12><14.4><17.28><20.74><24.88>cmbtt10}{}
\DeclareFontShape{OT1}{cmtex}{bx}{n}
  {<-> ssub * cmtt/bx/n}{}
\newcommand{\tex}[1]{\text{\texfamily#1}}	% NEU

\newcommand{\Sp}{\hskip.33334em\relax}


\newcommand{\Conid}[1]{\mathit{#1}}
\newcommand{\Varid}[1]{\mathit{#1}}
\newcommand{\anonymous}{\kern0.06em \vbox{\hrule\@width.5em}}
\newcommand{\plus}{\mathbin{+\!\!\!+}}
\newcommand{\bind}{\mathbin{>\!\!\!>\mkern-6.7mu=}}
\newcommand{\rbind}{\mathbin{=\mkern-6.7mu<\!\!\!<}}% suggested by Neil Mitchell
\newcommand{\sequ}{\mathbin{>\!\!\!>}}
\renewcommand{\leq}{\leqslant}
\renewcommand{\geq}{\geqslant}
\usepackage{polytable}

%mathindent has to be defined
\@ifundefined{mathindent}%
  {\newdimen\mathindent\mathindent\leftmargini}%
  {}%

\def\resethooks{%
  \global\let\SaveRestoreHook\empty
  \global\let\ColumnHook\empty}
\newcommand*{\savecolumns}[1][default]%
  {\g@addto@macro\SaveRestoreHook{\savecolumns[#1]}}
\newcommand*{\restorecolumns}[1][default]%
  {\g@addto@macro\SaveRestoreHook{\restorecolumns[#1]}}
\newcommand*{\aligncolumn}[2]%
  {\g@addto@macro\ColumnHook{\column{#1}{#2}}}

\resethooks

\newcommand{\onelinecommentchars}{\quad-{}- }
\newcommand{\commentbeginchars}{\enskip\{-}
\newcommand{\commentendchars}{-\}\enskip}

\newcommand{\visiblecomments}{%
  \let\onelinecomment=\onelinecommentchars
  \let\commentbegin=\commentbeginchars
  \let\commentend=\commentendchars}

\newcommand{\invisiblecomments}{%
  \let\onelinecomment=\empty
  \let\commentbegin=\empty
  \let\commentend=\empty}

\visiblecomments

\newlength{\blanklineskip}
\setlength{\blanklineskip}{0.66084ex}

\newcommand{\hsindent}[1]{\quad}% default is fixed indentation
\let\hspre\empty
\let\hspost\empty
\newcommand{\NB}{\textbf{NB}}
\newcommand{\Todo}[1]{$\langle$\textbf{To do:}~#1$\rangle$}

\EndFmtInput
\makeatother
%
%
%
%
%
%
% This package provides two environments suitable to take the place
% of hscode, called "plainhscode" and "arrayhscode". 
%
% The plain environment surrounds each code block by vertical space,
% and it uses \abovedisplayskip and \belowdisplayskip to get spacing
% similar to formulas. Note that if these dimensions are changed,
% the spacing around displayed math formulas changes as well.
% All code is indented using \leftskip.
%
% Changed 19.08.2004 to reflect changes in colorcode. Should work with
% CodeGroup.sty.
%
\ReadOnlyOnce{polycode.fmt}%
\makeatletter

\newcommand{\hsnewpar}[1]%
  {{\parskip=0pt\parindent=0pt\par\vskip #1\noindent}}

% can be used, for instance, to redefine the code size, by setting the
% command to \small or something alike
\newcommand{\hscodestyle}{}

% The command \sethscode can be used to switch the code formatting
% behaviour by mapping the hscode environment in the subst directive
% to a new LaTeX environment.

\newcommand{\sethscode}[1]%
  {\expandafter\let\expandafter\hscode\csname #1\endcsname
   \expandafter\let\expandafter\endhscode\csname end#1\endcsname}

% "compatibility" mode restores the non-polycode.fmt layout.

\newenvironment{compathscode}%
  {\par\noindent
   \advance\leftskip\mathindent
   \hscodestyle
   \let\\=\@normalcr
   \let\hspre\(\let\hspost\)%
   \pboxed}%
  {\endpboxed\)%
   \par\noindent
   \ignorespacesafterend}

\newcommand{\compaths}{\sethscode{compathscode}}

% "plain" mode is the proposed default.
% It should now work with \centering.
% This required some changes. The old version
% is still available for reference as oldplainhscode.

\newenvironment{plainhscode}%
  {\hsnewpar\abovedisplayskip
   \advance\leftskip\mathindent
   \hscodestyle
   \let\hspre\(\let\hspost\)%
   \pboxed}%
  {\endpboxed%
   \hsnewpar\belowdisplayskip
   \ignorespacesafterend}

\newenvironment{oldplainhscode}%
  {\hsnewpar\abovedisplayskip
   \advance\leftskip\mathindent
   \hscodestyle
   \let\\=\@normalcr
   \(\pboxed}%
  {\endpboxed\)%
   \hsnewpar\belowdisplayskip
   \ignorespacesafterend}

% Here, we make plainhscode the default environment.

\newcommand{\plainhs}{\sethscode{plainhscode}}
\newcommand{\oldplainhs}{\sethscode{oldplainhscode}}
\plainhs

% The arrayhscode is like plain, but makes use of polytable's
% parray environment which disallows page breaks in code blocks.

\newenvironment{arrayhscode}%
  {\hsnewpar\abovedisplayskip
   \advance\leftskip\mathindent
   \hscodestyle
   \let\\=\@normalcr
   \(\parray}%
  {\endparray\)%
   \hsnewpar\belowdisplayskip
   \ignorespacesafterend}

\newcommand{\arrayhs}{\sethscode{arrayhscode}}

% The mathhscode environment also makes use of polytable's parray 
% environment. It is supposed to be used only inside math mode 
% (I used it to typeset the type rules in my thesis).

\newenvironment{mathhscode}%
  {\parray}{\endparray}

\newcommand{\mathhs}{\sethscode{mathhscode}}

% texths is similar to mathhs, but works in text mode.

\newenvironment{texthscode}%
  {\(\parray}{\endparray\)}

\newcommand{\texths}{\sethscode{texthscode}}

% The framed environment places code in a framed box.

\def\codeframewidth{\arrayrulewidth}
\RequirePackage{calc}

\newenvironment{framedhscode}%
  {\parskip=\abovedisplayskip\par\noindent
   \hscodestyle
   \arrayrulewidth=\codeframewidth
   \tabular{@{}|p{\linewidth-2\arraycolsep-2\arrayrulewidth-2pt}|@{}}%
   \hline\framedhslinecorrect\\{-1.5ex}%
   \let\endoflinesave=\\
   \let\\=\@normalcr
   \(\pboxed}%
  {\endpboxed\)%
   \framedhslinecorrect\endoflinesave{.5ex}\hline
   \endtabular
   \parskip=\belowdisplayskip\par\noindent
   \ignorespacesafterend}

\newcommand{\framedhslinecorrect}[2]%
  {#1[#2]}

\newcommand{\framedhs}{\sethscode{framedhscode}}

% The inlinehscode environment is an experimental environment
% that can be used to typeset displayed code inline.

\newenvironment{inlinehscode}%
  {\(\def\column##1##2{}%
   \let\>\undefined\let\<\undefined\let\\\undefined
   \newcommand\>[1][]{}\newcommand\<[1][]{}\newcommand\\[1][]{}%
   \def\fromto##1##2##3{##3}%
   \def\nextline{}}{\) }%

\newcommand{\inlinehs}{\sethscode{inlinehscode}}

% The joincode environment is a separate environment that
% can be used to surround and thereby connect multiple code
% blocks.

\newenvironment{joincode}%
  {\let\orighscode=\hscode
   \let\origendhscode=\endhscode
   \def\endhscode{\def\hscode{\endgroup\def\@currenvir{hscode}\\}\begingroup}
   %\let\SaveRestoreHook=\empty
   %\let\ColumnHook=\empty
   %\let\resethooks=\empty
   \orighscode\def\hscode{\endgroup\def\@currenvir{hscode}}}%
  {\origendhscode
   \global\let\hscode=\orighscode
   \global\let\endhscode=\origendhscode}%

\makeatother
\EndFmtInput
%

\section{Overview}

This section informally introduces the main features of \name. In
particular, this section shows how the explicit casts in \name can be
used instead of the typical conversion rule present in calculi such as
the calculus of constructions. The formal details of \name are
presented in Section~\ref{sec:ecore}. \jeremy{to distinguish code from
  \sufcc and \name, we may want to use different fonts, e.g., {\tt
    typewriter font} for \sufcc}

\subsection{The Calculus of Constructions and the Conversion Rule}
\label{sec:coc}

The calculus of constructions (\coc)~\cite{coc} is a powerful
higher-order typed lambda calculus supporting dependent types (among
various other features).  A crutial
feature of \coc is the so-called \emph{conversion}
rule: \ottusedrule{\ottdruleTccXXConv{}}

%For the most part \name is similar to the \emph{Calculus of Constructions}
%(\coc)~\cite{coc}, which is a higher-order typed lambda calculus.
%However unlike \name and \coc is the
%absence of a conversion rule 

The conversion rule allows one to derive $e:\tau_{{\mathrm{2}}}$ from the
derivation of $e:\tau_{{\mathrm{1}}}$ and the $\beta$-equality of $\tau_{{\mathrm{1}}}$ and
$\tau_{{\mathrm{2}}}$. This rule is important to \emph{automatically} allows terms
with equivalent types to be considered type-compatible.  The following
example illustrates the use of the conversion rule:
\[
f \equiv \lambda  \ottmv{x}  \ottsym{:}  \ottsym{(}  \lambda  \ottmv{y}  \ottsym{:}  \star  \ottsym{.}  \ottmv{y}  \ottsym{)} \, \mathsf{Int}  \ottsym{.}  \ottmv{x}
\]
Here $f$ is a simple identity function. Notice that the type of $x$
(i.e., $\ottsym{(}  \lambda  \ottmv{y}  \ottsym{:}  \star  \ottsym{.}  \ottmv{y}  \ottsym{)} \, \mathsf{Int}$), which is the argument of $f$, is
interesting: it is an identity function on types, applied to an
integer.  Without the conversion rule, $f$ cannot be applied to, say
$3$ in \coc. However, given that $f$ is actually $\beta$-convertible
to $\lambda  \ottmv{x}  \ottsym{:}  \mathsf{Int}  \ottsym{.}  \ottmv{x}$, the conversion rule allows the application of $f$
to $3$ by implicitly converting $\lambda  \ottmv{x}  \ottsym{:}  \ottsym{(}  \lambda  \ottmv{y}  \ottsym{:}  \star  \ottsym{.}  \ottmv{y}  \ottsym{)} \, \mathsf{Int}  \ottsym{.}  \ottmv{x}$ to
$\lambda  \ottmv{x}  \ottsym{:}  \mathsf{Int}  \ottsym{.}  \ottmv{x}$.

\paragraph{Decidability of Type-Checking and Strong Normalization} 
While the conversion rule in \coc brings a lot of convenience, an
unfortunate consequence is that it couples decidability of
type-checking with strong normalization of the
calculus~\cite{coc:decidability}.  However strong normalization does
not hold with general recursion. This is because due to the conversion
rule, any non-terminating term would force the type checker to go into
an infinitely loop (by constantly applying the conversion rule without
termination), thus rendering the type system undecidable.

To illustrate the problem of the conversion rule with general
recursion, let us consider a somewhat contrived example. Suppose that
$d$ is a ``dependent type'' that has type $\mathsf{Int}  \rightarrow  \star$. With
general recursion at hand, we can image a term $z$ that has type
$d\,\mathsf{loop}$, where $\mathsf{loop}$ stands for any diverging
computation of type $ \mathsf{Int} $. What would happen if we try to type
check the following application: \[ \ottsym{(}  \lambda  \ottmv{x}  \ottsym{:}  \ottmv{d} \, 3  \ottsym{.}  \ottmv{x}  \ottsym{)} \, \ottmv{z}\]
Under the normal typing rules of \coc, the type checker would get
stuck as it tries to do $\beta$-equality on two terms: $d\,3$ and
$d\,\mathsf{loop}$, where the latter is non-terminating.  \bruno{show
  simple example. Explain issue better.} \jeremy{done!}

\subsection{An Alternative to the Conversion Rule: Explicit Casts}

\bruno{Mention somewhere that the \cast rules do \emph{one-step}
  reductions.} \jeremy{done! see last paragraph, also put beta
  reduction before beta expansion} In contrast to the implicit
reduction rules of \coc, \name makes it explicit as to when and where
to convert one type to another. Type conversions are explicit by
introducing two language constructs: $ \mathsf{cast}_{\downarrow} $ (beta reduction)
and $ \mathsf{cast}^{\uparrow} $ (beta expansion). The benefit of this approach is
that decidability of type-checking no longer is coupled with strong
normalization of the calculus.

\paragraph{Beta Reduction} The first of the two type conversions
$ \mathsf{cast}_{\downarrow} $, allows a type conversion provided that the resulting
type is a \emph{beta reduction} of the original type of the term. The
use of $ \mathsf{cast}_{\downarrow} $ is better explained by the following simple
example. Suppose that
\[ g \equiv \lambda  \ottmv{x}  \ottsym{:}  \mathsf{Int}  \ottsym{.}  \ottmv{x} \]
and term $z$ has type
\[ \ottsym{(}  \lambda  \ottmv{y}  \ottsym{:}  \star  \ottsym{.}  \ottmv{y}  \ottsym{)} \, \mathsf{Int} \]
$ g\,z $ is an ill-typed application, whereas $ g\,(\mathsf{cast}_{\downarrow} \, \ottmv{z}) $
is well-typed. This is witnessed by
$\ottsym{(}  \lambda  \ottmv{y}  \ottsym{:}  \star  \ottsym{.}  \ottmv{y}  \ottsym{)} \, \mathsf{Int} \rightarrow_{\beta}  \mathsf{Int} $, which is a beta
reduction of $\ottsym{(}  \lambda  \ottmv{y}  \ottsym{:}  \star  \ottsym{.}  \ottmv{y}  \ottsym{)} \, \mathsf{Int}$. \bruno{explain why this is a
  reduction} \jeremy{done!}

\paragraph{Beta Expansion} The dual operation of $ \mathsf{cast}_{\downarrow} $ is
$ \mathsf{cast}^{\uparrow} $, which allows a type conversion provided that the
resulting type is a \emph{beta expansion} of the original type of the
term.  Let us revisit the example from Section~\ref{sec:coc}. In \name,
$f\,3$ is an ill-typed application. Instead we must write the
application as
\[ f\,(\mathsf{cast}^{\uparrow} \, \ottsym{[}  \ottsym{(}  \lambda  \ottmv{y}  \ottsym{:}  \star  \ottsym{.}  \ottmv{y}  \ottsym{)} \, \mathsf{Int}  \ottsym{]} \,  3) \]
\bruno{how to put a space before $3$?} \jeremy{fixed!} Intuitively,
$ \mathsf{cast}^{\uparrow} $ is doing a type conversion, as the type of $ 3 $ is
$  \mathsf{Int}  $, and $ \ottsym{(}  \lambda  \ottmv{y}  \ottsym{:}  \star  \ottsym{.}  \ottmv{y}  \ottsym{)} \, \mathsf{Int} $ is the beta expansion of
$ \mathsf{Int} $ (witnessed by
$\ottsym{(}  \lambda  \ottmv{y}  \ottsym{:}  \star  \ottsym{.}  \ottmv{y}  \ottsym{)} \, \mathsf{Int} \rightarrow_{\beta}  \mathsf{Int} $). \bruno{explain why
  this is a beta expansion} \jeremy{done!} Notice that for
$ \mathsf{cast}^{\uparrow} $ to work, we need to provide the resulting type as
argument. This is because for the same term, there are more than one
choices for beta expansions (e.g., $1 + 2$ and $2 + 1$ are both the
beta expansions of $3$). \bruno{explain why for beta expansions we
  need to provide the resulting type as argument} \jeremy{done!}

A final point to make is that the \cast rules specify \emph{one-step}
reduction. This enables us to have more control over type-level
computation. The full technical details about \cast rules are presented
in Section~\ref{sec:ecore}.

\subsection{Decidability without Strong Normalization}

With explicit type conversion rules the decidability of type-checking 
no longer depends on the normalization property. 
A nice consequence of this is that the type system remains decidable
even in the presence of non-terminating programs at type level.
%As we will see in later sections. The
%ability to write non-terminating terms motivates us to have more
%control over type-level computation. 
% To illustrate, let us consider the following example. Suppose that $d$ is a ``dependent type'' where
% \[d : \mathsf{Int}  \rightarrow  \star\] so that $d\,3$ or $d\,100$ all yield the same
% type. With general recursion at hand, we can image a term $z$ that has
% type \[d\,\mathsf{loop}\] where $\mathsf{loop}$ stands for any
% diverging computation and of type $ \mathsf{Int} $. What would happen if we
% try to type check the following application: \[ \ottsym{(}  \lambda  \ottmv{x}  \ottsym{:}  \ottmv{d} \, 3  \ottsym{.}  \ottmv{x}  \ottsym{)} \, \ottmv{z}\]
% Under the normal typing rules of \coc, the type checker would get
% stuck as it tries to do $\beta$-equality on two terms: $d\,3$ and
% $d\,\mathsf{loop}$, where the latter is non-terminating.

To illustrate, let us consider the same example discussed in
Section~\ref{sec:coc}. Now the type checker will not get stuck when
type-checking the following application:
\[ \ottsym{(}  \lambda  \ottmv{x}  \ottsym{:}  \ottmv{d} \, 3  \ottsym{.}  \ottmv{x}  \ottsym{)} \, \ottmv{z} \]
where the type of $z$ is $d\,\mathsf{loop}$.  This is because in
\name, the type checker only does syntactic comparison between $d\,3$
and $d\,\mathsf{loop}$, instead of $\beta$-equality. Therefore it
rejects the above application as ill-typed. Indeed it is impossible to
type-check the application even with the use of $ \mathsf{cast}^{\uparrow} $ and/or
$ \mathsf{cast}_{\downarrow} $: one would need to write infinite number of
$ \mathsf{cast}_{\downarrow} $'s to make the type checker loop forever (e.g.,
$(\lambda  \ottmv{x}  \ottsym{:}  \ottmv{d} \, 3  \ottsym{.}  \ottmv{x})( \mathsf{cast}_{\downarrow} ( \mathsf{cast}_{\downarrow}  \dots z))$). But it is
impossible to write such program in reality.

In summary, \name achieves the decidability of type checking by
explicitly controlling type-level computation.  which is independent
of the normalization property, while supporting general recursion at
the same time.

\subsection{Recursion and Recursive Types}

\bruno{Show how in \name recursion and recursive types are unified.
Discuss that due to this unification the sensible choice for the
evaluation strategy is call-by-name. }

A simple extension to \name is to add a simple recursion construct.
With such an extension, it becomes possible to write standard
recursive programs at the term level. At the same time, the recursive
construct can also be used to model recursive types at the type-level.
Therefore, \name differs from other programming languages in that it
unifies both recursion and recursive types by the same $\mu$
primitive. With a single language construct we get two powerful
features!

\paragraph{Recursion}

The $\mu$ primitive can be used to define recursive functions.  For
example, the factorial function:
\begin{hscode}\SaveRestoreHook
\column{B}{@{}>{\hspre}l<{\hspost}@{}}%
\column{3}{@{}>{\hspre}l<{\hspost}@{}}%
\column{E}{@{}>{\hspre}l<{\hspost}@{}}%
\>[3]{}\mu\;\Varid{f}\mathbin{:}\Conid{Int}\to \Conid{Int}.\,\mathbf{if}\;\Varid{x}==\mathrm{0}\;\mathbf{then}\;\mathrm{1}\;\mathbf{else}\;\Varid{x}\;\times\;\Varid{f}\;(\Varid{x}\mathbin{-}\mathrm{1}){}\<[E]%
\ColumnHook
\end{hscode}\resethooks
The above recursive definition works because of the dynamic semantics of the
$\mu$ primitive: \ottusedrule{\ottdruleSXXMu{}} which is exactly doing
recursive unfolding of itself.

It is worth noting that the type $\tau$ in \ruleref{S\_Mu} is not
restricted to function types. This extra freedom allows us to define a
record of mutually recursive functions as the fixed point of a
function on records.

% The dynamic semantics of $\mu$ requires the recursive binder to satisfy (omit type annotations for clarity):  \[ \mu f.\,E = (\lambda f.\,E) (\mu f.\,E) \] which, however, does not terminate in strict languages. Therefore, to loosen the function-type restriction to allow any types, the sensible choice for the evaluation strategy is \emph{call-by-name}.

\subsubsection{Recursive types}
In the literature on type systems, there are two approaches to
recursive types, namely \emph{equi-recursive} and
\emph{iso-recursive}. The \emph{iso-recursive} approach treats a
recursive type and its unfolding as different, but isomorphic. The
isomorphism between a recursive type and its one step unfolding is
witnessed by traditionally \fold and \unfold operations. In \name, the
isomorphism is witnessed by first $ \mathsf{cast}^{\uparrow} $, then
$ \mathsf{cast}_{\downarrow} $. \bruno{Explain that the casts generalize fold and
  unfold!}  \jeremy{done!} At first sight, the \cast rules share some
similarities with \fold and \unfold, but $ \mathsf{cast}^{\uparrow} $ and
$ \mathsf{cast}_{\downarrow} $ actually generalize \fold and \unfold: they can convert
any types, not just recursive types. To demonstrate the use of the
\cast rules, let us consider a classic example of a recursive type,
the so-called ``hungry'' type~\cite{tapl}:
$H = \miu{\sigma}{\star}{\mathsf{Int} \rightarrow \sigma}$. A term $z$
of type $H$ can accept any number of integers and return a new
function that is hungry for more, as illustrated below:
\begin{align*}
\mathsf{cast}_{\downarrow} \, \ottmv{z} &:  \mathsf{Int}  \rightarrow H  \\
\mathsf{cast}_{\downarrow} \, \ottsym{(}  \mathsf{cast}_{\downarrow} \, \ottmv{z}  \ottsym{)} &:  \mathsf{Int}  \rightarrow  \mathsf{Int}  \rightarrow H \\
 \mathsf{cast}_{\downarrow} ( \mathsf{cast}_{\downarrow}  \dots z) &:  \mathsf{Int}  \rightarrow  \mathsf{Int}  \rightarrow \dots \rightarrow H
\end{align*}

% Due to the unification of recursive types and recursion, we can use
% the same $\mu$ primitive to write both recursive types and recursion
% with ease.

\paragraph{Call-by-Name}
Due to the unification, the \emph{call-by-value} evaluation strategy
does not fit in our setting. In call-by-value evaluation, recursion
can be expressed by the recursive binder $\mu$ as $\mu f : T
\rightarrow T.\, E$ (note that the type of $f$ is restricted to
function types). Since we don't want to pose restrictions on the
types, the \emph{call-by-name} evaluation is a sensible choice.
\bruno{Probably needs to be improved. I'll came back to this later!}

\subsection{Logical Inconsistency}

\bruno{Explain that the \name is inconsistent and discuss that this is
  a deliberate design decision, since we want to model languages like
  Haskell, which are logically inconsistent as well.} \bruno{Discuss
  the $* : *$ rule: since we already have inconsistency, having this
  rule adds expressiveness and simplifies the system.} \jeremy{added!}

One consequence of adding general recursion to the type system is that
the logical consistency of the system is broken. This is a deliberate
design decision, since our goal is to model languages like Haskell,
which are logically inconsistent as well.

In light of the fact that we decide to give up consistency, we take
another step further by declaring that the kind $\star$ is of type
$\star$. As it turns out, having this rule adds expressiveness and
simplifies our system. We return to this issue in Section~\ref{sec:related}.


\subsection{Encoding Datatypes}

The explicit type conversion rules and the $\mu$ primitive facilitates
the encoding of recursive datatypes and recursive functions over
datatypes. While inductive datatypes can be encoded using either the
Church or the Scott encoding, we adopt the Scott encoding as it
encodes case analysis, making it more convenient to encode pattern
matching. We demonstrate the encoding method using a simple datatype
as a running example: Peano numbers.

The datatype declaration for Peano numbers in Haskell is:
\begin{hscode}\SaveRestoreHook
\column{B}{@{}>{\hspre}l<{\hspost}@{}}%
\column{4}{@{}>{\hspre}l<{\hspost}@{}}%
\column{E}{@{}>{\hspre}l<{\hspost}@{}}%
\>[4]{}\mathbf{data}\;\Conid{Nat}\mathrel{=}\Conid{Z}\mid \Conid{S}\;\Conid{Nat}{}\<[E]%
\ColumnHook
\end{hscode}\resethooks
In the Scott encoding, the encoding of the \emph{Nat} datatype
reflects how its two constructors are going to be used. Since
\emph{Nat} is a recursive datatype, we have to use recursive types at
some point to reflect its recursive nature. As it turns out, the typed
Scott encoding of \emph{Nat} is:
\begin{hscode}\SaveRestoreHook
\column{B}{@{}>{\hspre}l<{\hspost}@{}}%
\column{3}{@{}>{\hspre}l<{\hspost}@{}}%
\column{E}{@{}>{\hspre}l<{\hspost}@{}}%
\>[3]{}\mu\;\Conid{X}\mathbin{:}\star.\,\Pi\;\Conid{B}\mathbin{:}\star.\,\Conid{B}\to (\Conid{X}\to \Conid{B})\to \Conid{B}{}\<[E]%
\ColumnHook
\end{hscode}\resethooks
The function type \ensuremath{\Conid{B}\to (\Conid{X}\to \Conid{B})\to \Conid{B}} demystifies the recursive
nature of \emph{Nat}: $B$ corresponds to the type of the constructor
\emph{Z}, and \ensuremath{\Conid{X}\to \Conid{B}} corresponds to the type of the constructor
\emph{S}. The intuition is that any recursive use of the datatype in
the data constructors is replaced with the variable ($X$ in the case)
bound by $\mu$, and we make the resulting variable ($B$ in this case)
universally quantified so that elements of the datatype with different
result types can be used in the same program~\cite{gadts}.

Its two constructors can be encoded correspondingly via the \cast rules:
\begin{hscode}\SaveRestoreHook
\column{B}{@{}>{\hspre}l<{\hspost}@{}}%
\column{3}{@{}>{\hspre}l<{\hspost}@{}}%
\column{E}{@{}>{\hspre}l<{\hspost}@{}}%
\>[3]{}\Conid{Z}\mathrel{=}\mathsf{cast}^\uparrow\;[\mskip1.5mu \Conid{Nat}\mskip1.5mu]\;(\lambda \Conid{B}\mathbin{:}\star.\,\lambda \Varid{z}\mathbin{:}\Conid{B}.\,\lambda \Varid{f}\mathbin{:}\Conid{Nat}\to \Conid{B}.\,\Varid{z}){}\<[E]%
\\
\>[3]{}\Conid{S}\mathrel{=}\lambda \Varid{n}\mathbin{:}\Conid{Nat}.\,\mathsf{cast}^\uparrow\;[\mskip1.5mu \Conid{Nat}\mskip1.5mu]\;(\lambda \Conid{B}\mathbin{:}\star.\,\lambda \Varid{z}\mathbin{:}\Conid{B}.\,\lambda \Varid{f}\mathbin{:}\Conid{Nat}\to \Conid{B}.\,\Varid{f}\;\Varid{n}){}\<[E]%
\ColumnHook
\end{hscode}\resethooks
Thanks to the \cast rules, we can make use of the $ \mathsf{cast}^{\uparrow} $
operation to do type conversion between the recursive type and its
unfolding.

The last example defines a recursive function that adds two natural
numbers:
\begin{hscode}\SaveRestoreHook
\column{B}{@{}>{\hspre}l<{\hspost}@{}}%
\column{3}{@{}>{\hspre}l<{\hspost}@{}}%
\column{7}{@{}>{\hspre}l<{\hspost}@{}}%
\column{E}{@{}>{\hspre}l<{\hspost}@{}}%
\>[3]{}\mu\;\Varid{f}\mathbin{:}\Conid{Nat}\to \Conid{Nat}\to \Conid{Nat}.\,\lambda \Varid{n}\mathbin{:}\Conid{Nat}.\,\lambda \Varid{m}\mathbin{:}\Conid{Nat}.\,{}\<[E]%
\\
\>[3]{}\hsindent{4}{}\<[7]%
\>[7]{}(\mathsf{cast}_\downarrow\;\Varid{n})\;\Conid{Nat}\;\Varid{m}\;(\lambda \Varid{n'}\mathbin{:}\Conid{Nat}.\,\Conid{S}\;(\Varid{f}\;\Varid{n'}\;\Varid{m})){}\<[E]%
\ColumnHook
\end{hscode}\resethooks
The above definition quite resembles case analysis commonly seen in
modern functional programming languages. (We formalize the encoding of
case analysis in Section~\ref{sec:surface}.)



%%% Local Variables:
%%% mode: latex
%%% TeX-master: "../main"
%%% End:


%%% !!! WARNING: AUTO GENERATED. DO NOT MODIFY !!! %%%
%% ODER: format ==         = "\mathrel{==}"
%% ODER: format /=         = "\neq "
%
%
\makeatletter
\@ifundefined{lhs2tex.lhs2tex.sty.read}%
  {\@namedef{lhs2tex.lhs2tex.sty.read}{}%
   \newcommand\SkipToFmtEnd{}%
   \newcommand\EndFmtInput{}%
   \long\def\SkipToFmtEnd#1\EndFmtInput{}%
  }\SkipToFmtEnd

\newcommand\ReadOnlyOnce[1]{\@ifundefined{#1}{\@namedef{#1}{}}\SkipToFmtEnd}
\usepackage{amstext}
\usepackage{amssymb}
\usepackage{stmaryrd}
\DeclareFontFamily{OT1}{cmtex}{}
\DeclareFontShape{OT1}{cmtex}{m}{n}
  {<5><6><7><8>cmtex8
   <9>cmtex9
   <10><10.95><12><14.4><17.28><20.74><24.88>cmtex10}{}
\DeclareFontShape{OT1}{cmtex}{m}{it}
  {<-> ssub * cmtt/m/it}{}
\newcommand{\texfamily}{\fontfamily{cmtex}\selectfont}
\DeclareFontShape{OT1}{cmtt}{bx}{n}
  {<5><6><7><8>cmtt8
   <9>cmbtt9
   <10><10.95><12><14.4><17.28><20.74><24.88>cmbtt10}{}
\DeclareFontShape{OT1}{cmtex}{bx}{n}
  {<-> ssub * cmtt/bx/n}{}
\newcommand{\tex}[1]{\text{\texfamily#1}}	% NEU

\newcommand{\Sp}{\hskip.33334em\relax}


\newcommand{\Conid}[1]{\mathit{#1}}
\newcommand{\Varid}[1]{\mathit{#1}}
\newcommand{\anonymous}{\kern0.06em \vbox{\hrule\@width.5em}}
\newcommand{\plus}{\mathbin{+\!\!\!+}}
\newcommand{\bind}{\mathbin{>\!\!\!>\mkern-6.7mu=}}
\newcommand{\rbind}{\mathbin{=\mkern-6.7mu<\!\!\!<}}% suggested by Neil Mitchell
\newcommand{\sequ}{\mathbin{>\!\!\!>}}
\renewcommand{\leq}{\leqslant}
\renewcommand{\geq}{\geqslant}
\usepackage{polytable}

%mathindent has to be defined
\@ifundefined{mathindent}%
  {\newdimen\mathindent\mathindent\leftmargini}%
  {}%

\def\resethooks{%
  \global\let\SaveRestoreHook\empty
  \global\let\ColumnHook\empty}
\newcommand*{\savecolumns}[1][default]%
  {\g@addto@macro\SaveRestoreHook{\savecolumns[#1]}}
\newcommand*{\restorecolumns}[1][default]%
  {\g@addto@macro\SaveRestoreHook{\restorecolumns[#1]}}
\newcommand*{\aligncolumn}[2]%
  {\g@addto@macro\ColumnHook{\column{#1}{#2}}}

\resethooks

\newcommand{\onelinecommentchars}{\quad-{}- }
\newcommand{\commentbeginchars}{\enskip\{-}
\newcommand{\commentendchars}{-\}\enskip}

\newcommand{\visiblecomments}{%
  \let\onelinecomment=\onelinecommentchars
  \let\commentbegin=\commentbeginchars
  \let\commentend=\commentendchars}

\newcommand{\invisiblecomments}{%
  \let\onelinecomment=\empty
  \let\commentbegin=\empty
  \let\commentend=\empty}

\visiblecomments

\newlength{\blanklineskip}
\setlength{\blanklineskip}{0.66084ex}

\newcommand{\hsindent}[1]{\quad}% default is fixed indentation
\let\hspre\empty
\let\hspost\empty
\newcommand{\NB}{\textbf{NB}}
\newcommand{\Todo}[1]{$\langle$\textbf{To do:}~#1$\rangle$}

\EndFmtInput
\makeatother
%
%
%
%
%
%
% This package provides two environments suitable to take the place
% of hscode, called "plainhscode" and "arrayhscode". 
%
% The plain environment surrounds each code block by vertical space,
% and it uses \abovedisplayskip and \belowdisplayskip to get spacing
% similar to formulas. Note that if these dimensions are changed,
% the spacing around displayed math formulas changes as well.
% All code is indented using \leftskip.
%
% Changed 19.08.2004 to reflect changes in colorcode. Should work with
% CodeGroup.sty.
%
\ReadOnlyOnce{polycode.fmt}%
\makeatletter

\newcommand{\hsnewpar}[1]%
  {{\parskip=0pt\parindent=0pt\par\vskip #1\noindent}}

% can be used, for instance, to redefine the code size, by setting the
% command to \small or something alike
\newcommand{\hscodestyle}{}

% The command \sethscode can be used to switch the code formatting
% behaviour by mapping the hscode environment in the subst directive
% to a new LaTeX environment.

\newcommand{\sethscode}[1]%
  {\expandafter\let\expandafter\hscode\csname #1\endcsname
   \expandafter\let\expandafter\endhscode\csname end#1\endcsname}

% "compatibility" mode restores the non-polycode.fmt layout.

\newenvironment{compathscode}%
  {\par\noindent
   \advance\leftskip\mathindent
   \hscodestyle
   \let\\=\@normalcr
   \let\hspre\(\let\hspost\)%
   \pboxed}%
  {\endpboxed\)%
   \par\noindent
   \ignorespacesafterend}

\newcommand{\compaths}{\sethscode{compathscode}}

% "plain" mode is the proposed default.
% It should now work with \centering.
% This required some changes. The old version
% is still available for reference as oldplainhscode.

\newenvironment{plainhscode}%
  {\hsnewpar\abovedisplayskip
   \advance\leftskip\mathindent
   \hscodestyle
   \let\hspre\(\let\hspost\)%
   \pboxed}%
  {\endpboxed%
   \hsnewpar\belowdisplayskip
   \ignorespacesafterend}

\newenvironment{oldplainhscode}%
  {\hsnewpar\abovedisplayskip
   \advance\leftskip\mathindent
   \hscodestyle
   \let\\=\@normalcr
   \(\pboxed}%
  {\endpboxed\)%
   \hsnewpar\belowdisplayskip
   \ignorespacesafterend}

% Here, we make plainhscode the default environment.

\newcommand{\plainhs}{\sethscode{plainhscode}}
\newcommand{\oldplainhs}{\sethscode{oldplainhscode}}
\plainhs

% The arrayhscode is like plain, but makes use of polytable's
% parray environment which disallows page breaks in code blocks.

\newenvironment{arrayhscode}%
  {\hsnewpar\abovedisplayskip
   \advance\leftskip\mathindent
   \hscodestyle
   \let\\=\@normalcr
   \(\parray}%
  {\endparray\)%
   \hsnewpar\belowdisplayskip
   \ignorespacesafterend}

\newcommand{\arrayhs}{\sethscode{arrayhscode}}

% The mathhscode environment also makes use of polytable's parray 
% environment. It is supposed to be used only inside math mode 
% (I used it to typeset the type rules in my thesis).

\newenvironment{mathhscode}%
  {\parray}{\endparray}

\newcommand{\mathhs}{\sethscode{mathhscode}}

% texths is similar to mathhs, but works in text mode.

\newenvironment{texthscode}%
  {\(\parray}{\endparray\)}

\newcommand{\texths}{\sethscode{texthscode}}

% The framed environment places code in a framed box.

\def\codeframewidth{\arrayrulewidth}
\RequirePackage{calc}

\newenvironment{framedhscode}%
  {\parskip=\abovedisplayskip\par\noindent
   \hscodestyle
   \arrayrulewidth=\codeframewidth
   \tabular{@{}|p{\linewidth-2\arraycolsep-2\arrayrulewidth-2pt}|@{}}%
   \hline\framedhslinecorrect\\{-1.5ex}%
   \let\endoflinesave=\\
   \let\\=\@normalcr
   \(\pboxed}%
  {\endpboxed\)%
   \framedhslinecorrect\endoflinesave{.5ex}\hline
   \endtabular
   \parskip=\belowdisplayskip\par\noindent
   \ignorespacesafterend}

\newcommand{\framedhslinecorrect}[2]%
  {#1[#2]}

\newcommand{\framedhs}{\sethscode{framedhscode}}

% The inlinehscode environment is an experimental environment
% that can be used to typeset displayed code inline.

\newenvironment{inlinehscode}%
  {\(\def\column##1##2{}%
   \let\>\undefined\let\<\undefined\let\\\undefined
   \newcommand\>[1][]{}\newcommand\<[1][]{}\newcommand\\[1][]{}%
   \def\fromto##1##2##3{##3}%
   \def\nextline{}}{\) }%

\newcommand{\inlinehs}{\sethscode{inlinehscode}}

% The joincode environment is a separate environment that
% can be used to surround and thereby connect multiple code
% blocks.

\newenvironment{joincode}%
  {\let\orighscode=\hscode
   \let\origendhscode=\endhscode
   \def\endhscode{\def\hscode{\endgroup\def\@currenvir{hscode}\\}\begingroup}
   %\let\SaveRestoreHook=\empty
   %\let\ColumnHook=\empty
   %\let\resethooks=\empty
   \orighscode\def\hscode{\endgroup\def\@currenvir{hscode}}}%
  {\origendhscode
   \global\let\hscode=\orighscode
   \global\let\endhscode=\origendhscode}%

\makeatother
\EndFmtInput
%


\section{\sufcc by Example}
\label{sec:app}

\bruno{Wrong title! This section is not about \name; it is about source languages that can be built on top of name!} \jeremy{this name for the moment}

\bruno{General comment is that, although the material is good, the text is a bit informally written.
Text needs to be polsihed. Also the text is lacking references.}

This sections shows a number of programs written in the surface
language \sufcc, which in built on top of \name. Most of these
examples either require non-trivial extensions of Haskell, or are
non-trivial to encode in dependently typed language like Coq or
Agda. The formalization of the surface language is presented in
Section~\ref{sec:surface}.

\begin{comment}
\subsection{Parametric HOAS}
\label{sec:phoas}

Parametric Higher Order Abstract Syntax (PHOAS) is a higher order
approach to represent binders, in which the function space of the
meta-language is used to encode the binders of the object language. We
show that \name can handle PHOAS by encoding lambda calculus as below:

\begin{figure}[h!]
\begin{hscode}\SaveRestoreHook
\column{B}{@{}>{\hspre}l<{\hspost}@{}}%
\column{4}{@{}>{\hspre}l<{\hspost}@{}}%
\column{E}{@{}>{\hspre}l<{\hspost}@{}}%
\>[B]{}\mathbf{data}\;\Conid{PLambda}\;(\Varid{a}\mathbin{:}\star)\mathrel{=}\Conid{Var}\;\Varid{a}{}\<[E]%
\\
\>[B]{}\hsindent{4}{}\<[4]%
\>[4]{}\mid \Conid{Num}\;\Varid{nat}{}\<[E]%
\\
\>[B]{}\hsindent{4}{}\<[4]%
\>[4]{}\mid \Conid{Lam}\;(\Varid{a}\to \Conid{PLambda}\;\Varid{a}){}\<[E]%
\\
\>[B]{}\hsindent{4}{}\<[4]%
\>[4]{}\mid \Conid{App}\;(\Conid{PLambda}\;\Varid{a})\;(\Conid{PLambda}\;\Varid{a});{}\<[E]%
\ColumnHook
\end{hscode}\resethooks
\end{figure}

Next we define the evaluator for our lambda calculus. One advantage of
PHOAS is that, environments are implicitly handled by the
meta-language, thus the type of the evaluator is simply \ensuremath{\Varid{plambda}\;\Varid{value}\to \Varid{value}}. The code is presented in Figure~\ref{fig:phoas}.

\begin{figure}[ht]
\begin{hscode}\SaveRestoreHook
\column{B}{@{}>{\hspre}l<{\hspost}@{}}%
\column{4}{@{}>{\hspre}l<{\hspost}@{}}%
\column{6}{@{}>{\hspre}l<{\hspost}@{}}%
\column{8}{@{}>{\hspre}l<{\hspost}@{}}%
\column{9}{@{}>{\hspre}l<{\hspost}@{}}%
\column{10}{@{}>{\hspre}l<{\hspost}@{}}%
\column{11}{@{}>{\hspre}l<{\hspost}@{}}%
\column{13}{@{}>{\hspre}l<{\hspost}@{}}%
\column{24}{@{}>{\hspre}l<{\hspost}@{}}%
\column{35}{@{}>{\hspre}l<{\hspost}@{}}%
\column{E}{@{}>{\hspre}l<{\hspost}@{}}%
\>[B]{}\mathbf{data}\;\Conid{Value}{}\<[13]%
\>[13]{}\mathrel{=}\Conid{VI}\;\Varid{nat}{}\<[E]%
\\
\>[B]{}\hsindent{4}{}\<[4]%
\>[4]{}\mid \Conid{VF}\;(\Conid{Value}\to \Conid{Value});{}\<[E]%
\\
\>[B]{}\mathbf{let}\;\Varid{eval}\mathbin{:}\Conid{PLambda}\;\Conid{Value}\to \Conid{Value}\mathrel{=}{}\<[E]%
\\
\>[B]{}\hsindent{4}{}\<[4]%
\>[4]{}\mathsf{mu}\;\Varid{ev}\mathbin{:}\Conid{PLambda}\;\Conid{Value}\to \Conid{Value}.\,{}\<[E]%
\\
\>[4]{}\hsindent{2}{}\<[6]%
\>[6]{}\lambda \Varid{e}\mathbin{:}\Conid{PLambda}\;\Conid{Value}.\,\mathbf{case}\;\Varid{e}\;\mathbf{of}{}\<[E]%
\\
\>[6]{}\hsindent{2}{}\<[8]%
\>[8]{}\Conid{Var}\;(\Varid{v}\mathbin{:}\Conid{Value})\Rightarrow \Varid{v}{}\<[E]%
\\
\>[4]{}\hsindent{2}{}\<[6]%
\>[6]{}\mid \Conid{Num}\;(\Varid{n}\mathbin{:}\Varid{nat}){}\<[24]%
\>[24]{}\Rightarrow \Conid{VI}\;\Varid{n}{}\<[E]%
\\
\>[4]{}\hsindent{2}{}\<[6]%
\>[6]{}\mid \Conid{Lam}\;(\Varid{f}\mathbin{:}\Conid{Value}\to \Conid{PLambda}\;\Conid{Value})\Rightarrow {}\<[E]%
\\
\>[6]{}\hsindent{4}{}\<[10]%
\>[10]{}\Conid{VF}\;(\lambda \Varid{x}\mathbin{:}\Conid{Value}.\,\Varid{ev}\;(\Varid{f}\;\Varid{x})){}\<[E]%
\\
\>[4]{}\hsindent{2}{}\<[6]%
\>[6]{}\mid \Conid{App}\;(\Varid{a}\mathbin{:}\Conid{PLambda}\;\Conid{Value})\;(\Varid{b}\mathbin{:}\Conid{PLambda}\;\Conid{Value})\Rightarrow {}\<[E]%
\\
\>[6]{}\hsindent{3}{}\<[9]%
\>[9]{}\mathbf{case}\;(\Varid{ev}\;\Varid{a})\;\mathbf{of}{}\<[E]%
\\
\>[9]{}\hsindent{2}{}\<[11]%
\>[11]{}\Conid{VI}\;(\Varid{n}\mathbin{:}\Varid{nat}){}\<[35]%
\>[35]{}\Rightarrow \Conid{VI}\;\Varid{n}\mbox{\onelinecomment  impossible to reach}{}\<[E]%
\\
\>[6]{}\hsindent{3}{}\<[9]%
\>[9]{}\mid \Conid{VF}\;(\Varid{f}\mathbin{:}\Conid{Value}\to \Conid{Value})\Rightarrow \Varid{f}\;(\Varid{ev}\;\Varid{b}){}\<[E]%
\\
\>[B]{}\mathbf{in}{}\<[E]%
\ColumnHook
\end{hscode}\resethooks
  \caption{Lambda Calculus in PHAOS}
  \label{fig:phoas}
\end{figure}

Now we can evaluate some lambda expression and get the result back as
in Figure~\ref{fig:pex}

\begin{figure}[ht]
\begin{hscode}\SaveRestoreHook
\column{B}{@{}>{\hspre}l<{\hspost}@{}}%
\column{3}{@{}>{\hspre}l<{\hspost}@{}}%
\column{5}{@{}>{\hspre}l<{\hspost}@{}}%
\column{7}{@{}>{\hspre}l<{\hspost}@{}}%
\column{29}{@{}>{\hspre}l<{\hspost}@{}}%
\column{E}{@{}>{\hspre}l<{\hspost}@{}}%
\>[B]{}\mathbf{let}\;\Varid{show}\mathbin{:}\Conid{Value}\to \Varid{nat}\mathrel{=}{}\<[E]%
\\
\>[B]{}\hsindent{3}{}\<[3]%
\>[3]{}\lambda \Varid{e}\mathbin{:}\Conid{Value}.\,\mathbf{case}\;\Varid{e}\;\mathbf{of}{}\<[E]%
\\
\>[3]{}\hsindent{2}{}\<[5]%
\>[5]{}\Conid{VI}\;(\Varid{n}\mathbin{:}\Varid{nat}){}\<[29]%
\>[29]{}\Rightarrow \Varid{n}{}\<[E]%
\\
\>[B]{}\hsindent{3}{}\<[3]%
\>[3]{}\mid \Conid{VF}\;(\Varid{f}\mathbin{:}\Conid{Value}\to \Conid{Value})\Rightarrow \mathrm{10000}\mbox{\onelinecomment  impossible to reach}{}\<[E]%
\\
\>[B]{}\mathbf{in}{}\<[E]%
\\
\>[B]{}\mathbf{let}\;\Varid{example}\mathbin{:}\Conid{PLambda}\;\Conid{Value}\mathrel{=}{}\<[E]%
\\
\>[B]{}\hsindent{3}{}\<[3]%
\>[3]{}\Conid{App}\;\Conid{Value}\;{}\<[E]%
\\
\>[3]{}\hsindent{4}{}\<[7]%
\>[7]{}(\Conid{Lam}\;\Conid{Value}\;(\lambda \Varid{x}\mathbin{:}\Conid{Value}.\,\Conid{Var}\;\Conid{Value}\;\Conid{X}))\;{}\<[E]%
\\
\>[3]{}\hsindent{4}{}\<[7]%
\>[7]{}(\Conid{Num}\;\Conid{Value}\;\mathrm{42}){}\<[E]%
\\
\>[B]{}\mathbf{in}\;\Varid{show}\;(\Varid{eval}\;\Varid{example})\mbox{\onelinecomment  return 42}{}\<[E]%
\ColumnHook
\end{hscode}\resethooks
\caption{Example of using PHOAS}
\label{fig:pex}
\end{figure}
\end{comment}

\subsubsection{Datatypes}
Conventional datatypes like natural numbers or polymorphic lists can
be easily defined in \sufcc, \bruno{This is not name; its the source
  language built on top of name!} \jeremy{changed} as in Haskell. For
example, below is the definition of polymorphic lists:
\begin{hscode}\SaveRestoreHook
\column{B}{@{}>{\hspre}l<{\hspost}@{}}%
\column{4}{@{}>{\hspre}l<{\hspost}@{}}%
\column{E}{@{}>{\hspre}l<{\hspost}@{}}%
\>[4]{}\mathbf{data}\;\Conid{List}\;(\Varid{a}\mathbin{:}\star)\mathrel{=}\Conid{Nil}\mid \Conid{Cons}\;(\Varid{x}\mathbin{:}\Varid{a})\;(\Varid{xs}\mathbin{:}\Conid{List}\;\Varid{a});{}\<[E]%
\ColumnHook
\end{hscode}\resethooks

Because \sufcc \bruno{You'll have to stop referring to \name in this
  section. You may want to consider giving the source language a
  name.} \jeremy{changed} is explicitly typed, each type parameter
needs to be accompanied with a corresponding kind expression. The use
of the above datatype is best illustrated by the \emph{length}
function:
\begin{hscode}\SaveRestoreHook
\column{B}{@{}>{\hspre}l<{\hspost}@{}}%
\column{3}{@{}>{\hspre}l<{\hspost}@{}}%
\column{5}{@{}>{\hspre}l<{\hspost}@{}}%
\column{8}{@{}>{\hspre}l<{\hspost}@{}}%
\column{E}{@{}>{\hspre}l<{\hspost}@{}}%
\>[3]{}\mathbf{letrec}\;\Varid{length}\mathbin{:}(\Varid{a}\mathbin{:}\star)\to \Conid{List}\;\Varid{a}\to \Varid{nat}\mathrel{=}{}\<[E]%
\\
\>[3]{}\hsindent{2}{}\<[5]%
\>[5]{}\lambda \Varid{a}\mathbin{:}\star.\,\lambda \Varid{l}\mathbin{:}\Conid{List}\;\Varid{a}.\,\mathbf{case}\;\Varid{l}\;\mathbf{of}{}\<[E]%
\\
\>[5]{}\hsindent{3}{}\<[8]%
\>[8]{}\Conid{Nil}\Rightarrow \mathrm{0}{}\<[E]%
\\
\>[3]{}\hsindent{2}{}\<[5]%
\>[5]{}\mid {}\<[8]%
\>[8]{}\Conid{Cons}\;(\Varid{x}\mathbin{:}\Varid{a})\;(\Varid{xs}\mathbin{:}\Conid{List}\;\Varid{a})\Rightarrow \mathrm{1}\mathbin{+}\Varid{length}\;\Varid{a}\;\Varid{xs}{}\<[E]%
\\
\>[3]{}\mathbf{in}{}\<[E]%
\\
\>[3]{}\mathbf{let}\;\Varid{test}\mathbin{:}\Conid{List}\;\Varid{nat}\mathrel{=}\Conid{Cons}\;\Varid{nat}\;\mathrm{1}\;(\Conid{Cons}\;\Varid{nat}\;\mathrm{2}\;(\Conid{Nil}\;\Varid{nat})){}\<[E]%
\\
\>[3]{}\mathbf{in}\;\Varid{length}\;\Varid{nat}\;\Varid{test}\mbox{\onelinecomment  return 2}{}\<[E]%
\ColumnHook
\end{hscode}\resethooks
\subsubsection{Higher-kinded Types}
Higher-kinded types are types that take other types and produce a new
type. To support higher-kinded types, languages like Haskell have to
extend their existing core languages to account for kind expressions.
(The existing core language of Haskell, System FC, is an extension of
System $F_{\omega}$~\cite{systemfw}, which naively supports
higher-kinded types.) \bruno{Probably want to mention $F_{\omega}$}
\jeremy{done!}  Given that \sufcc subsumes System $F_{\omega}$, we can
easily construct higher-kinded types. We show this by an example of
encoding the \emph{Functor} class:
\begin{hscode}\SaveRestoreHook
\column{B}{@{}>{\hspre}l<{\hspost}@{}}%
\column{3}{@{}>{\hspre}l<{\hspost}@{}}%
\column{5}{@{}>{\hspre}l<{\hspost}@{}}%
\column{E}{@{}>{\hspre}l<{\hspost}@{}}%
\>[3]{}\mathbf{rcrd}\;\Conid{Functor}\;(\Varid{f}\mathbin{:}\star\to \star)\mathrel{=}{}\<[E]%
\\
\>[3]{}\hsindent{2}{}\<[5]%
\>[5]{}\Conid{Func}\;\{\mskip1.5mu \Varid{fmap}\mathbin{:}(\Varid{a}\mathbin{:}\star)\to (\Varid{b}\mathbin{:}\star)\to (\Varid{a}\to \Varid{b})\to \Varid{f}\;\Varid{a}\to \Varid{f}\;\Varid{b}\mskip1.5mu\};{}\<[E]%
\ColumnHook
\end{hscode}\resethooks
A functor is just a record that has only one field \emph{fmap}. A
Functor instance of the \emph{Maybe} datatype is:
\begin{hscode}\SaveRestoreHook
\column{B}{@{}>{\hspre}l<{\hspost}@{}}%
\column{3}{@{}>{\hspre}l<{\hspost}@{}}%
\column{5}{@{}>{\hspre}l<{\hspost}@{}}%
\column{6}{@{}>{\hspre}c<{\hspost}@{}}%
\column{6E}{@{}l@{}}%
\column{7}{@{}>{\hspre}l<{\hspost}@{}}%
\column{9}{@{}>{\hspre}l<{\hspost}@{}}%
\column{E}{@{}>{\hspre}l<{\hspost}@{}}%
\>[3]{}\mathbf{let}\;\Varid{maybeInst}\mathbin{:}\Conid{Functor}\;\Conid{Maybe}\mathrel{=}{}\<[E]%
\\
\>[3]{}\hsindent{2}{}\<[5]%
\>[5]{}\Conid{Func}\;\Conid{Maybe}\;(\lambda \Varid{a}\mathbin{:}\star.\,\lambda \Varid{b}\mathbin{:}\star.\,\lambda \Varid{f}\mathbin{:}\Varid{a}\to \Varid{b}.\,\lambda \Varid{x}\mathbin{:}\Conid{Maybe}\;\Varid{a}.\,{}\<[E]%
\\
\>[5]{}\hsindent{2}{}\<[7]%
\>[7]{}\mathbf{case}\;\Varid{x}\;\mathbf{of}{}\<[E]%
\\
\>[7]{}\hsindent{2}{}\<[9]%
\>[9]{}\Conid{Nothing}\Rightarrow \Conid{Nothing}\;\Varid{b}{}\<[E]%
\\
\>[5]{}\hsindent{1}{}\<[6]%
\>[6]{}\mid {}\<[6E]%
\>[9]{}\Conid{Just}\;(\Varid{z}\mathbin{:}\Varid{a})\Rightarrow \Conid{Just}\;\Varid{b}\;(\Varid{f}\;\Varid{z})){}\<[E]%
\ColumnHook
\end{hscode}\resethooks
\subsubsection{HOAS}

\emph{Higher-order abstract syntax} is a representation of abstract
syntax where the function space of the meta-language is used to encode
the binders of the object language. Because of the recursive
occurrence of the datatype appears in a negative position (i.e., in
the left side of a function arrow) \bruno{explain where!}
\jeremy{done!}, systems like Coq and Agda would reject such programs using
HOAS due to the restrictiveness of their termination checkers. However
\sufcc is able to express HOAS in a straightforward way. We show an
example of encoding a simple lambda calculus:
\begin{hscode}\SaveRestoreHook
\column{B}{@{}>{\hspre}l<{\hspost}@{}}%
\column{3}{@{}>{\hspre}l<{\hspost}@{}}%
\column{5}{@{}>{\hspre}c<{\hspost}@{}}%
\column{5E}{@{}l@{}}%
\column{8}{@{}>{\hspre}l<{\hspost}@{}}%
\column{E}{@{}>{\hspre}l<{\hspost}@{}}%
\>[3]{}\mathbf{data}\;\Conid{Exp}\mathrel{=}\Conid{Num}\;(\Varid{n}\mathbin{:}\Varid{nat}){}\<[E]%
\\
\>[3]{}\hsindent{2}{}\<[5]%
\>[5]{}\mid {}\<[5E]%
\>[8]{}\Conid{Lam}\;(\Varid{f}\mathbin{:}\Conid{Exp}\to \Conid{Exp}){}\<[E]%
\\
\>[3]{}\hsindent{2}{}\<[5]%
\>[5]{}\mid {}\<[5E]%
\>[8]{}\Conid{App}\;(\Varid{a}\mathbin{:}\Conid{Exp})\;(\Varid{b}\mathbin{:}\Conid{Exp});{}\<[E]%
\ColumnHook
\end{hscode}\resethooks

Next we define the evaluator for our lambda calculus. As noted
by~\cite{Fegaras1996}, the evaluation function needs an extra function
\emph{reify} to invert the result of evaluation.
\begin{hscode}\SaveRestoreHook
\column{B}{@{}>{\hspre}l<{\hspost}@{}}%
\column{3}{@{}>{\hspre}l<{\hspost}@{}}%
\column{5}{@{}>{\hspre}l<{\hspost}@{}}%
\column{9}{@{}>{\hspre}l<{\hspost}@{}}%
\column{11}{@{}>{\hspre}l<{\hspost}@{}}%
\column{13}{@{}>{\hspre}l<{\hspost}@{}}%
\column{15}{@{}>{\hspre}l<{\hspost}@{}}%
\column{E}{@{}>{\hspre}l<{\hspost}@{}}%
\>[3]{}\mathbf{data}\;\Conid{Value}\mathrel{=}\Conid{VI}\;(\Varid{n}\mathbin{:}\Varid{nat})\mid \Conid{VF}\;(\Varid{f}\mathbin{:}\Conid{Value}\to \Conid{Value});{}\<[E]%
\\
\>[3]{}\mathbf{rcrd}\;\Conid{Eval}\mathrel{=}\Conid{Ev}\;\{\mskip1.5mu \Varid{eval'}\mathbin{:}\Conid{Exp}\to \Conid{Value},\Varid{reify'}\mathbin{:}\Conid{Value}\to \Conid{Exp}\mskip1.5mu\};{}\<[E]%
\\
\>[3]{}\mathbf{let}\;\Varid{f}\mathbin{:}\Conid{Eval}\mathrel{=}\mathsf{mu}\;\Varid{f'}\mathbin{:}\Conid{Eval}.\,{}\<[E]%
\\
\>[3]{}\hsindent{2}{}\<[5]%
\>[5]{}\Conid{Ev}\;{}\<[9]%
\>[9]{}(\lambda \Varid{e}\mathbin{:}\Conid{Exp}.\,\mathbf{case}\;\Varid{e}\;\mathbf{of}{}\<[E]%
\\
\>[9]{}\hsindent{2}{}\<[11]%
\>[11]{}\Conid{Num}\;(\Varid{n}\mathbin{:}\Varid{nat})\Rightarrow \Conid{VI}\;\Varid{n}{}\<[E]%
\\
\>[9]{}\mid \Conid{Lam}\;(\Varid{fun}\mathbin{:}\Conid{Exp}\to \Conid{Exp})\Rightarrow {}\<[E]%
\\
\>[9]{}\hsindent{4}{}\<[13]%
\>[13]{}\Conid{VF}\;(\lambda \Varid{e'}\mathbin{:}\Conid{Value}.\,\Varid{eval'}\;\Varid{f'}\;(\Varid{fun}\;(\Varid{reify'}\;\Varid{f'}\;\Varid{e'}))){}\<[E]%
\\
\>[9]{}\mid \Conid{App}\;(\Varid{a}\mathbin{:}\Conid{Exp})\;(\Varid{b}\mathbin{:}\Conid{Exp})\Rightarrow {}\<[E]%
\\
\>[9]{}\hsindent{4}{}\<[13]%
\>[13]{}\mathbf{case}\;\Varid{eval'}\;\Varid{f'}\;\Varid{a}\;\mathbf{of}{}\<[E]%
\\
\>[13]{}\hsindent{2}{}\<[15]%
\>[15]{}\Conid{VI}\;(\Varid{n}\mathbin{:}\Varid{nat})\Rightarrow \Varid{error}{}\<[E]%
\\
\>[9]{}\hsindent{4}{}\<[13]%
\>[13]{}\mid \Conid{VF}\;(\Varid{fun}\mathbin{:}\Conid{Value}\to \Conid{Value})\Rightarrow \Varid{fun}\;(\Varid{eval'}\;\Varid{f'}\;\Varid{b}))\;{}\<[E]%
\\
\>[9]{}(\lambda \Varid{v}\mathbin{:}\Conid{Value}.\,\mathbf{case}\;\Varid{v}\;\mathbf{of}{}\<[E]%
\\
\>[9]{}\hsindent{2}{}\<[11]%
\>[11]{}\Conid{VI}\;(\Varid{n}\mathbin{:}\Varid{nat})\Rightarrow \Conid{Num}\;\Varid{n}{}\<[E]%
\\
\>[9]{}\mid \Conid{VF}\;(\Varid{fun}\mathbin{:}\Conid{Value}\to \Conid{Value})\Rightarrow {}\<[E]%
\\
\>[9]{}\hsindent{4}{}\<[13]%
\>[13]{}\Conid{Lam}\;(\lambda \Varid{e'}\mathbin{:}\Conid{Exp}.\,(\Varid{reify'}\;\Varid{f'}\;(\Varid{fun}\;(\Varid{eval'}\;\Varid{f'}\;\Varid{e'}))))){}\<[E]%
\\
\>[3]{}\mathbf{in}\;\mathbf{let}\;\Varid{eval}\mathbin{:}\Conid{Exp}\to \Conid{Value}\mathrel{=}\Varid{eval'}\;\Varid{f}\;\mathbf{in}{}\<[E]%
\ColumnHook
\end{hscode}\resethooks

The definition of the evaluator is quite straightforward, although it
is worth noting that the evaluator is a partial function that can
cause run-time errors. Thanks to the flexibility of the $\mu$
primitive, mutual recursion can be encoded by using records!

Evaluation of a lambda expression proceeds as follows:
\begin{hscode}\SaveRestoreHook
\column{B}{@{}>{\hspre}l<{\hspost}@{}}%
\column{3}{@{}>{\hspre}l<{\hspost}@{}}%
\column{25}{@{}>{\hspre}l<{\hspost}@{}}%
\column{E}{@{}>{\hspre}l<{\hspost}@{}}%
\>[3]{}\mathbf{let}\;\Varid{test}\mathbin{:}\Conid{Exp}\mathrel{=}\Conid{App}\;{}\<[25]%
\>[25]{}(\Conid{Lam}\;(\lambda \Varid{f}\mathbin{:}\Conid{Exp}.\,\Conid{App}\;\Varid{f}\;(\Conid{Num}\;\mathrm{42})))\;{}\<[E]%
\\
\>[25]{}(\Conid{Lam}\;(\lambda \Varid{g}\mathbin{:}\Conid{Exp}.\,\Varid{g})){}\<[E]%
\\
\>[3]{}\mathbf{in}\;\Varid{show}\;(\Varid{eval}\;\Varid{test})\mbox{\onelinecomment  return 42}{}\<[E]%
\ColumnHook
\end{hscode}\resethooks
\subsubsection{Fix as a Datatype}
The type-level \emph{Fix} is a good example to demonstrate the
expressiveness of \sufcc. The definition is:
\begin{hscode}\SaveRestoreHook
\column{B}{@{}>{\hspre}l<{\hspost}@{}}%
\column{3}{@{}>{\hspre}l<{\hspost}@{}}%
\column{E}{@{}>{\hspre}l<{\hspost}@{}}%
\>[3]{}\mathbf{rcrd}\;\Conid{Fix}\;(\Varid{f}\mathbin{:}\star\to \star)\mathrel{=}\Conid{In}\;\{\mskip1.5mu \Varid{out}\mathbin{:}\Varid{f}\;(\Conid{Fix}\;\Varid{f})\mskip1.5mu\};{}\<[E]%
\ColumnHook
\end{hscode}\resethooks
The record notation also introduces the selector function:
\begin{hscode}\SaveRestoreHook
\column{B}{@{}>{\hspre}l<{\hspost}@{}}%
\column{3}{@{}>{\hspre}l<{\hspost}@{}}%
\column{E}{@{}>{\hspre}l<{\hspost}@{}}%
\>[3]{}\Varid{out}\mathbin{:}(\Varid{f}\mathbin{:}\star\to \star)\to \Conid{Fix}\;\Varid{f}\to \Varid{f}\;(\Conid{Fix}\;\Varid{f}){}\<[E]%
\ColumnHook
\end{hscode}\resethooks
The \emph{Fix} datatype is interesting in that now we can define
recursive datatypes in a non-recursive way. For instance, a
non-recursive definition for natural numbers is:
\begin{hscode}\SaveRestoreHook
\column{B}{@{}>{\hspre}l<{\hspost}@{}}%
\column{3}{@{}>{\hspre}l<{\hspost}@{}}%
\column{E}{@{}>{\hspre}l<{\hspost}@{}}%
\>[3]{}\mathbf{data}\;\Conid{NatF}\;(\Varid{self}\mathbin{:}\star)\mathrel{=}\Conid{Zero}\mid \Conid{Succ}\;\Varid{self};{}\<[E]%
\ColumnHook
\end{hscode}\resethooks
And the recursive version is just a synonym:
\begin{hscode}\SaveRestoreHook
\column{B}{@{}>{\hspre}l<{\hspost}@{}}%
\column{3}{@{}>{\hspre}l<{\hspost}@{}}%
\column{E}{@{}>{\hspre}l<{\hspost}@{}}%
\>[3]{}\mathbf{let}\;\Conid{Nat}\mathbin{:}\star\mathrel{=}\Conid{Fix}\;\Conid{NatF}{}\<[E]%
\ColumnHook
\end{hscode}\resethooks

Note that now we can use the above \emph{Nat} anywhere, including the
left-hand side of a function arrow, which is a potential source of
non-termination. The termination checker of Coq or Agda is so
conservative that it would brutally reject the definition of
\emph{Fix} to avoid the above situation. \bruno{show example?}
\jeremy{done!} However in \sufcc, where type-level computation is
explicitly controlled, we can safely use \emph{Fix} in the program.

Given \emph{fmap}, many recursive shcemes can be defined, for example
we can have \emph{catamorphism}~\cite{Meijer1991} \bruno{reference?}
\jeremy{done!} or generic function fold:
\begin{hscode}\SaveRestoreHook
\column{B}{@{}>{\hspre}l<{\hspost}@{}}%
\column{3}{@{}>{\hspre}l<{\hspost}@{}}%
\column{5}{@{}>{\hspre}l<{\hspost}@{}}%
\column{7}{@{}>{\hspre}l<{\hspost}@{}}%
\column{18}{@{}>{\hspre}l<{\hspost}@{}}%
\column{E}{@{}>{\hspre}l<{\hspost}@{}}%
\>[3]{}\mathbf{letrec}\;\Varid{cata}\mathbin{:}{}\<[18]%
\>[18]{}(\Varid{f}\mathbin{:}\star\to \star)\to (\Varid{a}\mathbin{:}\star)\to {}\<[E]%
\\
\>[18]{}\Conid{Functor}\;\Varid{f}\to (\Varid{f}\;\Varid{a}\to \Varid{a})\to \Conid{Fix}\;\Varid{f}\to \Varid{a}\mathrel{=}{}\<[E]%
\\
\>[3]{}\hsindent{2}{}\<[5]%
\>[5]{}\lambda \Varid{f}\mathbin{:}\star\to \star.\,\lambda \Varid{a}\mathbin{:}\star.\,\lambda \Varid{m}\mathbin{:}\Conid{Functor}\;\Varid{f}.\,\lambda \Varid{g}\mathbin{:}\Varid{f}\;\Varid{a}\to \Varid{a}.\,\lambda \Varid{t}\mathbin{:}\Conid{Fix}\;\Varid{f}.\,{}\<[E]%
\\
\>[5]{}\hsindent{2}{}\<[7]%
\>[7]{}\Varid{g}\;(\Varid{fmap}\;\Varid{f}\;\Varid{m}\;(\Conid{Fix}\;\Varid{f})\;\Varid{a}\;(\Varid{cata}\;\Varid{f}\;\Varid{a}\;\Varid{m}\;\Varid{g})\;(\Varid{out}\;\Varid{f}\;\Varid{t})){}\<[E]%
\ColumnHook
\end{hscode}\resethooks
\subsubsection{Kind Polymophism}
In Haskell, System FC~\cite{fc:pro} \bruno{reference} \jeremy{done!}
was proposed to support kind polymorphism. However it separates
expressions into terms, types and kinds, which complicates both the
implementation and future extensions. \sufcc natively allows datatype
definitions to have polymorphic kinds. Here is an example, taken
from~\cite{fc:pro}, of a datatype that benefits from kind polymophism:
a higher-kinded fixpoint combinator:
\begin{hscode}\SaveRestoreHook
\column{B}{@{}>{\hspre}l<{\hspost}@{}}%
\column{3}{@{}>{\hspre}l<{\hspost}@{}}%
\column{5}{@{}>{\hspre}l<{\hspost}@{}}%
\column{E}{@{}>{\hspre}l<{\hspost}@{}}%
\>[3]{}\mathbf{data}\;\Conid{Mu}\;(\Varid{k}\mathbin{:}\star)\;(\Varid{f}\mathbin{:}(\Varid{k}\to \star)\to \Varid{k}\to \star)\;(\Varid{a}\mathbin{:}\Varid{k})\mathrel{=}{}\<[E]%
\\
\>[3]{}\hsindent{2}{}\<[5]%
\>[5]{}\Conid{Roll}\;(\Varid{g}\mathbin{:}\Varid{f}\;(\Conid{Mu}\;\Varid{k}\;\Varid{f})\;\Varid{a});{}\<[E]%
\ColumnHook
\end{hscode}\resethooks
\emph{Mu} can be used to construct polymorphic recursive types of any kind, for instance:
\begin{hscode}\SaveRestoreHook
\column{B}{@{}>{\hspre}l<{\hspost}@{}}%
\column{3}{@{}>{\hspre}l<{\hspost}@{}}%
\column{5}{@{}>{\hspre}l<{\hspost}@{}}%
\column{E}{@{}>{\hspre}l<{\hspost}@{}}%
\>[3]{}\mathbf{data}\;\Conid{Listf}\;(\Varid{f}\mathbin{:}\star\to \star)\;(\Varid{a}\mathbin{:}\star)\mathrel{=}\Conid{Nil}{}\<[E]%
\\
\>[3]{}\hsindent{2}{}\<[5]%
\>[5]{}\mid \Conid{Cons}\;(\Varid{x}\mathbin{:}\Varid{a})\;(\Varid{xs}\mathbin{:}(\Varid{f}\;\Varid{a}));{}\<[E]%
\\
\>[3]{}\mathbf{let}\;\Conid{List}\mathbin{:}\star\to \star\mathrel{=}\lambda \Varid{a}\mathbin{:}\star.\,\Conid{Mu}\star\Conid{Listf}\;\Varid{a}{}\<[E]%
\ColumnHook
\end{hscode}\resethooks
\subsubsection{Nested Datatypes}

A nested datatype~\cite{nesteddt} \bruno{reference} \jeremy{done!},
also known as a \emph{non-regular} datatype, is a parametrised
datatype whose definition contains different instances of the type
parameters. Functions over nested datatypes usually involve
polymorphic recursion. We show that \sufcc is capable of defining all
useful functions over a nested datatype. A simple example would be the
type \emph{Pow} of power trees, whose size is exactly a power of two,
declared as follows:
\begin{hscode}\SaveRestoreHook
\column{B}{@{}>{\hspre}l<{\hspost}@{}}%
\column{3}{@{}>{\hspre}l<{\hspost}@{}}%
\column{5}{@{}>{\hspre}l<{\hspost}@{}}%
\column{E}{@{}>{\hspre}l<{\hspost}@{}}%
\>[3]{}\mathbf{data}\;\Conid{PairT}\;(\Varid{a}\mathbin{:}\star)\mathrel{=}\Conid{P}\;(\Varid{x}\mathbin{:}\Varid{a})\;(\Varid{x}\mathbin{:}\Varid{a});{}\<[E]%
\\
\>[3]{}\mathbf{data}\;\Conid{Pow}\;(\Varid{a}\mathbin{:}\star)\mathrel{=}\Conid{Zero}\;(\Varid{n}\mathbin{:}\Varid{a}){}\<[E]%
\\
\>[3]{}\hsindent{2}{}\<[5]%
\>[5]{}\mid \Conid{Succ}\;(\Varid{t}\mathbin{:}\Conid{Pow}\;(\Conid{PairT}\;\Varid{a}));{}\<[E]%
\ColumnHook
\end{hscode}\resethooks
Notice that the recursive mention of \emph{Pow} does not hold
\emph{a}, but \emph{PairT a}. This means every time we use a
\emph{Succ} constructor, the size of the pairs doubles. In case you
are curious about the encoding of \emph{Pow}, here is the one:
\begin{hscode}\SaveRestoreHook
\column{B}{@{}>{\hspre}l<{\hspost}@{}}%
\column{3}{@{}>{\hspre}l<{\hspost}@{}}%
\column{7}{@{}>{\hspre}l<{\hspost}@{}}%
\column{E}{@{}>{\hspre}l<{\hspost}@{}}%
\>[3]{}\mathbf{let}\;\Conid{Pow}\mathbin{:}\star\to \star\mathrel{=}\mathsf{mu}\;\Conid{X}\mathbin{:}\star\to \star.\,{}\<[E]%
\\
\>[3]{}\hsindent{4}{}\<[7]%
\>[7]{}\lambda \Varid{a}\mathbin{:}\star.\,(\Conid{B}\mathbin{:}\star)\to (\Varid{a}\to \Conid{B})\to (\Conid{X}\;(\Conid{PairT}\;\Varid{a})\to \Conid{B})\to \Conid{B}{}\<[E]%
\ColumnHook
\end{hscode}\resethooks
Notice how the higher-kinded type variable \ensuremath{\Conid{X}\mathbin{:}\star\to \star} helps encoding
nested datatypes. Below is a simple function \emph{toList} that
transforms a power tree into a list:
\begin{hscode}\SaveRestoreHook
\column{B}{@{}>{\hspre}l<{\hspost}@{}}%
\column{3}{@{}>{\hspre}l<{\hspost}@{}}%
\column{5}{@{}>{\hspre}l<{\hspost}@{}}%
\column{8}{@{}>{\hspre}l<{\hspost}@{}}%
\column{10}{@{}>{\hspre}l<{\hspost}@{}}%
\column{12}{@{}>{\hspre}l<{\hspost}@{}}%
\column{15}{@{}>{\hspre}l<{\hspost}@{}}%
\column{17}{@{}>{\hspre}l<{\hspost}@{}}%
\column{E}{@{}>{\hspre}l<{\hspost}@{}}%
\>[3]{}\mathbf{letrec}\;\Varid{toList}\mathbin{:}(\Varid{a}\mathbin{:}\star)\to \Conid{Pow}\;\Varid{a}\to \Conid{List}\;\Varid{a}\mathrel{=}{}\<[E]%
\\
\>[3]{}\hsindent{2}{}\<[5]%
\>[5]{}\lambda \Varid{a}\mathbin{:}\star.\,\lambda \Varid{t}\mathbin{:}\Conid{Pow}\;\Varid{a}.\,\mathbf{case}\;\Varid{t}\;\mathbf{of}{}\<[E]%
\\
\>[5]{}\hsindent{3}{}\<[8]%
\>[8]{}\Conid{Zero}\;(\Varid{x}\mathbin{:}\Varid{a})\Rightarrow \Conid{Cons}\;\Varid{a}\;\Varid{x}\;(\Conid{Nil}\;\Varid{a}){}\<[E]%
\\
\>[3]{}\hsindent{2}{}\<[5]%
\>[5]{}\mid {}\<[8]%
\>[8]{}\Conid{Succ}\;(\Varid{c}\mathbin{:}\Conid{Pow}\;(\Conid{PairT}\;\Varid{a}))\Rightarrow {}\<[E]%
\\
\>[8]{}\hsindent{2}{}\<[10]%
\>[10]{}\Varid{concatMap}\;(\Conid{PairT}\;\Varid{a})\;\Varid{a}\;{}\<[E]%
\\
\>[10]{}\hsindent{2}{}\<[12]%
\>[12]{}(\lambda \Varid{x}\mathbin{:}\Conid{PairT}\;\Varid{a}.\,\mathbf{case}\;\Varid{x}\;\mathbf{of}{}\<[E]%
\\
\>[12]{}\hsindent{3}{}\<[15]%
\>[15]{}\Conid{P}\;(\Varid{m}\mathbin{:}\Varid{a})\;(\Varid{n}\mathbin{:}\Varid{a})\Rightarrow {}\<[E]%
\\
\>[15]{}\hsindent{2}{}\<[17]%
\>[17]{}\Conid{Cons}\;\Varid{a}\;\Varid{m}\;(\Conid{Cons}\;\Varid{a}\;\Varid{n}\;(\Conid{Nil}\;\Varid{a})))\;{}\<[E]%
\\
\>[10]{}\hsindent{2}{}\<[12]%
\>[12]{}(\Varid{toList}\;(\Conid{PairT}\;\Varid{a})\;\Varid{c}){}\<[E]%
\ColumnHook
\end{hscode}\resethooks
\subsubsection{Data Promotion}
\bruno{what is the point that we are trying to make with this example? Title is wrong;
should be about the point, not about the particular example!} \jeremy{This section shows we can do data promotion much more easily than in Haskell}

Haskell needs sophisticated extensions~\cite{fc:pro} in order for
being able to use ordinary datatypes as kinds, and data constructors
as types. With the full power of dependent types, data promotion is
made trivial in \sufcc.

As a last example, we show a representation of a labeled binary tree,
where each node is labeled with its depth in the tree. Below is the
definition:
\begin{hscode}\SaveRestoreHook
\column{B}{@{}>{\hspre}l<{\hspost}@{}}%
\column{3}{@{}>{\hspre}l<{\hspost}@{}}%
\column{5}{@{}>{\hspre}l<{\hspost}@{}}%
\column{E}{@{}>{\hspre}l<{\hspost}@{}}%
\>[3]{}\mathbf{data}\;\Conid{PTree}\;(\Varid{n}\mathbin{:}\Conid{Nat})\mathrel{=}\Conid{Empty}{}\<[E]%
\\
\>[3]{}\hsindent{2}{}\<[5]%
\>[5]{}\mid \Conid{Fork}\;(\Varid{z}\mathbin{:}\Varid{nat})\;(\Varid{x}\mathbin{:}\Conid{PTree}\;(\Conid{S}\;\Varid{n}))\;(\Varid{y}\mathbin{:}\Conid{PTree}\;(\Conid{S}\;\Varid{n}));{}\<[E]%
\ColumnHook
\end{hscode}\resethooks
Notice how the datatype \emph{Nat} is ``promoted'' to be used in the
kind level. Next we can construct such a binary tree that keeps track
of its depth statically:\begin{hscode}\SaveRestoreHook
\column{B}{@{}>{\hspre}l<{\hspost}@{}}%
\column{3}{@{}>{\hspre}l<{\hspost}@{}}%
\column{E}{@{}>{\hspre}l<{\hspost}@{}}%
\>[3]{}\Conid{Fork}\;\Conid{Z}\;\mathrm{1}\;(\Conid{Empty}\;(\Conid{S}\;\Conid{Z}))\;(\Conid{Empty}\;(\Conid{S}\;\Conid{Z})){}\<[E]%
\ColumnHook
\end{hscode}\resethooks
If we accidentally write the wrong depth, for example:\begin{hscode}\SaveRestoreHook
\column{B}{@{}>{\hspre}l<{\hspost}@{}}%
\column{3}{@{}>{\hspre}l<{\hspost}@{}}%
\column{E}{@{}>{\hspre}l<{\hspost}@{}}%
\>[3]{}\Conid{Fork}\;\Conid{Z}\;\mathrm{1}\;(\Conid{Empty}\;(\Conid{S}\;\Conid{Z}))\;(\Conid{Empty}\;\Conid{Z}){}\<[E]%
\ColumnHook
\end{hscode}\resethooks
The above will fail to pass type checking.

\bruno{Two questions: firstly does it work? secondly do we support GADT syntax now?}  \jeremy{changed to a simple binary tree example}

\bruno{More examples? closed type families; dependent types?} \jeremy{had hard time thinking of a simple, non-recursive example for type families}

%include sections/example.lhs

%%% !!! WARNING: AUTO GENERATED. DO NOT MODIFY !!! %%%

\section{The Explicit Calculus of Constructions}
\label{sec:formal}

\bruno{Linus: can you write up this section? I think this section should be your priority.
First bring in all results and formalization: syntax; semantics; proofs ... then write text}

% This section formalizes the syntax and semantics of the explicit calculus 
% of constructions. This section also shows that how in the explicit 
% calculus of constructions decidability of the type system does not 
% depend on strong normalization.

% \begin{itemize}
% \item Give an overview of the core language and its syntax.
% \item Show the typing rules and operational semantics.
% \item The original formalization is suggested to rewrite using \textsf{ott}\footnote{\url{http://www.cl.cam.ac.uk/~pes20/ott/}} which is a standard in academia. For example, the formalization of GHC \url{https://github.com/ghc/ghc/tree/master/docs/core-spec}.
% \item Give formal proof of the soundness of the core language.
% \item Subject reduction and progress theorems will be proved.
% \end{itemize}

\newcommand{\expcc}{\textsf{ExpCC}\xspace}
\newcommand{\cc}{\textsf{CC}\xspace}
\newcommand{\gram}[1]{\ottgrammartabular{#1\ottafterlastrule}}
\newcommand{\ruleref}[1]{\ottdrulename{#1}}

In this section, we present a variant of the Calculus of Constructions (\cc), called \emph{explicit} Calculus of Constructions (\expcc), which is the foundation of our core language \name. \expcc can be regarded as \name without general recursion, so that has more straightforward properties and metatheory. It is suitable for illustrating the core idea of our design, that is to control $\beta$-reduction at the type level by introducing \emph{explicit} type conversion semantics. This also brings a benefit to type checking of \expcc, that the strong normalization is no long necessary to achieve the decidability of type checking. In the following part of this section, we give explanation of these properties by showing the syntax, static and dynamic semantics and the metatheory of \expcc.

\subsection{Syntax}\label{sec:ecc:syn}
The basic syntax of \expcc is shown in Figure \ref{fig:ecc:syntax}, which gives abstract syntax of expressions, sorts, contexts and values. Just like \cc, \expcc has two main advantages of keeping syntax concise when compared to the System $F$ families including System $F_\omega$ and $F_C$. One is that \expcc uses a single syntactic level to represent terms, types and kinds, which are usually distinguished in System $F$ families. This brings the economy that we can use a single set of rules for terms, types and kinds uniformly. We use metavariables $\ottnt{e}$ and $\tau$ when referring to a "term" and a "type" respectively. Note that without distinction of terms, types and kinds, the "term" can be a term, a type or a kind. For example, in $\alpha :  \star $, the "term" $\alpha$ is a type and the "type" of $\alpha$ is $ \star $, which is a kind.

Another advantage is that \expcc includes a product form $\Pi \, \ottmv{x}  \ottsym{:}  \tau_{{\mathrm{1}}}  \ottsym{.}  \tau_{{\mathrm{2}}}$ which is used to represent type of functions from values of type $\tau_{{\mathrm{1}}}$ to values of type $\tau_{{\mathrm{2}}}$. Compared with concepts in System $F$, $\Pi \, \ottmv{x}  \ottsym{:}  \tau_{{\mathrm{1}}}  \ottsym{.}  \tau_{{\mathrm{2}}}$ subsumes both the arrow of function types $\tau_{{\mathrm{1}}}  \rightarrow  \tau_{{\mathrm{2}}}$ (if $\ottmv{x}$ does not occur free in $\tau_{{\mathrm{2}}}$), and the universal quantification $\forall \ottmv{x}:\tau_{{\mathrm{1}}}.\tau_{{\mathrm{2}}}$. Moreover, if $\ottmv{x}$ occurs free in $\tau_{{\mathrm{2}}}$, the product becomes a dependent product, which allows to represent dependent types. The product $ \Pi $ keeps the syntax of \expcc simple and expressive at the same time.

The syntax difference of from \cc is that \expcc introduces two new explicit type conversion primitives, namely $ \mathsf{cast}_{\Uparrow} $ and $ \mathsf{cast}_{\Downarrow} $ (pronounced as "cast up" and "cast down"), in order to replace the implicit conversion rule of \cc. They represent two directions of type conversion operations: $ \mathsf{cast}_{\Downarrow} $ stands for the reduction of types while $ \mathsf{cast}_{\Uparrow} $ is the inverse. Specifically speaking, suppose we have $e:\sigma$, i.e. the type of expression $e$ is $\sigma$. $\mathsf{cast}_{\Uparrow} \, \ottsym{[}  \tau  \ottsym{]}  \ottnt{e}$ converts the type of $\ottnt{e}$ to $\tau$, if there exists a type $\tau$ such that it can be reduced to $\sigma$ in a single step, i.e. $\tau  \longrightarrow  \sigma$. $\mathsf{cast}_{\Downarrow} \, \ottnt{e}$ represents the one-step-reduced type of $e$, i.e. $(\mathsf{cast}_{\Downarrow} \, \ottnt{e}) : \sigma'$ if $\sigma  \longrightarrow  \sigma'$.

The intention of introducing two explicit cast primitives is that we can gain full control of computation at the type level by manually managing the type conversions. Later in \S \ref{sec:ecc:type} we will see dropping the implicit conversion rule of \cc simplifies the type checking and leads to syntax-directed typing rules. This also influences the requirements of decidable type checking, that strong normalization is no long necessary.

\begin{figure}[ht]
	\gram{\otte\ottinterrule
		\otts\ottinterrule
		\ottG\ottinterrule
		\ottv}
	\caption{Syntax of \expcc}
	\label{fig:ecc:syntax}
\end{figure}

\subsection{Syntactic sugar}
\linus{This part can be moved to the next section for \name.}

To keep the core language minimal and simplify the translation of surface language, we use syntactic sugar shown in Figure \ref{fig:ecc:sugar} for \expcc.

Let binding for $x=\ottnt{e_{{\mathrm{2}}}}$ in $\ottnt{e_{{\mathrm{1}}}}$ is equivalent to the substitution of $x$ in $\ottnt{e_{{\mathrm{1}}}}$ with $\ottnt{e_{{\mathrm{2}}}}$, which can be reduced from $\ottsym{(}  \lambda  \ottmv{x}  \ottsym{:}  \tau  \ottsym{.}  \ottnt{e_{{\mathrm{1}}}}  \ottsym{)} \, \ottnt{e_{{\mathrm{2}}}}$.

The syntactic sugar for the function type is discussed in \S \ref{sec:ecc:syn} for the functionality of the product $ \Pi $. The product $\Pi \, \ottmv{x}  \ottsym{:}  \tau_{{\mathrm{1}}}  \ottsym{.}  \tau_{{\mathrm{2}}}$ can also be simply denoted by $ \Pi  \_ : \tau_{{\mathrm{1}}} . \tau_{{\mathrm{2}}}$, where the underscore stands for an anonymous variable.

\begin{figure}[ht]
	\centering
	\[
	\begin{array}{llll}
	\text{Let binding} & \mathbf{let} \, \ottmv{x}  \ottsym{:}  \tau  \ottsym{=}  \ottnt{e_{{\mathrm{2}}}} \, \mathbf{in} \, \ottnt{e_{{\mathrm{1}}}} & \triangleq & \ottsym{(}  \lambda  \ottmv{x}  \ottsym{:}  \tau  \ottsym{.}  \ottnt{e_{{\mathrm{1}}}}  \ottsym{)} \, \ottnt{e_{{\mathrm{2}}}} \\
	\text{Function type} & \tau_{{\mathrm{1}}}  \rightarrow  \tau_{{\mathrm{2}}} & \triangleq & \Pi \, \ottmv{x}  \ottsym{:}  \tau_{{\mathrm{1}}}  \ottsym{.}  \tau_{{\mathrm{2}}}\\
	&&& \text{($x$ does not occur free in $\tau_{{\mathrm{2}}}$)}
	\end{array}
	\]
	\caption{Syntatic sugar}
	\label{fig:ecc:sugar}
\end{figure}

\subsection{Type system}\label{sec:ecc:type}
The type system for \expcc contains typing judgements and operational semantics. Figure \ref{fig:ecc:dynsem} lists operational semantics for \expcc that defines rules for one-step reduction, including the $\beta$-reduction rule and $ \mathsf{cast}_{\Downarrow} $ rules. The expressions will be reduced by applying rules one or more times. Rule \ruleref{S\_CastDown} prevents the reduction from stalling with $ \mathsf{cast}_{\Downarrow} $ and continues to reduce the inner expression. Rule \ruleref{S\_CastDownUp} states that $ \mathsf{cast}_{\Downarrow} $ cancels the $ \mathsf{cast}_{\Uparrow} $ of an expression.

\begin{figure}[ht]
	\ottdefnstep{}
	\caption{Operational semantics of \expcc}
	\label{fig:ecc:dynsem}
\end{figure}

\renewcommand{\ottpremise}[1]{\enskip #1 \enskip}

Figure \ref{fig:ecc:typerule} lists the typing judgements to check the validity of expressions. Most rules are straightforward and similar with the ones in \cc. For example, rule \ruleref{T\_Ax} states that the "type" of sort $ \star $ is a kind. This is derived from an axiom in \cc, that the highest sort is $ \Box $, making the type system predicative. Rule \ruleref{T\_Pi} allows us to type dependent products. There are four possible combinations of types of $\tau_{{\mathrm{1}}}$ and $\tau_{{\mathrm{2}}}$ in a product $\Pi \, \ottmv{x}  \ottsym{:}  \tau_{{\mathrm{1}}}  \ottsym{.}  \tau_{{\mathrm{2}}}$, i.e. $(s,t) \in \{ \star ,  \Box \} \times \{ \star ,  \Box \}$. For some $(\lambda  \ottmv{x}  \ottsym{:}  \tau_{{\mathrm{1}}}  \ottsym{.}  \ottnt{e}):(\Pi \, \ottmv{x}  \ottsym{:}  \tau_{{\mathrm{1}}}  \ottsym{.}  \tau_{{\mathrm{2}}})$, when $(s,t)=( \star , \Box )$, $x:\tau_{{\mathrm{1}}}: \star $, $e:\tau_{{\mathrm{2}}}: \Box $, so $x$ is a term and $e$ is a type. Thus, we have a type depending on a term which means the product is a dependent type.

The difference from \cc for typing rules of \expcc is that rule \ruleref{T\_CastUp} and \ruleref{T\_CastDown} are added to check the type conversion primitives $ \mathsf{cast}_{\Uparrow} $ and $ \mathsf{cast}_{\Downarrow} $, and the implicit type conversion rule of \cc is removed, which is the rule as follows:
\ottusedrule{\ottdruleTccXXConv{}}
This rule is necessary for \cc because of the premise requirements of the application rule \ruleref{T\_App}:
\ottusedrule{\ottdruleTXXApp{}}
Consider the following two cases of the term $\ottnt{e_{{\mathrm{1}}}} \, \ottnt{e_{{\mathrm{2}}}}$:
\begin{itemize}
\item $\ottnt{e_{{\mathrm{2}}}}$ can be an arbitrary term so its type $\tau_{{\mathrm{2}}}$ is not necessary in normal form which might break the type checking of $\ottnt{e_{{\mathrm{1}}}}$, e.g. suppose $\ottnt{e_{{\mathrm{1}}}}:\sigma  \rightarrow  \tau$ and $\ottnt{e_{{\mathrm{2}}}} : \tau_{{\mathrm{2}}}$, where $\tau_{{\mathrm{2}}}$ is an application $(\lambda  \ottmv{x}  \ottsym{:}  \star  \ottsym{.}  \ottmv{x})~\sigma$. By \ruleref{Tcc\_Conv}, $(\lambda  \ottmv{x}  \ottsym{:}  \star  \ottsym{.}  \ottmv{x})~\sigma$ is $\beta$-equivalent to $\sigma$, thus $\ottnt{e_{{\mathrm{2}}}} : \sigma$ and we can further use \ruleref{T\_App} to achieve $\ottnt{e_{{\mathrm{1}}}} \, \ottnt{e_{{\mathrm{2}}}} : \tau$.
\item The type of $\ottnt{e_{{\mathrm{1}}}}$ should be a product expression according to the premise. But without the conversion rule, the term fails to type check if the type of $\ottnt{e_{{\mathrm{1}}}}$ is an expression which can further evaluate to a product, e.g. $ \Pi  y:(\ottsym{(}  \lambda  \ottmv{x}  \ottsym{:}  \star  \ottsym{.}  \ottmv{x}  \ottsym{)} \, \tau_{{\mathrm{2}}}).\tau_{{\mathrm{1}}}$. After applying \ruleref{Tcc\_Conv}, the type of $\ottnt{e_{{\mathrm{1}}}}$ is converted to its $\beta$-equivalence $ \Pi  x:\tau_{{\mathrm{2}}}.\tau_{{\mathrm{1}}}$. Thus we can further apply the \ruleref{T\_App}.
\end{itemize}

We need to show that explicit type conversion rules with cast primitives can also satisfy the premises of rule \ruleref{T\_App}. Still consider the above two cases:
\begin{itemize}
\item Given $\ottnt{e_{{\mathrm{1}}}}:\sigma  \rightarrow  \tau$ and $\ottnt{e_{{\mathrm{2}}}} : (\lambda  \ottmv{x}  \ottsym{:}  \star  \ottsym{.}  \ottmv{x})~\sigma$, we do the application by term $\ottnt{e_{{\mathrm{1}}}} \, \ottsym{(}  \mathsf{cast}_{\Downarrow} \, \ottnt{e_{{\mathrm{2}}}}  \ottsym{)}$. Since $(\lambda  \ottmv{x}  \ottsym{:}  \star  \ottsym{.}  \ottmv{x})~\sigma  \longrightarrow  \sigma$, $\mathsf{cast}_{\Downarrow} \, \ottnt{e_{{\mathrm{2}}}} : \sigma$, the term $\ottnt{e_{{\mathrm{1}}}} \, \ottsym{(}  \mathsf{cast}_{\Downarrow} \, \ottnt{e_{{\mathrm{2}}}}  \ottsym{)}$ type-checks with the rule \ruleref{T\_App}.
\item Given $\ottnt{e_{{\mathrm{1}}}} : ( \Pi  y:(\ottsym{(}  \lambda  \ottmv{x}  \ottsym{:}  \star  \ottsym{.}  \ottmv{x}  \ottsym{)} \, \tau_{{\mathrm{2}}}).\tau_{{\mathrm{1}}})$ and $\ottnt{e_{{\mathrm{2}}}} : \tau_{{\mathrm{2}}}$, we do the application by term $\ottnt{e_{{\mathrm{1}}}} \, \ottsym{(}  \mathsf{cast}_{\Uparrow} \, \ottsym{[}  \ottsym{(}  \lambda  \ottmv{x}  \ottsym{:}  \star  \ottsym{.}  \ottmv{x}  \ottsym{)} \, \tau_{{\mathrm{2}}}  \ottsym{]}  \ottnt{e_{{\mathrm{2}}}}  \ottsym{)}$. Noting that $\ottsym{(}  \lambda  \ottmv{x}  \ottsym{:}  \star  \ottsym{.}  \ottmv{x}  \ottsym{)} \, \tau_{{\mathrm{2}}}  \longrightarrow  \tau_{{\mathrm{2}}}$, the term conforms to rule \ruleref{T\_CastUp}. Thus $\mathsf{cast}_{\Uparrow} \, \ottsym{[}  \ottsym{(}  \lambda  \ottmv{x}  \ottsym{:}  \star  \ottsym{.}  \ottmv{x}  \ottsym{)} \, \tau_{{\mathrm{2}}}  \ottsym{]}  \ottnt{e_{{\mathrm{2}}}} : \ottsym{(}  \ottsym{(}  \lambda  \ottmv{x}  \ottsym{:}  \star  \ottsym{.}  \ottmv{x}  \ottsym{)} \, \tau_{{\mathrm{2}}}  \ottsym{)}$ and the term $\ottnt{e_{{\mathrm{1}}}} \, \ottsym{(}  \mathsf{cast}_{\Uparrow} \, \ottsym{[}  \ottsym{(}  \lambda  \ottmv{x}  \ottsym{:}  \star  \ottsym{.}  \ottmv{x}  \ottsym{)} \, \tau_{{\mathrm{2}}}  \ottsym{]}  \ottnt{e_{{\mathrm{2}}}}  \ottsym{)}$ can be type-checked by the rule \ruleref{T\_App}.
\end{itemize}

Therefore, it is feasible to replace implicit conversion rules of \cc with explicit type conversion rules.

\begin{figure}[ht]
	\ottdefnexpr{}
	\caption{Typing rules of \expcc}
	\label{fig:ecc:typerule}
\end{figure}

\subsection{Decidability of type checking}
The conversion rule of \cc is not syntax-directed because it can be implicitly applied at any time in a derivation. The $\beta$-equality premise of the rule also leads to the decidability of type checking relying on the strong normalization property of \cc:

\begin{thm}[Necessity of strong normalization]
The strong normalization property of \cc is necessary for the decidability of its type checking.
\end{thm}

\begin{proof}
The proof is by contradiction. Suppose strong normalization does not hold in the type system, then we can find a type $\tau_{{\mathrm{1}}}$ such that there exists at least one reduction sequence which does not terminate. Notice that any type $\tau_{{\mathrm{2}}}$ in such reduction sequence holds for $ \tau_{{\mathrm{1}}}  =_{\beta}  \tau_{{\mathrm{2}}} $. Thus we can constantly apply the conversion rule without termination and the type checking will not stop, which means the type checking is undecidable.
\end{proof}

Requiring strong normalization to achieve the decidability of type checking makes it impossible to combine general recursion with \cc, because general recursion might cause nontermination which simply breaks the strong normalization property. So we use explicit type conversion rules by cast operations to relax the constraints of achieving decidable type checking. We have the following theorem:

\begin{thm}[Decidability of type chechking for \expcc]
Let $\Gamma$ be an environment, $\ottnt{e}$ and $\tau$ be expressions of \expcc such that $\Gamma  \vdash  \tau  \ottsym{:}  \ottnt{s}$. Then the problem of knowing if one has $\Gamma  \vdash  \ottnt{e}  \ottsym{:}  \tau$ is decidable.
\end{thm}

\begin{proof}
By induction on typing rules in Figure \ref{fig:ecc:typerule}.
\end{proof}

Notice that new explicit type conversion rules are syntax-directed and do not include the $\beta$-equality premise but one-step reduction instead. Because checking if one term is one-step-reducible to the other is always decidable by enumerating the reduction rules, type checking using these rules are always decidable. Therefore the proof of decidability for \expcc does not rely on the strong normalization. This also implies the possibility of introducing general recursion into the system with decidable type checking.

\subsection{Soundness}
\linus{Is it necessary to show soundness in this section? How about merging with the core language? Because we have to show the soundness of \name in later sections anyway.}

It is straightforward to obtain the soundness of \expcc by combining the following two theorems:

\begin{thm}[Subject Reduction]
  If $\Gamma  \vdash  \ottnt{e}  \ottsym{:}  \tau$ and $e  \longrightarrow  e'$ then $\Gamma  \vdash  \ottnt{e'}  \ottsym{:}  \tau$.
\end{thm}

\begin{proof}
	By induction on rules in Figure \ref{fig:ecc:dynsem}.
\end{proof}

\begin{thm}[Progress]
  If $\varnothing  \vdash  \ottnt{e}  \ottsym{:}  \tau$ then either $e$ is a value $v$ or there exists
  $e'$ such that $e  \longrightarrow  e'$.
\end{thm}

\begin{proof}
	By induction on rules in Figure \ref{fig:ecc:syntax}.
\end{proof}


%%% !!! WARNING: AUTO GENERATED. DO NOT MODIFY !!! %%%
\section{The Explicit Calculus of Constructions with Recursion}

\bruno{Linus and Jeremy, I think you should do this section together. Most work is on Linus though since he needs to work out the proofs. Jeremy is mostly for Linus to consult with here :).}

%This section shows how to extend $\lambda C\beta$ with recursion. This extension 
%allows the calculus to account for both: 1) recursive definitions; 2) recursive types. 
%The extension preserves the decidability and soundness of the type system.

%%% FIXME: Dirty hacks, may change later
\newcommand{\newsyntax}{
\ottrulehead{\ottnt{e}  ,\ \tau}{::=}{\ottcom{Expressions}}\ottprodnewline
\ottfirstprodline{|}{\dots}{}{}{}{}\ottprodnewline
\ottprodline{|}{\mu \, \ottmv{x}  \ottsym{:}  \tau  \ottsym{.}  \ottnt{e}}{}{}{}{\ottcom{General recursion}}}

We have shown that \expcc does not rely on strong normalization for decidable type checking and soundness. Thus it is safe to combine general recursion with \expcc under the control of explicit type conversion operations $ \kw{cast}^{\uparrow} $ and $ \kw{cast}_{\downarrow} $. We extend \expcc into \name by introducing one unified primitive called $\mu$-notation for general recursion. It functions as a fixed point at the term level as well as a recursive type at the type level.

\subsection{The $\mu$-notation}
Based on the syntax of \expcc, we add the following $\mu$-notation for \name (the same part as \expcc is left out):

\begin{figure}[ht]
	\centering
	\gram{\newsyntax}
\end{figure}

The $\mu$-notation is similar to the definition of recursive types, except that it is not only treated as types but also terms. This also corresponds to the property of \expcc that terms and types are not distinguished.

The typing rule and operational semantics of $\mu$-notation for terms and types are also unified, thus each one rule for static and dynamic semantics is only needed to add over \expcc. The new type checking rule of $\mu$-notation is as follows:

\begin{figure}[ht]
	\ottusedrule{\ottdruleTXXMu{}}
\end{figure}

And the one-step reduction rule is as follows:

\begin{figure}[ht]
	\ottusedrule{\ottdruleSXXMu{}}
\end{figure}

If $\mu \, \ottmv{x}  \ottsym{:}  \tau  \ottsym{.}  \ottnt{e}$ is a term, with the \ruleref{S\_Mu} rule, it is not treated as a value and can be further reduced, which is different from conventional iso-recursive types. The one-step reduced term of $\mu \, \ottmv{x}  \ottsym{:}  \tau  \ottsym{.}  \ottnt{e}$ is the substitution of $\ottmv{x}$ in $\ottnt{e}$ with itself, i.e. $\ottnt{e}  \ottsym{[}  \ottmv{x}  \mapsto  \mu \, \ottmv{x}  \ottsym{:}  \tau  \ottsym{.}  \ottnt{e}  \ottsym{]}$. Such behavior is just the same as the definition of a fixed point.

If $\mu \, \ottmv{x}  \ottsym{:}  \tau  \ottsym{.}  \ottnt{e}$ is a type, assume there exist $e_1 : \mu \, \ottmv{x}  \ottsym{:}  \tau  \ottsym{.}  \ottnt{e}$ and $e_2 : \ottnt{e}  \ottsym{[}  \ottmv{x}  \mapsto  \mu \, \ottmv{x}  \ottsym{:}  \tau  \ottsym{.}  \ottnt{e}  \ottsym{]}$. Notice that the types of $e_1$ and $e_2$ are equivalent by $\beta$-equivalence. But such result cannot be directly obtained because of the removal of implicit conversion rule. Instead, by using explicit cast operations of \expcc, we can obtain the following transformation between $e$ and $e'$:
\[\begin{array}{lll}
	&\kw{cast}^{\uparrow} \, \ottsym{[}  \mu \, \ottmv{x}  \ottsym{:}  \tau  \ottsym{.}  \ottnt{e}  \ottsym{]}  \ottnt{e_{{\mathrm{2}}}} &: \mu \, \ottmv{x}  \ottsym{:}  \tau  \ottsym{.}  \ottnt{e}\\
	& \kw{cast}_{\downarrow}  e_1 &: \ottsym{(}  \mu \, \ottmv{x}  \ottsym{:}  \tau  \ottsym{.}  \ottnt{e}  \ottsym{[}  \ottmv{x}  \mapsto  \mu \, \ottmv{x}  \ottsym{:}  \tau  \ottsym{.}  \ottnt{e}  \ottsym{]}  \ottsym{)}
\end{array}\]
For type-level $\mu$-notation, $ \kw{cast}^{\uparrow} $ and $ \kw{cast}_{\downarrow} $ work in the same way as $\textsf{fold}$ and $\textsf{unfold}$ operations in iso-recursive types to control recursion explicitly.

\subsection{Decidability and soundness}
\linus{Not finished. Needs thorough thinking about the proof of soundness.}

Due to the introduction of recursive types, \name is no long consistent so that not able to be used as a logic. But with the power of general recursion, the expressibility of \name is increased since more data types and functions can be mapped or encoded into \name. And more importantly, even with $\mu$-notation, \name can still be proved to have the same properties as \name in the sense of decidability of type checking and soundness.

As what we previously illustrate in Section \ref{sec:ecc:sound}, the type checking of \expcc can always terminate because the derivation is finite without the implicit conversion rule. With the $mu$-notation in \name, the decidability of type checking still holds because the type level recursion is explicitly controlled by cast operations. Notice that in the typing rule of $ \kw{cast}^{\uparrow} $ and $ \kw{cast}_{\downarrow} $, the reduction is performed by one step. Thus the reduction sequences are always finite. Also by adopting the definitional equality, to judge if two terms are equal in the type checking is also decidable. Therefore, the new \ruleref{T\_Mu} rule is decidable for type checking.

To prove the soundness, we only need to consider each one more case for subject reduction and progress, i.e. \ruleref{S\_Mu} and \ruleref{T\_Mu}. It is straightforward to verify these two rules still keeping the soundness.


%%% !!! WARNING: AUTO GENERATED. DO NOT MODIFY !!! %%%
\newcommand{\FV}{\mathsf{FV}}
\newcommand{\dom}{\mathsf{dom}}

\section{Surface language}
\label{sec:surface}

% \begin{itemize}
% \item Expand the core language with datatypes and pattern matching by encoding.
% \item Give translation rules.
% \item Encode GADTs and maybe other Haskell extensions? GADTs seems challenging, so perhaps some other examples would be datatypes like $Fix f$, and $Monad$ as a record. Could formalize records in Haskell style.
% \end{itemize}

In this section, we present a surface language \sufcc, built on top of
\name with features that are convenient for functional programming:
user-defined datatypes, and pattern matching. Thanks to the
expressiveness of \name, all these features can be elaborated into the
core language without extending the built-in language constructs of
\name. In what follows, we first give the syntax of the surface
language, followed by the extended typing rules, then we show the
formal translation rules that translates a surface language expression
to an expression in \name. Finally we prove the type-safety of the
translation.

\subsection{Extended Syntax}

% \newcommand{\case}{\mathbf{case}} \newcommand{\of}{\mathbf{of}}
% \newcommand{\data}{\mathbf{data}} \newcommand{\where}{\mathbf{where}}
% \newcommand{\letbb}{\mathbf{let}} \newcommand{\inb}{\mathbf{in}}

The full syntax of \sufcc is defined in Figure~\ref{fig:surface:syntax}. Compared with \name, \sufcc has a new syntax category: a program, consisting of a list of datatype declarations, followed by a expression. An \emph{algebraic data type} $D$ is introduced as a top-level \textbf{data} declaration with its \emph{data constructors}. For the purpose of presentation, we sometimes adopt the following syntactic convention:
\[
\overline{\tau}^n \rightarrow \tau_r \equiv \tau_1 \rightarrow \dots \rightarrow \tau_n \rightarrow \tau_r
\]
The type of a data constructor $K$ has the form:
\[
\ottmv{K}  \ottsym{:}  \ottsym{(}  \,\overline{  \ottnt{u}  \ottsym{:}  \kappa  }\,  \ottsym{)}  \rightarrow  \ottsym{(}  \,\overline{  \ottmv{x}  \ottsym{:}  T  }\,  \ottsym{)}  \rightarrow  \ottmv{D}    \,\overline{  \ottnt{u}  }\,
\]
\bruno{this looks a bit odd for a number of reasons: firstly why to
  insist on having the quantified variables in the same order as the
  arguments in the constructor? Secondly it seems that all other
  arguments cannot be dependently typed? 
It seems to me that 
\[
\ottmv{K}  \ottsym{:}  \ottsym{(}  \,\overline{  \ottmv{x}  \ottsym{:}  \kappa  }\,  \ottsym{)}  \rightarrow  \ottmv{D}    \,\overline{  \ottnt{u}  }\,
\]
where all variables $u$ are bound ($u \in \overline{x}$) would be
better.  } \jeremy{changed!  $\ottsym{(}  \,\overline{  \ottnt{u}  \ottsym{:}  \kappa  }\,  \ottsym{)}$ are for the arguments of
a type constructor, $\ottsym{(}  \,\overline{  \ottmv{x}  \ottsym{:}  T  }\,  \ottsym{)}$ are for the arguments of a data
constrcutor}
% The quantified type variables $\,\overline{  \ottnt{u}  }\,$ appear in the same order in
% the return type $\ottmv{D}    \,\overline{  \ottnt{u}  }\,$.
Note that the use of the dependent product in the type of a data
constructor (e.g., $\ottsym{(}  \,\overline{  \ottnt{u}  \ottsym{:}  \kappa  }\,  \ottsym{)}$) makes it possible to let the type
of some type constructor arguments depend on other type constructor
arguments, while in Haskell, this is not possible, becuase the arrow
$\rightarrow$ can be seen as an independent product type. The
\textbf{case} expression is conventional, used to break up values
built with data constructors.  The patterns of a case expression are
flat (no nested patterns), and bind value variables.

\begin{figure*}
\centering
\gram{\ottpgm\ottinterrule
\ottdecl\ottinterrule
\ottu\ottinterrule
\ottp\ottinterrule
\ottE\ottinterrule
\ottV\ottinterrule
\ottGs}
\[\ottsurfsugar\] % defined in otthelper.mng.tex
\caption{Syntax of the surface language}
\label{fig:surface:syntax}
\end{figure*}


% \begin{figure}[ht]
% \centering
% \small
% \[
%   \begin{array}{llll}
%     \textbf{Declarations} \\
%     pgm &::= & \overline{decl}; e & \text{Declarations} \\
%     decl &::= & \data\,D\,\overline{u : \kappa} = \overline{\mid K\,\overline{\tau}} & \text{Datatype} \\ \\
%     \textbf{Terms} \\
%     u &::= & x \mid K & \text{Variables and constructors} \\
%     e,\tau,\sigma,\upsilon,\kappa &::= & u & \text{Term atoms} \\
%     & & \dots \\
%     & \mid & \case\,e\,\of\,\overline{p \Rightarrow e} & \text{Case analysis} \\
%     p &::= & K\,\overline{x : \tau} & \text{Pattern} \\ \\
%     \textbf{Environments} \\
%     \Gamma &::= &\varnothing & \text{Empty} \\
%     & \mid & \Gamma,u:\tau & \text{Variable binding}
%   \end{array}
% \]
%   \caption{Syntax of \sufcc ($e$ for terms; $\tau,\sigma,\upsilon$ for types; $\kappa$ for kinds)}\label{fig:datasyn}
% \end{figure}

For the sake of programming, \sufcc employs some syntactic sugar. A
non-dependent function type can be written as $T_{{\mathrm{1}}}  \rightarrow  T_{{\mathrm{2}}}$. A
dependent function type $\Pi \, \ottmv{x}  \ottsym{:}  T_{{\mathrm{1}}}  \ottsym{.}  T_{{\mathrm{2}}}$ is abbreviated as
$\ottsym{(}  \ottmv{x}  \ottsym{:}  T_{{\mathrm{1}}}  \ottsym{)}  \rightarrow  T_{{\mathrm{2}}}$ for easy reading. We also introduce a
Haskell-like record syntax, which is desugared to datatypes with
accompanying selector functions.

% \begin{figure}[ht]
% \centering
% \[
%   \begin{array}{l}
%     \data\,R\,\overline{u : \kappa} = K\,\{\,\overline{S:\tau}\,\} \triangleq \\
%     \data\,R\,\overline{u : \kappa} = K\,\overline{\tau} \\
%     \letbb\,S_{i} : \Pi \overline{u : \kappa}.R\,\overline{u} \rightarrow \tau_{i} = \\
%     \quad \lambda \overline{(u:\kappa)}.\lam{l}{R\,\overline{u}}{\case\,l\,\of\, K\,\overline{x:\tau} \Rightarrow x_{i}} \\
%     \inb
%   \end{array}
% \]
% \caption{Syntactic sugar for records}\label{fig:records}
% \end{figure}

\subsection{Extended Typing Rules}
\bruno{For typing and translation show only one figure (Figure 8),
  since the typing figure is just a subset. We can use gray to
  highlight the parts which belong to the translation.}
\jeremy{adjusted!}

Figure~\ref{fig:source:translate} defines the type system of the
surface language (ignore the gray parts for the moment). Several new
typing judgments appear in the type system. The use of different
subscripts of the judgments is to be distinct from the one used in
\name. Most rules of the type system are standard for systems based on
\coc, including the rules for the well-formedness of contexts
(\ruleref{TRenv\_Empty}, \ruleref{TRenv\_Var}), inferring the types of
variables (\ruleref{TR\_Var}), and dependent application
(\ruleref{TR\_App}). Two judgments $ \Sigma  \labeledjudge{pg}  \ottnt{pgm}  :  T $ and
$ \Sigma  \labeledjudge{d}  \ottnt{decl}  :  \Sigma' $ are of the essence to the type checking of the
surface language. The former type checks a whole program, and the
latter type checks datatype declarations.

Rule \ruleref{TRpgm\_Pgm} type checks a whole program. It first
type-checks the declarations, which in return gives a new typing
environment. Combined with the original environment, it then continues
to type check the expression and return the result type. Rule
\ruleref{TRpgm\_Data} is used to type check datatype declarations. It
first ensures the well-formedness of the type of the type constructor
application (of kind $\star$). Note that since our system adopts
$\star : \star$, this means we can express kind polymophism for
datatypes. Finally it make sure the types of data constructors are
valid.

Rules \ruleref{TR\_Case} and \ruleref{TRpat\_Alt} handle the type
checking of case expressions. The conclusion of \ruleref{TS\_Case}
binds the right types to the scrutinee $\ottnt{E_{{\mathrm{1}}}}$ and alternatives
$\overline{p \Rightarrow E_2}$. The first premise of
\ruleref{Tpat\_Alt} binds the actual type constructor arguments to
$\,\overline{  \ottnt{u}  }\,$. The second premise checks whether the types of the
right-hand sides of each alternative, instantiated to the actual type
constructor arguments $\,\overline{  \ottnt{u}  }\,$, are equal. Finally the third
premise checks the well-formedness of the types of data constructor
arguments.  \bruno{Mention that we do not support refinement, as in
  GADTs?} \jeremy{done!}

As can be seen, currently \sufcc does not support refinement on the
final result of each data constructor, as in GADTs. However, our
encoding method does support some form of GADTs, as is discussed in
\S\ref{Discussion}.

\begin{figure*}
\ottdefnctxtrans{}\ottinterrule
\ottdefnpgmtrans{}\ottinterrule
\ottdefndecltrans{}
\[\hlmath{\ottdecltrans}\]\ottinterrule % defined in otthelper.mng.tex
\ottdefnpattrans{}\ottinterrule
\ottdefnexprtrans{}
\caption{Type directed translation rules of the surface language}
\label{fig:source:translate}
\end{figure*}


% \begin{figure*}
% \ottdefnctxsrc{}
% \ottdefnpgmsrc{}
% \ottdefndeclsrc{}
% \ottdefnpatsrc{}
% \ottdefnexprsrc{}
% \caption{Typing rules of surface language}
% \label{fig:surface:typing}
% \end{figure*}

% \newcommand{\ctx}[2][\Gamma]{#1 \vdash #2}
% \newcommand{\ctxz}[1]{\ctx[\varnothing]{#1}}
% \newcommand{\ctxw}[3][\Gamma]{#1,#2 \vdash #3}

% \begin{figure*}
%   \centering \small
%   \begin{tabular}{lc}
%     \framebox{$\Gamma \vdash pgm : \tau$} \\
%     (Pgm) & $\inferrule {\overline{\Gamma_{0} \vdash decl : \Gamma_{d}} \\ \Gamma = \Gamma_{0}, \overline{\Gamma_{d}} \\ \ctx{e:\tau}} {\Gamma_{0} \vdash \overline{decl}; e : \tau}$ \\
%     \framebox{$\Gamma \vdash decl : \Gamma_d$} \\
%     (Data) & $\inferrule {\Gamma \vdash \overline{\kappa} \rightarrow \star : \square \\ \overline{\Gamma, D:\overline{\kappa} \rightarrow \star, \overline{u : \kappa} \vdash \overline{\tau} \rightarrow D\,\overline{u}:\star}} {\ctx{(\data\,D\,\overline{u : \kappa} = \overline{\mid K\,\overline{\tau}}): (D : \overline{\kappa} \rightarrow \star, \overline{K : \Pi\overline{u : \kappa}.\overline{\tau} \rightarrow D\,\overline{u}})}}$ \\
%     \framebox{$\Gamma \vdash e : \tau$} \\
%     (Case) & $\inferrule {\ctx{e_{1}}:\sigma \\ \overline{\Gamma\vdash_{p} p \Rightarrow e_{2}:\sigma \rightarrow \tau}} {\Gamma\vdash\case\,e_{1}\,\of\,\overline{p \Rightarrow e_{2}}:\tau}$ \\
%     \framebox{$\Gamma \vdash_{p} p \Rightarrow e : \sigma \rightarrow \tau$} \\
%     (Alt) & $\inferrule {\theta=[\overline{u := \upsilon}] \\\\ K:\Pi\overline{u:\kappa}.\,\overline{\sigma} \rightarrow D\,\overline{u} \in \Gamma \\ \Gamma,\overline{x:\theta(\sigma)} \vdash e:\tau} {\Gamma \vdash_{p} K\,\overline{x:\theta(\sigma)} \Rightarrow e : D\,\overline{\upsilon} \rightarrow \tau}$
%   \end{tabular}
%   \caption{Typing rules of \sufcc}\label{fig:datatype}
% \end{figure*}

\subsection{Translation Overview}

We use a type-directed translation. The basic translation rules have the form:
\[
 \Sigma  \labeledjudge{s}  \ottnt{E}  :  T   \rightsquigarrow   \ottnt{e} 
\]
It states that \name expression $\ottnt{e}$ is the translation of the
surface expression $\ottnt{E}$ of type $T$.  The gray parts in
Figure~\ref{fig:source:translate} defines the translation
rules. \bruno{Any partiular reasons for this?} \jeremy{deleted}

Among others, Rules \ruleref{TRdecl\_Data}, \ruleref{TRpat\_Alt} and
\ruleref{TR\_Case} are of the essence to the translation. Rule
\ruleref{TR\_Case} translates case expressions into applications by
first translating the scrutinee expression, casting it down to the
right type. It is then applied to the result type of the body
expression and a list of translated \name expressions of its
alternatives. Rule \ruleref{TRpat\_Alt} tells how to translate each
alternative. Basically it translates an alternative into a lambda
abstraction, where each bound variable in the pattern corresponds to a
bound variable in the lambda abstraction in the same order. The body
in the alternative is recursively translated and treated as the body
of the lambda abstraction. Note that due to the rigidness of the
translation, pattern matching must be exhaustive, and the order of
patterns matters (the same order as in the datatype declaration).

Rule \ruleref{TRDecl\_Data} does the most heavy work and deserves
further explanation. First of all, it results in an incomplete
expression (as can be seen by the incomplete $\mathsf{let}$
expressions), The result expression is supposed to be prepended to the
translation of the last expression to form a complete \name
expression, as specified by Rule \ruleref{TRpgm\_Pgm}. Furthermore,
each type constructor is translated to a recursive type, of which the
body is a type-level lambda abstraction. What is interesting is that
each recursive mention of the datatype in the data constructor
parameters is replaced with the recursive variable $\ottmv{X}$. Note that
for the moment, the result type variable $\alpha$ is restricted to
have kind $\star$. This could pose difficulties when translating GADTs
as we will discussion in the future work. Each data constructor is
translated to a lambda abstraction. Notice how we use $ \mathsf{cast}^{\uparrow} $ in
the lambda body to get the right type.

The rest of the translation rules hold few surprises.

\subsection{Type-safefy of Translation}

\jeremy{put Linus's theorem here}

% \begin{figure*}
%   \renewcommand{\arraystretch}{2.5}
%   \centering \small
%   \begin{tabular}{lcl}
%     \framebox{$\Gamma \vdash e : \tau \rightsquigarrow E$} \\
%     (Ax) & $\inferrule { } {\ctxz{\star:\square \rightsquigarrow \star}}$ \\

%     (Var) & $\inferrule {x:\tau \in \Gamma} {\ctx{x:\tau \rightsquigarrow x}}$ \\

%     (App) & $\inferrule {\ctx{e_{1}:(\pai{x}{\tau_{2}}{\tau_{1}}) \rightsquigarrow E_{1}} \\ \ctx{e_{2}:\tau_{2} \rightsquigarrow E_{2}}} {\ctx{e_{1}e_{2}:\tau_{1}[x:=e_{2}] \rightsquigarrow E_{1}E_{2}}}$ \\

%     (Lam) & $\inferrule {\ctxw{x:\tau_{1}}{e:\tau_{2} \rightsquigarrow E} \\ \ctx{(\pai{x}{\tau_{1}}{\tau_{2}}):t}} {\ctx{(\lam{x}{\tau_{1}}{e})}:(\pai{x}{\tau_{1}}{\tau_{2}}) \rightsquigarrow \lam{x}{\tau_{1}}{E}}$ & $t \in \{\star, \square\}$ \\

%     (Pi) & $\inferrule {\ctx{\tau_{1}:s} \\ \ctxw{x:\tau_{1}}{\tau_{2}:t}} {\ctx{(\pai{x}{\tau_{1}}{\tau_{2}}):t \rightsquigarrow \pai{x}{\tau_{1}}{\tau_{2}}}}$ & $(s,t) \in \mathcal{R}$ \\

%     (Mu) & $\inferrule {\ctxw{x:\tau}{e:\tau} \rightsquigarrow E \\ \ctx{\tau:s}} {\ctx{(\miu{x}{\tau}{e}):\tau} \rightsquigarrow \miu{x}{\tau}{E}}$ & $s \in \{\star, \square\}$ \\

%     (Fold) & $\inferrule {\ctx{e:\tau_{2} \rightsquigarrow E} \\ \ctx{\tau_{1}:s} \\ \tau_{1} \longrightarrow \tau_{2}} {\ctx{(\fold{{\tau_{1}}}{e}):{\tau_{1}} \rightsquigarrow \fold{\tau_{1}}{E}}}$ \\

%     (Unfold) & $\inferrule {\ctx{e:\tau_{1} \rightsquigarrow E} \\ \ctx{\tau_{2}:s} \\ \tau_{1} \longrightarrow \tau_{2}} {\ctx{(\unfold{e}):\tau_{2} \rightsquigarrow \unfold{E}}}$ \\

%     (Case) & $\inferrule {\ctx{e_{1}:\sigma \rightsquigarrow E_{1}} \\ \overline{\Gamma\vdash_{p} p \Rightarrow e_{2}:\sigma \rightarrow \tau \rightsquigarrow E_{2}}} {\Gamma\vdash\case\,e_{1}\,\of\,\overline{p \Rightarrow e_{2}}:\tau \rightsquigarrow (\unfold{E_{1}})\,\tau\,\overline{E_{2}}}$ \\
%     \framebox{$\Gamma \vdash_{p} p \Rightarrow e : \sigma \rightarrow \tau \rightsquigarrow E$} \\
%     (Alt) & $\inferrule {\theta=[\overline{u := \upsilon}] \\\\ K:\Pi\overline{u:\kappa}.\,\overline{\sigma} \rightarrow D\,\overline{u} \in \Gamma \\ \Gamma,\overline{x:\theta(\sigma)} \vdash e:\tau \rightsquigarrow E} {\Gamma \vdash_{p} K\,\overline{x:\theta(\sigma)} \Rightarrow e : D\,\overline{\upsilon} \rightarrow \tau \rightsquigarrow \lambda(\overline{x:\theta(\sigma)}).E}$ \\
%     \framebox{$\Gamma \vdash decl : \Gamma_d \rightsquigarrow E$} \\
%     (Data) & $\inferrule {\Gamma \vdash \overline{\kappa} \rightarrow \star : \square \\ \overline{\Gamma, D:\overline{\kappa} \rightarrow \star, \overline{u : \kappa} \vdash \overline{\tau} \rightarrow D\,\overline{u}:\star}} {\ctx{(\data\,D\,\overline{u : \kappa} = \overline{\mid K\,\overline{\tau}}): (D : \overline{\kappa} \rightarrow \star, \overline{K : \Pi\overline{u : \kappa}.\overline{\tau} \rightarrow D\,\overline{u}}) \rightsquigarrow E}}$ \\
%          & \begingroup \renewcommand*{\arraystretch}{1.0} $\begin{array} {lll}
%                                                              E & ::= & \letbb\,D : \overline{\kappa} \rightarrow \star =\lambda\overline{u : \kappa}.\,\mu X : \star.\,\Pi b:\star.\,\overline{(\overline{\tau}[D\,\overline{u}:=X] \rightarrow b)} \rightarrow b\,\inb \\ & & \letbb\,K_{i} : \Pi\overline{u : \kappa}.\overline{\tau} \rightarrow D\,\overline{u} = \lambda \overline{(u : \kappa)}.\lambda\overline{(x : \tau)}.\\
%                                                                     & & \quad \fold{D\,\overline{u}}{(\lambda (b : \star)\overline{(c : \overline{\tau} \rightarrow b)} . c_{i}\,\overline{x})}\,\inb \end{array}$ \endgroup \\
%     \framebox{$\Gamma \vdash pgm : \tau \rightsquigarrow E$} \\
%     (Pgm) & $\inferrule {\overline{\Gamma_{0} \vdash decl : \Gamma_{d} \rightsquigarrow E_{1}} \\ \Gamma = \Gamma_{0}, \overline{\Gamma_{d}} \\ \ctx{e:\tau \rightsquigarrow E}} {\Gamma_{0} \vdash \overline{decl}; e : \tau \rightsquigarrow \overline{E_{1}} \oplus E}$
%   \end{tabular}
%   \caption{Type-directed translation from \sufcc to \name}\label{fig:datatrans}
% \end{figure*}

%%% Local Variables:
%%% mode: latex
%%% TeX-master: "../main"
%%% End:


\section{Related Work}
\label{sec:related}

There is a lot work on bring full-spectrum dependent types to the
practical programming world. Compared to existing work, the main
different of our work is that we propose the use of explicit casts to
control type-level computation. Moreover we also unify recursion and
recursive types in a single language construct. Throughout this
section, we refer to Figure~\ref{fig:related:comp} for comparison. Our
emphasis is that with considerably less language constructs than
System $F_C$, \name is a well-suited alternative as a core for
Haskell-like language.

\begin{figure*}
\begin{threeparttable}
\renewcommand{\arraystretch}{0.4}
\small
\centering
\begin{tabularx}{\textwidth}{XXXXXXl}
\toprule
&&&& \multicolumn{2}{c}{Complexity} & \\ \cmidrule{5-6}
Core Language & Surface \mbox{Language} & Decidable Type Checking & General \mbox{Recursion} & \# of Language Constructs\tnote{1} & \# of Syntactic Sorts & Logical Consistency \\ \midrule
\name & \sufcc & Yes & Yes & 8 & 1 & No \\
System $F_C$ & Haskell & Yes & Yes & 35 & 3 & No \\
\cc & N/A & Yes & No & 7 & 1 & Yes \\
$\lambda^\theta$ & \textsf{Zombie} & Yes & Yes & 24 & 1 & Yes, in \textsf{L} Fragment \\
Core Cayenne & Cayenne & No & Yes & 11 & 1 & No \\
$\mu F^\star$ & $F^\star$ & Yes & Yes & 37 & 3 & Yes, in \textsf{PURE} Fragment \\
$\Pi\Sigma$ & N/A & Unknown\tnote{2} & Yes & 18 & 1 & No \\ \bottomrule
% ~\cite{fc}
% ~\cite{handbook}
% ~\cite{zombie:popl14}
% ~\cite{cayenne}
% ~\cite{fstar}
% ~\cite{dep:pisigma}
\end{tabularx}
\begin{tablenotes}
\item[1] Literals such as integers are ignored.
\item[2] No metatheory given.
\end{tablenotes}
\end{threeparttable}
\caption{Comparison of Core Languages}
\label{fig:related:comp}
\end{figure*}

\paragraph{General recursion without decidable type checking} To our
knowledge, Cayenne~\cite{cayenne} is the first programming language
that integrates the full power of dependent types with general
recursion. It bears some similarities with \name: first, it also uses
one syntactic form for both terms and types; second, it allows
arbitrary computation to happen at type level; third, because of
unrestricted recursion allowed in the system, Cayenne is logically
inconsistent, thus cannot be used as a proof system. However, the most
crucial difference from \name is that type-checking in Cayenne is
undecidable. From a pragmatic view, this design choice simplifies the
implementation (the implemented type checker fixes an upper bound on
the number of reductions that it may perform), among
others. Nevertheless, Cayenne is able to type check many useful
programs. In our opinion, \name improves in this respect by preserving
decidable type checking under the presence of general recursion.


\paragraph{Restricted recursion with termination checking}

Dependently typed languages such as Coq~\cite{coqsite} and
Adga~\cite{agda}, on the other hand, are conservative as to what kind
of computation is allowed at type level. Coq, as a proof system,
requires all programs to terminate by means of a termination checker,
ensuring that recursive calls are only allowed on \emph{syntactic
  subterms} of the primary argument. This way, decidable type checking
is also preserved. The conservative, syntactic criteria, to which Coq
adheres, are insufficient to support a variety of important
programming paradigms. Agda and Idris~\cite{idris}, in addition, offer
an option to disable the termination checker to allow for writing
arbitrary functions. This, however, costs us the property of decidable
type checking. While the idea that all programs should terminate is
appealing, \name aims at a different goal: a new programming model
where dependent types, decidable type checking, and general recursion
coexist. Most of the time, programmers just want to write the function
definitions, not much of delicate reasoning and proof.

\paragraph{Stratified type system with general recursion on term level}

One way to allow general recursion and dependent types in the same
language and still have decidable type-checking is to use multiple
levels of syntax. In this way it is easy to have a term language with
powerful constructs, such as general recursion, but have a more
restricted type and/or kind language. On the other hand this brings
complexity to the language as multiple levels (possibly with similar
constructs) have to be managed.

Several early attempts of adding general recursion by using stratified
type systems include Twelf~\cite{lf:twelf} and
Delphin~\cite{lf:delphin}, both of which are implementations of the
Edinburgh Logical Framework (LF)~\cite{harper:lf}. In a nutshell, the
LF calculus is a three-level calculus for object, families, and kinds,
and as such, Twelf and Delphin both have multiple syntactic levels,
and this way they are able to preserve decidable type checking at the
presence of general recursion. In other words, decidable type checking
comes at a price of complexity and duplication of language
constructs. In contrast, \name unifies terms and types into a single
category, and still achieves decidable type checking, at a cost of
logical consistency. $F^{\star}$~\cite{Swamy2011} is a recently
proposed dependently typed language that supports writing
general-purpose programs with effects while maintaining a consistent
core language. Like Twelf and Delphin, it also has several
sub-languages -- for terms, proofs and so on. In $F^{\star}$, the use
of recursion is restricted in specifications and proofs while allowing
arbitrary recursion in the program. Another difference from \name is
that types in $F^{\star}$ can only contain values but no non-value
expressions, leading to its less expressiveness than \name.

\paragraph{Unified syntax and managed type-level computation}

Pure Type Systems (PTS)~\cite{pts} show how a while family of type
systems can be implemented using just a single syntactic form. PTS are
an obvious source of inspiration for our work. Although this paper
presents a specific system, it should be easy to generalize \name in
the same way as PTS and further show the applicability of our ideas to
other systems.

An early attempt of using a single syntax for an intermediate language
for functional programming was Henk~\cite{pts:henk}. The Henk proposal
was to use the \emph{lambda cube} as a typed intermediate language,
unifying all three levels. In Cayenne, there is also no syntactic
distinction between expressions and types. Another recent work on
dependently typed language based on the same syntactic category is
\textsf{Zombie}~\cite{zombie:popl14, zombie:thesis}. An interesting
aspect of Zombie is that it is composed of two fragments: a logical
fragment where every expression is known to terminate, and a
programmatic fragment that allows general recursion, so that it
supports both partial and total programming. Even though Zombie has
one syntactic category, it is still fairly complicated (with around 25
language constructs), as it tries to be both consistent as a logic and
pragmatic as a programming language. In constrast \name takes another
point of the design space, giving up logical consistency for
simplicity. $\Pi\Sigma$~\cite{dep:pisigma} is another small
dependently typed core language that resembles \name. It has no
syntactic difference between terms and types, while also supporting
general recursion. Like \name, $\Pi\Sigma$ uses one recursion
mechanism for both types and functions. The key idea relies on lifted
types and boxes: definitions are not unfolded inside boxes. One
concern of $\Pi\Sigma$ is that its metatheory is not yet formally
developed.

% \bruno{Maybe have a paragraph on recursive types?}

\paragraph{Adding dependent types to existing languages}

There has also been a lot work on adding dependent types to existing
programming languages. The current core language for Haskell, System
$F_{C}$~\cite{Eisenberg:2014}, already supports GADTs, datatype
promotion, type families, and soon even kind
equality~\cite{fc:kind}. Nowadays System $F_{C}$ has grown to be a
relatively large and complex core language with over 30 language
constructs. Indeed, one of our primal motivations is to develop a
simpler alternative to System $F_C$. Throughout the paper, we have
shown many features that are easy to implement in \name. That being
said, one feature that is missing in \name while widely used in System
$F_C$ is GADTs. While we believe it is possible to support GADTs in
\name, we leave the implementation for future work.

% In \name, we believe we have found a sweet spot, where
% there are fewer language constructs and quite a number of modern
% features found in Haskell.

% \paragraph{Unification of Terms, Types, and Kinds}
% Pure Type Systems (PTS)~\cite{pts} show how a while family of type systems
% can be implemented using just a single syntactic form. PTS are an
% obvious source of inspiration for our work. Although this paper
% presents a specific system, it should be easy to generalize \name 
% in the same way as PTS and further show the applicability of our 
% ideas to other systems. 

% An early attempt of using a single syntax for an intermediate language
% for functional programming was Henk~\cite{pts:henk}. The Henk proposal
% was to use the \emph{lambda cube} as a typed intermediate language,
% unifying all three levels. The system used in Henk
% it is not even a dependently typed
% language, as the authors intended to  disallow types to depend
% on terms. As for recursion, even though it has a full lambda calculus
% at the type level, recursion is disallowed. Moreover no meta-theoretic 
% results were proved for Henk.

% \begin{comment}
% Since the implicit conversion of the lambda
% cube is not syntax-directed, they come up with a approach to
% strategically distribute the conversion rule over the other typing
% rules. In retrospect, Henk is quite conservative in terms of
% type-level computation. Actually it is not even a dependently typed
% language, as they clearly state that they don't allow types to depend
% on terms. As for recursion, even though it has a full lambda calculus
% at the type level, recursion is disallowed. In Henk the authors have
% not attempted to prove any meta-theoretic results.
% \end{comment}

% Another recent work on dependently typed language based on the same
% syntactic category is \textsf{Zombie}~\cite{zombie:popl14,
%   zombie:thesis}, where terms, types and the single kind $\star$ all
% reside in the same level. The language is based on a call-by-value
% variant of lambda calculus. An interesting aspect of Zombie is that
% it is composed of two fragments: a logical fragment and a programmatic
% fragment, so that it supports both partial and total programming. Even
% though Zombie has one syntactic category, it is still fairly
% complicated, as it tries to be both consistent as a logic and
% pragmatic as a programming language. In constrast \name takes 
% another point of the design space, giving up logical consistency 
% for simplicity.

% $\Pi\Sigma$~\cite{dep:pisigma} is another recently proposed
% dependently typed core language that resembles \name, as there is no
% syntactic difference between terms and types.\bruno{So? What's the
%   difference?}
% \bruno{Cayenne? general recursion, but no decidable type-checking.}

% \paragraph{General Recursion and Managed Type-level Computation}
% One way to allow general recursion and dependent types in the same
% language and still have decidable type-checking is to use multiple
% levels of syntax. In this way it is easy to have a term language with 
% powerful constructs, such as general recursion, but have a more
% restricted type and/or kind language. On the other hand this brings 
% complexity to the language as multiple levels (possibly with similar
% constructs) have to be managed.

% \bruno{Very important reference: A Framework for Defining Logics}
% \bruno{integrate paragraph I wrote above better with subsequent text.}

% \begin{comment}
% As discussed in \S\ref{sec:rec}, bringing general
% recursion blindly into the dependently typed world causes more trouble
% than convenience. There are many dependently typed languages that
% allow general recursion. Zombie approaches general recursion by
% separating a consistent sub-language, in which all expressions are
% known to terminate, from a programmatic language that supports general
% recursion. What is interesting about Zombie is that those two
% seemingly conflicting worlds can interact with each other nicely,
% without compromising the consistency property. The key idea of this is
% to distinguish between these two fragments by using a
% \emph{consistency classifier $\theta$}. When $\theta$ is \textsf{L},
% it means the logical part, and \textsf{P} the program part. Like
% \name, Zombie uses \textsf{roll} and \textsf{unroll} for iso-recursive
% types. To ensure normalization (in order for decidable type checking),
% it forbids the use of \textsf{unroll} in \textsf{P}, where the
% potential non-termination could arise.\bruno{Zombie is being discussed
% in two different places.}
% \end{comment}

% $F^{\star}$~\cite{Swamy2011} also supports writing general-purpose
% programs with effects (e.g., state, exceptions, non-terminating, etc.)
% while maintaining a consistent core language. Unlike \name, it has
% several sub-languages -- for terms, proofs and so on. The interesting
% part of $F^{\star}$ lies in its kind system, which tracks the
% sub-languages and controls the interactions between them. The idea is
% to restrict the use of recursion in specifications and proofs while
% allowing arbitrary recursion in the program. They use $\star$ to
% denote program terms that may be effectful and divergent, and
% \textsf{P} for proofs that identify pure and total functions. In this
% way, they are able to ensure that fragments in a program used for
% building proof terms are never mixed with those that are potentially
% divergent. One difference from \name is that, types in $F^{\star}$ can
% only contain values but no non-value expressions, leading to its less
% expressiveness than \name.

% $\Pi\Sigma$ has a general mechanism for recursion. Like \name, it uses
% one recursion mechanism for the definition of both types and
% programs\bruno{Oh! So they have recursive types and recursion using 
% a single construct?}. The key idea relies on lifted types and boxes: definitions
% are not unfolded inside boxes. The way they achieve decidable type
% checking is to use boxing to stop the infinite unfolding of the
% recursive call, at the cost of additional annotations stating where to
% lift, box and force. One concern of $\Pi\Sigma$ is that its metatheory
% is not yet formally developed.\bruno{Maybe have a paragraph on
%   recursive types?}

\paragraph{Type in Type}
We are not the the first to embrace $\star : \star$ in the system. It
has been long known that systems with $\star : \star$ is inconsistent
as a logic~\cite{handbook}. The core language of the Glasgow Haskell
Compiler, System $F_{C}$~\cite{fc} has already been inconsistent,
since all kinds are inhabited. $\Pi\Sigma$ has a impredicative
universe of types with $\mathsf{Type} : \mathsf{Type}$ due to the
support of general recursion. The surface language of Zombie also has
the rule
$\Gamma \vdash \mathsf{Type} : \mathsf{Type}$~\cite{zombie:popl15}.

The $\star : \star$ axiom makes it convenient to support kind
polymorphism, among other language features. One concern is that it
often causes type checking to be undecidable in dependently typed
language.  However, as we explained in Section~\ref{sec:ecore}, this
is not the case for \name. Type checking in \name is decidable -- all
type-level computation is driven by finite cast operations, thus no
potentially infinite reductions can happen in reality.

\paragraph{Encoding Datatypes}
There is much work on encoding datatypes into various high-level
languages. The classic Church encoding of datatypes into System F is
detailed in the work of Bohm and Beraducci~\cite{Bohm1985}.  An
alternative encoding of datatypes is Scott
encoding~\cite{encoding:scott}. However Scott encoding is not typable
in System F, as it needs recursive types. \name has all it requires to
represent polymorphic and recursive datatypes.

Another line of related work is the \emph{inductive defined types} in
the Calculus of Inductive Constructions (CIC)~\cite{cic}, which is the
underlying formal language of Coq. In CIC, inductive defined types can
be represented by closed types in \coc, so are the primitive recursive
functionals over elements of the type. McBride et
al.~\cite{elim:pi:pattern} show that inductive families of datatypes
with dependent pattern matching can be translated into terms in Luo's
UTT~\cite{Luo:UTT}, extended with axiom K~\cite{axiomK}. The novelty
in his work is the introduction of \emph{splitting tree}, with which
explicit evidence for impossible case branches is recorded.


\section{Conclusions and Future Work}

This work proposes a small dependently typed language that allows the
same syntax for terms and types, supports type-level computation, and
preserves decidable type checking under the presence of general
recursion. We consider this as a well-suited core for Haskell-like
languages.

Of course much remains to be done. For one thing, intensive type-level
computation can be written in \name, but is inconvenient to use. This
is because in \name, type-level computation is driven by cast
operations, and the number of cast operation needs to be specified
beforehand. Currently, for simple non-recursive functions, it is easy
to deduce how many casts needs to be introduced, but for recursive
ones, it becomes inconvenient. However, we are optimistic in this
regard. As future work, we plan to add language constructs to the
surface language, in the same spirit as type families in Haskell, to
guide type-level computation by passing around cast operations to the
core language.

For another, as we mentioned in the related work, \name lacks support
for GADTs. In our experiments, we have succeeded in encoding some
examples of GADTs, including \emph{Fin}. As it turns out, the
translation rules for datatypes can be extended to account for
GADTs. The issues are manifested in two strands: 1) The notion of
equality-proofs. Currently \name only supports syntactic equality. We
plan to explore more expressive form of type-equality; 2) Injectivity
of type constructors. \jeremy{more?}


%% -- References --

\acks
Thanks to Blah. This work is supported by Blah.

\bibliographystyle{abbrvnat}
\nocite{*}
\bibliography{main}

%% -- Appendix --

\appendix
%%% !!! WARNING: AUTO GENERATED. DO NOT MODIFY !!! %%%
\section{Specification of core language}

\subsection{Syntax}
\gram{\otte\ottinterrule
        \otts\ottinterrule
        \ottG\ottinterrule
        \ottv}

\subsection{Operational semantics and expression typing}
\ottdefnstep{}
\ottusedrule{\ottdruleSXXMu{}}
\ottdefnexpr{}
\ottusedrule{\ottdruleTXXMu{}}

\section{Proofs of core language}
\subsection{Decidability of type checking}
\begin{lem}[Uniqueness of one-step reduction]
	The relation $ \longrightarrow $, i.e. one-step reduction, is \textbf{unique} in the sense that given $e$ there is at most one $e'$ such that $\ottnt{e}  \longrightarrow  \ottnt{e'}$.
\end{lem}

\begin{proof}
	By induction on the structure of $e$:
	\begin{description}
		\item[Case $e=v$:] $e$ has one of the following forms:
		\begin{inparaenum}[(1)]
			\item $\lambda  \ottmv{x}  \ottsym{:}  \tau  \ottsym{.}  \ottnt{e}$,
			\item $\Pi \, \ottmv{x}  \ottsym{:}  \tau_{{\mathrm{1}}}  \ottsym{.}  \tau_{{\mathrm{2}}}$,
			\item $\mathsf{cast}^{\uparrow} \, \ottsym{[}  \tau  \ottsym{]}  \ottnt{e}$,
		\end{inparaenum}
		which cannot match any rules of $ \longrightarrow $. Thus there is no $e'$ such that $\ottnt{e}  \longrightarrow  \ottnt{e'}$.
		\item[Case $e=\ottsym{(}  \lambda  \ottmv{x}  \ottsym{:}  \tau  \ottsym{.}  \ottnt{e_{{\mathrm{1}}}}  \ottsym{)} \, \ottnt{e_{{\mathrm{2}}}}$:] There is a unique $e'=\ottnt{e_{{\mathrm{1}}}}  \ottsym{[}  \ottmv{x}  \mapsto  \ottnt{e_{{\mathrm{2}}}}  \ottsym{]}$ by rule \ruleref{S\_Beta}.
		\item[Case $e=\mathsf{cast}_{\downarrow} \, \ottsym{(}  \mathsf{cast}^{\uparrow} \, \ottsym{[}  \tau  \ottsym{]}  \ottnt{e}  \ottsym{)}$:] There is a unique $e'=e$ by rule \ruleref{S\_CastDownUp}.
		\item[Case $e=\mu \, \ottmv{x}  \ottsym{:}  \tau  \ottsym{.}  \ottnt{e}$:] There is a unique $e'=\ottnt{e}  \ottsym{[}  \ottmv{x}  \mapsto  \mu \, \ottmv{x}  \ottsym{:}  \tau  \ottsym{.}  \ottnt{e}  \ottsym{]}$ by rule \ruleref{S\_Mu}.
		\item[Case $e=\ottnt{e_{{\mathrm{1}}}} \, \ottnt{e_{{\mathrm{2}}}}$ with $\ottnt{e_{{\mathrm{1}}}}  \longrightarrow  \ottnt{e'_{{\mathrm{1}}}}$:] $\ottnt{e_{{\mathrm{1}}}}$ cannot be a $\lambda$-term $\lambda  \ottmv{x}  \ottsym{:}  \tau  \ottsym{.}  \ottnt{e}$ which is a value that contradicts $\ottnt{e_{{\mathrm{1}}}}$ can be reduced to $\ottnt{e'_{{\mathrm{1}}}}$. By the induction hypothesis, $\ottnt{e'_{{\mathrm{1}}}}$ is unique reduction of $\ottnt{e_{{\mathrm{1}}}}$. Thus by rule \ruleref{S\_App}, $e'=\ottnt{e'_{{\mathrm{1}}}} \, \ottnt{e_{{\mathrm{2}}}}$ is the unique reduction for $e$.
		\item[Case $e=\mathsf{cast}_{\downarrow} \, \ottnt{e_{{\mathrm{1}}}}$ with $\ottnt{e_{{\mathrm{1}}}}  \longrightarrow  \ottnt{e'_{{\mathrm{1}}}}$:] $\ottnt{e_{{\mathrm{1}}}}$ cannot have the form $\mathsf{cast}^{\uparrow} \, \ottsym{[}  \tau  \ottsym{]}  \ottnt{e}$ which is a value that contradicts $\ottnt{e_{{\mathrm{1}}}}$ can be reduced to $\ottnt{e'_{{\mathrm{1}}}}$. By the induction hypothesis, $\ottnt{e'_{{\mathrm{1}}}}$ is unique reduction of $\ottnt{e_{{\mathrm{1}}}}$. Thus by rule \ruleref{S\_CastDown}, $e'=\mathsf{cast}_{\downarrow} \, \ottnt{e'_{{\mathrm{1}}}}$ is the unique reduction for $e$.
	\end{description}
\end{proof}

\begin{dfn}[Well-formed context]
	A \textbf{well-formed} context $\Gamma$ is defined by the following rules:
	
	\textnormal{\ottdefnctx{}}
\end{dfn}

\begin{lem}[Consistency of well-formed context]\label{lem:wfc}
	Given a well-formed initial context $\Gamma$, it remains well-formed through type checking.
\end{lem}

\begin{proof}
	Suppose $\Gamma$ is the initial context which is well-formed. To safely extend $\Gamma$ with a variable $x:\tau$, one should have $\Gamma  \vdash  \tau  \ottsym{:}  \ottnt{s}$ due to rule \ruleref{Env\_Var}. Note that when applying typing rules of $\Gamma  \vdash  \ottnt{e}  \ottsym{:}  \tau$, rule \ruleref{T\_Pi}, \ruleref{T\_Mu} and \ruleref{T\_Lam} will extend the context. We show that these rules cover the condition $\Gamma  \vdash  \tau  \ottsym{:}  \ottnt{s}$ with respect to $x:\tau$ as follows:
	\begin{description}
		\item[Case \ruleref{T\_Pi}:] \ottusedrule{\ottdruleTXXPi{}} For $x:\tau_{{\mathrm{1}}}$, $\Gamma  \vdash  \tau_{{\mathrm{1}}}  \ottsym{:}  \ottnt{s}$ is directly the premise of the rule.
		\item[Case \ruleref{T\_Mu}:] \ottusedrule{\ottdruleTXXMu{}} For $x:\tau$, $\Gamma  \vdash  \tau  \ottsym{:}  \ottnt{s}$ is directly the premise of the rule.
		\item[Case \ruleref{T\_Lam}:] \ottusedrule{\ottdruleTXXLam{}} For $x:\tau_{{\mathrm{1}}}$, note that the premise $\Gamma  \vdash  \ottsym{(}  \Pi \, \ottmv{x}  \ottsym{:}  \tau_{{\mathrm{1}}}  \ottsym{.}  \tau_{{\mathrm{2}}}  \ottsym{)}  \ottsym{:}  \ottnt{s}$ can be derived from rule \ruleref{T\_Pi}, which has the pre-condition $\Gamma  \vdash  \tau_{{\mathrm{1}}}  \ottsym{:}  \ottnt{s}$.
	\end{description}
\end{proof}

\begin{lem}[Valid context optimization]\label{lem:wfcopt}
	With a well-formed initial context $\Gamma$, the \ruleref{T\_Var} and \ruleref{T\_Weak} can be replaced by the following rule: \ottusedrule{\ottdruleTSXXVar{}}
\end{lem}

\begin{proof}
	By Lemma \ref{lem:wfc}, the context $\Gamma$ remains well-formed if it is initially well-formed. Thus, it is not necessary to use \ruleref{T\_Var} and \ruleref{T\_Weak} to check the well-formedness of $\Gamma$. In order to check the type of a variable $x$, it is necessary and sufficient to check if $\ottmv{x}  \ottsym{:}  \tau \, \in \, \Gamma$, which is simply rule \ruleref{TS\_Var}. \linus{This proof needs to be more formally written.}
\end{proof}

\begin{lem}[Decidability of type checking]
	There is a decidable algorithm which given $\Gamma, \ottnt{e}$ computes the unique $\tau$ such that $\Gamma  \vdash  \ottnt{e}  \ottsym{:}  \tau$ or reports there is no such $\tau$.
\end{lem}

\begin{proof}
	By induction on the derivation of $e$:
	\begin{description}
		\item[Case $e=x$:] By Lemma \ref{lem:wfcopt}, we only need to consider context $\Gamma$ that is well-formed. By rule \ruleref{TS\_Var}, if $\ottmv{x}  \ottsym{:}  \tau \, \in \, \Gamma$, $t$ is the unique type of $x$.
		\item[Case $e=\ottnt{e_{{\mathrm{1}}}} \, \ottnt{e_{{\mathrm{2}}}}$:] By rule \ruleref{T\_App}, 
	\end{description} 
\end{proof}

\subsection{Properties}
\newcommand{\FV}{\mathsf{FV}}
\newcommand{\dom}{\mathsf{dom}}

\begin{lem}[Free variables lemma]
If $\Gamma  \vdash  \ottnt{e}  \ottsym{:}  \tau$, then $\FV(\ottnt{e}),\FV(\tau) \subseteq \dom(\Gamma)$.
\end{lem}

\begin{lem}[Generation lemma]
	
\end{lem}

\begin{lem}[Substitution lemma]

\end{lem}

\subsection{Soundness}
\begin{lem}[Subject reduction]
If $\Gamma  \vdash  \ottnt{e}  \ottsym{:}  \tau$ and $e  \twoheadrightarrow  e'$ then $\Gamma  \vdash  \ottnt{e'}  \ottsym{:}  \tau$.
\end{lem}

\begin{lem}[Progress]
If $\varnothing  \vdash  \ottnt{e}  \ottsym{:}  \tau$ then either $e$ is a value $v$ or there exists $e'$ such that $e  \twoheadrightarrow  e'$.
\end{lem}



%% -- The end --

\end{document}

%%% Local Variables:
%%% mode: latex
%%% TeX-master: t
%%% End:
