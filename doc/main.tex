% SIGPLAN template
\documentclass[preprint]{sigplanconf}

%% -- Packages Imports --

% AMS stuff
\usepackage{amsmath}
\usepackage{amssymb}
\usepackage{amsthm}
\usepackage{xspace}

% Language
\usepackage{csquotes}
\usepackage[english]{babel}
\MakeOuterQuote{"}

\newcommand{\name}{{\bf $\lambda C_{\beta}$}\xspace}
\newcommand{\coc}{{\bf $\lambda C$}\xspace}
\newcommand{\expcc}{$\lambda C_{\mathsf{exp}}$\xspace}
\newcommand{\cc}{$\lambda C$\xspace}

\newcommand{\fold}[2]{\mathsf{cast}^\uparrow[#1]\,#2}
\newcommand{\unfold}[2][dummy]{\mathsf{cast}_\downarrow\,#2}

\newcommand{\lam}[3]{\lambda #1:#2.\,#3}
\newcommand{\pai}[3]{\Pi #1:#2.\,#3}
\newcommand{\miu}[3]{\mu #1:#2.\,#3}

\newcommand{\authornote}[3]{{\color{#2} {\sc #1}: #3}}
\newcommand\bruno[1]{\authornote{bruno}{red}{#1}}
\newcommand\jeremy[1]{\authornote{jeremy}{blue}{#1}}
\newcommand\linus[1]{\authornote{linus}{green}{#1}}

% Hyper links
\usepackage{hyperref}
\hypersetup{
   colorlinks,
   citecolor=black,
   filecolor=black,
   linkcolor=black,
   urlcolor=black
}

% Compact list
\usepackage{paralist}

% Figure import
\usepackage{graphicx}
\usepackage{float}

% Code highlighting
\usepackage{listings}

% Theorem
\usepackage{amsthm}
\newtheorem{thm}{Theorem}[section]
\newtheorem{lem}[thm]{Lemma}
\newtheorem{dfn}[thm]{Definition}

%% Typesetting inference rules
\usepackage{styles/mathpartir}  % by Didier Rémy (http://gallium.inria.fr/~remy/latex/mathpartir.html

% Ott includes
\usepackage{supertabular}
% generated by Ott 0.25 from: expcore.ott
\newcommand{\ottdrule}[4][]{{\displaystyle\frac{\begin{array}{l}#2\end{array}}{#3}\quad\ottdrulename{#4}}}
\newcommand{\ottusedrule}[1]{\[#1\]}
\newcommand{\ottpremise}[1]{ #1 \\}
\newenvironment{ottdefnblock}[3][]{ \framebox{\mbox{#2}} \quad #3 \\[0pt]}{}
\newenvironment{ottfundefnblock}[3][]{ \framebox{\mbox{#2}} \quad #3 \\[0pt]\begin{displaymath}\begin{array}{l}}{\end{array}\end{displaymath}}
\newcommand{\ottfunclause}[2]{ #1 \equiv #2 \\}
\newcommand{\ottnt}[1]{\mathit{#1}}
\newcommand{\ottmv}[1]{\mathit{#1}}
\newcommand{\ottkw}[1]{\mathbf{#1}}
\newcommand{\ottsym}[1]{#1}
\newcommand{\ottcom}[1]{\text{#1}}
\newcommand{\ottdrulename}[1]{\textsc{#1}}
\newcommand{\ottcomplu}[5]{\overline{#1}^{\,#2\in #3 #4 #5}}
\newcommand{\ottcompu}[3]{\overline{#1}^{\,#2<#3}}
\newcommand{\ottcomp}[2]{\overline{#1}^{\,#2}}
\newcommand{\ottgrammartabular}[1]{\begin{supertabular}{llcllllll}#1\end{supertabular}}
\newcommand{\ottmetavartabular}[1]{\begin{supertabular}{ll}#1\end{supertabular}}
\newcommand{\ottrulehead}[3]{$#1$ & & $#2$ & & & \multicolumn{2}{l}{#3}}
\newcommand{\ottprodline}[6]{& & $#1$ & $#2$ & $#3 #4$ & $#5$ & $#6$}
\newcommand{\ottfirstprodline}[6]{\ottprodline{#1}{#2}{#3}{#4}{#5}{#6}}
\newcommand{\ottlongprodline}[2]{& & $#1$ & \multicolumn{4}{l}{$#2$}}
\newcommand{\ottfirstlongprodline}[2]{\ottlongprodline{#1}{#2}}
\newcommand{\ottbindspecprodline}[6]{\ottprodline{#1}{#2}{#3}{#4}{#5}{#6}}
\newcommand{\ottprodnewline}{\\}
\newcommand{\ottinterrule}{\\[5.0mm]}
\newcommand{\ottafterlastrule}{\\}
  \newcommand{\labeledjudge}[1]{\vdash_{\!\!\mathsf{#1} } }

\newcommand{\ottmetavars}{
\ottmetavartabular{
 $ \ottmv{x} ,\, \ottmv{y} ,\, \ottmv{z} ,\, \ottmv{d} $ & \ottcom{Variable names} \\
}}

\newcommand{\otte}{
\ottrulehead{\ottnt{e}  ,\ \tau  ,\ \sigma}{::=}{\ottcom{Expressions}}\ottprodnewline
\ottfirstprodline{|}{\ottmv{x}}{}{}{}{\ottcom{Variable}}\ottprodnewline
\ottprodline{|}{\ottnt{s}}{}{}{}{\ottcom{Sort}}\ottprodnewline
\ottprodline{|}{\ottnt{e_{{\mathrm{1}}}} \, \ottnt{e_{{\mathrm{2}}}}}{}{}{}{\ottcom{Application}}\ottprodnewline
\ottprodline{|}{\lambda  \ottmv{x}  \ottsym{:}  \tau  \ottsym{.}  \ottnt{e}}{}{}{}{\ottcom{Abstraction}}\ottprodnewline
\ottprodline{|}{\Pi \, \ottmv{x}  \ottsym{:}  \tau_{{\mathrm{1}}}  \ottsym{.}  \tau_{{\mathrm{2}}}}{}{}{}{\ottcom{Product}}\ottprodnewline
\ottprodline{|}{\mathsf{cast}^{\uparrow} \, \ottsym{[}  \tau  \ottsym{]}  \ottnt{e}}{}{}{}{\ottcom{Cast up to type}}\ottprodnewline
\ottprodline{|}{\mathsf{cast}_{\downarrow} \, \ottnt{e}}{}{}{}{\ottcom{Cast down by reduction}}\ottprodnewline
\ottprodline{|}{\mu \, \ottmv{x}  \ottsym{:}  \tau  \ottsym{.}  \ottnt{e}}{}{}{}{\ottcom{General recursion}}}

\newcommand{\ottl}{
\ottrulehead{\ottnt{l}}{::=}{}}

\newcommand{\otts}{
\ottrulehead{\ottnt{s}  ,\ t}{::=}{\ottcom{Sorts}}\ottprodnewline
\ottfirstprodline{|}{\star}{}{}{}{\ottcom{Star}}\ottprodnewline
\ottprodline{|}{\Box}{}{}{}{\ottcom{Square}}}

\newcommand{\ottG}{
\ottrulehead{\Gamma}{::=}{\ottcom{Contexts}}\ottprodnewline
\ottfirstprodline{|}{\varnothing}{}{}{}{\ottcom{Empty}}\ottprodnewline
\ottprodline{|}{\Gamma  \ottsym{,}  \ottmv{x}  \ottsym{:}  \tau}{}{}{}{\ottcom{Variable binding}}}

\newcommand{\ottv}{
\ottrulehead{\ottnt{v}}{::=}{\ottcom{Values}}\ottprodnewline
\ottfirstprodline{|}{\lambda  \ottmv{x}  \ottsym{:}  \tau  \ottsym{.}  \ottnt{e}}{}{}{}{\ottcom{Abstraction}}\ottprodnewline
\ottprodline{|}{\Pi \, \ottmv{x}  \ottsym{:}  \tau_{{\mathrm{1}}}  \ottsym{.}  \tau_{{\mathrm{2}}}}{}{}{}{\ottcom{Product}}\ottprodnewline
\ottprodline{|}{\mathsf{cast}^{\uparrow} \, \ottsym{[}  \tau  \ottsym{]}  \ottnt{e}}{}{}{}{\ottcom{Cast up}}}

\newcommand{\ottee}{
\ottrulehead{e  ,\ \tau}{::=}{\ottcom{Expressions}}\ottprodnewline
\ottfirstprodline{|}{\ottmv{x}}{}{}{}{\ottcom{Variable}}\ottprodnewline
\ottprodline{|}{\ottnt{s}}{}{}{}{\ottcom{Sort}}\ottprodnewline
\ottprodline{|}{e_{{\mathrm{1}}} \, e_{{\mathrm{2}}}}{}{}{}{\ottcom{Application}}\ottprodnewline
\ottprodline{|}{\lambda  \ottmv{x}  \ottsym{:}  \tau  \ottsym{.}  e}{}{}{}{\ottcom{Abstraction}}\ottprodnewline
\ottprodline{|}{\Pi \, \ottmv{x}  \ottsym{:}  \tau_{{\mathrm{1}}}  \ottsym{.}  \tau_{{\mathrm{2}}}}{}{}{}{\ottcom{Product}}\ottprodnewline
\ottprodline{|}{\mathsf{cast}^{\uparrow} \, \ottsym{[}  \tau  \ottsym{]}  e}{}{}{}{\ottcom{Cast up to type}}\ottprodnewline
\ottprodline{|}{\mathsf{cast}_{\downarrow} \, e}{}{}{}{\ottcom{Cast down by reduction}}}

\newcommand{\ottgrammar}{\ottgrammartabular{
\otte\ottinterrule
\ottl\ottinterrule
\otts\ottinterrule
\ottG\ottinterrule
\ottv\ottinterrule
\ottee\ottafterlastrule
}}

% defnss
% defns Lint
%% defn coc
\newcommand{\ottdruleTXXConv}[1]{\ottdrule[#1]{%
{\Gamma  \vdash  \ottnt{e}  \ottsym{:}  \tau_{{\mathrm{1}}}}%
\\ {\Gamma  \vdash  \tau_{{\mathrm{2}}}  \ottsym{:}  \ottnt{s}}%
\\ {\tau_{{\mathrm{1}}}  =_{\beta}  \tau_{{\mathrm{2}}}}%
}{
\Gamma  \vdash  \ottnt{e}  \ottsym{:}  \tau_{{\mathrm{2}}}}{%
{\ottdrulename{T\_Conv}}{}%
}}

\newcommand{\ottdefncoc}[1]{\begin{ottdefnblock}[#1]{$\Gamma  \vdash  \ottnt{e}  \ottsym{:}  \tau$}{\ottcom{CoC typing}}
\ottusedrule{\ottdruleTXXConv{}}
\end{ottdefnblock}}

%% defn ext
\newcommand{\ottdruleTXXMu}[1]{\ottdrule[#1]{%
{\Gamma  \ottsym{,}  \ottmv{x}  \ottsym{:}  \tau  \vdash  \ottnt{e}  \ottsym{:}  \tau}%
\\ {\Gamma  \vdash  \tau  \ottsym{:}  \ottnt{s}}%
}{
\Gamma  \vdash  \ottsym{(}  \mu \, \ottmv{x}  \ottsym{:}  \tau  \ottsym{.}  \ottnt{e}  \ottsym{)}  \ottsym{:}  \tau}{%
{\ottdrulename{T\_Mu}}{}%
}}

\newcommand{\ottdefnext}[1]{\begin{ottdefnblock}[#1]{$\Gamma  \vdash  \ottnt{e}  \ottsym{:}  \tau$}{\ottcom{Extended expression typing}}
\ottusedrule{\ottdruleTXXMu{}}
\end{ottdefnblock}}

%% defn sur
\newcommand{\ottdruleTSXXVar}[1]{\ottdrule[#1]{%
{\vdash  \Gamma}%
\\ {\ottmv{x}  \ottsym{:}  \tau \, \in \, \Gamma}%
}{
\Gamma  \vdash  \ottmv{x}  \ottsym{:}  \tau}{%
{\ottdrulename{TS\_Var}}{}%
}}

\newcommand{\ottdefnsur}[1]{\begin{ottdefnblock}[#1]{$\Gamma  \vdash  \ottnt{e}  \ottsym{:}  \tau$}{\ottcom{Expression typing for source language}}
\ottusedrule{\ottdruleTSXXVar{}}
\end{ottdefnblock}}

%% defn ctx
\newcommand{\ottdruleEnvXXEmpty}[1]{\ottdrule[#1]{%
}{
\vdash  \varnothing}{%
{\ottdrulename{Env\_Empty}}{}%
}}


\newcommand{\ottdruleEnvXXVar}[1]{\ottdrule[#1]{%
{\vdash  \Gamma}%
\\ {\Gamma  \vdash  \tau  \ottsym{:}  \ottnt{s}}%
}{
\vdash  \Gamma  \ottsym{,}  \ottmv{x}  \ottsym{:}  \tau}{%
{\ottdrulename{Env\_Var}}{}%
}}

\newcommand{\ottdefnctx}[1]{\begin{ottdefnblock}[#1]{$\vdash  \Gamma$}{\ottcom{Well-formed context}}
\ottusedrule{\ottdruleEnvXXEmpty{}}
\ottusedrule{\ottdruleEnvXXVar{}}
\end{ottdefnblock}}

%% defn expr
\newcommand{\ottdruleTXXAx}[1]{\ottdrule[#1]{%
}{
\varnothing  \vdash  \star  \ottsym{:}  \Box}{%
{\ottdrulename{T\_Ax}}{}%
}}


\newcommand{\ottdruleTXXVar}[1]{\ottdrule[#1]{%
{\Gamma  \vdash  \tau  \ottsym{:}  \ottnt{s}}%
}{
\Gamma  \ottsym{,}  \ottmv{x}  \ottsym{:}  \tau  \vdash  \ottmv{x}  \ottsym{:}  \tau}{%
{\ottdrulename{T\_Var}}{}%
}}


\newcommand{\ottdruleTXXWeak}[1]{\ottdrule[#1]{%
{\Gamma  \vdash  \ottnt{e}  \ottsym{:}  \tau_{{\mathrm{2}}}}%
\\ {\Gamma  \vdash  \tau_{{\mathrm{1}}}  \ottsym{:}  \ottnt{s}}%
}{
\Gamma  \ottsym{,}  \ottmv{x}  \ottsym{:}  \tau_{{\mathrm{1}}}  \vdash  \ottnt{e}  \ottsym{:}  \tau_{{\mathrm{2}}}}{%
{\ottdrulename{T\_Weak}}{}%
}}


\newcommand{\ottdruleTXXApp}[1]{\ottdrule[#1]{%
{\Gamma  \vdash  \ottnt{e_{{\mathrm{1}}}}  \ottsym{:}  \ottsym{(}  \Pi \, \ottmv{x}  \ottsym{:}  \tau_{{\mathrm{2}}}  \ottsym{.}  \tau_{{\mathrm{1}}}  \ottsym{)}}%
\\ {\Gamma  \vdash  \ottnt{e_{{\mathrm{2}}}}  \ottsym{:}  \tau_{{\mathrm{2}}}}%
}{
\Gamma  \vdash  \ottnt{e_{{\mathrm{1}}}} \, \ottnt{e_{{\mathrm{2}}}}  \ottsym{:}  \tau_{{\mathrm{1}}}  \ottsym{[}  \ottmv{x}  \mapsto  \ottnt{e_{{\mathrm{2}}}}  \ottsym{]}}{%
{\ottdrulename{T\_App}}{}%
}}


\newcommand{\ottdruleTXXLam}[1]{\ottdrule[#1]{%
{\Gamma  \ottsym{,}  \ottmv{x}  \ottsym{:}  \tau_{{\mathrm{1}}}  \vdash  \ottnt{e}  \ottsym{:}  \tau_{{\mathrm{2}}}}%
\\ {\Gamma  \vdash  \ottsym{(}  \Pi \, \ottmv{x}  \ottsym{:}  \tau_{{\mathrm{1}}}  \ottsym{.}  \tau_{{\mathrm{2}}}  \ottsym{)}  \ottsym{:}  \ottnt{s}}%
}{
\Gamma  \vdash  \ottsym{(}  \lambda  \ottmv{x}  \ottsym{:}  \tau_{{\mathrm{1}}}  \ottsym{.}  \ottnt{e}  \ottsym{)}  \ottsym{:}  \ottsym{(}  \Pi \, \ottmv{x}  \ottsym{:}  \tau_{{\mathrm{1}}}  \ottsym{.}  \tau_{{\mathrm{2}}}  \ottsym{)}}{%
{\ottdrulename{T\_Lam}}{}%
}}


\newcommand{\ottdruleTXXPi}[1]{\ottdrule[#1]{%
{\Gamma  \vdash  \tau_{{\mathrm{1}}}  \ottsym{:}  \ottnt{s}}%
\\ {\Gamma  \ottsym{,}  \ottmv{x}  \ottsym{:}  \tau_{{\mathrm{1}}}  \vdash  \tau_{{\mathrm{2}}}  \ottsym{:}  t}%
}{
\Gamma  \vdash  \ottsym{(}  \Pi \, \ottmv{x}  \ottsym{:}  \tau_{{\mathrm{1}}}  \ottsym{.}  \tau_{{\mathrm{2}}}  \ottsym{)}  \ottsym{:}  t}{%
{\ottdrulename{T\_Pi}}{}%
}}


\newcommand{\ottdruleTXXCastUp}[1]{\ottdrule[#1]{%
{\Gamma  \vdash  \ottnt{e}  \ottsym{:}  \tau_{{\mathrm{2}}}}%
\\ {\Gamma  \vdash  \tau_{{\mathrm{1}}}  \ottsym{:}  \ottnt{s}}%
\\ {\tau_{{\mathrm{1}}}  \longrightarrow  \tau_{{\mathrm{2}}}}%
}{
\Gamma  \vdash  \ottsym{(}  \mathsf{cast}^{\uparrow} \, \ottsym{[}  \tau_{{\mathrm{1}}}  \ottsym{]}  \ottnt{e}  \ottsym{)}  \ottsym{:}  \tau_{{\mathrm{1}}}}{%
{\ottdrulename{T\_CastUp}}{}%
}}


\newcommand{\ottdruleTXXCastDown}[1]{\ottdrule[#1]{%
{\Gamma  \vdash  \ottnt{e}  \ottsym{:}  \tau_{{\mathrm{1}}}}%
\\ {\Gamma  \vdash  \tau_{{\mathrm{2}}}  \ottsym{:}  \ottnt{s}}%
\\ {\tau_{{\mathrm{1}}}  \longrightarrow  \tau_{{\mathrm{2}}}}%
}{
\Gamma  \vdash  \ottsym{(}  \mathsf{cast}_{\downarrow} \, \ottnt{e}  \ottsym{)}  \ottsym{:}  \tau_{{\mathrm{2}}}}{%
{\ottdrulename{T\_CastDown}}{}%
}}

\newcommand{\ottdefnexpr}[1]{\begin{ottdefnblock}[#1]{$\Gamma  \vdash  \ottnt{e}  \ottsym{:}  \tau$}{\ottcom{Expression typing}}
\ottusedrule{\ottdruleTXXAx{}}
\ottusedrule{\ottdruleTXXVar{}}
\ottusedrule{\ottdruleTXXWeak{}}
\ottusedrule{\ottdruleTXXApp{}}
\ottusedrule{\ottdruleTXXLam{}}
\ottusedrule{\ottdruleTXXPi{}}
\ottusedrule{\ottdruleTXXCastUp{}}
\ottusedrule{\ottdruleTXXCastDown{}}
\end{ottdefnblock}}


\newcommand{\ottdefnsLint}{
\ottdefncoc{}\ottdefnext{}\ottdefnsur{}\ottdefnctx{}\ottdefnexpr{}}

% defns OpSem
%% defn extstep
\newcommand{\ottdruleSXXMu}[1]{\ottdrule[#1]{%
}{
\mu \, \ottmv{x}  \ottsym{:}  \tau  \ottsym{.}  \ottnt{e}  \longrightarrow  \ottnt{e}  \ottsym{[}  \ottmv{x}  \mapsto  \mu \, \ottmv{x}  \ottsym{:}  \tau  \ottsym{.}  \ottnt{e}  \ottsym{]}}{%
{\ottdrulename{S\_Mu}}{}%
}}

\newcommand{\ottdefnextstep}[1]{\begin{ottdefnblock}[#1]{$\ottnt{e}  \longrightarrow  \ottnt{e'}$}{\ottcom{One-step reduction}}
\ottusedrule{\ottdruleSXXMu{}}
\end{ottdefnblock}}

%% defn step
\newcommand{\ottdruleSXXBeta}[1]{\ottdrule[#1]{%
}{
\ottsym{(}  \lambda  \ottmv{x}  \ottsym{:}  \tau  \ottsym{.}  \ottnt{e_{{\mathrm{1}}}}  \ottsym{)} \, \ottnt{e_{{\mathrm{2}}}}  \longrightarrow  \ottnt{e_{{\mathrm{1}}}}  \ottsym{[}  \ottmv{x}  \mapsto  \ottnt{e_{{\mathrm{2}}}}  \ottsym{]}}{%
{\ottdrulename{S\_Beta}}{}%
}}


\newcommand{\ottdruleSXXApp}[1]{\ottdrule[#1]{%
{\ottnt{e_{{\mathrm{1}}}}  \longrightarrow  \ottnt{e'_{{\mathrm{1}}}}}%
}{
\ottnt{e_{{\mathrm{1}}}} \, \ottnt{e_{{\mathrm{2}}}}  \longrightarrow  \ottnt{e'_{{\mathrm{1}}}} \, \ottnt{e_{{\mathrm{2}}}}}{%
{\ottdrulename{S\_App}}{}%
}}


\newcommand{\ottdruleSXXCastDown}[1]{\ottdrule[#1]{%
{\ottnt{e}  \longrightarrow  \ottnt{e'}}%
}{
\mathsf{cast}_{\downarrow} \, \ottnt{e}  \longrightarrow  \mathsf{cast}_{\downarrow} \, \ottnt{e'}}{%
{\ottdrulename{S\_CastDown}}{}%
}}


\newcommand{\ottdruleSXXCastDownUp}[1]{\ottdrule[#1]{%
}{
\mathsf{cast}_{\downarrow} \, \ottsym{(}  \mathsf{cast}^{\uparrow} \, \ottsym{[}  \tau  \ottsym{]}  \ottnt{e}  \ottsym{)}  \longrightarrow  \ottnt{e}}{%
{\ottdrulename{S\_CastDownUp}}{}%
}}

\newcommand{\ottdefnstep}[1]{\begin{ottdefnblock}[#1]{$\ottnt{e}  \longrightarrow  \ottnt{e'}$}{\ottcom{One-step reduction}}
\ottusedrule{\ottdruleSXXBeta{}}
\ottusedrule{\ottdruleSXXApp{}}
\ottusedrule{\ottdruleSXXCastDown{}}
\ottusedrule{\ottdruleSXXCastDownUp{}}
\end{ottdefnblock}}


\newcommand{\ottdefnsOpSem}{
\ottdefnextstep{}\ottdefnstep{}}

\newcommand{\ottdefnss}{
\ottdefnsLint
\ottdefnsOpSem
}

\newcommand{\ottall}{\ottmetavars\\[0pt]
\ottgrammar\\[5.0mm]
\ottdefnss}


% Hack to use mathpartir for ott
\newcommand{\ottlinebreak}{}
\renewcommand{\ottdrule}[4][]{{\inferrule{#2 }{#3}\quad\ottdrulename{#4}}}
\newcommand{\gram}[1]{\ottgrammartabular{#1\ottafterlastrule}}
\newcommand{\ruleref}[1]{\ottdrulename{#1}}

% lhs2tex
\usepackage{mylhs2tex}

%% -- Packages Imports --

% Main
\begin{document}

% Page size - US Letter
\special{papersize=8.5in,11in}
\setlength{\pdfpageheight}{\paperheight}
\setlength{\pdfpagewidth}{\paperwidth}

% Conference info
\conferenceinfo{CONF 'yy}{Month d--d, 20yy, City, ST, Country}
\copyrightyear{20yy}
\copyrightdata{978-1-nnnn-nnnn-n/yy/mm}
\doi{nnnnnnn.nnnnnnn}

% Title
\titlebanner{DRAFT} % Only for preprint
\preprintfooter{} % Only for preprint

\title{Type-Level Computation One Step at a Time}
%\title{A Dependently-typed Intermediate Language with General Recursion}
\subtitle{Or: Decidable Type-Checking in the presence of
  Type-Level General Recursion}

\authorinfo{Foo \and Bar \and Baz}
           {The University of Foo}
           {\{foo,bar,baz\}@foo.edu}

\maketitle

% Abstract
\begin{abstract}
Many type systems support a conversion rule that allows type-level
computation. In such type systems ensuring the \emph{decidability} of
type checking requires type-level computation to terminate.
For calculi where the syntax of types and terms is the same, the
decidability of type-checking is usually dependent on the strong normalization
of the calculus, which ensures termination. An unfortunate
consequence of this coupling between decidability and strong
normalization is that adding (unrestricted) general recursion to such
calculi is not possible.

This paper proposes an alternative to the conversion rule that allows
the same syntax for types and terms, type-level computation, and
preserves decidability of type-checking under the presence of general
recursion. The key idea, which is inspired by the traditional
treatment of \emph{iso-recursive types}, is to make each type-level
computation step explicit. Each beta reduction or expansion at the
type-level is introduced by a language construct. This allows control
over the type-level computation and ensures decidability of
type-checking even in the presence of non-terminating programs at the
type-level.  We realize this idea by presenting a variant of the
calculus of constructions with general recursion and recursive types.
Furthermore we show how many programming language features of
state-of-the-art functional languages (such as Haskell) can be encoded
in our minimalistic core calculus.
\end{abstract}

% Category, terms & keywords
\category{D.3.1}{Programming Languages}{Formal Definitions and Theory}
\terms Languages, Design
\keywords Dependent types, Intermediate langauge

%% -- Starting Point -- 

\section{Introduction}
\emph{These are definitely drafts and only some main points are listed in each section.}

\begin{enumerate}[a)]
\item Motivations:

\begin{itemize}
\item Because of the reluctance to introduce dependent types\footnote{This might be changed in the near future. See \url{https://ghc.haskell.org/trac/ghc/wiki/DependentHaskell/Phase1}.}, the current intermediate language of Haskell, namely System $F_C$ \cite{fc}, separates expressions as terms, types and kinds, which brings complexity to the implementation as well as further extensions \cite{fc:pro,fc:kind}.

\item Popular full-spectrum dependently typed languages, like Agda, Coq, Idris, have to ensure the termination of functions for the decidability of proofs. No general recursion and the limitation of enforcing termination checking make such languages impractical for general-purpose programming.

\item We would like to introduce a simple and compiler-friendly dependently typed core language with only one hierarchy, which supports general recursion at the same time.
\end{itemize}

\item Contribution:
\begin{itemize}
\item A core language based on Calculus of Constructions (CoC) that collapses terms, types and kinds into the same hierarchy.
\item General recursion by introducing recursive types for both terms and types by the same $\mu$ primitive.
\item Decidable type checking and managed type-level computation by replacing implicit conversion rule of CoC with generalized \textsf{fold}/\textsf{unfold} semantics.
\item First-class equality by coercion, which is used for encoding GADTs or newtypes without runtime overhead.
\item Surface language that supports datatypes, pattern matching and other language extensions for Haskell, and can be encoded into the core language.
\end{itemize}

\item Related work:

\begin{itemize}
\item Henk \cite{pts:henk} and one of its implementation \cite{pts:fp} show the simplicity of the Pure Type System (PTS). \cite{pts:rec} also tries to combine recursion with PTS.

\item \textsf{Zombie} \cite{zombie:popl14, zombie:thesis} is a language with two fragments supporting logics with non-termination. It limits the $\beta$-reduction for congruence closure \cite{zombie:popl15}.

\item $\Pi\Sigma$ \cite{dep:pisigma} is a simple, dependently-typed core language for expressing high-level constructions\footnote{But the paper didn't give any meta-theories about the langauge.}. UHC compiler \cite{fc:uhc} tries to use a simplified core language with coercion to encode GADTs.

\item System $F_C$ \cite{fc} has been extended with type promotion \cite{fc:pro} and kind equality \cite{fc:kind}. The latter one introduces a limited form of dependent types into the system\footnote{Richard A. Eisenberg is going to implement kind equality \cite{fc:kind} into GHC. The implementation is proposed at \url{https://phabricator.haskell.org/D808} and related paper is at \url{http://www.cis.upenn.edu/~eir/papers/2015/equalities/equalities-extended.pdf}.}, which mixes up types and kinds.
\end{itemize}

\end{enumerate}

%%% !!! WARNING: AUTO GENERATED. DO NOT MODIFY !!! %%%
%% ODER: format ==         = "\mathrel{==}"
%% ODER: format /=         = "\neq "
%
%
\makeatletter
\@ifundefined{lhs2tex.lhs2tex.sty.read}%
  {\@namedef{lhs2tex.lhs2tex.sty.read}{}%
   \newcommand\SkipToFmtEnd{}%
   \newcommand\EndFmtInput{}%
   \long\def\SkipToFmtEnd#1\EndFmtInput{}%
  }\SkipToFmtEnd

\newcommand\ReadOnlyOnce[1]{\@ifundefined{#1}{\@namedef{#1}{}}\SkipToFmtEnd}
\usepackage{amstext}
\usepackage{amssymb}
\usepackage{stmaryrd}
\DeclareFontFamily{OT1}{cmtex}{}
\DeclareFontShape{OT1}{cmtex}{m}{n}
  {<5><6><7><8>cmtex8
   <9>cmtex9
   <10><10.95><12><14.4><17.28><20.74><24.88>cmtex10}{}
\DeclareFontShape{OT1}{cmtex}{m}{it}
  {<-> ssub * cmtt/m/it}{}
\newcommand{\texfamily}{\fontfamily{cmtex}\selectfont}
\DeclareFontShape{OT1}{cmtt}{bx}{n}
  {<5><6><7><8>cmtt8
   <9>cmbtt9
   <10><10.95><12><14.4><17.28><20.74><24.88>cmbtt10}{}
\DeclareFontShape{OT1}{cmtex}{bx}{n}
  {<-> ssub * cmtt/bx/n}{}
\newcommand{\tex}[1]{\text{\texfamily#1}}	% NEU

\newcommand{\Sp}{\hskip.33334em\relax}


\newcommand{\Conid}[1]{\mathit{#1}}
\newcommand{\Varid}[1]{\mathit{#1}}
\newcommand{\anonymous}{\kern0.06em \vbox{\hrule\@width.5em}}
\newcommand{\plus}{\mathbin{+\!\!\!+}}
\newcommand{\bind}{\mathbin{>\!\!\!>\mkern-6.7mu=}}
\newcommand{\rbind}{\mathbin{=\mkern-6.7mu<\!\!\!<}}% suggested by Neil Mitchell
\newcommand{\sequ}{\mathbin{>\!\!\!>}}
\renewcommand{\leq}{\leqslant}
\renewcommand{\geq}{\geqslant}
\usepackage{polytable}

%mathindent has to be defined
\@ifundefined{mathindent}%
  {\newdimen\mathindent\mathindent\leftmargini}%
  {}%

\def\resethooks{%
  \global\let\SaveRestoreHook\empty
  \global\let\ColumnHook\empty}
\newcommand*{\savecolumns}[1][default]%
  {\g@addto@macro\SaveRestoreHook{\savecolumns[#1]}}
\newcommand*{\restorecolumns}[1][default]%
  {\g@addto@macro\SaveRestoreHook{\restorecolumns[#1]}}
\newcommand*{\aligncolumn}[2]%
  {\g@addto@macro\ColumnHook{\column{#1}{#2}}}

\resethooks

\newcommand{\onelinecommentchars}{\quad-{}- }
\newcommand{\commentbeginchars}{\enskip\{-}
\newcommand{\commentendchars}{-\}\enskip}

\newcommand{\visiblecomments}{%
  \let\onelinecomment=\onelinecommentchars
  \let\commentbegin=\commentbeginchars
  \let\commentend=\commentendchars}

\newcommand{\invisiblecomments}{%
  \let\onelinecomment=\empty
  \let\commentbegin=\empty
  \let\commentend=\empty}

\visiblecomments

\newlength{\blanklineskip}
\setlength{\blanklineskip}{0.66084ex}

\newcommand{\hsindent}[1]{\quad}% default is fixed indentation
\let\hspre\empty
\let\hspost\empty
\newcommand{\NB}{\textbf{NB}}
\newcommand{\Todo}[1]{$\langle$\textbf{To do:}~#1$\rangle$}

\EndFmtInput
\makeatother
%
%
%
%
%
%
% This package provides two environments suitable to take the place
% of hscode, called "plainhscode" and "arrayhscode". 
%
% The plain environment surrounds each code block by vertical space,
% and it uses \abovedisplayskip and \belowdisplayskip to get spacing
% similar to formulas. Note that if these dimensions are changed,
% the spacing around displayed math formulas changes as well.
% All code is indented using \leftskip.
%
% Changed 19.08.2004 to reflect changes in colorcode. Should work with
% CodeGroup.sty.
%
\ReadOnlyOnce{polycode.fmt}%
\makeatletter

\newcommand{\hsnewpar}[1]%
  {{\parskip=0pt\parindent=0pt\par\vskip #1\noindent}}

% can be used, for instance, to redefine the code size, by setting the
% command to \small or something alike
\newcommand{\hscodestyle}{}

% The command \sethscode can be used to switch the code formatting
% behaviour by mapping the hscode environment in the subst directive
% to a new LaTeX environment.

\newcommand{\sethscode}[1]%
  {\expandafter\let\expandafter\hscode\csname #1\endcsname
   \expandafter\let\expandafter\endhscode\csname end#1\endcsname}

% "compatibility" mode restores the non-polycode.fmt layout.

\newenvironment{compathscode}%
  {\par\noindent
   \advance\leftskip\mathindent
   \hscodestyle
   \let\\=\@normalcr
   \let\hspre\(\let\hspost\)%
   \pboxed}%
  {\endpboxed\)%
   \par\noindent
   \ignorespacesafterend}

\newcommand{\compaths}{\sethscode{compathscode}}

% "plain" mode is the proposed default.
% It should now work with \centering.
% This required some changes. The old version
% is still available for reference as oldplainhscode.

\newenvironment{plainhscode}%
  {\hsnewpar\abovedisplayskip
   \advance\leftskip\mathindent
   \hscodestyle
   \let\hspre\(\let\hspost\)%
   \pboxed}%
  {\endpboxed%
   \hsnewpar\belowdisplayskip
   \ignorespacesafterend}

\newenvironment{oldplainhscode}%
  {\hsnewpar\abovedisplayskip
   \advance\leftskip\mathindent
   \hscodestyle
   \let\\=\@normalcr
   \(\pboxed}%
  {\endpboxed\)%
   \hsnewpar\belowdisplayskip
   \ignorespacesafterend}

% Here, we make plainhscode the default environment.

\newcommand{\plainhs}{\sethscode{plainhscode}}
\newcommand{\oldplainhs}{\sethscode{oldplainhscode}}
\plainhs

% The arrayhscode is like plain, but makes use of polytable's
% parray environment which disallows page breaks in code blocks.

\newenvironment{arrayhscode}%
  {\hsnewpar\abovedisplayskip
   \advance\leftskip\mathindent
   \hscodestyle
   \let\\=\@normalcr
   \(\parray}%
  {\endparray\)%
   \hsnewpar\belowdisplayskip
   \ignorespacesafterend}

\newcommand{\arrayhs}{\sethscode{arrayhscode}}

% The mathhscode environment also makes use of polytable's parray 
% environment. It is supposed to be used only inside math mode 
% (I used it to typeset the type rules in my thesis).

\newenvironment{mathhscode}%
  {\parray}{\endparray}

\newcommand{\mathhs}{\sethscode{mathhscode}}

% texths is similar to mathhs, but works in text mode.

\newenvironment{texthscode}%
  {\(\parray}{\endparray\)}

\newcommand{\texths}{\sethscode{texthscode}}

% The framed environment places code in a framed box.

\def\codeframewidth{\arrayrulewidth}
\RequirePackage{calc}

\newenvironment{framedhscode}%
  {\parskip=\abovedisplayskip\par\noindent
   \hscodestyle
   \arrayrulewidth=\codeframewidth
   \tabular{@{}|p{\linewidth-2\arraycolsep-2\arrayrulewidth-2pt}|@{}}%
   \hline\framedhslinecorrect\\{-1.5ex}%
   \let\endoflinesave=\\
   \let\\=\@normalcr
   \(\pboxed}%
  {\endpboxed\)%
   \framedhslinecorrect\endoflinesave{.5ex}\hline
   \endtabular
   \parskip=\belowdisplayskip\par\noindent
   \ignorespacesafterend}

\newcommand{\framedhslinecorrect}[2]%
  {#1[#2]}

\newcommand{\framedhs}{\sethscode{framedhscode}}

% The inlinehscode environment is an experimental environment
% that can be used to typeset displayed code inline.

\newenvironment{inlinehscode}%
  {\(\def\column##1##2{}%
   \let\>\undefined\let\<\undefined\let\\\undefined
   \newcommand\>[1][]{}\newcommand\<[1][]{}\newcommand\\[1][]{}%
   \def\fromto##1##2##3{##3}%
   \def\nextline{}}{\) }%

\newcommand{\inlinehs}{\sethscode{inlinehscode}}

% The joincode environment is a separate environment that
% can be used to surround and thereby connect multiple code
% blocks.

\newenvironment{joincode}%
  {\let\orighscode=\hscode
   \let\origendhscode=\endhscode
   \def\endhscode{\def\hscode{\endgroup\def\@currenvir{hscode}\\}\begingroup}
   %\let\SaveRestoreHook=\empty
   %\let\ColumnHook=\empty
   %\let\resethooks=\empty
   \orighscode\def\hscode{\endgroup\def\@currenvir{hscode}}}%
  {\origendhscode
   \global\let\hscode=\orighscode
   \global\let\endhscode=\origendhscode}%

\makeatother
\EndFmtInput
%

\section{Overview}

This section informally introduces the main features of \name. In
particular, this section shows how the explicit casts in \name can be
used instead of the typical conversion rule present in calculi such as
the calculus of constructions. The formal details of \name are
presented in \S\ref{sec:ecore}. \jeremy{to distinguish code from
  \sufcc and \name, we may want to use different fonts, e.g., {\tt
    typewriter font} for \sufcc}

\subsection{The Calculus of Constructions and the Conversion Rule}
\label{sec:coc}

The calculus of constructions (\coc)~\cite{coc} is a powerful
higher-order typed lambda calculus supporting dependent types (among
various other features).  A crutial
feature of \coc is the so-called \emph{conversion}
rule: \ottusedrule{\ottdruleTccXXConv{}}

%For the most part \name is similar to the \emph{Calculus of Constructions}
%(\coc)~\cite{coc}, which is a higher-order typed lambda calculus.
%However unlike \name and \coc is the
%absence of a conversion rule 

The conversion rule allows one to derive $e:\tau_{{\mathrm{2}}}$ from the
derivation of $e:\tau_{{\mathrm{1}}}$ and the $\beta$-equality of $\tau_{{\mathrm{1}}}$ and
$\tau_{{\mathrm{2}}}$. This rule is important to \emph{automatically} allows 
terms with equivalent types to be considered type-compatible. 
To illustrate, let us consider a simple example. Suppose
we have a built-in base type $ \mathsf{Int} $ and: 
\[
f \equiv \lambda  \ottmv{x}  \ottsym{:}  \ottsym{(}  \lambda  \ottmv{y}  \ottsym{:}  \star  \ottsym{.}  \ottmv{y}  \ottsym{)} \, \mathsf{Int}  \ottsym{.}  \ottmv{x}
\]
Here $f$ is a simple identity function. However, the type 
of $x$ (i.e., $\ottsym{(}  \lambda  \ottmv{y}  \ottsym{:}  \star  \ottsym{.}  \ottmv{y}  \ottsym{)} \, \mathsf{Int}$), which is the argument of $f$, is interesting: it is 
an identity function on types, applied to an integer. 
Without the conversion rule, $f$ cannot be
applied to, say $3$ in \coc. However, given that $f$ is actually
$\beta$-convertible to $\lambda  \ottmv{x}  \ottsym{:}  \mathsf{Int}  \ottsym{.}  \ottmv{x}$, the conversion rule allows
the application of $f$ to $3$ by implicitly converting
$\lambda  \ottmv{x}  \ottsym{:}  \ottsym{(}  \lambda  \ottmv{y}  \ottsym{:}  \star  \ottsym{.}  \ottmv{y}  \ottsym{)} \, \mathsf{Int}  \ottsym{.}  \ottmv{x}$ to $\lambda  \ottmv{x}  \ottsym{:}  \mathsf{Int}  \ottsym{.}  \ottmv{x}$.

\paragraph{Decidability of Type-Checking and Strong Normalization} 
While the conversion rule in \coc brings a lot of convenience, an
unfortunate consequence is that it couples decidability of
type-checking with strong normalization of the
calculus~\cite{coc:decidability}.  However strong normalization does
not hold with general recursion. This is because due to the conversion
rule, any non-terminating term would force the type checker to go into
an infinitely loop (by constantly applying the conversion rule without
termination), thus rendering the type system undecidable.

To illustrate the problem of the conversion rule with general
recursion, let us consider a somewhat contrived example. Suppose that
$d$ is a ``dependent type'' where
\[d : \mathsf{Int}  \rightarrow  \star\]
and $d\,3$ or $d\,100$ all yield the same type. With general recursion
at hand, we can image a term $z$ that has type \[d\,\mathsf{loop}\]
where $\mathsf{loop}$ stands for any diverging computation of type
$ \mathsf{Int} $. What would happen if we try to type check the following
application: \[ \ottsym{(}  \lambda  \ottmv{x}  \ottsym{:}  \ottmv{d} \, 3  \ottsym{.}  \ottmv{x}  \ottsym{)} \, \ottmv{z}\]
Under the normal typing rules of \coc, the type checker would get
stuck as it tries to do $\beta$-equality on two terms: $d\,3$ and
$d\,\mathsf{loop}$, where the latter is non-terminating.  \bruno{show
  simple example. Explain issue better.} \jeremy{done!}

\subsection{An Alternative to the Conversion Rule: Explicit Casts}

\bruno{Mention somewhere that the cast rules do \emph{one-step}
  reductions.} \jeremy{done! see last paragraph, also put beta
  reduction before beta expansion} In contrast to the implicit
reduction rules of \coc, \name makes it explicit as to when and where
to convert one type to another. Type conversions are explicit by
introducing two language constructs: $ \mathsf{cast}_{\downarrow} $ (beta reduction)
and $ \mathsf{cast}^{\uparrow} $ (beta expansion). The benefit of this approach is
that decidability of type-checking no longer is coupled with strong
normalization of the calculus.

\paragraph{Beta Reduction} The first of the two type conversions
$ \mathsf{cast}_{\downarrow} $, allows a type conversion provided that the resulting
type is a \emph{beta reduction} of the original type of the term. The
use of $ \mathsf{cast}_{\downarrow} $ is better explained by the following simple
example. Suppose that
\[ g \equiv \lambda  \ottmv{x}  \ottsym{:}  \mathsf{Int}  \ottsym{.}  \ottmv{x} \]
and term $z$ has type
\[ \ottsym{(}  \lambda  \ottmv{y}  \ottsym{:}  \star  \ottsym{.}  \ottmv{y}  \ottsym{)} \, \mathsf{Int} \]
$ g\,z $ is an ill-typed application, whereas $ g\,(\mathsf{cast}_{\downarrow} \, \ottmv{z}) $
is type correct. This is witnessed by
$\ottsym{(}  \lambda  \ottmv{y}  \ottsym{:}  \star  \ottsym{.}  \ottmv{y}  \ottsym{)} \, \mathsf{Int} \rightarrow_{\beta}  \mathsf{Int} $, which is a beta
reduction for term $\ottsym{(}  \lambda  \ottmv{y}  \ottsym{:}  \star  \ottsym{.}  \ottmv{y}  \ottsym{)} \, \mathsf{Int}$. \bruno{explain why this is a
  reduction} \jeremy{done!}

\paragraph{Beta Expansion} The dual operation of $ \mathsf{cast}_{\downarrow} $ is
$ \mathsf{cast}^{\uparrow} $, which allows a type conversion provided that the resulting
type is a \emph{beta expansion} of the original type of the term.
Consider the same example from \S\ref{sec:coc}. In \name, $f\,3$ is an
ill-typed application. Instead we must write the application as
\[ f\,(\mathsf{cast}^{\uparrow} \, \ottsym{[}  \ottsym{(}  \lambda  \ottmv{y}  \ottsym{:}  \star  \ottsym{.}  \ottmv{y}  \ottsym{)} \, \mathsf{Int}  \ottsym{]} \,  3) \]
\bruno{how to put a space before $3$?} \jeremy{fixed!} The intuition
is that, $ \mathsf{cast}^{\uparrow} $ is actually doing a type conversion becuase the
type of $ 3 $ is $  \mathsf{Int}  $, and $ \ottsym{(}  \lambda  \ottmv{y}  \ottsym{:}  \star  \ottsym{.}  \ottmv{y}  \ottsym{)} \, \mathsf{Int} $ is the beta
expansion of $ \mathsf{Int} $ (witnessed by
$\ottsym{(}  \lambda  \ottmv{y}  \ottsym{:}  \star  \ottsym{.}  \ottmv{y}  \ottsym{)} \, \mathsf{Int} \rightarrow_{\beta}  \mathsf{Int} $). \bruno{explain why
  this is a beta expansion} \jeremy{done!} Notice that for
$ \mathsf{cast}^{\uparrow} $ to work, we need to provide the resulting type as
argument. This is because for the same term, there are more than one
choices for beta expansions (e.g., $1 + 2$ and $2 + 1$ are both the
beta expansions for $3$). \bruno{explain why for beta expansions we
  need to provide the resulting type as argument} \jeremy{done!}

A final point to make is that the cast rules specify \emph{one-step}
reduction. This enables us to have more control over type-level
computation. The full technical details about cast rules are presented
in \S\ref{sec:ecore}.

\subsection{Decidability without Strong Normalization}

With explicit type conversion rules the decidability of type-checking 
no longer depends on the normalization property. 
A nice consequence of this is that the type system remains decidable
even in the presence of non-terminating programs at type level.
%As we will see in later sections. The
%ability to write non-terminating terms motivates us to have more
%control over type-level computation. 
% To illustrate, let us consider the following example. Suppose that $d$ is a ``dependent type'' where
% \[d : \mathsf{Int}  \rightarrow  \star\] so that $d\,3$ or $d\,100$ all yield the same
% type. With general recursion at hand, we can image a term $z$ that has
% type \[d\,\mathsf{loop}\] where $\mathsf{loop}$ stands for any
% diverging computation and of type $ \mathsf{Int} $. What would happen if we
% try to type check the following application: \[ \ottsym{(}  \lambda  \ottmv{x}  \ottsym{:}  \ottmv{d} \, 3  \ottsym{.}  \ottmv{x}  \ottsym{)} \, \ottmv{z}\]
% Under the normal typing rules of \coc, the type checker would get
% stuck as it tries to do $\beta$-equality on two terms: $d\,3$ and
% $d\,\mathsf{loop}$, where the latter is non-terminating.

To illustrate, let us consider the same example discussed in
\S\ref{sec:coc}. Now the type checker won't get stuck when
type-checking the following application:
\[ \ottsym{(}  \lambda  \ottmv{x}  \ottsym{:}  \ottmv{d} \, 3  \ottsym{.}  \ottmv{x}  \ottsym{)} \, \ottmv{z} \]
where the type of $z$ is $d\,\mathsf{loop}$.  This is because in \name, type checker
only does syntactic comparison between $d\,3$ and $d\,\mathsf{loop}$,
instead of $\beta$-equality. Therefore it rejects the above
application as ill-typed. Indeed it is impossible to type-check the
application even with the use of $ \mathsf{cast}^{\uparrow} $ and/or $ \mathsf{cast}_{\downarrow} $:
one would need to write infinite number of $ \mathsf{cast}_{\downarrow} $'s to make
the type checker loop forever (e.g.,
$(\lambda  \ottmv{x}  \ottsym{:}  \ottmv{d} \, 3  \ottsym{.}  \ottmv{x})( \mathsf{cast}_{\downarrow} ( \mathsf{cast}_{\downarrow}  \dots z))$). But it is
impossible to write such program in reality.

In summary, \name achieves the decidability of type checking by
explicitly controlling type-level computation.  which is independent
of the normalization property, while supporting general recursion at
the same time.

\subsection{Recursion and Recursive Types}

\bruno{Show how in \name recursion and recursive types are unified.
Discuss that due to this unification the sensible choice for the
evaluation strategy is call-by-name. }

A simple extension to \name is to add a simple recursion construct.
With such an extension, it becomes possible to write standard
recursive programs at the term level. At the same time, the recursive
construct can also be used to model recursive types at the type-level.
Therefore, \name differs from other programming languages in that it
unifies both recursion and recursive types by the same $\mu$
primitive. With a single language construct we get two powerful
features!

\paragraph{Recursion}

The $\mu$ primitive can be used to define recursive functions.  For
example the factorial function is defined as follows:
\[
  \miu{f}{\mathsf{Int} \rightarrow \mathsf{Int}}{\lam{x}{\mathsf{Int}}{\mathsf{if}\,(x == 0)\,\mathsf{then}\,1\,\mathsf{else}\,x * f\,(x - 1)}}
\]
The above recursive definition works because of the dynamic semantics of the
$\mu$ primitive: \ottusedrule{\ottdruleSXXMu{}} which is exactly doing
recursive unfolding of the same term.

% The dynamic semantics of $\mu$ requires the recursive binder to satisfy (omit type annotations for clarity):  \[ \mu f.\,E = (\lambda f.\,E) (\mu f.\,E) \] which, however, does not terminate in strict languages. Therefore, to loosen the function-type restriction to allow any types, the sensible choice for the evaluation strategy is \emph{call-by-name}.

\subsubsection{Recursive types}
In the literature on type systems, there are two approaches to
recursive types. One is called \emph{equi-recursive}, the other
\emph{iso-recursive}. The \emph{iso-recursive} approach treats a
recursive type and its unfolding as different, but isomorphic. The
isomorphism between a recursive type and its one step unfolding is
witnessed by traditionally \fold and \unfold operations. In \name, the
isomorphism is witnessed by first $ \mathsf{cast}^{\uparrow} $, then
$ \mathsf{cast}_{\downarrow} $. \bruno{Explain that the casts generalize fold and
  unfold!}  \jeremy{done!} At first sight, the cast rules share some
similarities with \fold and \unfold, but $ \mathsf{cast}^{\uparrow} $ and
$ \mathsf{cast}_{\downarrow} $ actually generalize \fold and \unfold: they can convert
any types, not just recursive types. To demonstrate the use of the
cast rules, let us consider a classic example of a recursive type, the
so-called ``hungry'' type~\cite{tapl}:
$H = \miu{\sigma}{\star}{\mathsf{Int} \rightarrow \sigma}$. A term $z$
of type $H$ can accept any number of integers and return a new
function that is hungry for more, as illustrated below:

\begin{align*}
\mathsf{cast}_{\downarrow} \, \ottmv{z} &:  \mathsf{Int}  \rightarrow H  \\
\mathsf{cast}_{\downarrow} \, \ottsym{(}  \mathsf{cast}_{\downarrow} \, \ottmv{z}  \ottsym{)} &:  \mathsf{Int}  \rightarrow  \mathsf{Int}  \rightarrow H \\
 \mathsf{cast}_{\downarrow} ( \mathsf{cast}_{\downarrow}  \dots z) &:  \mathsf{Int}  \rightarrow  \mathsf{Int}  \rightarrow \dots \rightarrow H
\end{align*}

Due to the unification of recursive types and recursion, we can use
the same $\mu$ primitive to write both recursive types and recursion
with ease.

\paragraph{Call-by-Name}
Due to the unification, the \emph{call-by-value} evaluation strategy
does not fit in our setting. In call-by-value evaluation, recursion
can be expressed by the recursive binder $\mu$ as $\mu f : T
\rightarrow T.\, E$ (note that the type of $f$ is restricted to
function types). Since we don't want to pose restrictions on the
types, the \emph{call-by-name} evaluation is a sensible choice.
\bruno{Probably needs to be improved. I'll came back to this later!}

\subsection{Logical Inconsistency}

\bruno{Explain that the \name is inconsistent and discuss that this is
  a deliberate design decision, since we want to model languages like
  Haskell, which are logically inconsistent as well.} \bruno{Discuss
  the $* : *$ rule: since we already have inconsistency, having this
  rule adds expressiveness and simplifies the system.} \jeremy{added!}

One consequence of adding general recursion to the type system is that
the logical consistency of the system is broken. This is a deliberate
design decision, since our goal is to model languages like Haskell,
which are logically inconsistent as well.

In light of the fact that we decide to give up consistency, we take
another step further by declaring that the kind $\star$ is of type
$\star$. As it turns out, having this rule adds expressiveness and
simplifies our system. We return to this issue in \S\ref{sec:related}.


\subsection{Encoding Datatypes}


With the explicit type conversion rules and the $\mu$ primitive, it is
easy to encode recursive datatypes and recursive functions using
datatypes. While inductive datatypes can be encoded using either the
Church or the Scott encoding, we adopt the Scott encoding as it
encodes case analysis, making it more convenient to
encode pattern matching. We demonstrate the encoding method using a
simple datatype as a running example: Peano numbers.

The datatype declaration for Peano numbers in Haskell is:
\begin{hscode}\SaveRestoreHook
\column{B}{@{}>{\hspre}l<{\hspost}@{}}%
\column{4}{@{}>{\hspre}l<{\hspost}@{}}%
\column{E}{@{}>{\hspre}l<{\hspost}@{}}%
\>[4]{}\mathbf{data}\;\Conid{Nat}\mathrel{=}\Conid{Z}\mid \Conid{S}\;\Conid{Nat}{}\<[E]%
\ColumnHook
\end{hscode}\resethooks
In the Scott encoding, the encoding of the \emph{Nat} datatype reflects
how its two constructors are going to be used. Since \emph{Nat} is a
recursive datatype, we have to use recursive types at some point to
reflect its recursive nature. As it turns out, the \emph{Nat} datatype
can be represented as \ensuremath{\mu\;\Conid{X}\mathbin{:}\star.\,\Pi\;\Conid{B}\mathbin{:}\star.\,\Conid{B}\to (\Conid{X}\to \Conid{B})\to \Conid{B}}.

As can be seen, in the function arrow \ensuremath{\Conid{B}\to (\Conid{X}\to \Conid{B})\to \Conid{B}}, $B$
corresponds to the type of the constructor \emph{Z}, and \ensuremath{\Conid{X}\to \Conid{B}}
corresponds to the type of the constructor \emph{S}. The intuition is
that any use of the datatype in the data constructors is replaced with
the recursive type variable ($X$ in the case), and we make the
resulting type variable ($B$ in this case) universally quantified so
that elements of the datatype with different result types can be used
in the same program~\cite{gadts}.

Now its two constructors can be encoded correspondingly as below:
\begin{hscode}\SaveRestoreHook
\column{B}{@{}>{\hspre}l<{\hspost}@{}}%
\column{3}{@{}>{\hspre}l<{\hspost}@{}}%
\column{5}{@{}>{\hspre}l<{\hspost}@{}}%
\column{E}{@{}>{\hspre}l<{\hspost}@{}}%
\>[3]{}\mathbf{let}\;\Conid{Z}\mathbin{:}\Conid{Nat}\mathrel{=}\mathsf{cast}^\uparrow\;[\mskip1.5mu \Conid{Nat}\mskip1.5mu]\;(\lambda \Conid{B}\mathbin{:}\star.\,\lambda \Varid{z}\mathbin{:}\Conid{B}.\,\lambda \Varid{f}\mathbin{:}\Conid{Nat}\to \Conid{B}.\,\Varid{z}){}\<[E]%
\\
\>[3]{}\mathbf{in}{}\<[E]%
\\
\>[3]{}\mathbf{let}\;\Conid{S}\mathbin{:}\Conid{Nat}\to \Conid{Nat}\mathrel{=}\lambda \Varid{n}\mathbin{:}\Conid{Nat}.\,{}\<[E]%
\\
\>[3]{}\hsindent{2}{}\<[5]%
\>[5]{}\mathsf{cast}^\uparrow\;[\mskip1.5mu \Conid{Nat}\mskip1.5mu]\;(\lambda \Conid{B}\mathbin{:}\star.\,\lambda \Varid{z}\mathbin{:}\Conid{B}.\,\lambda \Varid{f}\mathbin{:}\Conid{Nat}\to \Conid{B}.\,\Varid{f}\;\Varid{n}){}\<[E]%
\ColumnHook
\end{hscode}\resethooks

Thanks to the explicit type conversion rules, we can make use of the
$ \mathsf{cast}^{\uparrow} $ operation to do type conversion between the recursive
type and its unfolding.

As the last example, let us see how we can define recursive functions
using the \emph{Nat} datatype. A simple example would be recursively
adding two natural numbers, which can be defined as below:
\begin{hscode}\SaveRestoreHook
\column{B}{@{}>{\hspre}l<{\hspost}@{}}%
\column{3}{@{}>{\hspre}l<{\hspost}@{}}%
\column{5}{@{}>{\hspre}l<{\hspost}@{}}%
\column{7}{@{}>{\hspre}l<{\hspost}@{}}%
\column{E}{@{}>{\hspre}l<{\hspost}@{}}%
\>[3]{}\mathbf{let}\;\Varid{add}\mathbin{:}\Conid{Nat}\to \Conid{Nat}\to \Conid{Nat}\mathrel{=}\mu\;\Varid{f}\mathbin{:}\Conid{Nat}\to \Conid{Nat}\to \Conid{Nat}.\,{}\<[E]%
\\
\>[3]{}\hsindent{2}{}\<[5]%
\>[5]{}\lambda \Varid{n}\mathbin{:}\Conid{Nat}.\,\lambda \Varid{m}\mathbin{:}\Conid{Nat}.\,{}\<[E]%
\\
\>[5]{}\hsindent{2}{}\<[7]%
\>[7]{}(\mathsf{cast}_\downarrow\;\Varid{n})\;\Conid{Nat}\;\Varid{m}\;(\lambda \Varid{n'}\mathbin{:}\Conid{Nat}.\,\Conid{S}\;(\Varid{f}\;\Varid{n'}\;\Varid{m})){}\<[E]%
\ColumnHook
\end{hscode}\resethooks
The above definition quite resembles case analysis commonly seen in
modern functional programming languages. (We formalize the encoding of
case analysis in \S\ref{sec:surface}.)



%%% Local Variables:
%%% mode: latex
%%% TeX-master: "../main"
%%% End:



\section{Applications}
\label{sec:app}

\jeremy{Fill in large examples like monad, Fix, HOAS, dependent types.}

%%% Local Variables:
%%% mode: latex
%%% TeX-master: "../main"
%%% End:

%include sections/example.lhs

\section{The Explicit Calculus of Constructions}

\bruno{Linus: can you write up this section? I think this section should be your priority.
First bring in all results and formalization: syntax; semantics; proofs ... then write text}

This section formalizes the syntax and semantics of the explicit calculus 
of constructions. This section also shows that how in the explicit 
calculus of constructions decidability of the type system does not 
depend on strong normalization.

\begin{itemize}
\item Give an overview of the core language and its syntax.
\item Show the typing rules and operational semantics.
\item The original formalization is suggested to rewrite using \textsf{ott}\footnote{\url{http://www.cl.cam.ac.uk/~pes20/ott/}} which is a standard in academia. For example, the formalization of GHC \url{https://github.com/ghc/ghc/tree/master/docs/core-spec}.
\item Give formal proof of the soundness of the core language.
\item Subject reduction and progress theorems will be proved.
\end{itemize}

\section{The Explicit Calculus of Constructions with Recursion}

This section shows how to extend $\lambda C\beta$ with recursion. This extension 
allows the calculus to account for both: 1) recursive definitions; 2) recursive types. 
The extension preserves the decidability and soundness of the type system.




%%% !!! WARNING: AUTO GENERATED. DO NOT MODIFY !!! %%%
\newcommand{\FV}{\mathsf{FV}}
\newcommand{\dom}{\mathsf{dom}}

\section{Surface language}
\label{sec:surface}

% \begin{itemize}
% \item Expand the core language with datatypes and pattern matching by encoding.
% \item Give translation rules.
% \item Encode GADTs and maybe other Haskell extensions? GADTs seems challenging, so perhaps some other examples would be datatypes like $Fix f$, and $Monad$ as a record. Could formalize records in Haskell style.
% \end{itemize}

In this section, we present a surface language built on top of \name with
features that are convenient for functional programming: user-defined
datatypes, and pattern matching. Thanks to the expressiveness of
\name, all these features can be elaborated into the core language
without extending the built-in language constructs of \name. In what
follows, we first give the syntax of the surface language, followed by
the extended typing rules, then we show the formal translation rules
that translates a surface language expression to an expression in
\name. Finally we prove the type-safety of the translation.

\subsection{Extended Syntax}

% \newcommand{\case}{\mathbf{case}} \newcommand{\of}{\mathbf{of}}
% \newcommand{\data}{\mathbf{data}} \newcommand{\where}{\mathbf{where}}
% \newcommand{\letbb}{\mathbf{let}} \newcommand{\inb}{\mathbf{in}}

The full syntax of the surface language is defined in Figure~\ref{fig:surface:syntax}. Compared with \name, the surface language has a new syntax category: a program, consisting of a list of datatype declarations, followed by a expression. An \emph{algebraic data type} $D$ is introduced as a top-level \textbf{data} declaration with its \emph{data constructors}. For the purpose of presentation, we sometimes adopt the following syntactic convention:
\[
\overline{\tau}^n \rightarrow \tau_r \equiv \tau_1 \rightarrow \dots \rightarrow \tau_n \rightarrow \tau_r
\]
The type of a data constructor $K$ has the form:
\[
\ottmv{K}  \ottsym{:}  \ottsym{(}  \,\overline{  \ottnt{u}  \ottsym{:}  \kappa  }\,  \ottsym{)}  \rightarrow  \,\overline{  T  }\,  \rightarrow  \ottmv{D}    \,\overline{  \ottnt{u}  }\,
\]
\bruno{this looks a bit odd for a number of reasons: firstly why to
  insist on having the quantified variables in the same order as the
  arguments in the constructor? Secondly it seems that all other
  arguments cannot be dependently typed? 
It seems to me that 
\[
\ottmv{K}  \ottsym{:}  \ottsym{(}  \,\overline{  \ottmv{x}  \ottsym{:}  \kappa  }\,  \ottsym{)}  \rightarrow  \ottmv{D}    \,\overline{  \ottnt{u}  }\,
\]
where all variables $u$ are bound ($u \in \overline{x}$) would be
better. 
}
The quantified type variables $\,\overline{  \ottnt{u}  }\,$ appear in the same order in
the return type $\ottmv{D}    \,\overline{  \ottnt{u}  }\,$. Note that the use of dependent product
to tie together the type constructor arguments ($\ottsym{(}  \,\overline{  \ottnt{u}  \ottsym{:}  \kappa  }\,  \ottsym{)}$) makes
it possible to let the type of some type constructor arguments depend
on other type constructor arguments. The \textbf{case} expression is
conventional, used to break up values built with data constructors.
The patterns of a case expression are flat (no nested patterns), and
bind value variables.

\begin{figure*}
\centering
\gram{\ottpgm\ottinterrule
\ottdecl\ottinterrule
\ottu\ottinterrule
\ottp\ottinterrule
\ottE\ottinterrule
\ottV\ottinterrule
\ottGs}
    \[
    \begin{array}{lllll}
     &&&& \text{Syntactic Sugar} \\
     T_{{\mathrm{1}}}  \rightarrow  T_{{\mathrm{2}}} & \triangleq & \Pi \, \ottmv{x}  \ottsym{:}  T_{{\mathrm{1}}}  \ottsym{.}  T_{{\mathrm{2}}} & x \not \in \FV(T_{{\mathrm{2}}}) & \quad\text{Function type} \\
     \ottsym{(}  \ottmv{x}  \ottsym{:}  T_{{\mathrm{1}}}  \ottsym{)}  \rightarrow  T_{{\mathrm{2}}} & \triangleq & \Pi \, \ottmv{x}  \ottsym{:}  T_{{\mathrm{1}}}  \ottsym{.}  T_{{\mathrm{2}}} & x \in \FV(T_{{\mathrm{2}}}) & \quad\text{Dependent function type} \\
     \kw{data} \, \ottmv{R}  \,\overline{  \ottnt{u}  \ottsym{:}  \kappa  }\,  \ottsym{=}  \ottmv{K}  \ottsym{\{}  \,\overline{  \ottmv{N}  \ottsym{:}  T  }\,  \ottsym{\}} & \triangleq &
                    \kw{data} \, \ottmv{R}  \,\overline{  \ottnt{u}  \ottsym{:}  \kappa  }\,  \ottsym{=}  \ottmv{K}  \,\overline{  T  }\,; && \quad\text{Record} \\
                  &&  \kw{let} ~\ottmv{N}_i : \ottsym{(}  \,\overline{  \ottnt{u}  \ottsym{:}  \kappa  }\,  \ottsym{)}  \rightarrow  \ottmv{R}    \,\overline{  \ottnt{u}  }\,  \rightarrow  T_i = & \forall i &  \\
                  && \lambda  \,\overline{  \ottnt{u}  \ottsym{:}  \kappa  }\,  \ottsym{.}  \lambda  \ottmv{y}  \ottsym{:}  \ottmv{R} \, \,\overline{  \ottnt{u}  }\,  \ottsym{.}  \kw{case} \, \ottmv{y} \, \kw{of} \, \ottmv{K}  \,\overline{  \ottmv{x}  \ottsym{:}  T  }\,  \Rightarrow  \ottmv{x}_i~ \kw{in}  && \\
    \end{array}
    \]
\caption{Syntax of the surface language}
\label{fig:surface:syntax}
\end{figure*}


% \begin{figure}[ht]
% \centering
% \small
% \[
%   \begin{array}{llll}
%     \textbf{Declarations} \\
%     pgm &::= & \overline{decl}; e & \text{Declarations} \\
%     decl &::= & \data\,D\,\overline{u : \kappa} = \overline{\mid K\,\overline{\tau}} & \text{Datatype} \\ \\
%     \textbf{Terms} \\
%     u &::= & x \mid K & \text{Variables and constructors} \\
%     e,\tau,\sigma,\upsilon,\kappa &::= & u & \text{Term atoms} \\
%     & & \dots \\
%     & \mid & \case\,e\,\of\,\overline{p \Rightarrow e} & \text{Case analysis} \\
%     p &::= & K\,\overline{x : \tau} & \text{Pattern} \\ \\
%     \textbf{Environments} \\
%     \Gamma &::= &\varnothing & \text{Empty} \\
%     & \mid & \Gamma,u:\tau & \text{Variable binding}
%   \end{array}
% \]
%   \caption{Syntax of \sufcc ($e$ for terms; $\tau,\sigma,\upsilon$ for types; $\kappa$ for kinds)}\label{fig:datasyn}
% \end{figure}

For the sake of programming, the surface language employs some
syntactic sugar. A non-dependent function type can be written as
$T_{{\mathrm{1}}}  \rightarrow  T_{{\mathrm{2}}}$. A dependent function type $\Pi \, \ottmv{x}  \ottsym{:}  T_{{\mathrm{1}}}  \ottsym{.}  T_{{\mathrm{2}}}$ is
abbreviated as $\ottsym{(}  \ottmv{x}  \ottsym{:}  T_{{\mathrm{1}}}  \ottsym{)}  \rightarrow  T_{{\mathrm{2}}}$ for easy reading. We also
introduce a Haskell-like record syntax, which is desugared to
datatypes with accompanying selector functions.

% \begin{figure}[ht]
% \centering
% \[
%   \begin{array}{l}
%     \data\,R\,\overline{u : \kappa} = K\,\{\,\overline{S:\tau}\,\} \triangleq \\
%     \data\,R\,\overline{u : \kappa} = K\,\overline{\tau} \\
%     \letbb\,S_{i} : \Pi \overline{u : \kappa}.R\,\overline{u} \rightarrow \tau_{i} = \\
%     \quad \lambda \overline{(u:\kappa)}.\lam{l}{R\,\overline{u}}{\case\,l\,\of\, K\,\overline{x:\tau} \Rightarrow x_{i}} \\
%     \inb
%   \end{array}
% \]
% \caption{Syntactic sugar for records}\label{fig:records}
% \end{figure}

\subsection{Extended Typing Rules}
\bruno{For typing and translation show only one figure (Figure 8), since the
  typing figure is just a subset. We can use gray to highlight the
  parts which belong to the translation.}

Figure~\ref{fig:surface:typing} defines the type system of the surface
language. Several new typing judgments appear in the type system. The
use of different subscripts of the judgments is to be distinct from
the one used in \name. Most rules of the type system are standard for
systems based on \coc, including the rules for the well-formedness of
contexts (\ruleref{EnvS\_Empty}, \ruleref{EnvS\_Var}), inferring the
types of variables (\ruleref{TS\_Var}), and dependent application
(\ruleref{TS\_App}). Two judgments $ \Sigma  \labeledjudge{pg}  \ottnt{pgm}  :  T $ and
$ \Sigma  \labeledjudge{d}  \ottnt{decl}  :  \Sigma' $ are of the essence to the type checking of the
surface language. The former type checks a whole program, and the
latter type checks datatype declarations.

Rule \ruleref{TSpgm\_Pgm} type checks a whole program. It first
type-checks the declarations, which in return gives a new typing
environment. Combined with the original environment, it then continues
to type check the expression and return the result type. Rule
\ruleref{TSpgm\_Data} is used to type check datatype declarations. It
first ensures the well-formedness of the type of the type constructor
application (of kind $\star$). Note that since our system adopts
$\star : \star$, this means we can express kind polymophism in the
type parameters. Finally it make sure the types of data constructors
are valid.

Rules \ruleref{TS\_Case} and \ruleref{Tpat\_Alt} handle the type
checking of case expressions. The conclusion of \ruleref{TS\_Case}
binds the right types to the scrutinee expression $\ottnt{E_{{\mathrm{1}}}}$ and
alternatives $\overline{p \Rightarrow E_2}$. The first premise of
\ruleref{Tpat\_Alt} binds the actual type constructor arguments to
$\,\overline{  \ottnt{u}  }\,$. The second premise checks whether the types of the
right-hand sides of each alternative, instantiated to the actual type
constructor arguments $\,\overline{  \ottnt{u}  }\,$, are equal. Finally the third
premise checks the well-formedness of the types of data constructor
arguments.
\bruno{Mention that we do not support refinement, as in GADTs?}

\begin{figure*}
\ottdefnctxsrc{}
\ottdefnpgmsrc{}
\ottdefndeclsrc{}
\ottdefnpatsrc{}
\ottdefnexprsrc{}
\caption{Typing rules of surface language}
\label{fig:surface:typing}
\end{figure*}

% \newcommand{\ctx}[2][\Gamma]{#1 \vdash #2}
% \newcommand{\ctxz}[1]{\ctx[\varnothing]{#1}}
% \newcommand{\ctxw}[3][\Gamma]{#1,#2 \vdash #3}

% \begin{figure*}
%   \centering \small
%   \begin{tabular}{lc}
%     \framebox{$\Gamma \vdash pgm : \tau$} \\
%     (Pgm) & $\inferrule {\overline{\Gamma_{0} \vdash decl : \Gamma_{d}} \\ \Gamma = \Gamma_{0}, \overline{\Gamma_{d}} \\ \ctx{e:\tau}} {\Gamma_{0} \vdash \overline{decl}; e : \tau}$ \\
%     \framebox{$\Gamma \vdash decl : \Gamma_d$} \\
%     (Data) & $\inferrule {\Gamma \vdash \overline{\kappa} \rightarrow \star : \square \\ \overline{\Gamma, D:\overline{\kappa} \rightarrow \star, \overline{u : \kappa} \vdash \overline{\tau} \rightarrow D\,\overline{u}:\star}} {\ctx{(\data\,D\,\overline{u : \kappa} = \overline{\mid K\,\overline{\tau}}): (D : \overline{\kappa} \rightarrow \star, \overline{K : \Pi\overline{u : \kappa}.\overline{\tau} \rightarrow D\,\overline{u}})}}$ \\
%     \framebox{$\Gamma \vdash e : \tau$} \\
%     (Case) & $\inferrule {\ctx{e_{1}}:\sigma \\ \overline{\Gamma\vdash_{p} p \Rightarrow e_{2}:\sigma \rightarrow \tau}} {\Gamma\vdash\case\,e_{1}\,\of\,\overline{p \Rightarrow e_{2}}:\tau}$ \\
%     \framebox{$\Gamma \vdash_{p} p \Rightarrow e : \sigma \rightarrow \tau$} \\
%     (Alt) & $\inferrule {\theta=[\overline{u := \upsilon}] \\\\ K:\Pi\overline{u:\kappa}.\,\overline{\sigma} \rightarrow D\,\overline{u} \in \Gamma \\ \Gamma,\overline{x:\theta(\sigma)} \vdash e:\tau} {\Gamma \vdash_{p} K\,\overline{x:\theta(\sigma)} \Rightarrow e : D\,\overline{\upsilon} \rightarrow \tau}$
%   \end{tabular}
%   \caption{Typing rules of \sufcc}\label{fig:datatype}
% \end{figure*}

\subsection{Translation Overview}

We use a type-directed translation. The basic translation rules have the form:
\[
 \Sigma  \labeledjudge{s}  \ottnt{E}  :  T   \rightsquigarrow   \ottnt{e} 
\]

It states that \name expression $\ottnt{e}$ is the translation of the
surface expression $\ottnt{E}$ of type $T$.
Figure~\ref{fig:source:translate} defines the translation rules, which
are the typing rules in Figure~\ref{fig:surface:typing} extended with
the resulting expression $\ottnt{e}$. In the translation, we require that
applications of constructors to be \emph{saturated}.\bruno{Any
  partiular reasons for this?}

Among others, Rules \ruleref{TRdecl\_Data}, \ruleref{TRpat\_Alt} and \ruleref{TR\_Case} are of the essence to the translation. Rule \ruleref{TR\_Case} translates case expressions into applications by first translating the scrutinee expression, casting it down to the right type. It is then applied to the result type of the body expression and a list of translated \name expressions of its alternatives. Rule \ruleref{TRpat\_Alt} tells how to translate each alternative. Basically it translates an alternative into a lambda abstraction, where each bound variable in the pattern corresponds to a bound variable in the lambda abstraction in the same order. The body in the alternative is recursively translated and treated as the body of the lambda abstraction.

Rule \ruleref{TRDecl\_Data} does the most heavy work and deserves further explanation. First of all, it results in an incomplete expression (as can be seen by the incomplete \emph{let} expressions), The result expression is supposed to be prepended to the translation of the last expression to form a complete \name expression, as specified by Rule \ruleref{TRpgm\_Pgm}. Furthermore, each type constructor is translated to a recursive type, of which the body is a type-level lambda abstraction. What is interesting is that each recursive mention of the datatype in the data constructor parameters is replaced with the recursive variable $\ottmv{X}$. Note that for the moment, the result type variable $\alpha$ is restricted to have kind $\star$. This could pose difficulties when translating GADTs as we will discussion in the future work. Each data constructor is translated to a lambda abstraction. Notice how we use \castup in the lambda body to get the right type.

The rest of the translation rules hold few surprises.

\begin{figure*}
\ottdefnctxtrans{}
\ottdefnpgmtrans{}
\ottdefndecltrans{}
\begin{align*}
\ottnt{e}  \triangleq ~  &  \kw{let} \, \ottmv{D}  \ottsym{:}  \ottsym{(}  \,\overline{  \ottnt{u}  \ottsym{:}  \sigma  }\,  \ottsym{)}  \rightarrow  \star  \ottsym{=}  \mu \, \ottmv{X}  \ottsym{:}  \ottsym{(}  \,\overline{  \ottnt{u}  \ottsym{:}  \sigma  }\,  \ottsym{)}  \rightarrow  \star  \ottsym{.}  \lambda  \,\overline{  \ottnt{u}  \ottsym{:}  \sigma  }\,  \ottsym{.}  \ottsym{(}  \alpha  \ottsym{:}  \star  \ottsym{)}  \rightarrow  \,\overline{  \ottsym{(}  \,\overline{  \tau  }\,  \ottsym{[}  \ottmv{D}  \mapsto  \ottmv{X}  \ottsym{]} \,  \rightarrow  \alpha  \ottsym{)}  }\,  \rightarrow  \alpha \, \kw{in} \, \\ &  \kw{let} \, \ottmv{K_{\ottmv{i}}}  \ottsym{:}  \ottsym{(}  \,\overline{  \ottnt{u}  \ottsym{:}  \sigma  }\,  \ottsym{)}  \rightarrow  \,\overline{  \tau  }\,  \rightarrow  \ottmv{D}    \,\overline{  \ottnt{u}  }\,  \ottsym{=}  \lambda  \,\overline{  \ottnt{u}  \ottsym{:}  \sigma  }\,  \ottsym{.}  \lambda  \,\overline{  \ottmv{x}  \ottsym{:}  \tau  }\,  \ottsym{.}  \kw{cast}_{\uparrow}^n \, \ottsym{[}  \ottmv{D}    \,\overline{  \ottnt{u}  }\,  \ottsym{]} \,  \ottsym{(}  \lambda  \alpha  \ottsym{:}  \star  \ottsym{.}  \lambda  \,\overline{  b  \ottsym{:}  \,\overline{  \tau  }\,  \rightarrow  \alpha  }\,  \ottsym{.}  b_{\ottmv{i}} \, \,\overline{  \ottmv{x}  }\,  \ottsym{)} \, \kw{in} \, 
\end{align*}
\ottdefnpattrans{}
\ottdefnexprtrans{}
\caption{Translation rules of source language}
\label{fig:source:translate}
\end{figure*}

\subsection{Type-safefy of Translation}

\jeremy{put Linus's theorem here}

% \begin{figure*}
%   \renewcommand{\arraystretch}{2.5}
%   \centering \small
%   \begin{tabular}{lcl}
%     \framebox{$\Gamma \vdash e : \tau \rightsquigarrow E$} \\
%     (Ax) & $\inferrule { } {\ctxz{\star:\square \rightsquigarrow \star}}$ \\

%     (Var) & $\inferrule {x:\tau \in \Gamma} {\ctx{x:\tau \rightsquigarrow x}}$ \\

%     (App) & $\inferrule {\ctx{e_{1}:(\pai{x}{\tau_{2}}{\tau_{1}}) \rightsquigarrow E_{1}} \\ \ctx{e_{2}:\tau_{2} \rightsquigarrow E_{2}}} {\ctx{e_{1}e_{2}:\tau_{1}[x:=e_{2}] \rightsquigarrow E_{1}E_{2}}}$ \\

%     (Lam) & $\inferrule {\ctxw{x:\tau_{1}}{e:\tau_{2} \rightsquigarrow E} \\ \ctx{(\pai{x}{\tau_{1}}{\tau_{2}}):t}} {\ctx{(\lam{x}{\tau_{1}}{e})}:(\pai{x}{\tau_{1}}{\tau_{2}}) \rightsquigarrow \lam{x}{\tau_{1}}{E}}$ & $t \in \{\star, \square\}$ \\

%     (Pi) & $\inferrule {\ctx{\tau_{1}:s} \\ \ctxw{x:\tau_{1}}{\tau_{2}:t}} {\ctx{(\pai{x}{\tau_{1}}{\tau_{2}}):t \rightsquigarrow \pai{x}{\tau_{1}}{\tau_{2}}}}$ & $(s,t) \in \mathcal{R}$ \\

%     (Mu) & $\inferrule {\ctxw{x:\tau}{e:\tau} \rightsquigarrow E \\ \ctx{\tau:s}} {\ctx{(\miu{x}{\tau}{e}):\tau} \rightsquigarrow \miu{x}{\tau}{E}}$ & $s \in \{\star, \square\}$ \\

%     (Fold) & $\inferrule {\ctx{e:\tau_{2} \rightsquigarrow E} \\ \ctx{\tau_{1}:s} \\ \tau_{1} \longrightarrow \tau_{2}} {\ctx{(\fold{{\tau_{1}}}{e}):{\tau_{1}} \rightsquigarrow \fold{\tau_{1}}{E}}}$ \\

%     (Unfold) & $\inferrule {\ctx{e:\tau_{1} \rightsquigarrow E} \\ \ctx{\tau_{2}:s} \\ \tau_{1} \longrightarrow \tau_{2}} {\ctx{(\unfold{e}):\tau_{2} \rightsquigarrow \unfold{E}}}$ \\

%     (Case) & $\inferrule {\ctx{e_{1}:\sigma \rightsquigarrow E_{1}} \\ \overline{\Gamma\vdash_{p} p \Rightarrow e_{2}:\sigma \rightarrow \tau \rightsquigarrow E_{2}}} {\Gamma\vdash\case\,e_{1}\,\of\,\overline{p \Rightarrow e_{2}}:\tau \rightsquigarrow (\unfold{E_{1}})\,\tau\,\overline{E_{2}}}$ \\
%     \framebox{$\Gamma \vdash_{p} p \Rightarrow e : \sigma \rightarrow \tau \rightsquigarrow E$} \\
%     (Alt) & $\inferrule {\theta=[\overline{u := \upsilon}] \\\\ K:\Pi\overline{u:\kappa}.\,\overline{\sigma} \rightarrow D\,\overline{u} \in \Gamma \\ \Gamma,\overline{x:\theta(\sigma)} \vdash e:\tau \rightsquigarrow E} {\Gamma \vdash_{p} K\,\overline{x:\theta(\sigma)} \Rightarrow e : D\,\overline{\upsilon} \rightarrow \tau \rightsquigarrow \lambda(\overline{x:\theta(\sigma)}).E}$ \\
%     \framebox{$\Gamma \vdash decl : \Gamma_d \rightsquigarrow E$} \\
%     (Data) & $\inferrule {\Gamma \vdash \overline{\kappa} \rightarrow \star : \square \\ \overline{\Gamma, D:\overline{\kappa} \rightarrow \star, \overline{u : \kappa} \vdash \overline{\tau} \rightarrow D\,\overline{u}:\star}} {\ctx{(\data\,D\,\overline{u : \kappa} = \overline{\mid K\,\overline{\tau}}): (D : \overline{\kappa} \rightarrow \star, \overline{K : \Pi\overline{u : \kappa}.\overline{\tau} \rightarrow D\,\overline{u}}) \rightsquigarrow E}}$ \\
%          & \begingroup \renewcommand*{\arraystretch}{1.0} $\begin{array} {lll}
%                                                              E & ::= & \letbb\,D : \overline{\kappa} \rightarrow \star =\lambda\overline{u : \kappa}.\,\mu X : \star.\,\Pi b:\star.\,\overline{(\overline{\tau}[D\,\overline{u}:=X] \rightarrow b)} \rightarrow b\,\inb \\ & & \letbb\,K_{i} : \Pi\overline{u : \kappa}.\overline{\tau} \rightarrow D\,\overline{u} = \lambda \overline{(u : \kappa)}.\lambda\overline{(x : \tau)}.\\
%                                                                     & & \quad \fold{D\,\overline{u}}{(\lambda (b : \star)\overline{(c : \overline{\tau} \rightarrow b)} . c_{i}\,\overline{x})}\,\inb \end{array}$ \endgroup \\
%     \framebox{$\Gamma \vdash pgm : \tau \rightsquigarrow E$} \\
%     (Pgm) & $\inferrule {\overline{\Gamma_{0} \vdash decl : \Gamma_{d} \rightsquigarrow E_{1}} \\ \Gamma = \Gamma_{0}, \overline{\Gamma_{d}} \\ \ctx{e:\tau \rightsquigarrow E}} {\Gamma_{0} \vdash \overline{decl}; e : \tau \rightsquigarrow \overline{E_{1}} \oplus E}$
%   \end{tabular}
%   \caption{Type-directed translation from \sufcc to \name}\label{fig:datatrans}
% \end{figure*}

%%% Local Variables:
%%% mode: latex
%%% TeX-master: "../main"
%%% End:


\section{Related Work}

\begin{itemize}
\item Henk \cite{pts:henk} and one of its implementation \cite{pts:fp} show the simplicity of the Pure Type System (PTS). \cite{pts:rec} also tries to combine recursion with PTS.

\item \textsf{Zombie} \cite{zombie:popl14, zombie:thesis} is a language with two fragments supporting logics with non-termination. It limits the $\beta$-reduction for congruence closure \cite{zombie:popl15}.

\item $\Pi\Sigma$ \cite{dep:pisigma} is a simple, dependently-typed core language for expressing high-level constructions\footnote{But the paper didn't give any meta-theories about the langauge.}. UHC compiler \cite{fc:uhc} tries to use a simplified core language with coercion to encode GADTs.

\item System $F_C$ \cite{fc} has been extended with type promotion \cite{fc:pro} and kind equality \cite{fc:kind}. The latter one introduces a limited form of dependent types into the system\footnote{Richard A. Eisenberg is going to implement kind equality \cite{fc:kind} into GHC. The implementation is proposed at \url{https://phabricator.haskell.org/D808} and related paper is at \url{http://www.cis.upenn.edu/~eir/papers/2015/equalities/equalities-extended.pdf}.}, which mixes up types and kinds.
\end{itemize}

\section{Conclusions and Future Work}

This work proposes a small dependently typed language that allows the
same syntax for terms and types, supports type-level computation, and
preserves decidable type checking under the presence of general
recursion. We consider this as a well-suited core for Haskell-like
languages.

Of course much remains to be done. For one thing, intensive type-level
computation can be written in \name, but is inconvenient to use. This
is because in \name, type-level computation is driven by cast
operations, and the number of cast operation needs to be specified
beforehand. Currently, for simple non-recursive functions, it is easy
to deduce how many casts needs to be introduced, but for recursive
ones, it becomes inconvenient. However, we are optimistic in this
regard. As future work, we plan to add language constructs to the
surface language, in the same spirit as type families in Haskell, to
guide type-level computation by passing around cast operations to the
core language.

For another, as we mentioned in the related work, \name lacks support
for GADTs. In our experiments, we have succeeded in encoding some
examples of GADTs, including \emph{Fin}. As it turns out, the
translation rules for datatypes can be extended to account for
GADTs. The issues are manifested in two strands: 1) The notion of
equality-proofs. Currently \name only supports syntactic equality. We
plan to explore more expressive form of type-equality; 2) Injectivity
of type constructors. \jeremy{more?}


%% -- References --

\acks
Thanks to Blah. This work is supported by Blah.

\bibliographystyle{abbrvnat}
\nocite{*}
\bibliography{main}

%% -- Appendix --

\appendix
%%% !!! WARNING: AUTO GENERATED. DO NOT MODIFY !!! %%%
\section{Specification of core language}

\subsection{Syntax}
\gram{\otte\ottinterrule
        \otts\ottinterrule
        \ottG\ottinterrule
        \ottv}

\subsection{Operational semantics and expression typing}
\ottdefnstep{}
\ottusedrule{\ottdruleSXXMu{}}
\ottdefnexpr{}
\ottusedrule{\ottdruleTXXMu{}}

\section{Proofs of core language}
\subsection{Decidability of type checking}
\begin{lem}[Uniqueness of one-step reduction]
	The relation $ \longrightarrow $, i.e. one-step reduction, is \textbf{unique} in the sense that given $e$ there is at most one $e'$ such that $\ottnt{e}  \longrightarrow  \ottnt{e'}$.
\end{lem}

\begin{proof}
	By induction on the structure of $e$:
	\begin{description}
		\item[Case $e=v$:] $e$ has one of the following forms:
		\begin{inparaenum}[(1)]
			\item $\lambda  \ottmv{x}  \ottsym{:}  \tau  \ottsym{.}  \ottnt{e}$,
			\item $\Pi \, \ottmv{x}  \ottsym{:}  \tau_{{\mathrm{1}}}  \ottsym{.}  \tau_{{\mathrm{2}}}$,
			\item $\mathsf{cast}^{\uparrow} \, \ottsym{[}  \tau  \ottsym{]}  \ottnt{e}$,
		\end{inparaenum}
		which cannot match any rules of $ \longrightarrow $. Thus there is no $e'$ such that $\ottnt{e}  \longrightarrow  \ottnt{e'}$.
		\item[Case $e=\ottsym{(}  \lambda  \ottmv{x}  \ottsym{:}  \tau  \ottsym{.}  \ottnt{e_{{\mathrm{1}}}}  \ottsym{)} \, \ottnt{e_{{\mathrm{2}}}}$:] There is a unique $e'=\ottnt{e_{{\mathrm{1}}}}  \ottsym{[}  \ottmv{x}  \mapsto  \ottnt{e_{{\mathrm{2}}}}  \ottsym{]}$ by rule \ruleref{S\_Beta}.
		\item[Case $e=\mathsf{cast}_{\downarrow} \, \ottsym{(}  \mathsf{cast}^{\uparrow} \, \ottsym{[}  \tau  \ottsym{]}  \ottnt{e}  \ottsym{)}$:] There is a unique $e'=e$ by rule \ruleref{S\_CastDownUp}.
		\item[Case $e=\mu \, \ottmv{x}  \ottsym{:}  \tau  \ottsym{.}  \ottnt{e}$:] There is a unique $e'=\ottnt{e}  \ottsym{[}  \ottmv{x}  \mapsto  \mu \, \ottmv{x}  \ottsym{:}  \tau  \ottsym{.}  \ottnt{e}  \ottsym{]}$ by rule \ruleref{S\_Mu}.
		\item[Case $e=\ottnt{e_{{\mathrm{1}}}} \, \ottnt{e_{{\mathrm{2}}}}$ with $\ottnt{e_{{\mathrm{1}}}}  \longrightarrow  \ottnt{e'_{{\mathrm{1}}}}$:] $\ottnt{e_{{\mathrm{1}}}}$ cannot be a $\lambda$-term $\lambda  \ottmv{x}  \ottsym{:}  \tau  \ottsym{.}  \ottnt{e}$ which is a value that contradicts $\ottnt{e_{{\mathrm{1}}}}$ can be reduced to $\ottnt{e'_{{\mathrm{1}}}}$. By the induction hypothesis, $\ottnt{e'_{{\mathrm{1}}}}$ is unique reduction of $\ottnt{e_{{\mathrm{1}}}}$. Thus by rule \ruleref{S\_App}, $e'=\ottnt{e'_{{\mathrm{1}}}} \, \ottnt{e_{{\mathrm{2}}}}$ is the unique reduction for $e$.
		\item[Case $e=\mathsf{cast}_{\downarrow} \, \ottnt{e_{{\mathrm{1}}}}$ with $\ottnt{e_{{\mathrm{1}}}}  \longrightarrow  \ottnt{e'_{{\mathrm{1}}}}$:] $\ottnt{e_{{\mathrm{1}}}}$ cannot have the form $\mathsf{cast}^{\uparrow} \, \ottsym{[}  \tau  \ottsym{]}  \ottnt{e}$ which is a value that contradicts $\ottnt{e_{{\mathrm{1}}}}$ can be reduced to $\ottnt{e'_{{\mathrm{1}}}}$. By the induction hypothesis, $\ottnt{e'_{{\mathrm{1}}}}$ is unique reduction of $\ottnt{e_{{\mathrm{1}}}}$. Thus by rule \ruleref{S\_CastDown}, $e'=\mathsf{cast}_{\downarrow} \, \ottnt{e'_{{\mathrm{1}}}}$ is the unique reduction for $e$.
	\end{description}
\end{proof}

\begin{dfn}[Well-formed context]
	A \textbf{well-formed} context $\Gamma$ is defined by the following rules:
	
	\textnormal{\ottdefnctx{}}
\end{dfn}

\begin{lem}[Consistency of well-formed context]\label{lem:wfc}
	Given a well-formed initial context $\Gamma$, it remains well-formed through type checking.
\end{lem}

\begin{proof}
	Suppose $\Gamma$ is the initial context which is well-formed. To safely extend $\Gamma$ with a variable $x:\tau$, one should have $\Gamma  \vdash  \tau  \ottsym{:}  \ottnt{s}$ due to rule \ruleref{Env\_Var}. Note that when applying typing rules of $\Gamma  \vdash  \ottnt{e}  \ottsym{:}  \tau$, rule \ruleref{T\_Pi}, \ruleref{T\_Mu} and \ruleref{T\_Lam} will extend the context. We show that these rules cover the condition $\Gamma  \vdash  \tau  \ottsym{:}  \ottnt{s}$ with respect to $x:\tau$ as follows:
	\begin{description}
		\item[Case \ruleref{T\_Pi}:] \ottusedrule{\ottdruleTXXPi{}} For $x:\tau_{{\mathrm{1}}}$, $\Gamma  \vdash  \tau_{{\mathrm{1}}}  \ottsym{:}  \ottnt{s}$ is directly the premise of the rule.
		\item[Case \ruleref{T\_Mu}:] \ottusedrule{\ottdruleTXXMu{}} For $x:\tau$, $\Gamma  \vdash  \tau  \ottsym{:}  \ottnt{s}$ is directly the premise of the rule.
		\item[Case \ruleref{T\_Lam}:] \ottusedrule{\ottdruleTXXLam{}} For $x:\tau_{{\mathrm{1}}}$, note that the premise $\Gamma  \vdash  \ottsym{(}  \Pi \, \ottmv{x}  \ottsym{:}  \tau_{{\mathrm{1}}}  \ottsym{.}  \tau_{{\mathrm{2}}}  \ottsym{)}  \ottsym{:}  \ottnt{s}$ can be derived from rule \ruleref{T\_Pi}, which has the pre-condition $\Gamma  \vdash  \tau_{{\mathrm{1}}}  \ottsym{:}  \ottnt{s}$.
	\end{description}
\end{proof}

\begin{lem}[Valid context optimization]\label{lem:wfcopt}
	With a well-formed initial context $\Gamma$, the \ruleref{T\_Var} and \ruleref{T\_Weak} can be replaced by the following rule: \ottusedrule{\ottdruleTSXXVar{}}
\end{lem}

\begin{proof}
	By Lemma \ref{lem:wfc}, the context $\Gamma$ remains well-formed if it is initially well-formed. Thus, it is not necessary to use \ruleref{T\_Var} and \ruleref{T\_Weak} to check the well-formedness of $\Gamma$. In order to check the type of a variable $x$, it is necessary and sufficient to check if $\ottmv{x}  \ottsym{:}  \tau \, \in \, \Gamma$, which is simply rule \ruleref{TS\_Var}. \linus{This proof needs to be more formally written.}
\end{proof}

\begin{lem}[Decidability of type checking]
	There is a decidable algorithm which given $\Gamma, \ottnt{e}$ computes the unique $\tau$ such that $\Gamma  \vdash  \ottnt{e}  \ottsym{:}  \tau$ or reports there is no such $\tau$.
\end{lem}

\begin{proof}
	By induction on the derivation of $e$:
	\begin{description}
		\item[Case $e=x$:] By Lemma \ref{lem:wfcopt}, we only need to consider context $\Gamma$ that is well-formed. By rule \ruleref{TS\_Var}, if $\ottmv{x}  \ottsym{:}  \tau \, \in \, \Gamma$, $t$ is the unique type of $x$.
		\item[Case $e=\ottnt{e_{{\mathrm{1}}}} \, \ottnt{e_{{\mathrm{2}}}}$:] By rule \ruleref{T\_App}, 
	\end{description} 
\end{proof}

\subsection{Properties}
\newcommand{\FV}{\mathsf{FV}}
\newcommand{\dom}{\mathsf{dom}}

\begin{lem}[Free variables lemma]
If $\Gamma  \vdash  \ottnt{e}  \ottsym{:}  \tau$, then $\FV(\ottnt{e}),\FV(\tau) \subseteq \dom(\Gamma)$.
\end{lem}

\begin{lem}[Generation lemma]
	
\end{lem}

\begin{lem}[Substitution lemma]

\end{lem}

\subsection{Soundness}
\begin{lem}[Subject reduction]
If $\Gamma  \vdash  \ottnt{e}  \ottsym{:}  \tau$ and $e  \twoheadrightarrow  e'$ then $\Gamma  \vdash  \ottnt{e'}  \ottsym{:}  \tau$.
\end{lem}

\begin{lem}[Progress]
If $\varnothing  \vdash  \ottnt{e}  \ottsym{:}  \tau$ then either $e$ is a value $v$ or there exists $e'$ such that $e  \twoheadrightarrow  e'$.
\end{lem}



%% -- The end --

\end{document}

%%% Local Variables:
%%% mode: latex
%%% TeX-master: t
%%% End:
