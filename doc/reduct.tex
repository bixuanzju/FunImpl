Since recursive types are introduced and due to the practical concern,
we use the \emph{call-by-name} reduction strategy, i.e. iteratively
reducing the \emph{left-most} \emph{outer-most}
redex. Figure~\ref{fig:mueval} shows the dynamic semantics with no
call-by-value specific premises or rules.

\begin{figure}[H]
  \centering
  \small
  \begin{syntax}
    \textbf{values:}\quad v &::= & \lam{x}{T}{E} & \ptext{abstraction} \\
    & \mid & \pai{x}{T_{1}}{T_{2}} & \ptext{product} \\
    & \mid & \hlmath{\fold{T}{E}} & \ptext{roll}
  \end{syntax}
  \begin{tabular}{lcl}
    (R-AppLam) & \ruleI{}{(\lam{x}{T}{E_{1}})E_{2} \tolong E_{1}[x:=E_{2}]} \\
    (R-AppL) & \ruleI{E_{1} \tolong E_{1}'}{E_{1}E_{2} \tolong E_{1}'E_{2}} \\
    (R-Unfold) & \hl{\ruleI{E \tolong E'}{\unfold{E} \tolong \unfold{E'}}} \\
    (R-Unfold-Fold) & \hl{\ruleI{}{\unfold{(\fold{T}{E})} \tolong E}} \\
    (R-Mu) & \hl{\ruleI{}{\miu{x}{T} \tolong T[x:=\miu{x}{T}]}}
  \end{tabular}
  \caption{Reduction rules for $\lambda C$}\label{fig:mueval}
\end{figure}

%%% Local Variables:
%%% mode: latex
%%% TeX-master: "pts"
%%% End:
