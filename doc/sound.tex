% Proof environments
\newtheorem{thm}{Theorem}[subsection]
\newtheorem{cor}{Corollary}[thm]
\newtheorem{lem}[thm]{Lemma}

\newcommand{\dotctx}[1]{\ctx[\cdot]{#1}}
\newcommand{\dotctxw}[2]{\ctxw[\cdot]{#1}{#2}}

\begin{lem}[Substitutions]\label{lem:subst}
  Assume we have
\begin{align}
  \Gamma,x:A &\vdash B:C \label{equ:s1}\\
  \Gamma &\vdash D:A, \label{equ:s2}
\end{align}
then \[\Gamma[x:=D] \vdash B[x:=D]:C[x:=D].\]
\end{lem}
\begin{proof}
  This is trivial by induction on the typing derivation of
  (\ref{equ:s1}) by typing rules in Fig.\ref{fig:mutype}. We only discuss two cases for example. Let $E^{*}$
  denote $E[x:=D]$. Consider following cases
\begin{itemize}
\item The last applied rule to obtain (\ref{equ:s1}) is \emph{Var}. There
  are 2 sub-cases:
\begin{enumerate}
\item It is derived by
\begin{center}
  \AxiomC{$\Gamma \vdash A:s$} \UnaryInfC{$\Gamma,x:A \vdash x:A$}
  \DisplayProof,
\end{center}
then we have $(B:C) \equiv (x:A)$. And
$\Gamma \vdash (x:A)^{*}\equiv (D:A)$ which holds by
(\ref{equ:s2}).

\item It is derived by
\begin{center}
  \AxiomC{$\Gamma,x:A \vdash E:s$} \UnaryInfC{$\Gamma,x:A,y:E \vdash y:E$}
  \DisplayProof,
\end{center}
then we need to show $\Gamma^{*},y:E^{*} \vdash y:E^{*}$. And it
directly follows the induction hypothesis, i.e.
$\Gamma^{*} \vdash E^{*}:s$.
\end{enumerate}

\item The last applied rule to obtain (\ref{equ:s1}) is \emph{App}, i.e.
\begin{center}
  \AxiomC{$\Gamma,x:A \vdash B_{1}:(\Pi y:C_{1}.\ C_{2})$}
  \AxiomC{$\Gamma,x:A \vdash B_{2}:C_{1}$}
  \BinaryInfC{$\Gamma,x:A \vdash (B_{1}B_{2}):C_{2}[y:=B_{2}]$}
  \DisplayProof.
\end{center}
By the induction hypothesis, we can obtain
$ \Gamma^{*} \vdash B_{1}^{*}:(\Pi y:C_{1}^{*}.C_{2}^{*})$ and
$\Gamma^{*} \vdash B_{2}^{*}:C_{1}^{*}$. Thus,
$\Gamma^{*} \vdash (B_{1}^{*}B_{2}^{*}):(C_{2}^{*}[y:=B_{2}^{*}])$, i.e.
$\Gamma^{*} \vdash (B_{1}B_{2})^{*}:(C_{2}[y:=B_{2}])^{*}$.
\end{itemize}
\end{proof}

\begin{thm}[Subject Reduction]
  If $\Gamma \vdash A:B$ and $A \tolong A'$ then
  $\Gamma \vdash A':B'$ for some $B'$ such that either $B' \equiv B$ or $B' \tolong B$.
\end{thm}

\begin{proof}
  Let $\mathcal{D}$ be the derivation of $\Gamma \vdash A:B$. The
  proof is by induction on dynamic semantics shown in Fig.\ref{fig:mueval}.

\begin{description}
\item[case \emph{R-AppLam}:] \ruleI{}{(\lam{x}{A}{M})N \tolong M[x:=N]}.

  Derivation $\mathcal{D}$ has the following form
\begin{center}
  \AxiomC{$\Gamma,x:A \vdash M:A'$} \RightLabel{\emph{Lam}}
  \UnaryInfC{$\Gamma \vdash (\lambda x:A.M):(\Pi x:A.A')$}
  \RightLabel{\emph{App}} \AxiomC{$\Gamma \vdash N:A$}
  \BinaryInfC{$\Gamma \vdash (\lambda x:A.M)N:A'$}
  \DisplayProof
\end{center}
Thus, by Lemma \ref{lem:subst} we can obtain
$\Gamma \vdash M[x:=N]:A'$.

\item[case \emph{R-AppL}:]
  \ruleI{M \tolong M'}{MN \tolong M'N}.

  Derivation $\mathcal{D}$ has the following form
\begin{center}
  \AxiomC{$\Gamma \vdash M:(\Pi x:A.A')$} \AxiomC{$\Gamma \vdash N:A$}
  \RightLabel{\emph{App}} \BinaryInfC{$\Gamma \vdash MN:A'$}
  \DisplayProof
\end{center}
By the induction hypothesis we have $\Gamma \vdash M':(\Pi x:A.A')$. Hence,
\begin{center}
  \AxiomC{$\Gamma \vdash M':(\Pi x:A.A')$} \AxiomC{$\Gamma \vdash N:A$}
  \RightLabel{\emph{App}} \BinaryInfC{$\Gamma \vdash M'N:A'$}
  \DisplayProof
\end{center}

\item[case \emph{R-Unfold}:] \ruleI{M \tolong M'}{\unfold{M} \tolong \unfold{M'}}.

  Derivation $\mathcal{D}$ has the following form
\begin{center}
  \AxiomC{$\Gamma \vdash M:\mu x.A$} \RightLabel{\emph{Unfold}}
  \UnaryInfC{$\Gamma \vdash (\unfold[\mu x.A]{M}):A[x:=\mu x.A]$}
  \DisplayProof
\end{center}
By the induction hypothesis we have
$\Gamma \vdash M':\mu x.A$. Hence,
\begin{center}
  \AxiomC{$\Gamma \vdash M':\mu x.A$} \RightLabel{\emph{Unfold}}
  \UnaryInfC{$\Gamma \vdash (\unfold[\mu x.A]{M'}):A[x:=\mu x.A]$}
  \DisplayProof
\end{center}

\item[case \emph{R-Unfold-Fold}:]
  \ruleI{}{\unfold{(\fold{\miu{x}{A}}{M})} \tolong M}.

  Derivation $\mathcal{D}$ has the following form
\begin{center}
  \AxiomC{$\Gamma \vdash M:(A[x:=\mu x.A])$} \RightLabel{\emph{Fold}}
  \UnaryInfC{$\Gamma \vdash (\fold{\miu{x}{A}}{M}):\mu x.A$}
  \RightLabel{\emph{Unfold}}
  \UnaryInfC{$\Gamma \vdash \unfold[N]{(\fold{\miu{x}{A}}{M})}:(A[x:=\mu
    x.A])$} \DisplayProof
\end{center}

\item[case \emph{R-Mu}:] \ruleI{}{\miu{x}{M} \tolong M[x:=\miu{x}{M}]}.

  Derivation $\mathcal{D}$ has the following form
\begin{center}
  \ruleLabel{\emph{Mu}}\ruleI{\ctxw{x:s}{M:s}}{\ctx{(\miu{x}{M}):s}}
\end{center}
Hence, by Lemma \ref{lem:subst} we have \ruleII{\ctxw{x:s}{M:s}}{\ctx{\miu{x}{M}:s}}{\ctx{(M[x:=\miu{x}{M}]):s}}.

\item[case \emph{R-Beta}:] \ruleI{}{\bet{M} \tolong M}.

  Derivation $\mathcal{D}$ has the following form
\begin{center}
  \ruleLabel{\it Beta}\ruleIII{\ctx{M:A}}{\ctx{B:s}}{B \tolong A}{\ctx{(\bet{M}):B}}
\end{center}
By the induction hypothesis we have
$\ctx{M':A}$ and $B \tolong A$. Hence,
\begin{center}
  \ruleLabel{\it Beta}\ruleIII{\ctx{M':A}}{\ctx{B:s}}{B \tolong A}{\ctx{(\bet{M'}):B}}
\end{center}

\end{description}
\end{proof}

\begin{thm}[Progress]
  If $\cdot \vdash A:B$ then either $A$ is a value $v$ or
  there exists $A'$ such that $A \tolong A'$.
\end{thm}

\begin{proof}
We can give the proof by induction on the derivation of
$\cdot \vdash A:B$ by typing rules in Fig.\ref{fig:mutype}:

\begin{description}
\item[case \emph{Var}:] \ruleI{\dotctx{A:s}}{\dotctxw{x:A}{x:A}}.

  This case cannot be reached. Proof is by contradiction. If we have $\dotctx{x:A}$ then $x$ is assigned with type $A$ from a context "$\cdot$" without $A$, which is not possible.

\item[case \emph{Weak}:] \ruleII{\dotctx{b:B}}{\dotctx{A:s}}{\dotctxw{x:A}{b:B}}.

  The result is trivial by induction hypothesis.

\item[case \emph{App}:] \AxiomC{$\cdot \vdash M:(\Pi x:A.B)$}
  \AxiomC{$\cdot \vdash N:A$} \BinaryInfC{$\cdot \vdash MN:B$}
  \DisplayProof.

  By induction hypothesis on $\cdot \vdash M:(\Pi x:A.B)$, there are
  two possible cases.
  \begin{enumerate}
  \item $M=v$ is a value. Hence $v=\lambda x:A.M'$ where
    $\cdot \vdash M':B$. Then $MN=vN=(\lambda x:A.M')N=M'[x:=N]$. By
    the substitution lemma, $\cdot \vdash (M'[x:=N]):B$ which is just
    $\cdot \vdash MN:B$.
  \item $M \tolong M'$. The result is obvious by
    the operational semantic
    \AxiomC{$M \tolong M'$} \RightLabel{\emph{R-AppL}} \UnaryInfC{$MN \tolong M'N$}
    \DisplayProof.
  \end{enumerate}
\item[case \emph{Lam}:] \AxiomC{$\dots$}
  \UnaryInfC{$\cdot \vdash (\lambda x:A.M):(\Pi x:A.B)$} \DisplayProof. 

  The result is trivial if let $v=\lambda x:A.M$.
\item[case \emph{Pi}:] \ruleII{\dotctx{A:s}}{\dotctxw{x:A}{B:t}}{\dotctx{(\pai{x}{A}{B}):t}}. 

  The result is trivial if let $v=\pai{x}{A}{B}$.

\item[case \emph{Mu}:] \ruleI{\dots}{\dotctx{(\miu{x}{A}):s}}.

  The result is trivial since we always have such reduction $\miu x A \tolong A[x:=\miu x A]$.

\item[case \emph{Fold}:] \AxiomC{$\dots$}
  \UnaryInfC{$\cdot \vdash (\fold{\mu x.A}{M}):\mu x.A$}
  \DisplayProof.

  The result is trivial if let $v=\fold{\mu x.A}{M}$.
\item[case \emph{Unfold}:] \ruleII{\dotctx{a:\miu{x}{A}}}{\dotctx{A[x:=\miu{x}{A}]:s}}%
                      {\dotctx{(\unfold{a}):A[x:=\miu{x}{A}]}}.

  By induction hypothesis on $\dotctx{a:\miu{x}{A}}$, there are
  two possible cases.
  \begin{enumerate}
  \item $a=v$ is a value. Hence $a=\fold{\miu{x}{A}}{b}$ where $\dotctx{b:(A[x:=\miu{x}{A}])}$. Then by the \emph{R-Unfold-Fold} rule, $\unfold{a}=\unfold{(\fold{\miu{x}{A}}{b})} = b$. Thus $\dotctx{(\unfold{a}):A[x:=\miu{x}{A}]}$.
  \item $a \tolong a'$. The result is obvious by the
    reduction rule \ruleLabel{\emph{R-Unfold}}
    \ruleI{M \tolong M'}{\unfold{M} \tolong \unfold{M'}}.
  \end{enumerate}
\item[case \emph{Beta}:] 
    \ruleI{\cdots}{\dotctx{(\bet{a}):B}}.

  The result is trivial since we always have such reduction $\bet{a} \tolong a$.
\end{description}

\end{proof}
