\newcommand{\Nat}{\mathsf{Nat}} \newcommand{\zero}{\mathsf{zero}}
\newcommand{\suc}{\mathsf{suc}} \newcommand{\plus}{\mathsf{plus}}
\newcommand{\List}{\mathsf{List}} \newcommand{\nil}{\mathsf{nil}}
\newcommand{\cons}{\mathsf{cons}}
\newcommand{\length}{\mathsf{length}}
\newcommand{\PlFunc}{\Nat\rightarrow\Nat\rightarrow\Nat}
\newcommand{\case}{\mathsf{case}} \newcommand{\of}{\mathsf{of}}

\subsubsection{Examples of Simple Datatypes}

\begin{itemize}

\item We can encode the type of natural numbers as follow:
  \[
  \Nat = \mu X.\ \Pi(a : \star).\ a \rightarrow (X \rightarrow a)
  \rightarrow a
\]
then we can define $\zero$ and $\suc$ as follows:
\begin{align*}
  \zero &: \Nat \\
  \zero &= \fold{\Nat}{(\lambda (a : \star) (z : a) (f : \Nat \rightarrow a).\,z)}\\
  \suc &: \Nat \rightarrow \Nat\\
  \suc &= \lambda (n : \Nat).\,\fold{\Nat}{(\lambda (a : \star) (z : a) (f : \Nat \rightarrow a).\,f\,n)}
\end{align*}
Using $\mathsf{fix}$, we can define a recursive function $\plus$ as
follow:
\begin{align*}
  \plus &:\PlFunc\\
  \plus &=\mathsf{fix}\,(\PlFunc)\,(\lambda(p : \PlFunc)(n : \Nat)(m : \Nat).\\
        &\qquad (\unfold[\Nat]{n})\,\Nat\,m\,(\lambda (n^{\prime} : \Nat).\,\suc\,(p\,n^{\prime}\,m)))
\end{align*}
\item We can encode the type of lists of a certain type:
  \[
  \List = \mu X.\,\Pi(a : \star).\,a \rightarrow (\Pi (b : \star).\,b
  \rightarrow X \rightarrow a) \rightarrow a
\]
then we can define $\nil$ and $\cons$ as follows:
\begin{align*}
  \nil &: \List\\
  \nil &= \fold{\List}{(\lambda (a : \star) (z : a) (f : \Pi (b : \star).\,b \rightarrow \List \rightarrow a).\ z)}\\
  \cons &: \Pi (b : \star).\,b \rightarrow \List \rightarrow \List\\
  \cons &= \lambda(b : \star)(x : b)(xs : \List).\\
       &\qquad \fold{\List}{(\lambda(a : \star)(z : a)(f : \Pi (b : \star).\,b\rightarrow \List \rightarrow a).\,f\,b\,x\,xs)}
\end{align*}
Using $\mathsf{fix}$, we can define a recursive function $\length$ as
follow:
\begin{align*}
  \length &: \List \rightarrow \Nat\\
  \length &= \mathsf{fix}\,(\List \rightarrow \Nat)\,(\lambda(l : \List
            \rightarrow \Nat)(xs : \List).\\
          &\qquad (\unfold[\List]{xs})\,\Nat\,\zero\,(\lambda(b : \star)(y : b)(ys : \List).\,\suc\,(l\,ys)))
\end{align*}
% \item The rule $(\mathsf{Mu})$ doesn't allow me to express something
%   like $(\mu x.\,A) : \Nat \rightarrow \star$
\end{itemize}

\subsubsection{Elaboration of Datatypes}

We can extend $\lambda C_{\mu}$ with \emph{first-order}
datatypes~\cite{geuvers2014church}:
\[
  \mathbf{data} \quad D = K_{1}\,T_{1}^{1}(D) \dots
  T_{\mathsf{ar}(1)}^{1}(D) \mid \cdots \mid K_{n}\,T_{1}^{n}(D) \dots
  T_{\mathsf{ar}(n)}^{n}(D)
\]
where each of the $T_{i}^{j}(X)$ is either $X$ or a type expression
that does not contain $X$. This defines an algebraic datatype $D$ with
$n$ constructors. Each constructor $K_{i}$ has arity $\mathsf{ar}(i)$,
which can be zero.

We adopt the following convention: we write $T^{1}(X)$ for
$T_{1}^{1}(X) \dots T_{\mathsf{ar}(1)}^{1}(X)$ etc. So each data
constructor has the following types:
\begingroup
\renewcommand*{\arraystretch}{1.0}
\begin{table}[h]
  \centering
  \begin{tabular}{lll}
    $K_{1}$ &:& $T^{1}(D) \rightarrow D$ \\
            && \dots \\
    $K_{n}$ &:& $T^{n}(D) \rightarrow D$
  \end{tabular}
\end{table}
\endgroup

Next we show how datatypes can be translated to our system with
recursive types.

Given a datatype $D$, with constructors $K_{1},\dots,K_{n}$, the
encoding of $D$ in our system is given by:
\[
  D ::= \mu \beta.\,\Pi(\alpha : \star).\,(T^{1}(\beta) \rightarrow
  \alpha) \rightarrow \dots \rightarrow (T^{n}(\beta) \rightarrow
  \alpha) \rightarrow \alpha
\]

The constructors are encoded by:
\begin{align*}
  K_{i} &::= \lambda(x_{1}:T_{1}^{i}(D))\dots(x_{\mathsf{ar}(i)}:T_{\mathsf{ar}(i)}^{i}(D)).\\
        &\quad \fold{D}{(\lambda(\alpha:\star)(c_{1}:T^{1}(D) \rightarrow \alpha)\dots(c_{n}:T^{n}(D) \rightarrow \alpha).\,c_{i}\,x_{1} \dots x_{\mathsf{ar}(i)})}
\end{align*}

\subsubsection{Elaboration of Case Analysis}

The set of expressions $A$ of $\lambda C_{\mu}$ extended with case
analysis is defined by
\[
  \renewcommand*{\arraystretch}{1.0}%
  \begin{array}{lcl}
    A &::= & x \mid \star \mid \square \\
      & \mid & AA \mid \lambda x:A.A \mid \Pi x:A.A \\
      & \mid & \mu x.A \mid \fold{A}{A} \mid \unfold[A]{A} \\
      & \mid & \bet{A} \\
      & \mid & \case \,A\,\of\,\{x\,x_{1}\,x_{2} \dots \Rightarrow A; \dots\}
  \end{array}
\]

Suppose we have
\begin{align*}
  \case\,&x\,\of\,\{\\
         &K_{1}\,x_{1}\dots x_{\mathsf{ar}(1)} \Rightarrow r_{1}\\
         &\dots\\
         &K_{n}\,x_{1}\dots x_{\mathsf{ar}(n)} \Rightarrow r_{n}\\
         &\}
\end{align*}
where $x : D$ and $r_{1},\dots,r_{n} : T$ ($T$ is some known type).

This can be translated to our system as follows:
\begin{align*}
  (\unfold[D]{x})\,T\,&(\lambda(x_{1}:T_{1}^{1}(D))\dots(x_{\mathsf{ar}(1)}:T_{\mathsf{ar}(1)}^{1}(D)).\,r_{1})\\
                      &\dots\\
                      &(\lambda(x_{1}:T_{1}^{n}(D))\dots(x_{\mathsf{ar}(n)}:T_{\mathsf{ar}(n)}^{n}(D)).\,r_{n})
\end{align*}
