We extend Calculus of Constructions ($\lambda C$) with recursive
types, namely $\lambda C_{\mu}$. Differences with $\lambda C$ are
highlighted. Figure~\ref{fig:musyn} shows the extended syntax.

\begin{figure}[ht]
  \small
  \begin{syntax}
    \textbf{Terms} \\
    E,T &::= & x & \ptext{variable} \\
    & \mid & \star & \ptext{star} \\
    & \mid & \square & \ptext{square} \\
    & \mid & E \ E & \ptext{application} \\
    & \mid & \lam{x}{T}{E} & \ptext{abstraction} \\
    & \mid & \pai{x}{T}{T} & \ptext{product} \\
    & \mid & \hlmath{\miu{x}{T}{E}} & \ptext{recursive type} \\
    & \mid & \hlmath{\fold{T}{E}} & \ptext{generalized roll} \\
    & \mid & \hlmath{\unfold{E}} & \ptext{generalized unroll} \\
    \textbf{Environments} \\
    \Gamma &::= &\varnothing & \ptext{empty} \\
    & \mid & \Gamma,x:T & \ptext{variable binding} \\
    \textbf{Syntactic sugar} \\
    \letb{x}{T}{E_{1}}{E_{2}} &::= & (\lam{x}{T}{E_{2}}) \ E_{1}
  \end{syntax}
  \caption{Syntax of $\lambda C_\mu$}\label{fig:musyn}
\end{figure}

%%% Local Variables:
%%% mode: latex
%%% TeX-master: "pts"
%%% End:
