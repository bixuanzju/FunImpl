%%% !!! WARNING: AUTO GENERATED. DO NOT MODIFY !!! %%%
\section{Specification of core language}

\subsection{Syntax}
\gram{\otte\ottinterrule
        \otts\ottinterrule
        \ottG\ottinterrule
        \ottv}

\subsection{Operational semantics and expression typing}
\ottdefnstep{}
\ottusedrule{\ottdruleSXXMu{}}
\ottdefnexpr{}
\ottusedrule{\ottdruleTXXMu{}}

\section{Proofs of core language}
\subsection{Properties}
\begin{lem}[Generation lemma]\label{lem:gen}
$\quad$
\begin{enumerate}
	\item If $\Gamma  \vdash  \ottmv{x}  \ottsym{:}  \ottmv{T}$, then there exist an expression $\tau$ and a sort $s$ such that $ \tau  =_{\alpha}  \ottmv{T} $, $\Gamma  \vdash  \tau  \ottsym{:}  \ottnt{s}$ and $\ottmv{x}  \ottsym{:}  \tau \, \in \, \Gamma$.
	\item If $\Gamma  \vdash  \ottnt{e_{{\mathrm{1}}}} \, \ottnt{e_{{\mathrm{2}}}}  \ottsym{:}  \ottmv{T}$, then there exist expressions $\tau_{{\mathrm{1}}}$ and $\tau_{{\mathrm{2}}}$ such that $\Gamma  \vdash  \ottnt{e_{{\mathrm{1}}}}  \ottsym{:}  \ottsym{(}  \Pi \, \ottmv{x}  \ottsym{:}  \tau_{{\mathrm{1}}}  \ottsym{.}  \tau_{{\mathrm{2}}}  \ottsym{)}$, $\Gamma  \vdash  \ottnt{e_{{\mathrm{2}}}}  \ottsym{:}  \tau_{{\mathrm{2}}}$ and $ \ottmv{T}  =_{\alpha}  \tau_{{\mathrm{1}}}  \ottsym{[}  \ottmv{x}  \mapsto  \ottnt{e_{{\mathrm{2}}}}  \ottsym{]} $.
	\item If $\Gamma  \vdash  \ottsym{(}  \lambda  \ottmv{x}  \ottsym{:}  \tau_{{\mathrm{1}}}  \ottsym{.}  \ottnt{e}  \ottsym{)}  \ottsym{:}  \ottmv{T}$, then there exist a sort $s$ and an expression $\tau_{{\mathrm{2}}}$ such that $ \ottmv{T}  =_{\alpha}  \Pi \, \ottmv{x}  \ottsym{:}  \tau_{{\mathrm{1}}}  \ottsym{.}  \tau_{{\mathrm{2}}} $ where $\Gamma  \vdash  \ottsym{(}  \Pi \, \ottmv{x}  \ottsym{:}  \tau_{{\mathrm{1}}}  \ottsym{.}  \tau_{{\mathrm{2}}}  \ottsym{)}  \ottsym{:}  \ottnt{s}$ and $\Gamma  \ottsym{,}  \ottmv{x}  \ottsym{:}  \tau_{{\mathrm{1}}}  \vdash  \ottnt{e}  \ottsym{:}  \tau_{{\mathrm{2}}}$.
	\item If $\Gamma  \vdash  \ottsym{(}  \Pi \, \ottmv{x}  \ottsym{:}  \tau_{{\mathrm{1}}}  \ottsym{.}  \tau_{{\mathrm{2}}}  \ottsym{)}  \ottsym{:}  \ottmv{T}$, then there exist sorts $\ottnt{s_{{\mathrm{1}}}}$ and $\ottnt{s_{{\mathrm{2}}}}$ such that $\ottmv{T}  \equiv  \ottnt{s_{{\mathrm{2}}}}$, $\Gamma  \vdash  \tau_{{\mathrm{1}}}  \ottsym{:}  \ottnt{s_{{\mathrm{1}}}}$ and $\Gamma  \ottsym{,}  \ottmv{x}  \ottsym{:}  \tau_{{\mathrm{1}}}  \vdash  \tau_{{\mathrm{2}}}  \ottsym{:}  \ottnt{s_{{\mathrm{2}}}}$.
	\item If $\Gamma  \vdash  \ottsym{(}  \mu \, \ottmv{x}  \ottsym{:}  \tau  \ottsym{.}  \ottnt{e}  \ottsym{)}  \ottsym{:}  \ottmv{T}$, then there exists a sort $\ottnt{s}$ such that $\Gamma  \vdash  \tau  \ottsym{:}  \ottnt{s}$, $ \ottmv{T}  =_{\alpha}  \tau $ and $\Gamma  \ottsym{,}  \ottmv{x}  \ottsym{:}  \tau  \vdash  \ottnt{e}  \ottsym{:}  \tau$.
	\item IF $\Gamma  \vdash  \ottsym{(}  \mathsf{cast}^{\uparrow} \, \ottsym{[}  \tau_{{\mathrm{1}}}  \ottsym{]}  \ottnt{e}  \ottsym{)}  \ottsym{:}  \ottmv{T}$, then there exist an expression $\tau_{{\mathrm{2}}}$ and a sort $s$ such that $\Gamma  \vdash  \ottnt{e}  \ottsym{:}  \tau_{{\mathrm{2}}}$, $\Gamma  \vdash  \tau_{{\mathrm{1}}}  \ottsym{:}  \ottnt{s}$, $\tau_{{\mathrm{1}}}  \longrightarrow  \tau_{{\mathrm{2}}}$ and $ \ottmv{T}  =_{\alpha}  \tau_{{\mathrm{1}}} $.
	\item If $\Gamma  \vdash  \ottsym{(}  \mathsf{cast}_{\downarrow} \, \ottnt{e}  \ottsym{)}  \ottsym{:}  \ottmv{T}$, then there exist expressions $\tau_{{\mathrm{1}}},\tau_{{\mathrm{2}}}$ and a sort $s$ such that $\Gamma  \vdash  \ottnt{e}  \ottsym{:}  \tau_{{\mathrm{1}}}$, $\Gamma  \vdash  \tau_{{\mathrm{2}}}  \ottsym{:}  \ottnt{s}$, $\tau_{{\mathrm{1}}}  \longrightarrow  \tau_{{\mathrm{2}}}$ and $ \ottmv{T}  =_{\alpha}  \tau_{{\mathrm{2}}} $.
\end{enumerate}
\end{lem}

\begin{lem}[Substitution lemma]
	If $\Gamma_{{\mathrm{1}}}, x:\tau_{{\mathrm{1}}}, \Gamma_{{\mathrm{2}}}  \vdash  \ottnt{e_{{\mathrm{1}}}}  \ottsym{:}  \tau_{{\mathrm{2}}}$ and $\Gamma_{{\mathrm{1}}}  \vdash  \ottnt{e_{{\mathrm{2}}}}  \ottsym{:}  \tau_{{\mathrm{1}}}$, then $\Gamma_{{\mathrm{1}}}, \Gamma_{{\mathrm{2}}} [x  \mapsto  \ottnt{e_{{\mathrm{2}}}} ]  \vdash  \ottnt{e_{{\mathrm{1}}}}  \ottsym{[}  \ottmv{x}  \mapsto  \ottnt{e_{{\mathrm{2}}}}  \ottsym{]} : \tau_{{\mathrm{2}}}  \ottsym{[}  \ottmv{x}  \mapsto  \ottnt{e_{{\mathrm{2}}}}  \ottsym{]}$.
\end{lem}

\begin{dfn}[Well-formed context]
	A \textbf{well-formed} context $\Gamma$ is defined by the following rules:
	
	\textnormal{\ottdefnctx{}}
\end{dfn}

\begin{lem}[Consistency of well-formed context]\label{lem:wfc}
	Given a well-formed initial context $\Gamma$, it remains well-formed through type checking.
\end{lem}

\begin{proof}
	Suppose $\Gamma$ is the initial context which is well-formed. To safely extend $\Gamma$ with a variable $x:\tau$, one should have $\Gamma  \vdash  \tau  \ottsym{:}  \ottnt{s}$ due to rule \ruleref{Env\_Var}. Note that when applying typing rules of $\Gamma  \vdash  \ottnt{e}  \ottsym{:}  \tau$, rule \ruleref{T\_Pi}, \ruleref{T\_Mu} and \ruleref{T\_Lam} will extend the context. We show that these rules cover the condition $\Gamma  \vdash  \tau  \ottsym{:}  \ottnt{s}$ with respect to $x:\tau$ as follows:
	\begin{description}
		\item[Case \ruleref{T\_Pi}:] \ottusedrule{\ottdruleTXXPi{}} For $x:\tau_{{\mathrm{1}}}$, $\Gamma  \vdash  \tau_{{\mathrm{1}}}  \ottsym{:}  \ottnt{s}$ is directly the premise of the rule.
		\item[Case \ruleref{T\_Mu}:] \ottusedrule{\ottdruleTXXMu{}} For $x:\tau$, $\Gamma  \vdash  \tau  \ottsym{:}  \ottnt{s}$ is directly the premise of the rule.
		\item[Case \ruleref{T\_Lam}:] \ottusedrule{\ottdruleTXXLam{}} For $x:\tau_{{\mathrm{1}}}$, note that the premise $\Gamma  \vdash  \ottsym{(}  \Pi \, \ottmv{x}  \ottsym{:}  \tau_{{\mathrm{1}}}  \ottsym{.}  \tau_{{\mathrm{2}}}  \ottsym{)}  \ottsym{:}  \ottnt{s}$ can be derived from rule \ruleref{T\_Pi}, which has the pre-condition $\Gamma  \vdash  \tau_{{\mathrm{1}}}  \ottsym{:}  \ottnt{s}$.
	\end{description}
\end{proof}

\begin{lem}[Valid context optimization]\label{lem:wfcopt}
	With a well-formed initial context $\Gamma$, the \ruleref{T\_Var} and \ruleref{T\_Weak} can be replaced by the following rule: \ottusedrule{\ottdruleTSXXVar{}}
\end{lem}

\begin{proof}
	By Lemma \ref{lem:wfc}, the context $\Gamma$ remains well-formed if it is initially well-formed. Thus, it is not necessary to use \ruleref{T\_Var} and \ruleref{T\_Weak} to check the well-formedness of $\Gamma$. By Lemma \ref{lem:gen}, in order to check the type of a variable $x$, it is necessary and sufficient to check if $\ottmv{x}  \ottsym{:}  \tau \, \in \, \Gamma$, which is simply rule \ruleref{TS\_Var}.
\end{proof}

\subsection{Decidability of type checking}
\begin{lem}[Uniqueness of one-step reduction]\label{lem:unired}
	The relation $ \longrightarrow $, i.e. one-step reduction, is \textbf{unique} in the sense that given $e$ there is at most one $e'$ such that $\ottnt{e}  \longrightarrow  \ottnt{e'}$.
\end{lem}

\begin{proof}
	By induction on the structure of $e$:
	\begin{description}
		\item[Case $e=v$:] $e$ has one of the following forms:
		\begin{inparaenum}[(1)]
			\item $\lambda  \ottmv{x}  \ottsym{:}  \tau  \ottsym{.}  \ottnt{e}$,
			\item $\Pi \, \ottmv{x}  \ottsym{:}  \tau_{{\mathrm{1}}}  \ottsym{.}  \tau_{{\mathrm{2}}}$,
			\item $\mathsf{cast}^{\uparrow} \, \ottsym{[}  \tau  \ottsym{]}  \ottnt{e}$,
		\end{inparaenum}
		which cannot match any rules of $ \longrightarrow $. Thus there is no $e'$ such that $\ottnt{e}  \longrightarrow  \ottnt{e'}$.
		\item[Case $e=\ottsym{(}  \lambda  \ottmv{x}  \ottsym{:}  \tau  \ottsym{.}  \ottnt{e_{{\mathrm{1}}}}  \ottsym{)} \, \ottnt{e_{{\mathrm{2}}}}$:] There is a unique $e'=\ottnt{e_{{\mathrm{1}}}}  \ottsym{[}  \ottmv{x}  \mapsto  \ottnt{e_{{\mathrm{2}}}}  \ottsym{]}$ by rule \ruleref{S\_Beta}.
		\item[Case $e=\mathsf{cast}_{\downarrow} \, \ottsym{(}  \mathsf{cast}^{\uparrow} \, \ottsym{[}  \tau  \ottsym{]}  \ottnt{e}  \ottsym{)}$:] There is a unique $e'=e$ by rule \ruleref{S\_CastDownUp}.
		\item[Case $e=\mu \, \ottmv{x}  \ottsym{:}  \tau  \ottsym{.}  \ottnt{e}$:] There is a unique $e'=\ottnt{e}  \ottsym{[}  \ottmv{x}  \mapsto  \mu \, \ottmv{x}  \ottsym{:}  \tau  \ottsym{.}  \ottnt{e}  \ottsym{]}$ by rule \ruleref{S\_Mu}.
		\item[Case $e=\ottnt{e_{{\mathrm{1}}}} \, \ottnt{e_{{\mathrm{2}}}}$ with $\ottnt{e_{{\mathrm{1}}}}  \longrightarrow  \ottnt{e'_{{\mathrm{1}}}}$:] $\ottnt{e_{{\mathrm{1}}}}$ cannot be a $\lambda$-term $\lambda  \ottmv{x}  \ottsym{:}  \tau  \ottsym{.}  \ottnt{e}$ which is a value that contradicts $\ottnt{e_{{\mathrm{1}}}}$ can be reduced to $\ottnt{e'_{{\mathrm{1}}}}$. By the induction hypothesis, $\ottnt{e'_{{\mathrm{1}}}}$ is unique reduction of $\ottnt{e_{{\mathrm{1}}}}$. Thus by rule \ruleref{S\_App}, $e'=\ottnt{e'_{{\mathrm{1}}}} \, \ottnt{e_{{\mathrm{2}}}}$ is the unique reduction for $e$.
		\item[Case $e=\mathsf{cast}_{\downarrow} \, \ottnt{e_{{\mathrm{1}}}}$ with $\ottnt{e_{{\mathrm{1}}}}  \longrightarrow  \ottnt{e'_{{\mathrm{1}}}}$:] $\ottnt{e_{{\mathrm{1}}}}$ cannot have the form $\mathsf{cast}^{\uparrow} \, \ottsym{[}  \tau  \ottsym{]}  \ottnt{e}$ which is a value that contradicts $\ottnt{e_{{\mathrm{1}}}}$ can be reduced to $\ottnt{e'_{{\mathrm{1}}}}$. By the induction hypothesis, $\ottnt{e'_{{\mathrm{1}}}}$ is unique reduction of $\ottnt{e_{{\mathrm{1}}}}$. Thus by rule \ruleref{S\_CastDown}, $e'=\mathsf{cast}_{\downarrow} \, \ottnt{e'_{{\mathrm{1}}}}$ is the unique reduction for $e$.
	\end{description}
\end{proof}

\begin{lem}[Decidability of type checking]
	There is a decidable algorithm which given $\Gamma, \ottnt{e}$ computes the unique $\tau$ such that $\Gamma  \vdash  \ottnt{e}  \ottsym{:}  \tau$ or reports there is no such $\tau$.
\end{lem}

\begin{proof}
	By induction on the derivation of $e$:
	\begin{description}
		\item[Case $e=x$:] By Lemma \ref{lem:wfcopt}, we only need to consider context $\Gamma$ that is well-formed. By rule \ruleref{TS\_Var}, if $\ottmv{x}  \ottsym{:}  \tau \, \in \, \Gamma$, $t$ is the unique type of $x$.
		\item[Case $e=\ottnt{e_{{\mathrm{1}}}} \, \ottnt{e_{{\mathrm{2}}}}$, or $\lambda  \ottmv{x}  \ottsym{:}  \tau_{{\mathrm{1}}}  \ottsym{.}  \ottnt{e_{{\mathrm{1}}}}$, or $\Pi \, \ottmv{x}  \ottsym{:}  \tau_{{\mathrm{1}}}  \ottsym{.}  \tau_{{\mathrm{2}}}$, or $\mu \, \ottmv{x}  \ottsym{:}  \tau  \ottsym{.}  \ottnt{e_{{\mathrm{1}}}}$:] Trivial according to Lemma \ref{lem:gen} by applying rule \ruleref{T\_App}, \ruleref{T\_Lam}, \ruleref{T\_Pi}, or \ruleref{T\_Mu} respectively.
		\item[Case $e=\mathsf{cast}^{\uparrow} \, \ottsym{[}  \tau_{{\mathrm{1}}}  \ottsym{]}  \ottnt{e_{{\mathrm{1}}}}$:] From rule \ruleref{T\_CastUp}, by induction hypothesis, we can derive the type of $\ottnt{e_{{\mathrm{1}}}}$ as $\tau_{{\mathrm{2}}}$, and check whether $\tau_{{\mathrm{1}}}$ is legal, i.e. its sorts is either $ \star $ or $ \Box $. If $\tau_{{\mathrm{1}}}$ is legal, by Lemma \ref{lem:unired}, there is at most one $\tau'_{{\mathrm{1}}}$ such that $\tau_{{\mathrm{1}}}  \longrightarrow  \tau'_{{\mathrm{1}}}$. If such $\tau'_{{\mathrm{1}}}$ does not exist, then we report the type checking is failed. Otherwise, we examine if $\tau'_{{\mathrm{1}}}$ is syntactically equal to $\tau_{{\mathrm{2}}}$, i.e. to check the $\alpha$-equality $ \tau'_{{\mathrm{1}}}  =_{\alpha}  \tau_{{\mathrm{2}}} $. If the equality holds, we obtain the unique type of $\ottnt{e}$ which is $\tau_{{\mathrm{1}}}$. Otherwise, we report $\ottnt{e}$ fails to type check.
		\item[Case $e=\mathsf{cast}_{\downarrow} \, \ottnt{e_{{\mathrm{1}}}}$:] From rule \ruleref{T\_CastDown}, by induction hypothesis, we can derive the type of $\ottnt{e_{{\mathrm{1}}}}$ as $\tau_{{\mathrm{1}}}$. By Lemma \ref{lem:unired}, there is at most one $\tau_{{\mathrm{2}}}$ such that $\tau_{{\mathrm{1}}}  \longrightarrow  \tau_{{\mathrm{2}}}$. If such $\tau_{{\mathrm{2}}}$ exists and its sorts is either $ \star $ or $ \Box $, we have found the unique type of $\ottnt{e}$ is $\tau_{{\mathrm{2}}}$. Otherwise, we report $\ottnt{e}$ fails to type check.
	\end{description}
\end{proof}

\subsection{Soundness}
\begin{lem}[Subject reduction]
If $\Gamma  \vdash  \ottnt{e}  \ottsym{:}  \tau$ and $e  \twoheadrightarrow  e'$ then $\Gamma  \vdash  \ottnt{e'}  \ottsym{:}  \tau$.
\end{lem}

\begin{lem}[Progress]
If $\varnothing  \vdash  \ottnt{e}  \ottsym{:}  \tau$ then either $e$ is a value $v$ or there exists $e'$ such that $e  \twoheadrightarrow  e'$.
\end{lem}

