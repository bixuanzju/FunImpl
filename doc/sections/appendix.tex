%%% !!! WARNING: AUTO GENERATED. DO NOT MODIFY !!! %%%
\section{Full Specification of Core Language}

\subsection{Syntax}
\gram{\otte\ottinterrule
        \ottG\ottinterrule
        \ottv}
\[\ottcoresugar\] % defined in otthelper.mng.tex

\subsection{Operational Semantics}
\ottdefnstep{}
\ottusedrule{\ottdruleSXXMu{}}

\subsection{Typing}
\ottdefnctx{}\ottinterrule
\ottdefnexpr{}
\ottusedrule{\ottdruleTXXMu{}}

\section{Proofs about Core Language}
\subsection{Properties}
\begin{lem}[Free Variable]\label{lem:appendix:free}
    If $\Gamma  \vdash  \ottnt{e}  \ottsym{:}  \tau$, then $\FV(e) \subseteq \dom(\Gamma)$ and $\FV(\tau)
\subseteq \dom(\Gamma)$.
\end{lem}

\begin{proof}
    By induction on the derivation of $\Gamma  \vdash  \ottnt{e}  \ottsym{:}  \tau$. We only treat cases
\ruleref{T\_Mu}, \ruleref{T\_CastUp} and \ruleref{T\_CastDown} (since proofs of
other cases are the same as \cc \cite{handbook}):
    \begin{description}
        \item[Case \ruleref{T\_Mu}:] From premises of $\Gamma  \vdash  \ottsym{(}  \mu \, \ottmv{x}  \ottsym{:}  \tau  \ottsym{.}  \ottnt{e_{{\mathrm{1}}}}  \ottsym{)}  \ottsym{:}  \tau$, by induction hypothesis, we have $\FV(e_1) \subseteq \dom(\Gamma) \cup
\{\ottmv{x}\}$ and $\FV(\tau) \subseteq \dom(\Gamma)$. Thus the result follows by
$\FV(\mu \, \ottmv{x}  \ottsym{:}  \tau  \ottsym{.}  \ottnt{e_{{\mathrm{1}}}})=\FV(e_1) \setminus \{\ottmv{x}\} \subseteq \dom(\Gamma)$ and
$\FV(\tau) \subseteq \dom(\Gamma)$.
        \item[Case \ruleref{T\_CastUp}:] Since $\FV(\mathsf{cast}^{\uparrow} \, \ottsym{[}  \tau  \ottsym{]} \,  \ottnt{e_{{\mathrm{1}}}})=\FV(\ottnt{e_{{\mathrm{1}}}})$, the result follows directly by the induction hypothesis.
        \item[Case \ruleref{T\_CastDown}:] Since $\FV(\mathsf{cast}_{\downarrow} \, \ottnt{e_{{\mathrm{1}}}})=\FV(\ottnt{e_{{\mathrm{1}}}})$, the result follows directly by the induction hypothesis.
    \end{description}
\end{proof}

\begin{lem}[Thinning]\label{lem:appendix:thin}
    Let $\Gamma$ and $\Gamma'$ be legal contexts such that $\Gamma \subseteq
\Gamma'$. If $\Gamma  \vdash  \ottnt{e}  \ottsym{:}  \tau$ then $\Gamma'  \vdash  \ottnt{e}  \ottsym{:}  \tau$.
\end{lem}

\begin{proof}
    By trivial induction on the derivation of $\Gamma  \vdash  \ottnt{e}  \ottsym{:}  \tau$.
\end{proof}

\begin{lem}[Substitution]\label{lem:appendix:subst}
	If $\Gamma_{{\mathrm{1}}}  \ottsym{,}  \ottmv{x}  \ottsym{:}  \sigma  \ottsym{,}  \Gamma_{{\mathrm{2}}}  \vdash  \ottnt{e_{{\mathrm{1}}}}  \ottsym{:}  \tau$ and $\Gamma_{{\mathrm{1}}}  \vdash  \ottnt{e_{{\mathrm{2}}}}  \ottsym{:}  \sigma$, then $\Gamma_{{\mathrm{1}}}  \ottsym{,}  \Gamma_{{\mathrm{2}}}  \ottsym{[}  \ottmv{x}  \mapsto  \ottnt{e_{{\mathrm{2}}}}  \ottsym{]} \,  \vdash  \ottnt{e_{{\mathrm{1}}}}  \ottsym{[}  \ottmv{x}  \mapsto  \ottnt{e_{{\mathrm{2}}}}  \ottsym{]} \,  \ottsym{:}  \tau  \ottsym{[}  \ottmv{x}  \mapsto  \ottnt{e_{{\mathrm{2}}}}  \ottsym{]} \,$.
\end{lem}

\begin{proof}
    By induction on the derivation of $\Gamma_{{\mathrm{1}}}  \ottsym{,}  \ottmv{x}  \ottsym{:}  \sigma  \ottsym{,}  \Gamma_{{\mathrm{2}}}  \vdash  \ottnt{e_{{\mathrm{1}}}}  \ottsym{:}  \tau$. Let $\ottnt{e}  ^{*}  \equiv  \ottnt{e}  \ottsym{[}  \ottmv{x}  \mapsto  \ottnt{e_{{\mathrm{2}}}}  \ottsym{]} \,$. Then the result can be written as $\Gamma_{{\mathrm{1}}}  \ottsym{,}  \Gamma_{{\mathrm{2}}}  ^{*}  \vdash  \ottnt{e_{{\mathrm{1}}}}  ^{*}  \ottsym{:}  \tau  ^{*}$.
We only treat cases \ruleref{T\_Mu}, \ruleref{T\_CastUp} and
\ruleref{T\_CastDown}. Consider the last step of derivation of the following
cases:
    \begin{description}
        \item[Case \ruleref{T\_Mu}:] $\inferrule{\Gamma_{{\mathrm{1}}}  \ottsym{,}  \ottmv{x}  \ottsym{:}  \sigma  \ottsym{,}  \Gamma_{{\mathrm{2}}}  \vdash  \ottnt{e_{{\mathrm{1}}}}  \ottsym{:}  \tau \\
\Gamma_{{\mathrm{1}}}  \ottsym{,}  \ottmv{x}  \ottsym{:}  \sigma  \ottsym{,}  \Gamma_{{\mathrm{2}}}  \vdash  \tau  \ottsym{:}  \star}{\Gamma_{{\mathrm{1}}}  \ottsym{,}  \ottmv{x}  \ottsym{:}  \sigma  \ottsym{,}  \Gamma_{{\mathrm{2}}}  \vdash  \ottsym{(}  \mu \, \ottmv{y}  \ottsym{:}  \tau  \ottsym{.}  \ottnt{e_{{\mathrm{1}}}}  \ottsym{)}  \ottsym{:}  \tau}$ 
        
        By induction hypothesis, we have $\Gamma_{{\mathrm{1}}}  \ottsym{,}  \Gamma_{{\mathrm{2}}}  ^{*}  \vdash  \ottnt{e_{{\mathrm{1}}}}  ^{*}  \ottsym{:}  \tau  ^{*}$ and $\Gamma_{{\mathrm{1}}}  \ottsym{,}  \Gamma_{{\mathrm{2}}}  ^{*}  \vdash  \tau  ^{*}  \ottsym{:}  \star$. Then by the deviation rule, $\Gamma_{{\mathrm{1}}}  \ottsym{,}  \Gamma_{{\mathrm{2}}}  ^{*}  \vdash  \ottsym{(}  \mu \, \ottmv{y}  \ottsym{:}  \tau  ^{*}  \ottsym{.}  \ottnt{e_{{\mathrm{1}}}}  ^{*}  \ottsym{)}  \ottsym{:}  \tau  ^{*}$. Thus we have $\Gamma_{{\mathrm{1}}}  \ottsym{,}  \Gamma_{{\mathrm{2}}}  ^{*}  \vdash  \ottsym{(}  \mu \, \ottmv{y}  \ottsym{:}  \tau  \ottsym{.}  \ottnt{e_{{\mathrm{1}}}}  \ottsym{)}  ^{*}  \ottsym{:}  \tau  ^{*}$ which is just
the result.
        \item[Case \ruleref{T\_CastUp}:] $\inferrule{\Gamma_{{\mathrm{1}}}  \ottsym{,}  \ottmv{x}  \ottsym{:}  \sigma  \ottsym{,}  \Gamma_{{\mathrm{2}}}  \vdash  \ottnt{e_{{\mathrm{1}}}}  \ottsym{:}  \tau_{{\mathrm{2}}}
\\ \Gamma_{{\mathrm{1}}}  \ottsym{,}  \ottmv{x}  \ottsym{:}  \sigma  \ottsym{,}  \Gamma_{{\mathrm{2}}}  \vdash  \tau_{{\mathrm{1}}}  \ottsym{:}  \star \\ \tau_{{\mathrm{1}}}  \longrightarrow  \tau_{{\mathrm{2}}}}{\Gamma_{{\mathrm{1}}}  \ottsym{,}  \ottmv{x}  \ottsym{:}  \sigma  \ottsym{,}  \Gamma_{{\mathrm{2}}}  \vdash  \ottsym{(}  \mathsf{cast}^{\uparrow} \, \ottsym{[}  \tau_{{\mathrm{1}}}  \ottsym{]} \,  \ottnt{e_{{\mathrm{1}}}}  \ottsym{)}  \ottsym{:}  \tau_{{\mathrm{1}}}}$ 
        
        By induction hypothesis, we have $\Gamma_{{\mathrm{1}}}  \ottsym{,}  \Gamma_{{\mathrm{2}}}  ^{*}  \vdash  \ottnt{e_{{\mathrm{1}}}}  ^{*}  \ottsym{:}  \tau_{{\mathrm{2}}}  ^{*}$, $\Gamma_{{\mathrm{1}}}  \ottsym{,}  \Gamma_{{\mathrm{2}}}  ^{*}  \vdash  \tau_{{\mathrm{1}}}  ^{*}  \ottsym{:}  \star$ and $\tau_{{\mathrm{1}}}  \longrightarrow  \tau_{{\mathrm{2}}}$. By the definition of substitution, we can
obtain $\tau_{{\mathrm{1}}}  ^{*}  \longrightarrow  \tau_{{\mathrm{2}}}  ^{*}$ by $\tau_{{\mathrm{1}}}  \longrightarrow  \tau_{{\mathrm{2}}}$. Then by the deviation rule, $\Gamma_{{\mathrm{1}}}  \ottsym{,}  \Gamma_{{\mathrm{2}}}  ^{*}  \vdash  \ottsym{(}  \mathsf{cast}^{\uparrow} \, \ottsym{[}  \tau_{{\mathrm{1}}}  ^{*}  \ottsym{]} \,  \ottnt{e_{{\mathrm{1}}}}  ^{*}  \ottsym{)}  \ottsym{:}  \tau_{{\mathrm{1}}}  ^{*}$. Thus we have $\Gamma_{{\mathrm{1}}}  \ottsym{,}  \Gamma_{{\mathrm{2}}}  ^{*}  \vdash  \ottsym{(}  \mathsf{cast}^{\uparrow} \, \ottsym{[}  \tau_{{\mathrm{1}}}  \ottsym{]} \,  \ottnt{e_{{\mathrm{1}}}}  \ottsym{)}  ^{*}  \ottsym{:}  \tau_{{\mathrm{1}}}  ^{*}$ which is just the result.
        \item[Case \ruleref{T\_CastDown}:] $\inferrule{\Gamma_{{\mathrm{1}}}  \ottsym{,}  \ottmv{x}  \ottsym{:}  \sigma  \ottsym{,}  \Gamma_{{\mathrm{2}}}  \vdash  \ottnt{e_{{\mathrm{1}}}}  \ottsym{:}  \tau_{{\mathrm{1}}}
\\ \Gamma_{{\mathrm{1}}}  \ottsym{,}  \ottmv{x}  \ottsym{:}  \sigma  \ottsym{,}  \Gamma_{{\mathrm{2}}}  \vdash  \tau_{{\mathrm{2}}}  \ottsym{:}  \star \\ \tau_{{\mathrm{1}}}  \longrightarrow  \tau_{{\mathrm{2}}}}{\Gamma_{{\mathrm{1}}}  \ottsym{,}  \ottmv{x}  \ottsym{:}  \sigma  \ottsym{,}  \Gamma_{{\mathrm{2}}}  \vdash  \ottsym{(}  \mathsf{cast}_{\downarrow} \, \ottnt{e_{{\mathrm{1}}}}  \ottsym{)}  \ottsym{:}  \tau_{{\mathrm{2}}}}$ 
        
        By induction hypothesis, we have $\Gamma_{{\mathrm{1}}}  \ottsym{,}  \Gamma_{{\mathrm{2}}}  ^{*}  \vdash  \ottnt{e_{{\mathrm{1}}}}  ^{*}  \ottsym{:}  \tau_{{\mathrm{1}}}  ^{*}$, $\Gamma_{{\mathrm{1}}}  \ottsym{,}  \Gamma_{{\mathrm{2}}}  ^{*}  \vdash  \tau_{{\mathrm{2}}}  ^{*}  \ottsym{:}  \star$ and $\tau_{{\mathrm{1}}}  \longrightarrow  \tau_{{\mathrm{2}}}$ thus $\tau_{{\mathrm{1}}}  ^{*}  \longrightarrow  \tau_{{\mathrm{2}}}  ^{*}$. Then by the
deviation rule, $\Gamma_{{\mathrm{1}}}  \ottsym{,}  \Gamma_{{\mathrm{2}}}  ^{*}  \vdash  \ottsym{(}  \mathsf{cast}_{\downarrow} \, \ottnt{e_{{\mathrm{1}}}}  ^{*}  \ottsym{)}  \ottsym{:}  \tau_{{\mathrm{2}}}  ^{*}$. Thus we have $\Gamma_{{\mathrm{1}}}  \ottsym{,}  \Gamma_{{\mathrm{2}}}  ^{*}  \vdash  \ottsym{(}  \mathsf{cast}_{\downarrow} \, \ottnt{e_{{\mathrm{1}}}}  \ottsym{)}  ^{*}  \ottsym{:}  \tau_{{\mathrm{2}}}  ^{*}$ which is just the result.
    \end{description}
\end{proof}

\begin{lem}[Generation]\label{lem:appendix:gen}
$\quad$
\begin{enumerate}[(1)]
	\item If $\Gamma  \vdash  \ottmv{x}  \ottsym{:}  \sigma$, then there exist an expression $\tau$ such that $\tau  \equiv  \sigma$, $\Gamma  \vdash  \tau  \ottsym{:}  \star$ and $\ottmv{x}  \ottsym{:}  \tau \, \in \, \Gamma$.
	\item If $\Gamma  \vdash  \ottnt{e_{{\mathrm{1}}}} \, \ottnt{e_{{\mathrm{2}}}}  \ottsym{:}  \sigma$, then there exist expressions $\tau_{{\mathrm{1}}}$ and
$\tau_{{\mathrm{2}}}$ such that $\Gamma  \vdash  \ottnt{e_{{\mathrm{1}}}}  \ottsym{:}  \ottsym{(}  \Pi \, \ottmv{x}  \ottsym{:}  \tau_{{\mathrm{1}}}  \ottsym{.}  \tau_{{\mathrm{2}}}  \ottsym{)}$, $\Gamma  \vdash  \ottnt{e_{{\mathrm{2}}}}  \ottsym{:}  \tau_{{\mathrm{1}}}$ and $\sigma  \equiv  \tau_{{\mathrm{2}}}  \ottsym{[}  \ottmv{x}  \mapsto  \ottnt{e_{{\mathrm{2}}}}  \ottsym{]} \,$.
	\item If $\Gamma  \vdash  \ottsym{(}  \lambda  \ottmv{x}  \ottsym{:}  \tau_{{\mathrm{1}}}  \ottsym{.}  \ottnt{e}  \ottsym{)}  \ottsym{:}  \sigma$, then there exist an expression $\tau_{{\mathrm{2}}}$ such
that $\sigma  \equiv  \Pi \, \ottmv{x}  \ottsym{:}  \tau_{{\mathrm{1}}}  \ottsym{.}  \tau_{{\mathrm{2}}}$ where $\Gamma  \vdash  \ottsym{(}  \Pi \, \ottmv{x}  \ottsym{:}  \tau_{{\mathrm{1}}}  \ottsym{.}  \tau_{{\mathrm{2}}}  \ottsym{)}  \ottsym{:}  \star$ and $\Gamma  \ottsym{,}  \ottmv{x}  \ottsym{:}  \tau_{{\mathrm{1}}}  \vdash  \ottnt{e}  \ottsym{:}  \tau_{{\mathrm{2}}}$.
    \item If $\Gamma  \vdash  \ottsym{(}  \Pi \, \ottmv{x}  \ottsym{:}  \tau_{{\mathrm{1}}}  \ottsym{.}  \tau_{{\mathrm{2}}}  \ottsym{)}  \ottsym{:}  \sigma$, then $\sigma  \equiv  \star$, $\Gamma  \vdash  \tau_{{\mathrm{1}}}  \ottsym{:}  \star$ and
$\Gamma  \ottsym{,}  \ottmv{x}  \ottsym{:}  \tau_{{\mathrm{1}}}  \vdash  \tau_{{\mathrm{2}}}  \ottsym{:}  \star$.
	\item If $\Gamma  \vdash  \ottsym{(}  \mu \, \ottmv{x}  \ottsym{:}  \tau  \ottsym{.}  \ottnt{e}  \ottsym{)}  \ottsym{:}  \sigma$, then $\Gamma  \vdash  \tau  \ottsym{:}  \star$, $\sigma  \equiv  \tau$ and $\Gamma  \ottsym{,}  \ottmv{x}  \ottsym{:}  \tau  \vdash  \ottnt{e}  \ottsym{:}  \tau$.
	\item If $\Gamma  \vdash  \ottsym{(}  \mathsf{cast}^{\uparrow} \, \ottsym{[}  \tau_{{\mathrm{1}}}  \ottsym{]} \,  \ottnt{e}  \ottsym{)}  \ottsym{:}  \sigma$, then there exist an expression $\tau_{{\mathrm{2}}}$
such that $\Gamma  \vdash  \ottnt{e}  \ottsym{:}  \tau_{{\mathrm{2}}}$, $\Gamma  \vdash  \tau_{{\mathrm{1}}}  \ottsym{:}  \star$, $\tau_{{\mathrm{1}}}  \longrightarrow  \tau_{{\mathrm{2}}}$ and $\sigma  \equiv  \tau_{{\mathrm{1}}}$.
	\item If $\Gamma  \vdash  \ottsym{(}  \mathsf{cast}_{\downarrow} \, \ottnt{e}  \ottsym{)}  \ottsym{:}  \sigma$, then there exist expressions
$\tau_{{\mathrm{1}}},\tau_{{\mathrm{2}}}$ such that $\Gamma  \vdash  \ottnt{e}  \ottsym{:}  \tau_{{\mathrm{1}}}$, $\Gamma  \vdash  \tau_{{\mathrm{2}}}  \ottsym{:}  \star$, $\tau_{{\mathrm{1}}}  \longrightarrow  \tau_{{\mathrm{2}}}$ and
$\sigma  \equiv  \tau_{{\mathrm{2}}}$.
\end{enumerate}
\end{lem}

\begin{proof}
    Consider a derivation of $\Gamma  \vdash  \ottnt{e}  \ottsym{:}  \sigma$ for one of cases in the lemma. We
can follow the process of derivation until expression $\ottnt{e}$ is introduced the
first time. The last step of derivation can be done by
    \begin{itemize}
        \item rule \ruleref{T\_Var} for case 1;
        \item rule \ruleref{T\_App} for case 2;
        \item rule \ruleref{T\_Lam} for case 3;
        \item rule \ruleref{T\_Pi} for case 4;
        \item rule \ruleref{T\_Mu} for case 5;
        \item rule \ruleref{T\_CastUp} for case 6;
        \item rule \ruleref{T\_CastDown} for case 7.
    \end{itemize}
    In each case, assume the conclusion of the rule is $\Gamma'  \vdash  \ottnt{e}  \ottsym{:}  \tau'$ where
$\Gamma' \subseteq \Gamma$ and $\tau'  \equiv  \sigma$. Then by inspection of used
derivation rules and Lemma \ref{lem:appendix:thin}, it can be shown that the
statement of the lemma holds and is the only possible case.
\end{proof}

\begin{lem}[Correctness of Types]\label{lem:appendix:corrtyp}
    If $\Gamma  \vdash  \ottnt{e}  \ottsym{:}  \tau$ then $\tau  \equiv  \star$ or $\Gamma  \vdash  \tau  \ottsym{:}  \star$.
\end{lem}

\begin{proof}
    Trivial induction on the derivation of $\Gamma  \vdash  \ottnt{e}  \ottsym{:}  \tau$ using Lemma
\ref{lem:appendix:gen}.
\end{proof}

\subsection{Decidability of Type Checking}
\begin{lem}[Decidability of One-step Reduction]\label{lem:appendix:unired}
	The one-step reduction $ \longrightarrow $ is called decidable if 
given $\ottnt{e}$ there is a unique $\ottnt{e'}$ such that $\ottnt{e}  \longrightarrow  \ottnt{e'}$ or no such $\ottnt{e'}$.
\end{lem}

\begin{proof}
	By induction on the structure of $\ottnt{e}$:
	\begin{description}
		\item[Case $\ottnt{e}  \ottsym{=}  \ottnt{v}$:] $\ottnt{e}$ has one of the following forms:
		\begin{inparaenum}[(1)]
		    \item $\star$,
			\item $\lambda  \ottmv{x}  \ottsym{:}  \tau  \ottsym{.}  \ottnt{e}$,
			\item $\Pi \, \ottmv{x}  \ottsym{:}  \tau_{{\mathrm{1}}}  \ottsym{.}  \tau_{{\mathrm{2}}}$,
			\item $\mathsf{cast}^{\uparrow} \, \ottsym{[}  \tau  \ottsym{]} \,  \ottnt{e}$,
		\end{inparaenum}
		which cannot match any rules of $ \longrightarrow $. Thus there is no $\ottnt{e}'$ such
that $\ottnt{e}  \longrightarrow  \ottnt{e'}$.
		\item[Case $\ottnt{e}=\ottsym{(}  \lambda  \ottmv{x}  \ottsym{:}  \tau  \ottsym{.}  \ottnt{e_{{\mathrm{1}}}}  \ottsym{)} \, \ottnt{e_{{\mathrm{2}}}}$:] There is a unique $\ottnt{e}'=\ottnt{e_{{\mathrm{1}}}}  \ottsym{[}  \ottmv{x}  \mapsto  \ottnt{e_{{\mathrm{2}}}}  \ottsym{]} \,$ by rule \ruleref{S\_Beta}.
		\item[Case $\ottnt{e}=\mathsf{cast}_{\downarrow} \, \ottsym{(}  \mathsf{cast}^{\uparrow} \, \ottsym{[}  \tau  \ottsym{]} \,  \ottnt{e}  \ottsym{)}$:] There is a unique $\ottnt{e}'=e$
by rule \ruleref{S\_CastDownUp}.
		\item[Case $\ottnt{e}=\mu \, \ottmv{x}  \ottsym{:}  \tau  \ottsym{.}  \ottnt{e}$:] There is a unique $\ottnt{e}'=\ottnt{e}  \ottsym{[}  \ottmv{x}  \mapsto  \mu \, \ottmv{x}  \ottsym{:}  \tau  \ottsym{.}  \ottnt{e}  \ottsym{]} \,$ by rule \ruleref{S\_Mu}.
		\item[Case $\ottnt{e}=\ottnt{e_{{\mathrm{1}}}} \, \ottnt{e_{{\mathrm{2}}}}$ and $\ottnt{e_{{\mathrm{1}}}}$ is not a $\lambda$-term:] If
$\ottnt{e_{{\mathrm{1}}}}=v$, there is no $\ottnt{e'_{{\mathrm{1}}}}$ such that $\ottnt{e_{{\mathrm{1}}}}  \longrightarrow  \ottnt{e'_{{\mathrm{1}}}}$. Since $\ottnt{e_{{\mathrm{1}}}}$ is
not a $\lambda$-term, there is no rule to reduce $\ottnt{e}$. Thus there is no
$\ottnt{e}'$ such that $\ottnt{e}  \longrightarrow  \ottnt{e'}$.
		
		Otherwise, there exists some $\ottnt{e'_{{\mathrm{1}}}}$ such that $\ottnt{e_{{\mathrm{1}}}}  \longrightarrow  \ottnt{e'_{{\mathrm{1}}}}$. By the
induction hypothesis, $\ottnt{e'_{{\mathrm{1}}}}$ is unique reduction of $\ottnt{e_{{\mathrm{1}}}}$. Thus by rule
\ruleref{S\_App}, $\ottnt{e}'=\ottnt{e'_{{\mathrm{1}}}} \, \ottnt{e_{{\mathrm{2}}}}$ is the unique reduction for $\ottnt{e}$.
		\item[Case $\ottnt{e}=\mathsf{cast}_{\downarrow} \, \ottnt{e_{{\mathrm{1}}}}$ and $\ottnt{e_{{\mathrm{1}}}}$ is not a $ \mathsf{cast}^{\uparrow} $-term:]
If $\ottnt{e_{{\mathrm{1}}}}=v$, there is no $\ottnt{e'_{{\mathrm{1}}}}$ such that $\ottnt{e_{{\mathrm{1}}}}  \longrightarrow  \ottnt{e'_{{\mathrm{1}}}}$. Since $\ottnt{e_{{\mathrm{1}}}}$
is not a $ \mathsf{cast}^{\uparrow} $-term, there is no rule to reduce $\ottnt{e}$. Thus there is
no $\ottnt{e}'$ such that $\ottnt{e}  \longrightarrow  \ottnt{e'}$.
		
		Otherwise, there exists some $\ottnt{e'_{{\mathrm{1}}}}$ such that $\ottnt{e_{{\mathrm{1}}}}  \longrightarrow  \ottnt{e'_{{\mathrm{1}}}}$. By the
induction hypothesis, $\ottnt{e'_{{\mathrm{1}}}}$ is unique reduction of $\ottnt{e_{{\mathrm{1}}}}$. Thus by rule
\ruleref{S\_CastDown}, $\ottnt{e}'=\mathsf{cast}_{\downarrow} \, \ottnt{e'_{{\mathrm{1}}}}$ is the unique reduction for
$\ottnt{e}$.
	\end{description}
\end{proof}

\begin{lem}[Decidability of $n$-step Reduction]\label{lem:appendix:uniquen}
    The $n$-step reduction $ \longrightarrow_n $ is called decidable if 
    given $\ottnt{e}$ there is a unique $\ottnt{e'}$ such that $\ottnt{e}  \longrightarrow_n  \ottnt{e'}$ or no such $\ottnt{e'}$.
\end{lem}

\begin{proof}
	Immediate from Lemma \ref{lem:appendix:unired}, by induction on the number of
reduction steps.
\end{proof}

\begin{thm}[Decidability of Type Checking]
	There is an algorithm which given $\Gamma, \ottnt{e}$ computes the unique
$\tau$ such that $\Gamma  \vdash  \ottnt{e}  \ottsym{:}  \tau$ or reports there is no such $\tau$.
\end{thm}

\begin{proof}
	By induction on the structure of $\ottnt{e}$:
	\begin{description}
	    \item[Case $\ottnt{e}  \ottsym{=}  \star$:] Trivial by applying \ruleref{T\_Ax} and $\tau  \equiv  \star$.
		\item[Case $\ottnt{e}  \ottsym{=}  \ottmv{x}$:] Trivial by rule \ruleref{T\_Var} and $\tau$ is the
unique type of $\ottmv{x}$ if $\ottmv{x}  \ottsym{:}  \tau \, \in \, \Gamma$.
		\item[Case $\ottnt{e}=\ottnt{e_{{\mathrm{1}}}} \, \ottnt{e_{{\mathrm{2}}}}$:] By rule \ruleref{T\_App} and introduction
hypothesis, there exist unique expressions $\tau_{{\mathrm{1}}}$ and $\tau_{{\mathrm{2}}}$ such that $\Gamma  \vdash  \ottnt{e_{{\mathrm{1}}}}  \ottsym{:}  \ottsym{(}  \Pi \, \ottmv{x}  \ottsym{:}  \tau_{{\mathrm{1}}}  \ottsym{.}  \tau_{{\mathrm{2}}}  \ottsym{)}$, $\Gamma  \vdash  \ottnt{e_{{\mathrm{2}}}}  \ottsym{:}  \tau_{{\mathrm{1}}}$. Thus, from Lemma
\ref{lem:appendix:gen}, $\tau_{{\mathrm{2}}}  \ottsym{[}  \ottmv{x}  \mapsto  \ottnt{e_{{\mathrm{2}}}}  \ottsym{]} \,$ is the unique type of $\ottnt{e}$.
		\item[Case $\ottnt{e}  \ottsym{=}  \lambda  \ottmv{x}  \ottsym{:}  \tau_{{\mathrm{1}}}  \ottsym{.}  \ottnt{e_{{\mathrm{1}}}}$:] By rule \ruleref{T\_Lam} and introduction
hypothesis, there exist unique expressions $\tau_{{\mathrm{2}}}$ such that $\Gamma  \vdash  \ottsym{(}  \Pi \, \ottmv{x}  \ottsym{:}  \tau_{{\mathrm{1}}}  \ottsym{.}  \tau_{{\mathrm{2}}}  \ottsym{)}  \ottsym{:}  \star$ and $\Gamma  \ottsym{,}  \ottmv{x}  \ottsym{:}  \tau_{{\mathrm{1}}}  \vdash  \ottnt{e}  \ottsym{:}  \tau_{{\mathrm{2}}}$. Thus, from Lemma
\ref{lem:appendix:gen}, $\Pi \, \ottmv{x}  \ottsym{:}  \tau_{{\mathrm{1}}}  \ottsym{.}  \tau_{{\mathrm{2}}}$ is the unique type of $\ottnt{e}$.
		\item[Case $\ottnt{e}  \ottsym{=}  \Pi \, \ottmv{x}  \ottsym{:}  \tau_{{\mathrm{1}}}  \ottsym{.}  \tau_{{\mathrm{2}}}$:] By rule \ruleref{T\_Pi} and introduction
hypothesis, we have $\Gamma  \vdash  \tau_{{\mathrm{1}}}  \ottsym{:}  \star$ and $\Gamma  \ottsym{,}  \ottmv{x}  \ottsym{:}  \tau_{{\mathrm{1}}}  \vdash  \tau_{{\mathrm{2}}}  \ottsym{:}  \star$. Thus, from Lemma
\ref{lem:appendix:gen}, $\star$ is the unique type of $\ottnt{e}$.
		\item[Case $\ottnt{e}  \ottsym{=}  \mu \, \ottmv{x}  \ottsym{:}  \tau  \ottsym{.}  \ottnt{e_{{\mathrm{1}}}}$:] By rule \ruleref{T\_Mu} and introduction
hypothesis, we have $\Gamma  \vdash  \tau  \ottsym{:}  \star$ and $\Gamma  \ottsym{,}  \ottmv{x}  \ottsym{:}  \tau  \vdash  \ottnt{e}  \ottsym{:}  \tau$. Thus, from Lemma
\ref{lem:appendix:gen}, $\tau$ is the unique type of $\ottnt{e}$.
		\item[Case $\ottnt{e}=\mathsf{cast}^{\uparrow} \, \ottsym{[}  \tau_{{\mathrm{1}}}  \ottsym{]} \,  \ottnt{e_{{\mathrm{1}}}}$:] From the premises of rule
\ruleref{T\_CastUp}, by induction hypothesis, we can derive the type of
$\ottnt{e_{{\mathrm{1}}}}$ as $\tau_{{\mathrm{2}}}$, and check whether $\tau_{{\mathrm{1}}}$ is legal, i.e. its sorts is
$\star$. If $\tau_{{\mathrm{1}}}$ is legal, by Lemma \ref{lem:appendix:unired}, there is
at most one $\tau'_{{\mathrm{1}}}$ such that $\tau_{{\mathrm{1}}}  \longrightarrow  \tau'_{{\mathrm{1}}}$. If such $\tau'_{{\mathrm{1}}}$ does not
exist, then we report type checking fails. Otherwise, we examine if $\tau'_{{\mathrm{1}}}$
is syntactically equal to $\tau_{{\mathrm{2}}}$, i.e. $\tau'_{{\mathrm{1}}}  \equiv  \tau_{{\mathrm{2}}}$. If the equality
holds, we obtain the unique type of $\ottnt{e}$ which is $\tau_{{\mathrm{1}}}$. Otherwise, we
report $\ottnt{e}$ fails to type check.
		\item[Case $\ottnt{e}=\mathsf{cast}_{\downarrow} \, \ottnt{e_{{\mathrm{1}}}}$:] From the premises of rule
\ruleref{T\_CastDown}, by induction hypothesis, we can derive the type of
$\ottnt{e_{{\mathrm{1}}}}$ as $\tau_{{\mathrm{1}}}$. By Lemma \ref{lem:appendix:unired}, there is at most one
$\tau_{{\mathrm{2}}}$ such that $\tau_{{\mathrm{1}}}  \longrightarrow  \tau_{{\mathrm{2}}}$. If such $\tau_{{\mathrm{2}}}$ exists and its sorts is
$\star$, we have found the unique type of $\ottnt{e}$ is $\tau_{{\mathrm{2}}}$. Otherwise, we
report $\ottnt{e}$ fails to type check.
	\end{description}
\end{proof}

\subsection{Soundness}
\begin{dfn}[Multi-step reduction]
    The relation $ \twoheadrightarrow $ is the transitive and reflexive closure of
$ \longrightarrow $.
\end{dfn}

\begin{thm}[Subject Reduction]
If $\Gamma  \vdash  \ottnt{e}  \ottsym{:}  \sigma$ and $\ottnt{e}  \twoheadrightarrow  e'$ then $\Gamma  \vdash  \ottnt{e'}  \ottsym{:}  \sigma$.
\end{thm}

\begin{proof}
    We prove the case for one-step reduction, i.e. $\ottnt{e}  \longrightarrow  \ottnt{e'}$. The lemma
can follow by induction on the number of one-step reductions of $\ottnt{e}  \twoheadrightarrow 
\ottnt{e'}$.
    The proof is by induction with respect to the definition of one-step
reduction $ \longrightarrow $ as follows:
    \begin{description}
        \item[Case $\ottdruleSXXBeta{}$:] $\quad$ \\
        Suppose $\Gamma  \vdash  \ottsym{(}  \lambda  \ottmv{x}  \ottsym{:}  \tau_{{\mathrm{1}}}  \ottsym{.}  \ottnt{e_{{\mathrm{1}}}}  \ottsym{)} \, \ottnt{e_{{\mathrm{2}}}}  \ottsym{:}  \sigma$ and $\Gamma  \vdash  \ottnt{e_{{\mathrm{1}}}}  \ottsym{[}  \ottmv{x}  \mapsto  \ottnt{e_{{\mathrm{2}}}}  \ottsym{]} \,  \ottsym{:}  \sigma'$. By
Lemma \ref{lem:appendix:gen}(2), there exist expressions $\tau'_{{\mathrm{1}}}$ and $\tau_{{\mathrm{2}}}$
such that 
        \begin{align}
            &\Gamma  \vdash  \ottsym{(}  \lambda  \ottmv{x}  \ottsym{:}  \tau_{{\mathrm{1}}}  \ottsym{.}  \ottnt{e_{{\mathrm{1}}}}  \ottsym{)}  \ottsym{:}  \ottsym{(}  \Pi \, \ottmv{x}  \ottsym{:}  \tau'_{{\mathrm{1}}}  \ottsym{.}  \tau_{{\mathrm{2}}}  \ottsym{)} \label{equ:lam} \\
            &\Gamma  \vdash  \ottnt{e_{{\mathrm{2}}}}  \ottsym{:}  \tau'_{{\mathrm{1}}} \nonumber \\
            &\sigma  \equiv  \tau_{{\mathrm{2}}}  \ottsym{[}  \ottmv{x}  \mapsto  \ottnt{e_{{\mathrm{2}}}}  \ottsym{]} \, \nonumber
        \end{align}
        By Lemma \ref{lem:appendix:gen}(3), the judgement (\ref{equ:lam})
implies that there exists an expression $\tau'_{{\mathrm{2}}}$ such that
        \begin{align}
            &\Pi \, \ottmv{x}  \ottsym{:}  \tau'_{{\mathrm{1}}}  \ottsym{.}  \tau_{{\mathrm{2}}}  \equiv  \Pi \, \ottmv{x}  \ottsym{:}  \tau_{{\mathrm{1}}}  \ottsym{.}  \tau'_{{\mathrm{2}}} \label{equ:lameq}\\
            &\Gamma  \ottsym{,}  \ottmv{x}  \ottsym{:}  \tau_{{\mathrm{1}}}  \vdash  \ottnt{e_{{\mathrm{1}}}}  \ottsym{:}  \tau'_{{\mathrm{2}}} \nonumber
        \end{align}
        Hence, by (\ref{equ:lameq}) we have $\tau_{{\mathrm{1}}}  \equiv  \tau'_{{\mathrm{1}}}$ and $\tau_{{\mathrm{2}}}  \equiv  \tau'_{{\mathrm{2}}}$. Then we can obtain $\Gamma  \ottsym{,}  \ottmv{x}  \ottsym{:}  \tau_{{\mathrm{1}}}  \vdash  \ottnt{e_{{\mathrm{1}}}}  \ottsym{:}  \tau_{{\mathrm{2}}}$ and $\Gamma  \vdash  \ottnt{e_{{\mathrm{2}}}}  \ottsym{:}  \tau_{{\mathrm{1}}}$. By
Lemma \ref{lem:appendix:subst}, we have $\Gamma  \vdash  \ottnt{e_{{\mathrm{1}}}}  \ottsym{[}  \ottmv{x}  \mapsto  \ottnt{e_{{\mathrm{2}}}}  \ottsym{]} \,  \ottsym{:}  \tau_{{\mathrm{2}}}  \ottsym{[}  \ottmv{x}  \mapsto  \ottnt{e_{{\mathrm{2}}}}  \ottsym{]} \,$. Therefore, we conclude with $\sigma'  \equiv  \tau_{{\mathrm{2}}}  \ottsym{[}  \ottmv{x}  \mapsto  \ottnt{e_{{\mathrm{2}}}}  \ottsym{]} \,  \equiv  \sigma$.
        
        \item[Case $\ottdruleSXXApp{}$:] $\quad$ \\
        Suppose $\Gamma  \vdash  \ottnt{e_{{\mathrm{1}}}} \, \ottnt{e_{{\mathrm{2}}}}  \ottsym{:}  \sigma$ and $\Gamma  \vdash  \ottnt{e'_{{\mathrm{1}}}} \, \ottnt{e_{{\mathrm{2}}}}  \ottsym{:}  \sigma'$. By Lemma
\ref{lem:appendix:gen}(2), there exist expressions $\tau_{{\mathrm{1}}}$ and $\tau_{{\mathrm{2}}}$ such
that 
        \begin{align*}
            &\Gamma  \vdash  \ottnt{e_{{\mathrm{1}}}}  \ottsym{:}  \ottsym{(}  \Pi \, \ottmv{x}  \ottsym{:}  \tau_{{\mathrm{1}}}  \ottsym{.}  \tau_{{\mathrm{2}}}  \ottsym{)} \\
            &\Gamma  \vdash  \ottnt{e_{{\mathrm{2}}}}  \ottsym{:}  \tau_{{\mathrm{1}}}\\
            &\sigma  \equiv  \tau_{{\mathrm{2}}}  \ottsym{[}  \ottmv{x}  \mapsto  \ottnt{e_{{\mathrm{2}}}}  \ottsym{]} \,
        \end{align*}
        By induction hypothesis, we have $\Gamma  \vdash  \ottnt{e'_{{\mathrm{1}}}}  \ottsym{:}  \ottsym{(}  \Pi \, \ottmv{x}  \ottsym{:}  \tau_{{\mathrm{1}}}  \ottsym{.}  \tau_{{\mathrm{2}}}  \ottsym{)}$. By rule
\ruleref{T\_App}, we obtain $\Gamma  \vdash  \ottnt{e'_{{\mathrm{1}}}} \, \ottnt{e_{{\mathrm{2}}}}  \ottsym{:}  \tau_{{\mathrm{2}}}  \ottsym{[}  \ottmv{x}  \mapsto  \ottnt{e_{{\mathrm{2}}}}  \ottsym{]} \,$. Therefore, $\sigma'  \equiv  \tau_{{\mathrm{2}}}  \ottsym{[}  \ottmv{x}  \mapsto  \ottnt{e_{{\mathrm{2}}}}  \ottsym{]} \,  \equiv  \sigma$.
        
        \item[Case $\ottdruleSXXCastDown{}$:] $\quad$ \\
        Suppose $\Gamma  \vdash  \mathsf{cast}_{\downarrow} \, \ottnt{e}  \ottsym{:}  \sigma$ and $\Gamma  \vdash  \mathsf{cast}_{\downarrow} \, \ottnt{e'}  \ottsym{:}  \sigma'$. By
Lemma \ref{lem:appendix:gen}(7), there exist expressions $\tau_{{\mathrm{1}}}, \tau_{{\mathrm{2}}}$ such
that 
        \begin{align*}
            &\Gamma  \vdash  \ottnt{e}  \ottsym{:}  \tau_{{\mathrm{1}}} \qquad \Gamma  \vdash  \tau_{{\mathrm{2}}}  \ottsym{:}  \star \\
            &\tau_{{\mathrm{1}}}  \longrightarrow  \tau_{{\mathrm{2}}} \qquad \sigma  \equiv  \tau_{{\mathrm{2}}}
        \end{align*}
        By induction hypothesis, we have $\Gamma  \vdash  \ottnt{e'}  \ottsym{:}  \tau_{{\mathrm{1}}}$. By rule
\ruleref{T\_CastDown}, we obtain $\Gamma  \vdash  \mathsf{cast}_{\downarrow} \, \ottnt{e'}  \ottsym{:}  \tau_{{\mathrm{2}}}$. Therefore, $\sigma'  \equiv  \tau_{{\mathrm{2}}}  \equiv  \sigma$.
        
        \item[Case $\ottdruleSXXCastDownUp{}$:] $\quad$ \\
        Suppose $\Gamma  \vdash  \mathsf{cast}_{\downarrow} \, \ottsym{(}  \mathsf{cast}^{\uparrow} \, \ottsym{[}  \tau_{{\mathrm{1}}}  \ottsym{]} \,  \ottnt{e}  \ottsym{)}  \ottsym{:}  \sigma$ and $\Gamma  \vdash  \ottnt{e}  \ottsym{:}  \sigma'$. By
Lemma \ref{lem:appendix:gen}(7), there exist expressions $\tau'_{{\mathrm{1}}}, \tau_{{\mathrm{2}}}$ such
that 
        \begin{align}
            &\Gamma  \vdash  \ottsym{(}  \mathsf{cast}^{\uparrow} \, \ottsym{[}  \tau_{{\mathrm{1}}}  \ottsym{]} \,  \ottnt{e}  \ottsym{)}  \ottsym{:}  \tau'_{{\mathrm{1}}} \label{equ:fold} \\
            &\tau'_{{\mathrm{1}}}  \longrightarrow  \tau_{{\mathrm{2}}} \label{equ:foldeq1} \\
            &\sigma  \equiv  \tau_{{\mathrm{2}}} \label{equ:foldeq4}
        \end{align}
        By Lemma \ref{lem:appendix:gen}(6), the judgement (\ref{equ:fold})
implies that there exists an expression $\tau'_{{\mathrm{2}}}$ such that
        \begin{align}
            &\Gamma  \vdash  \ottnt{e}  \ottsym{:}  \tau'_{{\mathrm{2}}} \label{equ:foldr} \\
            &\tau_{{\mathrm{1}}}  \longrightarrow  \tau'_{{\mathrm{2}}} \label{equ:foldeq2} \\
            &\tau'_{{\mathrm{1}}}  \equiv  \tau_{{\mathrm{1}}} \label{equ:foldeq3}
        \end{align}
        By (\ref{equ:foldeq1}, \ref{equ:foldeq2}, \ref{equ:foldeq3}) and Lemma
\ref{lem:appendix:unired} we obtain $\tau_{{\mathrm{2}}}  \equiv  \tau'_{{\mathrm{2}}}$. From (\ref{equ:foldr}) we
have $\sigma'  \equiv  \tau'_{{\mathrm{2}}}$. Therefore, by (\ref{equ:foldeq4}), $\sigma'  \equiv  \tau'_{{\mathrm{2}}}
 \equiv  \tau_{{\mathrm{2}}}  \equiv  \sigma$.
        
        \item[Case $\ottdruleSXXMu{}$:] $\quad$ \\
        Suppose $\Gamma  \vdash  \ottsym{(}  \mu \, \ottmv{x}  \ottsym{:}  \tau  \ottsym{.}  \ottnt{e}  \ottsym{)}  \ottsym{:}  \sigma$ and $\Gamma  \vdash  \ottnt{e}  \ottsym{[}  \ottmv{x}  \mapsto  \mu \, \ottmv{x}  \ottsym{:}  \tau  \ottsym{.}  \ottnt{e}  \ottsym{]} \,  \ottsym{:}  \sigma'$.
By Lemma \ref{lem:appendix:gen}(5), we have $\sigma  \equiv  \tau$ and $\Gamma  \ottsym{,}  \ottmv{x}  \ottsym{:}  \tau  \vdash  \ottnt{e}  \ottsym{:}  \tau$. Then we obtain $\Gamma  \vdash  \ottsym{(}  \mu \, \ottmv{x}  \ottsym{:}  \tau  \ottsym{.}  \ottnt{e}  \ottsym{)}  \ottsym{:}  \tau$. Thus by Lemma
\ref{lem:appendix:subst}, we have $\Gamma  \vdash  \ottnt{e}  \ottsym{[}  \ottmv{x}  \mapsto  \mu \, \ottmv{x}  \ottsym{:}  \tau  \ottsym{.}  \ottnt{e}  \ottsym{]} \,  \ottsym{:}  \tau  \ottsym{[}  \ottmv{x}  \mapsto  \mu \, \ottmv{x}  \ottsym{:}  \tau  \ottsym{.}  \ottnt{e}  \ottsym{]} \,$.
        
        Note that $\ottmv{x}:\tau$, i.e. the type of $\ottmv{x}$ is $\tau$, then
$\ottmv{x} \notin \FV(\tau)$ holds implicitly. Hence, by the definition of
substitution, we obtain $\tau  \ottsym{[}  \ottmv{x}  \mapsto  \mu \, \ottmv{x}  \ottsym{:}  \tau  \ottsym{.}  \ottnt{e}  \ottsym{]} \,  \equiv  \tau$. Therefore, $\sigma'  \equiv  \tau  \ottsym{[}  \ottmv{x}  \mapsto  \mu \, \ottmv{x}  \ottsym{:}  \tau  \ottsym{.}  \ottnt{e}  \ottsym{]} \,  \equiv  \tau  \equiv  \sigma$.
    \end{description}
\end{proof}

\begin{thm}[Progress]
If $\varnothing  \vdash  \ottnt{e}  \ottsym{:}  \sigma$ then either $\ottnt{e}$ is a value $v$ or there exists $\ottnt{e}'$
such that $\ottnt{e}  \longrightarrow  \ottnt{e'}$.
\end{thm}

\begin{proof}
    By induction on the derivation of $\varnothing  \vdash  \ottnt{e}  \ottsym{:}  \sigma$ as follows:
    \begin{description}
        \item[Case $\ottnt{e}  \ottsym{=}  \ottmv{x}$:] Impossible, because the context is empty.
        \item[Case $\ottnt{e}  \ottsym{=}  \ottnt{v}$:] Trivial, since $\ottnt{e}$ is already a value that
has one of the following forms:
		\begin{inparaenum}[(1)]
		    \item $\star$,
			\item $\lambda  \ottmv{x}  \ottsym{:}  \tau  \ottsym{.}  \ottnt{e}$,
			\item $\Pi \, \ottmv{x}  \ottsym{:}  \tau_{{\mathrm{1}}}  \ottsym{.}  \tau_{{\mathrm{2}}}$,
			\item $\mathsf{cast}^{\uparrow} \, \ottsym{[}  \tau  \ottsym{]} \,  \ottnt{e}$.
		\end{inparaenum}
		\item[Case $\ottnt{e}=\ottnt{e_{{\mathrm{1}}}} \, \ottnt{e_{{\mathrm{2}}}}$:] By Lemma \ref{lem:appendix:gen}(2), there
exist expressions $\tau_{{\mathrm{1}}}$ and $\tau_{{\mathrm{2}}}$ such that $\varnothing  \vdash  \ottnt{e_{{\mathrm{1}}}}  \ottsym{:}  \ottsym{(}  \Pi \, \ottmv{x}  \ottsym{:}  \tau_{{\mathrm{1}}}  \ottsym{.}  \tau_{{\mathrm{2}}}  \ottsym{)}$ and
$\varnothing  \vdash  \ottnt{e_{{\mathrm{2}}}}  \ottsym{:}  \tau_{{\mathrm{1}}}$. Consider whether $\ottnt{e_{{\mathrm{1}}}}$ is a value:
    		\begin{itemize}
    		    \item If $\ottnt{e_{{\mathrm{1}}}}=v$, by Lemma \ref{lem:appendix:gen}(3), it must be a
$\lambda$-term such that $\ottnt{e_{{\mathrm{1}}}}  \equiv  \lambda  \ottmv{x}  \ottsym{:}  \tau_{{\mathrm{1}}}  \ottsym{.}  \ottnt{e'_{{\mathrm{1}}}}$ for some $\ottnt{e'_{{\mathrm{1}}}}$ satisfying
$\varnothing  \vdash  \ottnt{e'_{{\mathrm{1}}}}  \ottsym{:}  \tau_{{\mathrm{2}}}$. Then by rule \ruleref{S\_Beta}, we have $\ottsym{(}  \lambda  \ottmv{x}  \ottsym{:}  \tau_{{\mathrm{1}}}  \ottsym{.}  \ottnt{e'_{{\mathrm{1}}}}  \ottsym{)} \, \ottnt{e_{{\mathrm{2}}}}  \longrightarrow  \ottnt{e'_{{\mathrm{1}}}}  \ottsym{[}  \ottmv{x}  \mapsto  \ottnt{e_{{\mathrm{2}}}}  \ottsym{]} \,$. Thus, there exists $\ottnt{e'}  \equiv  \ottnt{e'_{{\mathrm{1}}}}  \ottsym{[}  \ottmv{x}  \mapsto  \ottnt{e_{{\mathrm{2}}}}  \ottsym{]} \,$ such that
$\ottnt{e}  \longrightarrow  \ottnt{e'}$.
    		    \item Otherwise, by induction hypothesis, there exists $\ottnt{e'_{{\mathrm{1}}}}$ such
that $\ottnt{e_{{\mathrm{1}}}}  \longrightarrow  \ottnt{e'_{{\mathrm{1}}}}$. Then by rule \ruleref{S\_App}, we have $\ottnt{e_{{\mathrm{1}}}} \, \ottnt{e_{{\mathrm{2}}}}  \longrightarrow  \ottnt{e'_{{\mathrm{1}}}} \, \ottnt{e_{{\mathrm{2}}}}$. Thus, there exists $\ottnt{e'}  \equiv  \ottnt{e'_{{\mathrm{1}}}} \, \ottnt{e_{{\mathrm{2}}}}$ such that $\ottnt{e}  \longrightarrow  \ottnt{e'}$.
    		\end{itemize}
		\item[Case $\ottnt{e}=\mathsf{cast}_{\downarrow} \, \ottnt{e_{{\mathrm{1}}}}$:] By Lemma \ref{lem:appendix:gen}(7),
there exist expressions $\tau_{{\mathrm{1}}}$ and $\tau_{{\mathrm{2}}}$ such that $\varnothing  \vdash  \ottnt{e_{{\mathrm{1}}}}  \ottsym{:}  \tau_{{\mathrm{1}}}$ and
$\tau_{{\mathrm{1}}}  \longrightarrow  \tau_{{\mathrm{2}}}$. Consider whether $\ottnt{e_{{\mathrm{1}}}}$ is a value:
		     \begin{itemize}
    		    \item If $\ottnt{e_{{\mathrm{1}}}}=v$, by Lemma \ref{lem:appendix:gen}(6), it must be a
$ \mathsf{cast}^{\uparrow} $-term such that $\ottnt{e_{{\mathrm{1}}}}  \equiv  \mathsf{cast}^{\uparrow} \, \ottsym{[}  \tau_{{\mathrm{1}}}  \ottsym{]} \,  \ottnt{e'_{{\mathrm{1}}}}$ for some $\ottnt{e'_{{\mathrm{1}}}}$
satisfying $\varnothing  \vdash  \ottnt{e'_{{\mathrm{1}}}}  \ottsym{:}  \tau_{{\mathrm{2}}}$. Then by rule \ruleref{S\_CastDownUp}, we can obtain
$\mathsf{cast}_{\downarrow} \, \ottsym{(}  \mathsf{cast}^{\uparrow} \, \ottsym{[}  \tau_{{\mathrm{1}}}  \ottsym{]} \,  \ottnt{e'_{{\mathrm{1}}}}  \ottsym{)}  \longrightarrow  \ottnt{e'_{{\mathrm{1}}}}$. Thus, there exists $\ottnt{e'}  \equiv  \ottnt{e'_{{\mathrm{1}}}}$
such that $\ottnt{e}  \longrightarrow  \ottnt{e'}$.
    		    \item Otherwise, by induction hypothesis, there exists $\ottnt{e'_{{\mathrm{1}}}}$ such
that $\ottnt{e_{{\mathrm{1}}}}  \longrightarrow  \ottnt{e'_{{\mathrm{1}}}}$. Then by rule \ruleref{S\_CastDown}, we have $\mathsf{cast}_{\downarrow} \, \ottnt{e_{{\mathrm{1}}}}  \longrightarrow  \mathsf{cast}_{\downarrow} \, \ottnt{e'_{{\mathrm{1}}}}$. Thus, there exists $\ottnt{e'}  \equiv  \mathsf{cast}_{\downarrow} \, \ottnt{e'_{{\mathrm{1}}}}$ such that
$\ottnt{e}  \longrightarrow  \ottnt{e'}$.
    		\end{itemize}
		\item[Case $\ottnt{e}=\mu \, \ottmv{x}  \ottsym{:}  \tau  \ottsym{.}  \ottnt{e_{{\mathrm{1}}}}$:] By rule \ruleref{S\_Mu}, there always
exists $\ottnt{e'}  \equiv  \ottnt{e_{{\mathrm{1}}}}  \ottsym{[}  \ottmv{x}  \mapsto  \mu \, \ottmv{x}  \ottsym{:}  \tau  \ottsym{.}  \ottnt{e_{{\mathrm{1}}}}  \ottsym{]} \,$.
    \end{description}
\end{proof}

\section{Full Specification of Surface Language}
\subsection{Syntax}
See Figure \ref{fig:appendix:syntax}.
\begin{figure*}
\centering
\gram{\ottpgm\ottinterrule
\ottdecl\ottinterrule
\ottu\ottinterrule
\ottp\ottinterrule
\ottE\ottinterrule
\ottGs}
\[\ottsurfsugar\] % defined in otthelper.mng.tex
\caption{Syntax of the surface language}
\label{fig:appendix:syntax}
\end{figure*}

\subsection{Expression Typing}
See Figure \ref{fig:appendix:typing}.
\begin{figure*}
\renewcommand{\hlmath}[1]{}
\renewcommand{\ottdrulename}[1]{\textsc{\replace{#1}{TR}{TS}}}
\renewcommand{\ottcom}[1]{\text{\replace{#1}{translation}{typing}}}
\ottdefnctxtrans{}\ottinterrule
\ottdefnpgmtrans{}\ottinterrule
\ottdefndecltrans{}\ottinterrule % defined in otthelper.mng.tex
\ottdefnpattrans{}\ottinterrule
\ottdefnexprtrans{}
\caption{Typing rules of the surface language}
\label{fig:appendix:typing}
\end{figure*}

\subsection{Translation to the Core}
See Figure \ref{fig:appendix:translate}.
\begin{figure*}
\ottdefnctxtrans{}\ottinterrule
\ottdefnpgmtrans{}\ottinterrule
\ottdefndecltrans{}
\[\hlmath{\ottdecltrans}\]\ottinterrule % defined in otthelper.mng.tex
\ottdefnpattrans{}\ottinterrule
\ottdefnexprtrans{}
\caption{Translation rules of the surface language}
\label{fig:appendix:translate}
\end{figure*}

\section{Proofs about Surface Language}
\subsection{Type-safety of Translation}

\begin{thm}[Type-safety of Expression Translation]
If $ \Sigma  \labeledjudge{s}  \ottnt{E}  :  T   \rightsquigarrow   \ottnt{e} $, $ \Sigma  \labeledjudge{s}  T  :  \star   \rightsquigarrow   \tau $ and $ \labeledjudge{wf}  \Sigma   \rightsquigarrow   \Gamma $, then
$\Gamma  \vdash  \ottnt{e}  \ottsym{:}  \tau$.
\end{thm}

\begin{proof}
    By induction on the derivation of $ \Sigma  \labeledjudge{s}  \ottnt{E}  :  T   \rightsquigarrow   \ottnt{e} $ . Suppose there is
a core language context $\Gamma$ such that $ \labeledjudge{wf}  \Sigma   \rightsquigarrow   \Gamma $.
    \begin{description}
        \item[Case \ruleref{TR\_Ax}:] Trivial. $\ottnt{e} = \tau = \star$ and
$ \Sigma  \labeledjudge{s}  \star  :  \star $ holds by rule \ruleref{T\_Ax}.
        \item[Case \ruleref{TR\_Var}:] Trivial. By rule \ruleref{T\_Var}, we
have $ \labeledjudge{wf}  \Sigma   \rightsquigarrow   \Gamma $, then $\ottmv{x}:\tau  \in  \Gamma$ where $ \Sigma  \labeledjudge{s}  T  :  \star   \rightsquigarrow   \tau $.
        \item[Case \ruleref{TR\_App}:] Suppose
            \[\begin{array}{l}
             \Sigma  \labeledjudge{s}  \ottnt{E_{{\mathrm{1}}}} \, \ottnt{E_{{\mathrm{2}}}}  :  T_{{\mathrm{1}}}  \ottsym{[}  \ottmv{x}  \mapsto  \ottnt{E_{{\mathrm{2}}}}  \ottsym{]} \,   \rightsquigarrow   \ottnt{e_{{\mathrm{1}}}} \, \ottnt{e_{{\mathrm{2}}}}  \\
             \Sigma  \labeledjudge{s}  T_{{\mathrm{1}}}  \ottsym{[}  \ottmv{x}  \mapsto  \ottnt{E_{{\mathrm{2}}}}  \ottsym{]} \,  :  \star   \rightsquigarrow   \tau_{{\mathrm{1}}}  \ottsym{[}  \ottmv{x}  \mapsto  \ottnt{e_{{\mathrm{2}}}}  \ottsym{]} \, .
            \end{array} \]
            By induction
            hypothesis, we have 
            $
            \Gamma  \vdash  \ottnt{e_{{\mathrm{1}}}}  \ottsym{:}  \ottsym{(}  \Pi \, \ottmv{x}  \ottsym{:}  \tau_{{\mathrm{2}}}  \ottsym{.}  \tau_{{\mathrm{1}}}  \ottsym{)},
            \Gamma  \vdash  \ottnt{e_{{\mathrm{2}}}}  \ottsym{:}  \tau_{{\mathrm{2}}},
            $
            where
            \[\begin{array}{l}
              \Sigma  \labeledjudge{s}  \ottnt{E_{{\mathrm{1}}}}  :  \ottsym{(}  \Pi \, \ottmv{x}  \ottsym{:}  T_{{\mathrm{2}}}  \ottsym{.}  T_{{\mathrm{1}}}  \ottsym{)}   \rightsquigarrow   \ottnt{e_{{\mathrm{1}}}}  \\
               \Sigma  \labeledjudge{s}  \ottsym{(}  \Pi \, \ottmv{x}  \ottsym{:}  T_{{\mathrm{2}}}  \ottsym{.}  T_{{\mathrm{1}}}  \ottsym{)}  :  \star   \rightsquigarrow   \ottsym{(}  \Pi \, \ottmv{x}  \ottsym{:}  \tau_{{\mathrm{2}}}  \ottsym{.}  \tau_{{\mathrm{1}}}  \ottsym{)}  \\
               \Sigma  \labeledjudge{s}  \ottnt{E_{{\mathrm{2}}}}  :  T_{{\mathrm{2}}}   \rightsquigarrow   \ottnt{e_{{\mathrm{2}}}}  \\
               \Sigma  \labeledjudge{s}  T_{{\mathrm{2}}}  :  \star   \rightsquigarrow   \tau_{{\mathrm{2}}} .
            \end{array}\] Thus by rule \ruleref{T\_App}, we have $\Gamma  \vdash  \ottnt{e_{{\mathrm{1}}}} \, \ottnt{e_{{\mathrm{2}}}}  \ottsym{:}  \tau_{{\mathrm{1}}}  \ottsym{[}  \ottmv{x}  \mapsto  \ottnt{e_{{\mathrm{2}}}}  \ottsym{]} \,$.
        \item[Case \ruleref{TR\_Lam}:] Suppose
            \[\begin{array}{l}
             \Sigma  \labeledjudge{s}  \ottsym{(}  \lambda  \ottmv{x}  \ottsym{:}  T_{{\mathrm{1}}}  \ottsym{.}  \ottnt{E}  \ottsym{)}  :  \ottsym{(}  \Pi \, \ottmv{x}  \ottsym{:}  T_{{\mathrm{1}}}  \ottsym{.}  T_{{\mathrm{2}}}  \ottsym{)}   \rightsquigarrow   \lambda  \ottmv{x}  \ottsym{:}  \tau_{{\mathrm{1}}}  \ottsym{.}  \ottnt{e}  \\ 
             \Sigma  \labeledjudge{s}  \Pi \, \ottmv{x}  \ottsym{:}  T_{{\mathrm{1}}}  \ottsym{.}  T_{{\mathrm{2}}}  :  \star   \rightsquigarrow   \Pi \, \ottmv{x}  \ottsym{:}  \tau_{{\mathrm{1}}}  \ottsym{.}  \tau_{{\mathrm{2}}} .
            \end{array} \]
            By induction hypothesis, we have 
            $
            \Gamma  \ottsym{,}  \ottmv{x}  \ottsym{:}  \tau_{{\mathrm{1}}}  \vdash  \ottnt{e}  \ottsym{:}  \tau_{{\mathrm{2}}},
            \Gamma  \vdash  \Pi \, \ottmv{x}  \ottsym{:}  \tau_{{\mathrm{1}}}  \ottsym{.}  \tau_{{\mathrm{2}}}  \ottsym{:}  \star
            $
            where 
            \[
            \begin{array}{ll}
             \Sigma  \ottsym{,}  \ottmv{x}  \ottsym{:}  T_{{\mathrm{1}}}  \labeledjudge{s}  \ottnt{E}  :  T_{{\mathrm{2}}}   \rightsquigarrow   \ottnt{e}  & \\
             \Sigma  \labeledjudge{s}  T_{{\mathrm{1}}}  :  \star   \rightsquigarrow   \tau_{{\mathrm{1}}}  &  \Sigma  \labeledjudge{s}  T_{{\mathrm{2}}}  :  \star   \rightsquigarrow   \tau_{{\mathrm{2}}}  \\
             \Sigma  \labeledjudge{s}  \ottsym{(}  \Pi \, \ottmv{x}  \ottsym{:}  T_{{\mathrm{1}}}  \ottsym{.}  T_{{\mathrm{2}}}  \ottsym{)}  :  \star   \rightsquigarrow   \Pi \, \ottmv{x}  \ottsym{:}  \tau_{{\mathrm{1}}}  \ottsym{.}  \tau_{{\mathrm{2}}}  &
            \end{array}
            \]
            Thus by rule \ruleref{T\_Lam}, we have $\Gamma  \vdash  \ottsym{(}  \lambda  \ottmv{x}  \ottsym{:}  \tau_{{\mathrm{1}}}  \ottsym{.}  \ottnt{e}  \ottsym{)}  \ottsym{:}  \ottsym{(}  \Pi \, \ottmv{x}  \ottsym{:}  \tau_{{\mathrm{1}}}  \ottsym{.}  \tau_{{\mathrm{2}}}  \ottsym{)}$.
        \item[Case \ruleref{TR\_Pi}:] Suppose 
                \[  \Sigma  \labeledjudge{s}  \ottsym{(}  \Pi \, \ottmv{x}  \ottsym{:}  T_{{\mathrm{1}}}  \ottsym{.}  T_{{\mathrm{2}}}  \ottsym{)}  :  \star   \rightsquigarrow   \Pi \, \ottmv{x}  \ottsym{:}  \tau_{{\mathrm{1}}}  \ottsym{.}  \tau_{{\mathrm{2}}} . \] 
            By induction hypothesis, we have 
            $
                \Gamma  \vdash  \tau_{{\mathrm{1}}}  \ottsym{:}  \star, \Gamma  \ottsym{,}  \ottmv{x}  \ottsym{:}  \tau_{{\mathrm{1}}}  \vdash  \tau_{{\mathrm{2}}}  \ottsym{:}  \star
            $
            where
            $
                 \Sigma  \labeledjudge{s}  T_{{\mathrm{1}}}  :  \star   \rightsquigarrow   \tau_{{\mathrm{1}}} ,  \Sigma  \ottsym{,}  \ottmv{x}  \ottsym{:}  T_{{\mathrm{1}}}  \labeledjudge{s}  T_{{\mathrm{2}}}  :  \star   \rightsquigarrow   \tau_{{\mathrm{2}}} 
            $
            Thus by rule \ruleref{T\_Pi} we have $\Gamma  \vdash  \ottsym{(}  \Pi \, \ottmv{x}  \ottsym{:}  \tau_{{\mathrm{1}}}  \ottsym{.}  \tau_{{\mathrm{2}}}  \ottsym{)}  \ottsym{:}  \star$.
        \item[Case \ruleref{TR\_Mu}:] Suppose 
                \[\begin{array}{l}
                     \Sigma  \labeledjudge{s}  \ottsym{(}  \mu \, \ottmv{x}  \ottsym{:}  T  \ottsym{.}  \ottnt{E}  \ottsym{)}  :  T   \rightsquigarrow   \mu \, \ottmv{x}  \ottsym{:}  \tau  \ottsym{.}  \ottnt{e}  \\
                     \Sigma  \labeledjudge{s}  T  :  \star   \rightsquigarrow   \tau . 
                \end{array}\]
            By induction hypothesis, we have 
                \[ \Gamma  \ottsym{,}  \ottmv{x}  \ottsym{:}  \tau  \vdash  \ottnt{e}  \ottsym{:}  \tau,\text{ where } \Sigma  \ottsym{,}  \ottmv{x}  \ottsym{:}  T  \labeledjudge{s}  \ottnt{E}  :  T   \rightsquigarrow   \ottnt{e} . \] 
            Thus by rule \ruleref{T\_Mu}, we have $\Gamma  \vdash  \ottsym{(}  \mu \, \ottmv{x}  \ottsym{:}  \tau  \ottsym{.}  \ottnt{e}  \ottsym{)}  \ottsym{:}  \tau$.
        \item[Case \ruleref{TR\_Case}:] Suppose 
            \[\begin{array}{l}
                 \Sigma  \labeledjudge{s}  \kw{case} \, \ottnt{E_{{\mathrm{1}}}} \, \kw{of} \, \,\overline{  \ottnt{p}  \Rightarrow  \ottnt{E_{{\mathrm{2}}}}  }\,  :  S   \rightsquigarrow   \ottsym{(}  \mathsf{cast}_{\downarrow}^{n+1} \, \ottnt{e_{{\mathrm{1}}}}  \ottsym{)} \, \sigma \, \,\overline{  \ottnt{e_{{\mathrm{2}}}}  }\,  \\
                 \Sigma  \labeledjudge{s}  S  :  \star   \rightsquigarrow   \sigma .
            \end{array}\]
            By induction hypothesis, we have 
            \[\begin{array}{ll}
                 \Sigma  \labeledjudge{s}  \ottnt{E_{{\mathrm{1}}}}  :  T_{{\mathrm{1}}}   \rightsquigarrow   \ottnt{e_{{\mathrm{1}}}}  &
                 \Sigma  \labeledjudge{s}  T_{{\mathrm{1}}}  :  \star   \rightsquigarrow   \tau_{{\mathrm{1}}}  \\
                \Gamma  \vdash  \ottnt{e_{{\mathrm{1}}}}  \ottsym{:}  \tau_{{\mathrm{1}}} &
                \,\overline{   \Sigma  \labeledjudge{p}  \ottnt{p}   \Rightarrow   \ottnt{E_{{\mathrm{2}}}}  :  T_{{\mathrm{1}}}   \rightarrow   S   \rightsquigarrow   \ottnt{e_{{\mathrm{2}}}}   }\,            
            \end{array}\]
            By rule \ruleref{TRpat\_Alt}, we have
            \begin{align*}
                \ottnt{p} & \equiv  \ottmv{K}  \,\overline{  \ottmv{x}  \ottsym{:}  T  \ottsym{[}  \,\overline{  \ottnt{u}  \mapsto  \upsilon  }\,  \ottsym{]} \,  }\, \\
                T_{{\mathrm{1}}} & \equiv  \ottmv{D}    \,\overline{  \upsilon  }^{n}\, \\
                \,\overline{  \ottnt{e_{{\mathrm{2}}}}  }\, & \equiv  \,\overline{  \lambda  \,\overline{  \ottmv{x}  \ottsym{:}  \tau'  }\,  \ottsym{.}  \ottnt{e}  }\,
            \end{align*}
            where
            \[\begin{array}{ll}
                \,\overline{   \Sigma  \labeledjudge{s}  \ottnt{E_{{\mathrm{2}}}}  :  S   \rightsquigarrow   \ottnt{e}   }\, &
                \,\overline{  \Gamma  \vdash  \ottnt{e}  \ottsym{:}  \sigma  }\, \\
                \,\overline{   \Sigma  \labeledjudge{s}  T  \ottsym{[}  \,\overline{  \ottnt{u}  \mapsto  \upsilon  }\,  \ottsym{]} \,  :  \star   \rightsquigarrow   \tau  \ottsym{[}  u  \mapsto  u'  \ottsym{]} \,   }\, &
                \,\overline{   \Sigma  \labeledjudge{s}  \upsilon  :  \star   \rightsquigarrow   u'   }\, \\
                \tau'  \equiv  \tau  \ottsym{[}  u  \mapsto  u'  \ottsym{]} \,
            \end{array}\]
            By rule \ruleref{TRdecl\_Data}, we have $\ottmv{D}   \equiv  \ottdeclD$. Thus,
            \[ \tau_{{\mathrm{1}}}  \equiv  \ottmv{D} \,\overline{  u'  }\,^n,\text{ where }\,\overline{  \Gamma  \vdash  u'  \ottsym{:}  \rho  }\,.\] 
            Note that by operational semantics, the following reduction sequence follows for $\tau_{{\mathrm{1}}}$:
            \begin{align*}
                \ottmv{D} \,\overline{  u'  }\,^n~
                & \longrightarrow ~ \mathscale[0.7]{\ottsym{(}  \lambda  \,\overline{  \ottnt{u}  \ottsym{:}  \rho  }^{n}\,  \ottsym{.}  \ottsym{(}  b  \ottsym{:}  \star  \ottsym{)}  \rightarrow  \,\overline{  \ottsym{(}  \ottsym{(}  \,\overline{  \ottmv{x}  \ottsym{:}  \tau  \ottsym{[}  \ottmv{D}  \mapsto  \ottmv{X}  \ottsym{]} \,  \ottsym{[}  \ottmv{X}  \mapsto  \ottmv{D}  \ottsym{]} \,  }\,  \ottsym{)}  \rightarrow  b  \ottsym{)}  }\,  \rightarrow  b  \ottsym{)}\,\overline{  u'  }\,^n}\\
                & \longrightarrow_n ~ \ottsym{(}  b  \ottsym{:}  \star  \ottsym{)}  \rightarrow  \,\overline{  \ottsym{(}  \,\overline{  \ottmv{x}  \ottsym{:}  \tau'  }\,  \ottsym{)}  \rightarrow  b  }\,  \rightarrow  b
            \end{align*}
            Then by
            rule \ruleref{T\_CastDown} and the definition of $n$-step cast operator, the
            type of $\mathsf{cast}_{\downarrow}^{n+1} \, \ottnt{e_{{\mathrm{1}}}}$ is \[ \ottsym{(}  b  \ottsym{:}  \star  \ottsym{)}  \rightarrow  \,\overline{  \ottsym{(}  \,\overline{  \ottmv{x}  \ottsym{:}  \tau'  }\,  \ottsym{)}  \rightarrow  b  }\,  \rightarrow  b.\] Note
            that by rule \ruleref{T\_Lam}, $\Gamma  \vdash  \ottnt{e_{{\mathrm{2}}}}  \ottsym{:}  \ottsym{(}  \,\overline{  \ottmv{x}  \ottsym{:}  \tau'  }\,  \ottsym{)}  \rightarrow  \sigma$. Therefore, by rule
            \ruleref{T\_App}, we obtain $\Gamma  \vdash  \ottsym{(}  \mathsf{cast}_{\downarrow}^{n+1} \, \ottnt{e_{{\mathrm{1}}}}  \ottsym{)} \, \sigma \, \,\overline{  \ottnt{e_{{\mathrm{2}}}}  }\,  \ottsym{:}  \sigma$, which follows
            the result.
    \end{description}
\end{proof}

