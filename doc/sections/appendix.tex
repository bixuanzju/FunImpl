%%% !!! WARNING: AUTO GENERATED. DO NOT MODIFY !!! %%%
\section{Specification of core language}

\newcommand{\FV}{\mathsf{FV}}
\newcommand{\dom}{\mathsf{dom}}

\subsection{Syntax}
\gram{\otte\ottinterrule
        \otts\ottinterrule
        \ottG\ottinterrule
        \ottv}

\subsection{Operational semantics and expression typing}
\ottdefnstep{}
\ottusedrule{\ottdruleSXXMu{}}
\ottdefnexpr{}
\ottusedrule{\ottdruleTXXMu{}}

\section{Specification of source language}
\subsection{Syntax}
See Figure \ref{fig:appendix:syntax}.
\begin{figure*}[ht]
\centering
\gram{\ottpgm\ottinterrule
\ottdecl\ottinterrule
\ottu\ottinterrule
\ottp\ottinterrule
\otts\ottinterrule
\ottE\ottinterrule
\ottGs}
    \[
    \begin{array}{lllll}
    \text{Syntactic Sugar} &&&& \\
    \quad\text{Let binding} & \kw{let} \, \ottmv{x}  \ottsym{:}  T  \ottsym{=}  \ottnt{E_{{\mathrm{2}}}} \, \kw{in} \, \ottnt{E_{{\mathrm{1}}}} & \triangleq & \ottsym{(}  \lambda  \ottmv{x}  \ottsym{:}  T  \ottsym{.}  \ottnt{E_{{\mathrm{1}}}}  \ottsym{)} \, \ottnt{E_{{\mathrm{2}}}} & \\
    \quad\text{Function type} & T_{{\mathrm{1}}}  \rightarrow  T_{{\mathrm{2}}} & \triangleq & \Pi \, \ottmv{x}  \ottsym{:}  T_{{\mathrm{1}}}  \ottsym{.}  T_{{\mathrm{2}}} & x \not \in \FV(T_{{\mathrm{2}}}) \\
    \quad\text{Dependent function type} & \ottsym{(}  \ottmv{x}  \ottsym{:}  T_{{\mathrm{1}}}  \ottsym{)}  \rightarrow  T_{{\mathrm{2}}} & \triangleq & \Pi \, \ottmv{x}  \ottsym{:}  T_{{\mathrm{1}}}  \ottsym{.}  T_{{\mathrm{2}}} & x \in \FV(T_{{\mathrm{2}}})\\
    \quad\text{Record} & \kw{data} \, \ottmv{R} \, \ottcomp{\ottnt{u_{\ottmv{l}}}  \ottsym{:}  \kappa_{\ottmv{l}}}{\ottmv{l}} \, \ottsym{=}  \ottmv{K}  \ottsym{\{} \, \ottcomp{\ottmv{N_{\ottmv{i}}}  \ottsym{:}  T_{\ottmv{i}}}{\ottmv{i}} \, \ottsym{\}} & \triangleq &
                    \kw{data} \, \ottmv{R} \, \ottcomp{\ottnt{u_{\ottmv{l}}}  \ottsym{:}  \kappa_{\ottmv{l}}}{\ottmv{l}} \, \ottsym{=}  \ottmv{K} \, \ottcomp{T_{\ottmv{i}}}{\ottmv{i}}; & \\
                  &&&  \kw{let} ~\ottmv{N}_i : \ottsym{(} \, \ottcomp{\ottnt{u_{\ottmv{j}}}  \ottsym{:}  \kappa_{\ottmv{j}}}{\ottmv{j}} \, \ottsym{)}  \rightarrow  \ottmv{R} \, \ottcomp{\ottnt{u_{\ottmv{l}}}}{\ottmv{l}} \, \rightarrow  T_i = & \forall \ottmv{N}_i \\
                  &&& \lambda \, \ottcomp{\ottnt{u_{\ottmv{j}}}  \ottsym{:}  \kappa_{\ottmv{j}}}{\ottmv{j}} . \lambda  \ottmv{y}  \ottsym{:}  \ottmv{R} \, \ottcomp{\ottnt{u_{\ottmv{l}}}}{\ottmv{l}} . \kw{case} \, \ottmv{y} \, \kw{of} \, \ottmv{K} \, \ottcomp{\ottmv{x_{\ottmv{i}}}  \ottsym{:}  T_{\ottmv{i}}}{\ottmv{i}}  \Rightarrow  \ottmv{x}_i~ \kw{in}  & \\
    \end{array}
    \]
\caption{Syntax of source language}
\label{fig:appendix:syntax}
\end{figure*}

\subsection{Expression typing}
\ottdefnctxsrc{}
\ottdefnexprsrc{}

\section{Proofs about core language}
\subsection{Properties}
\begin{lem}[Free variable lemma]\label{lem:free}
    If $\Gamma  \vdash  \ottnt{e}  \ottsym{:}  \tau$, then $\FV(e) \subseteq \dom(\Gamma)$ and $\FV(\tau) \subseteq \dom(\Gamma)$.
\end{lem}

\begin{proof}
    By induction on the derivation of $\Gamma  \vdash  \ottnt{e}  \ottsym{:}  \tau$. We only treat cases \ruleref{T\_Mu}, \ruleref{T\_CastUp} and \ruleref{T\_CastDown} (since proofs of other cases are the same as \cc \cite{handbook}):
    \begin{description}
        \item[Case \ruleref{T\_Mu}:] From premises of $\Gamma  \vdash  \ottsym{(}  \mu \, \ottmv{x}  \ottsym{:}  \tau  \ottsym{.}  \ottnt{e_{{\mathrm{1}}}}  \ottsym{)}  \ottsym{:}  \tau$, by induction hypothesis, we have $\FV(e_1) \subseteq \dom(\Gamma) \cup \{x\}$ and $\FV(\tau) \subseteq \dom(\Gamma)$. Thus the result follows by $\FV(\mu \, \ottmv{x}  \ottsym{:}  \tau  \ottsym{.}  \ottnt{e_{{\mathrm{1}}}})=\FV(e_1) \setminus \{x\} \subseteq \dom(\Gamma)$ and $\FV(\tau) \subseteq \dom(\Gamma)$.
        \item[Case \ruleref{T\_CastUp}:] Since $\FV(\kw{cast}^{\uparrow} \, \ottsym{[}  \tau  \ottsym{]}  \ottnt{e_{{\mathrm{1}}}})=\FV(\ottnt{e_{{\mathrm{1}}}})$, the result follows directly by the induction hypothesis.
        \item[Case \ruleref{T\_CastDown}:] Since $\FV(\kw{cast}_{\downarrow} \, \ottnt{e_{{\mathrm{1}}}})=\FV(\ottnt{e_{{\mathrm{1}}}})$, the result follows directly by the induction hypothesis.
    \end{description}
\end{proof}

\begin{lem}[Substitution lemma]\label{lem:subst}
	If $\Gamma_{{\mathrm{1}}}  \ottsym{,}  \ottmv{x}  \ottsym{:}  \sigma  \ottsym{,}  \Gamma_{{\mathrm{2}}}  \vdash  \ottnt{e_{{\mathrm{1}}}}  \ottsym{:}  \tau$ and $\Gamma_{{\mathrm{1}}}  \vdash  \ottnt{e_{{\mathrm{2}}}}  \ottsym{:}  \sigma$, then $\Gamma_{{\mathrm{1}}}  \ottsym{,}  \Gamma_{{\mathrm{2}}}  \ottsym{[}  \ottmv{x}  \mapsto  \ottnt{e_{{\mathrm{2}}}}  \ottsym{]}  \vdash  \ottnt{e_{{\mathrm{1}}}}  \ottsym{[}  \ottmv{x}  \mapsto  \ottnt{e_{{\mathrm{2}}}}  \ottsym{]}  \ottsym{:}  \tau  \ottsym{[}  \ottmv{x}  \mapsto  \ottnt{e_{{\mathrm{2}}}}  \ottsym{]}$.
\end{lem}

\begin{proof}
    By induction on the derivation of $\Gamma_{{\mathrm{1}}}  \ottsym{,}  \ottmv{x}  \ottsym{:}  \sigma  \ottsym{,}  \Gamma_{{\mathrm{2}}}  \vdash  \ottnt{e_{{\mathrm{1}}}}  \ottsym{:}  \tau$. Let $\ottnt{e}  ^{*}  \equiv  \ottnt{e}  \ottsym{[}  \ottmv{x}  \mapsto  \ottnt{e_{{\mathrm{2}}}}  \ottsym{]}$. Then the result can be written as $\Gamma_{{\mathrm{1}}}  \ottsym{,}  \Gamma_{{\mathrm{2}}}  ^{*}  \vdash  \ottnt{e_{{\mathrm{1}}}}  ^{*}  \ottsym{:}  \tau  ^{*}$. We only treat cases \ruleref{T\_Mu}, \ruleref{T\_CastUp} and \ruleref{T\_CastDown}. Consider the last step of derivation of the following cases:
    \begin{description}
        \item[Case \ruleref{T\_Mu}:] $\inferrule{\Gamma_{{\mathrm{1}}}  \ottsym{,}  \ottmv{x}  \ottsym{:}  \sigma  \ottsym{,}  \Gamma_{{\mathrm{2}}}  \vdash  \ottnt{e_{{\mathrm{1}}}}  \ottsym{:}  \tau \\ \Gamma_{{\mathrm{1}}}  \ottsym{,}  \ottmv{x}  \ottsym{:}  \sigma  \ottsym{,}  \Gamma_{{\mathrm{2}}}  \vdash  \tau  \ottsym{:}  \ottnt{s}}{\Gamma_{{\mathrm{1}}}  \ottsym{,}  \ottmv{x}  \ottsym{:}  \sigma  \ottsym{,}  \Gamma_{{\mathrm{2}}}  \vdash  \ottsym{(}  \mu \, \ottmv{y}  \ottsym{:}  \tau  \ottsym{.}  \ottnt{e_{{\mathrm{1}}}}  \ottsym{)}  \ottsym{:}  \tau}$ 
        
        By induction hypothesis, we have $\Gamma_{{\mathrm{1}}}  \ottsym{,}  \Gamma_{{\mathrm{2}}}  ^{*}  \vdash  \ottnt{e_{{\mathrm{1}}}}  ^{*}  \ottsym{:}  \tau  ^{*}$ and $\Gamma_{{\mathrm{1}}}  \ottsym{,}  \Gamma_{{\mathrm{2}}}  ^{*}  \vdash  \tau  ^{*}  \ottsym{:}  \ottnt{s}  ^{*}$. Then by the deviation rule, $\Gamma_{{\mathrm{1}}}  \ottsym{,}  \Gamma_{{\mathrm{2}}}  ^{*}  \vdash  \ottsym{(}  \mu \, \ottmv{y}  \ottsym{:}  \tau  ^{*}  \ottsym{.}  \ottnt{e_{{\mathrm{1}}}}  ^{*}  \ottsym{)}  \ottsym{:}  \tau  ^{*}$. Thus we have $\Gamma_{{\mathrm{1}}}  \ottsym{,}  \Gamma_{{\mathrm{2}}}  ^{*}  \vdash  \ottsym{(}  \mu \, \ottmv{y}  \ottsym{:}  \tau  \ottsym{.}  \ottnt{e_{{\mathrm{1}}}}  \ottsym{)}  ^{*}  \ottsym{:}  \tau  ^{*}$ which is just the result.
        \item[Case \ruleref{T\_CastUp}:] $\inferrule{\Gamma_{{\mathrm{1}}}  \ottsym{,}  \ottmv{x}  \ottsym{:}  \sigma  \ottsym{,}  \Gamma_{{\mathrm{2}}}  \vdash  \ottnt{e_{{\mathrm{1}}}}  \ottsym{:}  \tau_{{\mathrm{2}}} \\ \Gamma_{{\mathrm{1}}}  \ottsym{,}  \ottmv{x}  \ottsym{:}  \sigma  \ottsym{,}  \Gamma_{{\mathrm{2}}}  \vdash  \tau_{{\mathrm{1}}}  \ottsym{:}  \ottnt{s} \\ \tau_{{\mathrm{1}}}  \longrightarrow  \tau_{{\mathrm{2}}}}{\Gamma_{{\mathrm{1}}}  \ottsym{,}  \ottmv{x}  \ottsym{:}  \sigma  \ottsym{,}  \Gamma_{{\mathrm{2}}}  \vdash  \ottsym{(}  \kw{cast}^{\uparrow} \, \ottsym{[}  \tau_{{\mathrm{1}}}  \ottsym{]}  \ottnt{e_{{\mathrm{1}}}}  \ottsym{)}  \ottsym{:}  \tau_{{\mathrm{1}}}}$ 
        
        By induction hypothesis, we have $\Gamma_{{\mathrm{1}}}  \ottsym{,}  \Gamma_{{\mathrm{2}}}  ^{*}  \vdash  \ottnt{e_{{\mathrm{1}}}}  ^{*}  \ottsym{:}  \tau_{{\mathrm{2}}}  ^{*}$, $\Gamma_{{\mathrm{1}}}  \ottsym{,}  \Gamma_{{\mathrm{2}}}  ^{*}  \vdash  \tau_{{\mathrm{1}}}  ^{*}  \ottsym{:}  \ottnt{s}  ^{*}$ and $\tau_{{\mathrm{1}}}  \longrightarrow  \tau_{{\mathrm{2}}}$. By the definition of substitution, we can obtain $\tau_{{\mathrm{1}}}  ^{*}  \longrightarrow  \tau_{{\mathrm{2}}}  ^{*}$ by $\tau_{{\mathrm{1}}}  \longrightarrow  \tau_{{\mathrm{2}}}$. Then by the deviation rule, $\Gamma_{{\mathrm{1}}}  \ottsym{,}  \Gamma_{{\mathrm{2}}}  ^{*}  \vdash  \ottsym{(}  \kw{cast}^{\uparrow} \, \ottsym{[}  \tau_{{\mathrm{1}}}  ^{*}  \ottsym{]}  \ottnt{e_{{\mathrm{1}}}}  ^{*}  \ottsym{)}  \ottsym{:}  \tau_{{\mathrm{1}}}  ^{*}$. Thus we have $\Gamma_{{\mathrm{1}}}  \ottsym{,}  \Gamma_{{\mathrm{2}}}  ^{*}  \vdash  \ottsym{(}  \kw{cast}^{\uparrow} \, \ottsym{[}  \tau_{{\mathrm{1}}}  \ottsym{]}  \ottnt{e_{{\mathrm{1}}}}  \ottsym{)}  ^{*}  \ottsym{:}  \tau_{{\mathrm{1}}}  ^{*}$ which is just the result.
        \item[Case \ruleref{T\_CastDown}:] $\inferrule{\Gamma_{{\mathrm{1}}}  \ottsym{,}  \ottmv{x}  \ottsym{:}  \sigma  \ottsym{,}  \Gamma_{{\mathrm{2}}}  \vdash  \ottnt{e_{{\mathrm{1}}}}  \ottsym{:}  \tau_{{\mathrm{1}}} \\ \Gamma_{{\mathrm{1}}}  \ottsym{,}  \ottmv{x}  \ottsym{:}  \sigma  \ottsym{,}  \Gamma_{{\mathrm{2}}}  \vdash  \tau_{{\mathrm{2}}}  \ottsym{:}  \ottnt{s} \\ \tau_{{\mathrm{1}}}  \longrightarrow  \tau_{{\mathrm{2}}}}{\Gamma_{{\mathrm{1}}}  \ottsym{,}  \ottmv{x}  \ottsym{:}  \sigma  \ottsym{,}  \Gamma_{{\mathrm{2}}}  \vdash  \ottsym{(}  \kw{cast}_{\downarrow} \, \ottnt{e_{{\mathrm{1}}}}  \ottsym{)}  \ottsym{:}  \tau_{{\mathrm{2}}}}$ 
        
        By induction hypothesis, we have $\Gamma_{{\mathrm{1}}}  \ottsym{,}  \Gamma_{{\mathrm{2}}}  ^{*}  \vdash  \ottnt{e_{{\mathrm{1}}}}  ^{*}  \ottsym{:}  \tau_{{\mathrm{1}}}  ^{*}$, $\Gamma_{{\mathrm{1}}}  \ottsym{,}  \Gamma_{{\mathrm{2}}}  ^{*}  \vdash  \tau_{{\mathrm{2}}}  ^{*}  \ottsym{:}  \ottnt{s}  ^{*}$ and $\tau_{{\mathrm{1}}}  \longrightarrow  \tau_{{\mathrm{2}}}$ thus $\tau_{{\mathrm{1}}}  ^{*}  \longrightarrow  \tau_{{\mathrm{2}}}  ^{*}$. Then by the deviation rule, $\Gamma_{{\mathrm{1}}}  \ottsym{,}  \Gamma_{{\mathrm{2}}}  ^{*}  \vdash  \ottsym{(}  \kw{cast}_{\downarrow} \, \ottnt{e_{{\mathrm{1}}}}  ^{*}  \ottsym{)}  \ottsym{:}  \tau_{{\mathrm{2}}}  ^{*}$. Thus we have $\Gamma_{{\mathrm{1}}}  \ottsym{,}  \Gamma_{{\mathrm{2}}}  ^{*}  \vdash  \ottsym{(}  \kw{cast}_{\downarrow} \, \ottnt{e_{{\mathrm{1}}}}  \ottsym{)}  ^{*}  \ottsym{:}  \tau_{{\mathrm{2}}}  ^{*}$ which is just the result.
    \end{description}
\end{proof}

\begin{lem}[Generation lemma]\label{lem:gen}
$\quad$
\begin{enumerate}[(1)]
	\item If $\Gamma  \vdash  \ottmv{x}  \ottsym{:}  \sigma$, then there exist an expression $\tau$ and a sort $s$ such that $\tau  \equiv  \sigma$, $\Gamma  \vdash  \tau  \ottsym{:}  \ottnt{s}$ and $\ottmv{x}  \ottsym{:}  \tau \, \in \, \Gamma$.
	\item If $\Gamma  \vdash  \ottnt{e_{{\mathrm{1}}}} \, \ottnt{e_{{\mathrm{2}}}}  \ottsym{:}  \sigma$, then there exist expressions $\tau_{{\mathrm{1}}}$ and $\tau_{{\mathrm{2}}}$ such that $\Gamma  \vdash  \ottnt{e_{{\mathrm{1}}}}  \ottsym{:}  \ottsym{(}  \Pi \, \ottmv{x}  \ottsym{:}  \tau_{{\mathrm{1}}}  \ottsym{.}  \tau_{{\mathrm{2}}}  \ottsym{)}$, $\Gamma  \vdash  \ottnt{e_{{\mathrm{2}}}}  \ottsym{:}  \tau_{{\mathrm{1}}}$ and $\sigma  \equiv  \tau_{{\mathrm{2}}}  \ottsym{[}  \ottmv{x}  \mapsto  \ottnt{e_{{\mathrm{2}}}}  \ottsym{]}$.
	\item If $\Gamma  \vdash  \ottsym{(}  \lambda  \ottmv{x}  \ottsym{:}  \tau_{{\mathrm{1}}}  \ottsym{.}  \ottnt{e}  \ottsym{)}  \ottsym{:}  \sigma$, then there exist a sort $s$ and an expression $\tau_{{\mathrm{2}}}$ such that $\sigma  \equiv  \Pi \, \ottmv{x}  \ottsym{:}  \tau_{{\mathrm{1}}}  \ottsym{.}  \tau_{{\mathrm{2}}}$ where $\Gamma  \vdash  \ottsym{(}  \Pi \, \ottmv{x}  \ottsym{:}  \tau_{{\mathrm{1}}}  \ottsym{.}  \tau_{{\mathrm{2}}}  \ottsym{)}  \ottsym{:}  \ottnt{s}$ and $\Gamma  \ottsym{,}  \ottmv{x}  \ottsym{:}  \tau_{{\mathrm{1}}}  \vdash  \ottnt{e}  \ottsym{:}  \tau_{{\mathrm{2}}}$.
	\item If $\Gamma  \vdash  \ottsym{(}  \Pi \, \ottmv{x}  \ottsym{:}  \tau_{{\mathrm{1}}}  \ottsym{.}  \tau_{{\mathrm{2}}}  \ottsym{)}  \ottsym{:}  \sigma$, then there exist sorts $\ottnt{s_{{\mathrm{1}}}}$ and $\ottnt{s_{{\mathrm{2}}}}$ such that $\sigma  \equiv  \ottnt{s_{{\mathrm{2}}}}$, $\Gamma  \vdash  \tau_{{\mathrm{1}}}  \ottsym{:}  \ottnt{s_{{\mathrm{1}}}}$ and $\Gamma  \ottsym{,}  \ottmv{x}  \ottsym{:}  \tau_{{\mathrm{1}}}  \vdash  \tau_{{\mathrm{2}}}  \ottsym{:}  \ottnt{s_{{\mathrm{2}}}}$.
	\item If $\Gamma  \vdash  \ottsym{(}  \mu \, \ottmv{x}  \ottsym{:}  \tau  \ottsym{.}  \ottnt{e}  \ottsym{)}  \ottsym{:}  \sigma$, then there exists a sort $\ottnt{s}$ such that $\Gamma  \vdash  \tau  \ottsym{:}  \ottnt{s}$, $\sigma  \equiv  \tau$ and $\Gamma  \ottsym{,}  \ottmv{x}  \ottsym{:}  \tau  \vdash  \ottnt{e}  \ottsym{:}  \tau$.
	\item If $\Gamma  \vdash  \ottsym{(}  \kw{cast}^{\uparrow} \, \ottsym{[}  \tau_{{\mathrm{1}}}  \ottsym{]}  \ottnt{e}  \ottsym{)}  \ottsym{:}  \sigma$, then there exist an expression $\tau_{{\mathrm{2}}}$ and a sort $s$ such that $\Gamma  \vdash  \ottnt{e}  \ottsym{:}  \tau_{{\mathrm{2}}}$, $\Gamma  \vdash  \tau_{{\mathrm{1}}}  \ottsym{:}  \ottnt{s}$, $\tau_{{\mathrm{1}}}  \longrightarrow  \tau_{{\mathrm{2}}}$ and $\sigma  \equiv  \tau_{{\mathrm{1}}}$.
	\item If $\Gamma  \vdash  \ottsym{(}  \kw{cast}_{\downarrow} \, \ottnt{e}  \ottsym{)}  \ottsym{:}  \sigma$, then there exist expressions $\tau_{{\mathrm{1}}},\tau_{{\mathrm{2}}}$ and a sort $s$ such that $\Gamma  \vdash  \ottnt{e}  \ottsym{:}  \tau_{{\mathrm{1}}}$, $\Gamma  \vdash  \tau_{{\mathrm{2}}}  \ottsym{:}  \ottnt{s}$, $\tau_{{\mathrm{1}}}  \longrightarrow  \tau_{{\mathrm{2}}}$ and $\sigma  \equiv  \tau_{{\mathrm{2}}}$.
\end{enumerate}
\end{lem}

\begin{proof}
    Consider a derivation of $\Gamma  \vdash  \ottnt{e}  \ottsym{:}  \sigma$ for one of cases in the lemma. Note that rule \ruleref{T\_Weak} does not change $e$, then we can follow the process of derivation until expression $e$ is introduced the first time. The last step of derivation can be done by
    \begin{itemize}
        \item rule \ruleref{T\_Var} for case 1;
        \item rule \ruleref{T\_App} for case 2;
        \item rule \ruleref{T\_Lam} for case 3;
        \item rule \ruleref{T\_Pi} for case 4;
        \item rule \ruleref{T\_Mu} for case 5;
        \item rule \ruleref{T\_CastUp} for case 6;
        \item rule \ruleref{T\_CastDown} for case 7.
    \end{itemize}
    In each case, assume the conclusion of the rule is $\Gamma'  \vdash  \ottnt{e}  \ottsym{:}  \tau'$ where $\Gamma' \subseteq \Gamma$ and $\tau'  \equiv  \sigma$. Then by inspection of used derivation rules, it can be shown that the statement of the lemma holds and is the only possible case.
\end{proof}

\begin{lem}[Correctness of types]\label{lem:corrtyp}
    If $\Gamma  \vdash  \ottnt{e}  \ottsym{:}  \tau$ then there exists a sort $s$ such that $\tau  \equiv  \ottnt{s}$ or $\Gamma  \vdash  \tau  \ottsym{:}  \ottnt{s}$.
\end{lem}

\begin{proof}
    Trivial induction on the derivation of $\Gamma  \vdash  \ottnt{e}  \ottsym{:}  \tau$ using Lemma \ref{lem:gen}.
\end{proof}

\begin{dfn}[Well-formed context]
	A \textbf{well-formed} context $\Gamma$ is defined by the following rules:
	
	\textnormal{\ottdefnctx{}}
\end{dfn}

\begin{lem}[Consistency of well-formed context]\label{lem:wfc}
	Given a well-formed initial context $\Gamma$, it remains well-formed through type checking.
\end{lem}

\begin{proof}
	Suppose $\Gamma$ is the initial context which is well-formed. To safely extend $\Gamma$ with a variable $x:\tau$, one should have $\Gamma  \vdash  \tau  \ottsym{:}  \ottnt{s}$ due to rule \ruleref{Env\_Var}. Note that when applying typing rules of $\Gamma  \vdash  \ottnt{e}  \ottsym{:}  \tau$, rule \ruleref{T\_Pi}, \ruleref{T\_Mu} and \ruleref{T\_Lam} will extend the context. We show that these rules cover the condition $\Gamma  \vdash  \tau  \ottsym{:}  \ottnt{s}$ with respect to $x:\tau$ as follows:
	\begin{description}
		\item[Case \ruleref{T\_Pi}:] To extend $\Gamma$ with $x:\tau_{{\mathrm{1}}}$, $\Gamma  \vdash  \tau_{{\mathrm{1}}}  \ottsym{:}  \ottnt{s}$ is already the premise of the rule.
		\item[Case \ruleref{T\_Mu}:] To extend $\Gamma$ with $x:\tau$, $\Gamma  \vdash  \tau  \ottsym{:}  \ottnt{s}$ is already the premise of the rule.
		\item[Case \ruleref{T\_Lam}:] To extend $\Gamma$ with $x:\tau_{{\mathrm{1}}}$, note that the premise $\Gamma  \vdash  \ottsym{(}  \Pi \, \ottmv{x}  \ottsym{:}  \tau_{{\mathrm{1}}}  \ottsym{.}  \tau_{{\mathrm{2}}}  \ottsym{)}  \ottsym{:}  \ottnt{s}$ can be derived from rule \ruleref{T\_Pi}, which has the premise $\Gamma  \vdash  \tau_{{\mathrm{1}}}  \ottsym{:}  \ottnt{s}$.
	\end{description}
\end{proof}

\begin{lem}[Valid context optimization]\label{lem:wfcopt}
	With a well-formed initial context $\Gamma$, the \ruleref{T\_Var} and \ruleref{T\_Weak} can be replaced by the following rule: \ottusedrule{\ottdruleTXXVarEnv{}}
\end{lem}

\begin{proof}
	By Lemma \ref{lem:wfc}, the context $\Gamma$ remains well-formed if it is initially well-formed. Thus, the well-formedness of $\Gamma$ keeps without checking by rule \ruleref{T\_Var} and \ruleref{T\_Weak}. By Lemma \ref{lem:gen}, if $\Gamma  \vdash  \ottmv{x}  \ottsym{:}  \tau$, then $\ottmv{x}  \ottsym{:}  \tau \, \in \, \Gamma$. Thus, in order to check the type of a variable $x$, it is sufficient to check its bound type $\tau$ in the context, which is simply rule \ruleref{T\_VarEnv}.
\end{proof}

\subsection{Decidability of type checking}
\begin{lem}[Uniqueness of one-step reduction]\label{lem:unired}
	The relation $ \longrightarrow $, i.e. one-step reduction, is \textbf{unique} in the sense that given $e$ there is at most one $e'$ such that $\ottnt{e}  \longrightarrow  \ottnt{e'}$.
\end{lem}

\begin{proof}
	By induction on the structure of $e$:
	\begin{description}
	    \item[Case $e=s$, or $e=x$]: No such $e'$ exists since it is impossible to reduce a sort or a variable.
		\item[Case $e=v$:] $e$ has one of the following forms:
		\begin{inparaenum}[(1)]
			\item $\lambda  \ottmv{x}  \ottsym{:}  \tau  \ottsym{.}  \ottnt{e}$,
			\item $\Pi \, \ottmv{x}  \ottsym{:}  \tau_{{\mathrm{1}}}  \ottsym{.}  \tau_{{\mathrm{2}}}$,
			\item $\kw{cast}^{\uparrow} \, \ottsym{[}  \tau  \ottsym{]}  \ottnt{e}$,
		\end{inparaenum}
		which cannot match any rules of $ \longrightarrow $. Thus there is no $e'$ such that $\ottnt{e}  \longrightarrow  \ottnt{e'}$.
		\item[Case $e=\ottsym{(}  \lambda  \ottmv{x}  \ottsym{:}  \tau  \ottsym{.}  \ottnt{e_{{\mathrm{1}}}}  \ottsym{)} \, \ottnt{e_{{\mathrm{2}}}}$:] There is a unique $e'=\ottnt{e_{{\mathrm{1}}}}  \ottsym{[}  \ottmv{x}  \mapsto  \ottnt{e_{{\mathrm{2}}}}  \ottsym{]}$ by rule \ruleref{S\_Beta}.
		\item[Case $e=\kw{cast}_{\downarrow} \, \ottsym{(}  \kw{cast}^{\uparrow} \, \ottsym{[}  \tau  \ottsym{]}  \ottnt{e}  \ottsym{)}$:] There is a unique $e'=e$ by rule \ruleref{S\_CastDownUp}.
		\item[Case $e=\mu \, \ottmv{x}  \ottsym{:}  \tau  \ottsym{.}  \ottnt{e}$:] There is a unique $e'=\ottnt{e}  \ottsym{[}  \ottmv{x}  \mapsto  \mu \, \ottmv{x}  \ottsym{:}  \tau  \ottsym{.}  \ottnt{e}  \ottsym{]}$ by rule \ruleref{S\_Mu}.
		\item[Case $e=\ottnt{e_{{\mathrm{1}}}} \, \ottnt{e_{{\mathrm{2}}}}$ and $\ottnt{e_{{\mathrm{1}}}}$ is not a $\lambda$-term:] If $\ottnt{e_{{\mathrm{1}}}}=v$, there is no $\ottnt{e'_{{\mathrm{1}}}}$ such that $\ottnt{e_{{\mathrm{1}}}}  \longrightarrow  \ottnt{e'_{{\mathrm{1}}}}$. Since $\ottnt{e_{{\mathrm{1}}}}$ is not a $\lambda$-term, there is no rule to reduce $e$. Thus there is no $e'$ such that $\ottnt{e}  \longrightarrow  \ottnt{e'}$.
		
		Otherwise, there exists some $\ottnt{e'_{{\mathrm{1}}}}$ such that $\ottnt{e_{{\mathrm{1}}}}  \longrightarrow  \ottnt{e'_{{\mathrm{1}}}}$. By the induction hypothesis, $\ottnt{e'_{{\mathrm{1}}}}$ is unique reduction of $\ottnt{e_{{\mathrm{1}}}}$. Thus by rule \ruleref{S\_App}, $e'=\ottnt{e'_{{\mathrm{1}}}} \, \ottnt{e_{{\mathrm{2}}}}$ is the unique reduction for $e$.
		\item[Case $e=\kw{cast}_{\downarrow} \, \ottnt{e_{{\mathrm{1}}}}$ and $\ottnt{e_{{\mathrm{1}}}}$ is not a $ \kw{cast}^{\uparrow} $-term:] If $\ottnt{e_{{\mathrm{1}}}}=v$, there is no $\ottnt{e'_{{\mathrm{1}}}}$ such that $\ottnt{e_{{\mathrm{1}}}}  \longrightarrow  \ottnt{e'_{{\mathrm{1}}}}$. Since $\ottnt{e_{{\mathrm{1}}}}$ is not a $ \kw{cast}^{\uparrow} $-term, there is no rule to reduce $e$. Thus there is no $e'$ such that $\ottnt{e}  \longrightarrow  \ottnt{e'}$.
		
		Otherwise, there exists some $\ottnt{e'_{{\mathrm{1}}}}$ such that $\ottnt{e_{{\mathrm{1}}}}  \longrightarrow  \ottnt{e'_{{\mathrm{1}}}}$. By the induction hypothesis, $\ottnt{e'_{{\mathrm{1}}}}$ is unique reduction of $\ottnt{e_{{\mathrm{1}}}}$. Thus by rule \ruleref{S\_CastDown}, $e'=\kw{cast}_{\downarrow} \, \ottnt{e'_{{\mathrm{1}}}}$ is the unique reduction for $e$.
	\end{description}
\end{proof}

\begin{lem}[Decidability of type checking]
	There is a decidable algorithm which given $\Gamma, \ottnt{e}$ computes the unique $\tau$ such that $\Gamma  \vdash  \ottnt{e}  \ottsym{:}  \tau$ or reports there is no such $\tau$.
\end{lem}

\begin{proof}
	By induction on the structure of $e$:
	\begin{description}
	    \item[Case $e= \Box $:] Impossible case and report error.
	    \item[Case $e= \star $:] Trivial by applying \ruleref{T\_Ax} and $\tau  \equiv  \Box$.
		\item[Case $e=x$:] By Lemma \ref{lem:wfcopt}, we only need to consider context $\Gamma$ that is well-formed. By rule \ruleref{TS\_Var}, if $\ottmv{x}  \ottsym{:}  \tau \, \in \, \Gamma$, $\tau$ is the unique type of $x$.
		\item[Case $e=\ottnt{e_{{\mathrm{1}}}} \, \ottnt{e_{{\mathrm{2}}}}$, or $\lambda  \ottmv{x}  \ottsym{:}  \tau_{{\mathrm{1}}}  \ottsym{.}  \ottnt{e_{{\mathrm{1}}}}$, or $\Pi \, \ottmv{x}  \ottsym{:}  \tau_{{\mathrm{1}}}  \ottsym{.}  \tau_{{\mathrm{2}}}$, or $\mu \, \ottmv{x}  \ottsym{:}  \tau  \ottsym{.}  \ottnt{e_{{\mathrm{1}}}}$:] Trivial according to Lemma \ref{lem:gen} by using rule \ruleref{T\_App}, \ruleref{T\_Lam}, \ruleref{T\_Pi}, or \ruleref{T\_Mu} respectively.
		\item[Case $e=\kw{cast}^{\uparrow} \, \ottsym{[}  \tau_{{\mathrm{1}}}  \ottsym{]}  \ottnt{e_{{\mathrm{1}}}}$:] From the premises of rule \ruleref{T\_CastUp}, by induction hypothesis, we can derive the type of $\ottnt{e_{{\mathrm{1}}}}$ as $\tau_{{\mathrm{2}}}$, and check whether $\tau_{{\mathrm{1}}}$ is legal, i.e. its sorts is either $ \star $ or $ \Box $. If $\tau_{{\mathrm{1}}}$ is legal, by Lemma \ref{lem:unired}, there is at most one $\tau'_{{\mathrm{1}}}$ such that $\tau_{{\mathrm{1}}}  \longrightarrow  \tau'_{{\mathrm{1}}}$. If such $\tau'_{{\mathrm{1}}}$ does not exist, then we report the type checking is failed. Otherwise, we examine if $\tau'_{{\mathrm{1}}}$ is syntactically equal to $\tau_{{\mathrm{2}}}$, i.e. $\tau'_{{\mathrm{1}}}  \equiv  \tau_{{\mathrm{2}}}$. If the equality holds, we obtain the unique type of $\ottnt{e}$ which is $\tau_{{\mathrm{1}}}$. Otherwise, we report $\ottnt{e}$ fails to type check.
		\item[Case $e=\kw{cast}_{\downarrow} \, \ottnt{e_{{\mathrm{1}}}}$:] From the premises of rule \ruleref{T\_CastDown}, by induction hypothesis, we can derive the type of $\ottnt{e_{{\mathrm{1}}}}$ as $\tau_{{\mathrm{1}}}$. By Lemma \ref{lem:unired}, there is at most one $\tau_{{\mathrm{2}}}$ such that $\tau_{{\mathrm{1}}}  \longrightarrow  \tau_{{\mathrm{2}}}$. If such $\tau_{{\mathrm{2}}}$ exists and its sorts is either $ \star $ or $ \Box $, we have found the unique type of $\ottnt{e}$ is $\tau_{{\mathrm{2}}}$. Otherwise, we report $\ottnt{e}$ fails to type check.
	\end{description}
\end{proof}

\subsection{Soundness}
\begin{dfn}[Multi-step reduction]
    The relation $ \twoheadrightarrow $ is the transitive and reflexive closure of $ \longrightarrow $.
\end{dfn}

\begin{lem}[Subject reduction]
If $\Gamma  \vdash  \ottnt{e}  \ottsym{:}  \sigma$ and $e  \twoheadrightarrow  e'$ then $\Gamma  \vdash  \ottnt{e'}  \ottsym{:}  \sigma$.
\end{lem}

\begin{proof}
    We prove the case for one-step reduction, i.e. $\ottnt{e}  \longrightarrow  \ottnt{e'}$. The lemma can follow by induction on the number of one-step reductions of $e  \twoheadrightarrow  e'$.
    The proof is by induction with respect to the definition of one-step reduction $ \longrightarrow $ as follows:
    \begin{description}
        \item[Case $\ottdruleSXXBeta{}$:] $\quad$ \\
        Suppose $\Gamma  \vdash  \ottsym{(}  \lambda  \ottmv{x}  \ottsym{:}  \tau_{{\mathrm{1}}}  \ottsym{.}  \ottnt{e_{{\mathrm{1}}}}  \ottsym{)} \, \ottnt{e_{{\mathrm{2}}}}  \ottsym{:}  \sigma$ and $\Gamma  \vdash  \ottnt{e_{{\mathrm{1}}}}  \ottsym{[}  \ottmv{x}  \mapsto  \ottnt{e_{{\mathrm{2}}}}  \ottsym{]}  \ottsym{:}  \sigma'$. By Lemma \ref{lem:gen}(2), there exist expressions $\tau'_{{\mathrm{1}}}$ and $\tau_{{\mathrm{2}}}$ such that 
        \begin{align}
            &\Gamma  \vdash  \ottsym{(}  \lambda  \ottmv{x}  \ottsym{:}  \tau_{{\mathrm{1}}}  \ottsym{.}  \ottnt{e_{{\mathrm{1}}}}  \ottsym{)}  \ottsym{:}  \ottsym{(}  \Pi \, \ottmv{x}  \ottsym{:}  \tau'_{{\mathrm{1}}}  \ottsym{.}  \tau_{{\mathrm{2}}}  \ottsym{)} \label{equ:lam} \\
            &\Gamma  \vdash  \ottnt{e_{{\mathrm{2}}}}  \ottsym{:}  \tau'_{{\mathrm{1}}} \nonumber \\
            &\sigma  \equiv  \tau_{{\mathrm{2}}}  \ottsym{[}  \ottmv{x}  \mapsto  \ottnt{e_{{\mathrm{2}}}}  \ottsym{]} \nonumber
        \end{align}
        By Lemma \ref{lem:gen}(3), the judgement (\ref{equ:lam}) implies that there exists an expression $\tau'_{{\mathrm{2}}}$ such that
        \begin{align}
            &\Pi \, \ottmv{x}  \ottsym{:}  \tau'_{{\mathrm{1}}}  \ottsym{.}  \tau_{{\mathrm{2}}}  \equiv  \Pi \, \ottmv{x}  \ottsym{:}  \tau_{{\mathrm{1}}}  \ottsym{.}  \tau'_{{\mathrm{2}}} \label{equ:lameq}\\
            &\Gamma  \ottsym{,}  \ottmv{x}  \ottsym{:}  \tau_{{\mathrm{1}}}  \vdash  \ottnt{e_{{\mathrm{1}}}}  \ottsym{:}  \tau'_{{\mathrm{2}}} \nonumber
        \end{align}
        Hence, by (\ref{equ:lameq}) we have $\tau_{{\mathrm{1}}}  \equiv  \tau'_{{\mathrm{1}}}$ and $\tau_{{\mathrm{2}}}  \equiv  \tau'_{{\mathrm{2}}}$. Then we can obtain $\Gamma  \ottsym{,}  \ottmv{x}  \ottsym{:}  \tau_{{\mathrm{1}}}  \vdash  \ottnt{e_{{\mathrm{1}}}}  \ottsym{:}  \tau_{{\mathrm{2}}}$ and $\Gamma  \vdash  \ottnt{e_{{\mathrm{2}}}}  \ottsym{:}  \tau_{{\mathrm{1}}}$. By Lemma \ref{lem:subst}, we have $\Gamma  \vdash  \ottnt{e_{{\mathrm{1}}}}  \ottsym{[}  \ottmv{x}  \mapsto  \ottnt{e_{{\mathrm{2}}}}  \ottsym{]}  \ottsym{:}  \tau_{{\mathrm{2}}}  \ottsym{[}  \ottmv{x}  \mapsto  \ottnt{e_{{\mathrm{2}}}}  \ottsym{]}$. Therefore, we conclude with $\sigma'  \equiv  \tau_{{\mathrm{2}}}  \ottsym{[}  \ottmv{x}  \mapsto  \ottnt{e_{{\mathrm{2}}}}  \ottsym{]}  \equiv  \sigma$.
        
        \item[Case $\ottdruleSXXApp{}$:] $\quad$ \\
        Suppose $\Gamma  \vdash  \ottnt{e_{{\mathrm{1}}}} \, \ottnt{e_{{\mathrm{2}}}}  \ottsym{:}  \sigma$ and $\Gamma  \vdash  \ottnt{e'_{{\mathrm{1}}}} \, \ottnt{e_{{\mathrm{2}}}}  \ottsym{:}  \sigma'$. By Lemma \ref{lem:gen}(2), there exist expressions $\tau_{{\mathrm{1}}}$ and $\tau_{{\mathrm{2}}}$ such that 
        \begin{align*}
            &\Gamma  \vdash  \ottnt{e_{{\mathrm{1}}}}  \ottsym{:}  \ottsym{(}  \Pi \, \ottmv{x}  \ottsym{:}  \tau_{{\mathrm{1}}}  \ottsym{.}  \tau_{{\mathrm{2}}}  \ottsym{)} \\
            &\Gamma  \vdash  \ottnt{e_{{\mathrm{2}}}}  \ottsym{:}  \tau_{{\mathrm{1}}}\\
            &\sigma  \equiv  \tau_{{\mathrm{2}}}  \ottsym{[}  \ottmv{x}  \mapsto  \ottnt{e_{{\mathrm{2}}}}  \ottsym{]}
        \end{align*}
        By induction hypothesis, we have $\Gamma  \vdash  \ottnt{e'_{{\mathrm{1}}}}  \ottsym{:}  \ottsym{(}  \Pi \, \ottmv{x}  \ottsym{:}  \tau_{{\mathrm{1}}}  \ottsym{.}  \tau_{{\mathrm{2}}}  \ottsym{)}$. By rule \ruleref{T\_App}, we obtain $\Gamma  \vdash  \ottnt{e'_{{\mathrm{1}}}} \, \ottnt{e_{{\mathrm{2}}}}  \ottsym{:}  \tau_{{\mathrm{2}}}  \ottsym{[}  \ottmv{x}  \mapsto  \ottnt{e_{{\mathrm{2}}}}  \ottsym{]}$. Therefore, $\sigma'  \equiv  \tau_{{\mathrm{2}}}  \ottsym{[}  \ottmv{x}  \mapsto  \ottnt{e_{{\mathrm{2}}}}  \ottsym{]}  \equiv  \sigma$.
        
        \item[Case $\ottdruleSXXCastDown{}$:] $\quad$ \\
        Suppose $\Gamma  \vdash  \kw{cast}_{\downarrow} \, \ottnt{e}  \ottsym{:}  \sigma$ and $\Gamma  \vdash  \kw{cast}_{\downarrow} \, \ottnt{e'}  \ottsym{:}  \sigma'$. By Lemma \ref{lem:gen}(7), there exist expressions $\tau_{{\mathrm{1}}}, \tau_{{\mathrm{2}}}$ and a sort $s$ such that 
        \begin{align*}
            &\Gamma  \vdash  \ottnt{e}  \ottsym{:}  \tau_{{\mathrm{1}}} \qquad \Gamma  \vdash  \tau_{{\mathrm{2}}}  \ottsym{:}  \ottnt{s} \\
            &\tau_{{\mathrm{1}}}  \longrightarrow  \tau_{{\mathrm{2}}} \qquad \sigma  \equiv  \tau_{{\mathrm{2}}}
        \end{align*}
        By induction hypothesis, we have $\Gamma  \vdash  \ottnt{e'}  \ottsym{:}  \tau_{{\mathrm{1}}}$. By rule \ruleref{T\_CastDown}, we obtain $\Gamma  \vdash  \kw{cast}_{\downarrow} \, \ottnt{e'}  \ottsym{:}  \tau_{{\mathrm{2}}}$. Therefore, $\sigma'  \equiv  \tau_{{\mathrm{2}}}  \equiv  \sigma$.
        
        \item[Case $\ottdruleSXXCastDownUp{}$:] $\quad$ \\
        Suppose $\Gamma  \vdash  \kw{cast}_{\downarrow} \, \ottsym{(}  \kw{cast}^{\uparrow} \, \ottsym{[}  \tau_{{\mathrm{1}}}  \ottsym{]}  \ottnt{e}  \ottsym{)}  \ottsym{:}  \sigma$ and $\Gamma  \vdash  \ottnt{e}  \ottsym{:}  \sigma'$. By Lemma \ref{lem:gen}(7), there exist expressions $\tau'_{{\mathrm{1}}}, \tau_{{\mathrm{2}}}$ such that 
        \begin{align}
            &\Gamma  \vdash  \ottsym{(}  \kw{cast}^{\uparrow} \, \ottsym{[}  \tau_{{\mathrm{1}}}  \ottsym{]}  \ottnt{e}  \ottsym{)}  \ottsym{:}  \tau'_{{\mathrm{1}}} \label{equ:fold} \\
            &\tau'_{{\mathrm{1}}}  \longrightarrow  \tau_{{\mathrm{2}}} \label{equ:foldeq1} \\
            &\sigma  \equiv  \tau_{{\mathrm{2}}} \label{equ:foldeq4}
        \end{align}
        By Lemma \ref{lem:gen}(6), the judgement (\ref{equ:fold}) implies that there exists an expression $\tau'_{{\mathrm{2}}}$ such that
        \begin{align}
            &\Gamma  \vdash  \ottnt{e}  \ottsym{:}  \tau'_{{\mathrm{2}}} \label{equ:foldr} \\
            &\tau_{{\mathrm{1}}}  \longrightarrow  \tau'_{{\mathrm{2}}} \label{equ:foldeq2} \\
            &\tau'_{{\mathrm{1}}}  \equiv  \tau_{{\mathrm{1}}} \label{equ:foldeq3}
        \end{align}
        By (\ref{equ:foldeq1}, \ref{equ:foldeq2}, \ref{equ:foldeq3}) and Lemma \ref{lem:unired} we obtain $\tau_{{\mathrm{2}}}  \equiv  \tau'_{{\mathrm{2}}}$. From (\ref{equ:foldr}) we have $\sigma'  \equiv  \tau'_{{\mathrm{2}}}$. Therefore, by (\ref{equ:foldeq4}), $\sigma'  \equiv  \tau'_{{\mathrm{2}}}  \equiv  \tau_{{\mathrm{2}}}  \equiv  \sigma$.
        
        \item[Case $\ottdruleSXXMu{}$:] $\quad$ \\
        Suppose $\Gamma  \vdash  \ottsym{(}  \mu \, \ottmv{x}  \ottsym{:}  \tau  \ottsym{.}  \ottnt{e}  \ottsym{)}  \ottsym{:}  \sigma$ and $\Gamma  \vdash  \ottnt{e}  \ottsym{[}  \ottmv{x}  \mapsto  \mu \, \ottmv{x}  \ottsym{:}  \tau  \ottsym{.}  \ottnt{e}  \ottsym{]}  \ottsym{:}  \sigma'$. By Lemma \ref{lem:gen}(5), we have $\sigma  \equiv  \tau$ and $\Gamma  \ottsym{,}  \ottmv{x}  \ottsym{:}  \tau  \vdash  \ottnt{e}  \ottsym{:}  \tau$. Then we obtain $\Gamma  \vdash  \ottsym{(}  \mu \, \ottmv{x}  \ottsym{:}  \tau  \ottsym{.}  \ottnt{e}  \ottsym{)}  \ottsym{:}  \tau$. Thus by Lemma \ref{lem:subst}, we have $\Gamma  \vdash  \ottnt{e}  \ottsym{[}  \ottmv{x}  \mapsto  \mu \, \ottmv{x}  \ottsym{:}  \tau  \ottsym{.}  \ottnt{e}  \ottsym{]}  \ottsym{:}  \tau  \ottsym{[}  \ottmv{x}  \mapsto  \mu \, \ottmv{x}  \ottsym{:}  \tau  \ottsym{.}  \ottnt{e}  \ottsym{]}$.
        
        Note that $x:\tau$, i.e. the type of $x$ is $\tau$, then $x \notin \FV(\tau)$ holds implicitly. Hence, by the definition of substitution, we obtain $\tau  \ottsym{[}  \ottmv{x}  \mapsto  \mu \, \ottmv{x}  \ottsym{:}  \tau  \ottsym{.}  \ottnt{e}  \ottsym{]}  \equiv  \tau$. Therefore, $\sigma'  \equiv  \tau  \ottsym{[}  \ottmv{x}  \mapsto  \mu \, \ottmv{x}  \ottsym{:}  \tau  \ottsym{.}  \ottnt{e}  \ottsym{]}  \equiv  \tau  \equiv  \sigma$.
    \end{description}
\end{proof}

\begin{lem}[Progress]
If $\vdash  \ottnt{e}  \ottsym{:}  \sigma$ then either $e$ is a value $v$ or there exists $e'$ such that $\ottnt{e}  \longrightarrow  \ottnt{e'}$.
\end{lem}

\begin{proof}
    By induction on the derivation of $\vdash  \ottnt{e}  \ottsym{:}  \sigma$ as follows:
    \begin{description}
        \item[Case $e= \star $:] Trivial by rule \ruleref{T\_Ax} where $\sigma  \equiv  \Box$.
        \item[Case $e=x$:] Impossible, since the context is empty.
        \item[Case $e=v$:] Trivial, since $e$ is already a value that has one of the following forms:
		\begin{inparaenum}[(1)]
			\item $\lambda  \ottmv{x}  \ottsym{:}  \tau  \ottsym{.}  \ottnt{e}$,
			\item $\Pi \, \ottmv{x}  \ottsym{:}  \tau_{{\mathrm{1}}}  \ottsym{.}  \tau_{{\mathrm{2}}}$,
			\item $\kw{cast}^{\uparrow} \, \ottsym{[}  \tau  \ottsym{]}  \ottnt{e}$.
		\end{inparaenum}
		\item[Case $e=\ottnt{e_{{\mathrm{1}}}} \, \ottnt{e_{{\mathrm{2}}}}$:] By Lemma \ref{lem:gen}(2), there exist expressions $\tau_{{\mathrm{1}}}$ and $\tau_{{\mathrm{2}}}$ such that $\vdash  \ottnt{e_{{\mathrm{1}}}}  \ottsym{:}  \ottsym{(}  \Pi \, \ottmv{x}  \ottsym{:}  \tau_{{\mathrm{1}}}  \ottsym{.}  \tau_{{\mathrm{2}}}  \ottsym{)}$ and $\vdash  \ottnt{e_{{\mathrm{2}}}}  \ottsym{:}  \tau_{{\mathrm{1}}}$. Consider whether $\ottnt{e_{{\mathrm{1}}}}$ is a value:
    		\begin{itemize}
    		    \item If $\ottnt{e_{{\mathrm{1}}}}=v$, by Lemma \ref{lem:gen}(3), it must be a $\lambda$-term such that $\ottnt{e_{{\mathrm{1}}}}  \equiv  \lambda  \ottmv{x}  \ottsym{:}  \tau_{{\mathrm{1}}}  \ottsym{.}  \ottnt{e'_{{\mathrm{1}}}}$ for some $\ottnt{e'_{{\mathrm{1}}}}$ satisfying $\vdash  \ottnt{e'_{{\mathrm{1}}}}  \ottsym{:}  \tau_{{\mathrm{2}}}$. Then by rule \ruleref{S\_Beta}, we have $\ottsym{(}  \lambda  \ottmv{x}  \ottsym{:}  \tau_{{\mathrm{1}}}  \ottsym{.}  \ottnt{e'_{{\mathrm{1}}}}  \ottsym{)} \, \ottnt{e_{{\mathrm{2}}}}  \longrightarrow  \ottnt{e'_{{\mathrm{1}}}}  \ottsym{[}  \ottmv{x}  \mapsto  \ottnt{e_{{\mathrm{2}}}}  \ottsym{]}$. Thus, there exists $\ottnt{e'}  \equiv  \ottnt{e'_{{\mathrm{1}}}}  \ottsym{[}  \ottmv{x}  \mapsto  \ottnt{e_{{\mathrm{2}}}}  \ottsym{]}$ such that $\ottnt{e}  \longrightarrow  \ottnt{e'}$.
    		    \item Otherwise, by induction hypothesis, there exists $\ottnt{e'_{{\mathrm{1}}}}$ such that $\ottnt{e_{{\mathrm{1}}}}  \longrightarrow  \ottnt{e'_{{\mathrm{1}}}}$. Then by rule \ruleref{S\_App}, we have $\ottnt{e_{{\mathrm{1}}}} \, \ottnt{e_{{\mathrm{2}}}}  \longrightarrow  \ottnt{e'_{{\mathrm{1}}}} \, \ottnt{e_{{\mathrm{2}}}}$. Thus, there exists $\ottnt{e'}  \equiv  \ottnt{e'_{{\mathrm{1}}}} \, \ottnt{e_{{\mathrm{2}}}}$ such that $\ottnt{e}  \longrightarrow  \ottnt{e'}$.
    		\end{itemize}
		\item[Case $e=\kw{cast}_{\downarrow} \, \ottnt{e_{{\mathrm{1}}}}$:] By Lemma \ref{lem:gen}(7), there exist expressions $\tau_{{\mathrm{1}}}$ and $\tau_{{\mathrm{2}}}$ such that $\vdash  \ottnt{e_{{\mathrm{1}}}}  \ottsym{:}  \tau_{{\mathrm{1}}}$ and $\tau_{{\mathrm{1}}}  \longrightarrow  \tau_{{\mathrm{2}}}$. Consider whether $\ottnt{e_{{\mathrm{1}}}}$ is a value:
		     \begin{itemize}
    		    \item If $\ottnt{e_{{\mathrm{1}}}}=v$, by Lemma \ref{lem:gen}(6), it must be a $ \kw{cast}^{\uparrow} $-term such that $\ottnt{e_{{\mathrm{1}}}}  \equiv  \kw{cast}^{\uparrow} \, \ottsym{[}  \tau_{{\mathrm{1}}}  \ottsym{]}  \ottnt{e'_{{\mathrm{1}}}}$ for some $\ottnt{e'_{{\mathrm{1}}}}$ satisfying $\vdash  \ottnt{e'_{{\mathrm{1}}}}  \ottsym{:}  \tau_{{\mathrm{2}}}$. Then by rule \ruleref{S\_CastDownUp}, we can obtain $\kw{cast}_{\downarrow} \, \ottsym{(}  \kw{cast}^{\uparrow} \, \ottsym{[}  \tau_{{\mathrm{1}}}  \ottsym{]}  \ottnt{e'_{{\mathrm{1}}}}  \ottsym{)}  \longrightarrow  \ottnt{e'_{{\mathrm{1}}}}$. Thus, there exists $\ottnt{e'}  \equiv  \ottnt{e'_{{\mathrm{1}}}}$ such that $\ottnt{e}  \longrightarrow  \ottnt{e'}$.
    		    \item Otherwise, by induction hypothesis, there exists $\ottnt{e'_{{\mathrm{1}}}}$ such that $\ottnt{e_{{\mathrm{1}}}}  \longrightarrow  \ottnt{e'_{{\mathrm{1}}}}$. Then by rule \ruleref{S\_CastDown}, we have $\kw{cast}_{\downarrow} \, \ottnt{e_{{\mathrm{1}}}}  \longrightarrow  \kw{cast}_{\downarrow} \, \ottnt{e'_{{\mathrm{1}}}}$. Thus, there exists $\ottnt{e'}  \equiv  \kw{cast}_{\downarrow} \, \ottnt{e'_{{\mathrm{1}}}}$ such that $\ottnt{e}  \longrightarrow  \ottnt{e'}$.
    		\end{itemize}
		\item[Case $e=\mu \, \ottmv{x}  \ottsym{:}  \tau  \ottsym{.}  \ottnt{e_{{\mathrm{1}}}}$:] By rule \ruleref{S\_Mu}, there always exists $\ottnt{e'}  \equiv  \ottnt{e_{{\mathrm{1}}}}  \ottsym{[}  \ottmv{x}  \mapsto  \mu \, \ottmv{x}  \ottsym{:}  \tau  \ottsym{.}  \ottnt{e_{{\mathrm{1}}}}  \ottsym{]}$.
    \end{description}
\end{proof}

