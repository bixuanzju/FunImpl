%% ODER: format ==         = "\mathrel{==}"
%% ODER: format /=         = "\neq "
%
%
\makeatletter
\@ifundefined{lhs2tex.lhs2tex.sty.read}%
  {\@namedef{lhs2tex.lhs2tex.sty.read}{}%
   \newcommand\SkipToFmtEnd{}%
   \newcommand\EndFmtInput{}%
   \long\def\SkipToFmtEnd#1\EndFmtInput{}%
  }\SkipToFmtEnd

\newcommand\ReadOnlyOnce[1]{\@ifundefined{#1}{\@namedef{#1}{}}\SkipToFmtEnd}
\usepackage{amstext}
\usepackage{amssymb}
\usepackage{stmaryrd}
\DeclareFontFamily{OT1}{cmtex}{}
\DeclareFontShape{OT1}{cmtex}{m}{n}
  {<5><6><7><8>cmtex8
   <9>cmtex9
   <10><10.95><12><14.4><17.28><20.74><24.88>cmtex10}{}
\DeclareFontShape{OT1}{cmtex}{m}{it}
  {<-> ssub * cmtt/m/it}{}
\newcommand{\texfamily}{\fontfamily{cmtex}\selectfont}
\DeclareFontShape{OT1}{cmtt}{bx}{n}
  {<5><6><7><8>cmtt8
   <9>cmbtt9
   <10><10.95><12><14.4><17.28><20.74><24.88>cmbtt10}{}
\DeclareFontShape{OT1}{cmtex}{bx}{n}
  {<-> ssub * cmtt/bx/n}{}
\newcommand{\tex}[1]{\text{\texfamily#1}}	% NEU

\newcommand{\Sp}{\hskip.33334em\relax}


\newcommand{\Conid}[1]{\mathit{#1}}
\newcommand{\Varid}[1]{\mathit{#1}}
\newcommand{\anonymous}{\kern0.06em \vbox{\hrule\@width.5em}}
\newcommand{\plus}{\mathbin{+\!\!\!+}}
\newcommand{\bind}{\mathbin{>\!\!\!>\mkern-6.7mu=}}
\newcommand{\rbind}{\mathbin{=\mkern-6.7mu<\!\!\!<}}% suggested by Neil Mitchell
\newcommand{\sequ}{\mathbin{>\!\!\!>}}
\renewcommand{\leq}{\leqslant}
\renewcommand{\geq}{\geqslant}
\usepackage{polytable}

%mathindent has to be defined
\@ifundefined{mathindent}%
  {\newdimen\mathindent\mathindent\leftmargini}%
  {}%

\def\resethooks{%
  \global\let\SaveRestoreHook\empty
  \global\let\ColumnHook\empty}
\newcommand*{\savecolumns}[1][default]%
  {\g@addto@macro\SaveRestoreHook{\savecolumns[#1]}}
\newcommand*{\restorecolumns}[1][default]%
  {\g@addto@macro\SaveRestoreHook{\restorecolumns[#1]}}
\newcommand*{\aligncolumn}[2]%
  {\g@addto@macro\ColumnHook{\column{#1}{#2}}}

\resethooks

\newcommand{\onelinecommentchars}{\quad-{}- }
\newcommand{\commentbeginchars}{\enskip\{-}
\newcommand{\commentendchars}{-\}\enskip}

\newcommand{\visiblecomments}{%
  \let\onelinecomment=\onelinecommentchars
  \let\commentbegin=\commentbeginchars
  \let\commentend=\commentendchars}

\newcommand{\invisiblecomments}{%
  \let\onelinecomment=\empty
  \let\commentbegin=\empty
  \let\commentend=\empty}

\visiblecomments

\newlength{\blanklineskip}
\setlength{\blanklineskip}{0.66084ex}

\newcommand{\hsindent}[1]{\quad}% default is fixed indentation
\let\hspre\empty
\let\hspost\empty
\newcommand{\NB}{\textbf{NB}}
\newcommand{\Todo}[1]{$\langle$\textbf{To do:}~#1$\rangle$}

\EndFmtInput
\makeatother
%


\section{Applications}
\label{sec:app}

In this section, we show how some large examples using \name

\subsection{Parametric HOAS}
\label{sec:phoas}

Parametric Higher Order Abstract Syntax (PHOAS) is a higher order approach to represent binders, in which the function space of the meta-language is used to encode the binders of the object language. We show that \name can handle PHOAS by encoding lambda calculus as below:

\begingroup\par\noindent\advance\leftskip\mathindent\(
\begin{pboxed}\SaveRestoreHook
\column{B}{@{}>{\hspre}l<{\hspost}@{}}%
\column{3}{@{}>{\hspre}l<{\hspost}@{}}%
\column{E}{@{}>{\hspre}l<{\hspost}@{}}%
\>[B]{}\mathbf{data}\;\Varid{plambda}\;(\Varid{a}\mathbin{:}\mathbin{*})\mathrel{=}\Varid{var}\;\Varid{a}{}\<[E]%
\\
\>[B]{}\hsindent{3}{}\<[3]%
\>[3]{}\mid \Varid{num}\;\Varid{nat}{}\<[E]%
\\
\>[B]{}\hsindent{3}{}\<[3]%
\>[3]{}\mid \Varid{lam}\;(\Varid{a}\to (\Varid{plambda}\;\Varid{a})){}\<[E]%
\\
\>[B]{}\hsindent{3}{}\<[3]%
\>[3]{}\mid \Varid{app}\;(\Varid{plambda}\;\Varid{a})\;(\Varid{plambda}\;\Varid{a}){}\<[E]%
\ColumnHook
\end{pboxed}
\)\par\noindent\endgroup\resethooks

Next we define the evaluator for our lambda calculus. One advantage of PHOAS is that, environments are implicitly handled by the meta-language, thus the type of the evaluator is simply \ensuremath{\Varid{plambda}\;\Varid{value}\to \Varid{value}}. The code is presented as below:

\begingroup\par\noindent\advance\leftskip\mathindent\(
\begin{pboxed}\SaveRestoreHook
\column{B}{@{}>{\hspre}l<{\hspost}@{}}%
\column{3}{@{}>{\hspre}l<{\hspost}@{}}%
\column{4}{@{}>{\hspre}l<{\hspost}@{}}%
\column{6}{@{}>{\hspre}l<{\hspost}@{}}%
\column{8}{@{}>{\hspre}l<{\hspost}@{}}%
\column{9}{@{}>{\hspre}l<{\hspost}@{}}%
\column{10}{@{}>{\hspre}l<{\hspost}@{}}%
\column{11}{@{}>{\hspre}l<{\hspost}@{}}%
\column{13}{@{}>{\hspre}l<{\hspost}@{}}%
\column{E}{@{}>{\hspre}l<{\hspost}@{}}%
\>[B]{}\mathbf{data}\;\Varid{value}{}\<[13]%
\>[13]{}\mathrel{=}\Varid{vi}\;\Varid{nat}{}\<[E]%
\\
\>[B]{}\hsindent{3}{}\<[3]%
\>[3]{}\mid \Varid{vf}\;(\Varid{value}\to \Varid{value});{}\<[E]%
\\
\>[B]{}\mathbf{let}\;\Varid{eval}\mathbin{:}\Varid{plambda}\;\Varid{value}\to \Varid{value}\mathrel{=}{}\<[E]%
\\
\>[B]{}\hsindent{4}{}\<[4]%
\>[4]{}\Varid{mu}\;\Varid{ev}\mathbin{:}\Varid{plambda}\;\Varid{value}\to \Varid{value}\mathbin{\circ}{}\<[E]%
\\
\>[4]{}\hsindent{2}{}\<[6]%
\>[6]{}\lambda \Varid{e}\mathbin{:}\Varid{plambda}\;\Varid{value}\mathbin{\circ}\mathbf{case}\;\Varid{e}\;\mathbf{of}{}\<[E]%
\\
\>[6]{}\hsindent{2}{}\<[8]%
\>[8]{}\Varid{var}\;(\Varid{v}\mathbin{:}\Varid{value})\Rightarrow \Varid{v}{}\<[E]%
\\
\>[4]{}\hsindent{2}{}\<[6]%
\>[6]{}\mid \Varid{num}\;(\Varid{n}\mathbin{:}\Varid{nat})\Rightarrow (\Varid{vi}\;\Varid{n}){}\<[E]%
\\
\>[4]{}\hsindent{2}{}\<[6]%
\>[6]{}\mid \Varid{lam}\;(\Varid{f}\mathbin{:}\Varid{value}\mathbin{→}\Varid{plambda}\;\Varid{value})\Rightarrow {}\<[E]%
\\
\>[6]{}\hsindent{4}{}\<[10]%
\>[10]{}\Varid{vf}\;(\lambda \Varid{x}\mathbin{:}\Varid{value}\mathbin{\circ}\Varid{ev}\;(\Varid{f}\;\Varid{x})){}\<[E]%
\\
\>[4]{}\hsindent{2}{}\<[6]%
\>[6]{}\mid \Varid{app}\;(\Varid{a}\mathbin{:}\Varid{plambda}\;\Varid{value})\;(\Varid{b}\mathbin{:}\Varid{plambda}\;\Varid{value})\Rightarrow {}\<[E]%
\\
\>[6]{}\hsindent{3}{}\<[9]%
\>[9]{}\mathbf{case}\;(\Varid{ev}\;\Varid{a})\;\mathbf{of}{}\<[E]%
\\
\>[9]{}\hsindent{2}{}\<[11]%
\>[11]{}\Varid{vi}\;(\Varid{n}\mathbin{:}\Varid{nat})\Rightarrow \Varid{vi}\;\Varid{n}\mbox{\onelinecomment  absurd value}{}\<[E]%
\\
\>[6]{}\hsindent{3}{}\<[9]%
\>[9]{}\mid \Varid{vf}\;(\Varid{f}\mathbin{:}\Varid{value}\mathbin{→}\Varid{value})\Rightarrow \Varid{f}\;(\Varid{ev}\;\Varid{b}){}\<[E]%
\\
\>[B]{}\mathbf{in}{}\<[E]%
\ColumnHook
\end{pboxed}
\)\par\noindent\endgroup\resethooks

Now we can evaluate some lambda expression and get the result back:

\begingroup\par\noindent\advance\leftskip\mathindent\(
\begin{pboxed}\SaveRestoreHook
\column{B}{@{}>{\hspre}l<{\hspost}@{}}%
\column{3}{@{}>{\hspre}l<{\hspost}@{}}%
\column{5}{@{}>{\hspre}l<{\hspost}@{}}%
\column{7}{@{}>{\hspre}l<{\hspost}@{}}%
\column{E}{@{}>{\hspre}l<{\hspost}@{}}%
\>[B]{}\mathbf{let}\;\Varid{show}\mathbin{:}\Varid{value}\to \Varid{nat}\mathrel{=}{}\<[E]%
\\
\>[B]{}\hsindent{3}{}\<[3]%
\>[3]{}\lambda \Varid{e}\mathbin{:}\Varid{value}\mathbin{\circ}\mathbf{case}\;\Varid{e}\;\mathbf{of}{}\<[E]%
\\
\>[3]{}\hsindent{2}{}\<[5]%
\>[5]{}\Varid{vi}\;(\Varid{n}\mathbin{:}\Varid{nat})\Rightarrow \Varid{n}{}\<[E]%
\\
\>[B]{}\hsindent{3}{}\<[3]%
\>[3]{}\mid \Varid{vf}\;(\Varid{f}\mathbin{:}\Varid{value}\to \Varid{value})\Rightarrow \mathrm{10000}\mbox{\onelinecomment  absurd value}{}\<[E]%
\\
\>[B]{}\mathbf{in}{}\<[E]%
\\
\>[B]{}\mathbf{let}\;\Varid{example}\mathbin{:}\Varid{plambda}\;\Varid{value}\mathrel{=}{}\<[E]%
\\
\>[B]{}\hsindent{3}{}\<[3]%
\>[3]{}\Varid{app}\;\Varid{value}\;{}\<[E]%
\\
\>[3]{}\hsindent{4}{}\<[7]%
\>[7]{}(\Varid{lam}\;\Varid{value}\;(\lambda \Varid{x}\mathbin{:}\Varid{value}\mathbin{\circ}\Varid{var}\;\Varid{value}\;\Varid{x}))\;{}\<[E]%
\\
\>[3]{}\hsindent{4}{}\<[7]%
\>[7]{}(\Varid{num}\;\Varid{value}\;\mathrm{4}){}\<[E]%
\\
\>[B]{}\mathbf{in}{}\<[E]%
\\
\>[B]{}\Varid{show}\;(\Varid{eval}\;\Varid{example})\mbox{\onelinecomment  return 4}{}\<[E]%
\ColumnHook
\end{pboxed}
\)\par\noindent\endgroup\resethooks
