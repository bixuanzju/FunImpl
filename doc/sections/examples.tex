
\section{Applications}
\label{sec:app}

\jeremy{Fill in large examples like monad, Fix, HOAS, dependent types.}

\subsection{Monad}
\label{sec:monad}

In this section, we show how we can encode monad in \name.

Monad definition in Haskell:
\begin{lstlisting}
class Monad m where
  return :: a -> m a
  bind :: m a -> (a -> m b) -> m b
\end{lstlisting}

Translated in \name as a record:

\begin{lstlisting}
rec monad (m : * -> *) = mo
  {
    return : pi a : * . a -> m a,
    bind : pi a : *. pi b : *.
           m a -> (a -> m b) -> m b
  }
\end{lstlisting}

The monad instance of \emph{Maybe} datatype in Haskell:

\begin{lstlisting}
instance Monad Maybe where
  return x = Just x
  Nothing >>= f = Nothing
  Just x >>= f  = f x
\end{lstlisting}

And in \name:
\begin{figure}[ht]
  \centering
\begin{lstlisting}
let inst : monad maybe =
  (mo maybe
      (lam a : * . lam x : a . nothing a)
      (lam a : *. lam b : *.
       lam x : maybe a . lam f : a -> maybe b .
       case x of
         nothing => nothing b
       | just (y : a) => f y))
in
\end{lstlisting}
\end{figure}

\subsection{Fix as a datatype}
\label{sec:fix}

In Haskell, we can make a fix datatype:
\begin{lstlisting}
newtype Fix f = In {out :: f (Fix f)}
\end{lstlisting}

And in \name:
\begin{lstlisting}
rec Fix (f : * -> *) = In {out : f (Fix f)}
\end{lstlisting}


%%% Local Variables:
%%% mode: latex
%%% TeX-master: "../main"
%%% End:
