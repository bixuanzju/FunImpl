
\section{A Dependently Typed Calculus with Casts}\label{sec:ecore}

In this section, we present the \ecore calculus. This calculus is very 
close to the calculus of contructions, except for two key differences: 
1) the absence of the $\Box$ constant (due to use of the "type-in-type"
axiom); 2) the existence of two \cast operators. Like the calculus 
of constructions, \ecore has decidable type-checking. However, unlike
\cc the proof of decidability of type-checking does not require 
the strong normalization of the calculus. 
In rest of this
section, we demonstrate the syntax, operational semantics, typing
rules and meta-theory of \ecore.\bruno{Mention where to find the proofs!}

%\ecore is a subset of the core
%language \name without general recursion. By explicitly controlling the
%type-level
%computation with \cast operators, \ecore has decidable type-checking
%without requiring strong normalization. 

\subsection{Syntax}\label{sec:ecore:syn}

Figure \ref{fig:ecore:syntax} shows the syntax of \ecore, including
expressions, contexts and values. \ecore uses a unified syntactic
representation for different levels of expressions by following the
\emph{pure type system} (PTS) representation of \cc. Therefore, there
is no syntactic distinction between terms, types or kinds. This design
brings economy for type checking, since one set of rules can cover
all syntactic levels. By convention, we use metavariables $[[t]]$ and
$[[T]]$ for an expression on the type-level position and $e$ for one
on the term level.

\paragraph{Type of Types}
Traditionally in \cc, there are two distinct
sorts $[[star]]$ and $[[square]]$ representing the type of
\emph{types} and \emph{sorts} respectively, and an axiom
$[[star]]:[[square]]$ specifying the relation between the two sorts. In \ecore, we further
merge types and kinds together by including only a single sort
$[[star]]$ and an impredicative axiom $[[star]]:[[star]]$. 


\paragraph{Explicit Type Conversion}

We introduce two new primitives $[[castup]]$ and $[[castdown]]$
(pronounced as "cast up" and "cast down") to replace the implicit
conversion rule of \cc with a \emph{one-step} explicit type
conversion. They represent two directions of type conversion:
$[[castdown]]$ stands for the $\beta$-reduction of types, while
$[[castup]]$ is the inverse. % (Examples will be given in later typing sections).
Though \cast primitives make the syntax verbose when type conversion
is heavily used, the implementation of type checking is simplified:
the typing rules of \ecore become syntax-directed without \cc's
implicit conversion rule. \bruno{I think the following point should be
made in the source language section, not here.}
% Considering the core language is
% compiler-oriented, end-users will not directly use them. 
% Some type
% conversions can be generated through the translation of the source
% language (\S \ref{sec:surface}).
\linus{Revised. Section 6 has mentioned this.}

\begin{figure}
    \gram{\ottec\ottinterrule
        \ottG\ottinterrule
        \ottv}
    \caption{Syntax of \ecore}
    \label{fig:ecore:syntax}
\end{figure}\bruno{Product should be dependent product. }\linus{Corrected in all figures.}

\subsection{Operational Semantics}\label{sec:ecore:opsem}

Figure \ref{fig:ecore:opsem} shows the \emph{call-by-name} operational
semantics, defined by one-step reduction. Two base cases include
\ruleref{S\_Beta} for $\beta$-reduction and \ruleref{S\_CastDownUp}
for cast canceling. Two inductive cases, \ruleref{S\_App} and
\ruleref{S\_CastDown}, define reduction in the head position of an
application, and in the $[[castdown]]$ inner expression
respectively. The reduction rules are \emph{weak} in the sense that they
are not allowed to reduce inside a $\lambda$-term or $[[castup]]$-term
which is viewed as a value (see Figure \ref{fig:ecore:syntax}).

To evaluate the value of a term-level expression, we apply the
one-step reduction multiple times. The number of evaluation steps is
not restricted, which is possible to be infinite. The multi-step
reduction can be defined as follows:

\begin{dfn}[Multi-step reduction]
    The relation $[[->>]]$ is the transitive and reflexive closure of
    the one-step reduction $[[-->]]$.
\end{dfn}

%\bruno{The following point should be made when talking about typing, 
%not here!}
%But for a type-level expression, the evaluation is driven by \cast
%operators, which is finite. 
For a consecutive sequence of reductions
with $n$ steps, we use the notation $[[-->>]]$ to denote the relation
between the initial and final expressions:

\begin{dfn}[$n$-step reduction]
    The $n$-step reduction is denoted by $[[e0]] [[-->>]] [[en]]$, if
    there exists a sequence of one-step reductions $[[e0]] [[-->]]
    [[e1]] [[-->]] [[e2]] [[-->]] \dots [[-->]] [[en]]$, where $n$ is
    a positive integer and $[[ei]]\,(i=0,1,\dots,n)$ are valid
    expressions.
\end{dfn}

\begin{figure}
    \ottdefnstep{}
    \caption{Operational semantics of \ecore}
    \label{fig:ecore:opsem}
\end{figure}

\subsection{Typing}\label{sec:ecore:type}

Figure \ref{fig:ecore:typing} gives the \emph{syntax-directed} typing
rules of \ecore, including rules of context well-formedness $[[|- G]]$
and expression typing $[[G |- e : t]]$. Note that there is only a
single set of rules for expression typing, because there is no
distinction of different syntactic levels.

Most typing rules are quite standard. We write $[[|- G]]$ if a context
$[[G]]$ is well-formed. Note that there is only a single sort
$[[star]]$, we use $[[G |- t : star]]$ to check if $[[t]]$ is a
well-formed type. Rule \ruleref{T\_Ax} is the "type-in-type"
axiom. Rule \ruleref{T\_Var} checks the type of variable $[[x]]$ from
the valid context. Rules \ruleref{T\_App} and \ruleref{T\_Lam} check
the validity of application and abstraction. Rules \ruleref{T\_Pi}
check the type well-formedness of the dependent function.

\paragraph{The Cast Rules}
We focus on rules \ruleref{T\_CastUp} and \ruleref{T\_CastDown} that
define the semantics of \cast operators and replace the conversion
rule of \cc~(see Section~\ref{sec:coc}). The relation between the original
and converted type is defined by one-step reduction (see Figure
\ref{fig:ecore:opsem}). Given a judgement
$[[G |- e : t2]]$ and relation $[[t1 --> t2]] [[-->]] [[t3]]$, then
$[[castup [t1] e]]$ expands the type of $[[e]]$ from $[[t2]]$ to
$[[t1]]$, while $[[castdown e]]$ reduces the type of $[[e]]$ from
$[[t2]]$ to $[[t3]]$. For example, assume $[[G |- e1 : int]]$ and $[[G |- e2 :
(\x : star.x) int]]$. Note that the following reduction holds:
\[ [[(\x : star.x) int --> int]]\]
Thus, we can obtain the following derivations:
\[
\inferrule{[[G |- e1 : int]] \\ [[G |- (\x : star.x) int : star]] \\ [[(\x :
star.x) int --> int]]}{[[G |- (castup[(\x : star.x) int] e1):(\x : star.x)
int]]}
\]
\[
\inferrule{[[G |- e2 : (\x : star.x) int]] \\ [[G |- int : star]] \\ [[(\x :
star.x) int --> int]]}{[[G |- (castdown e2):int]]}
\]

Importantly, in \ecore term-level and type-level computation are treated 
differently. Term-level computation is dealt in the usual way, by 
using multi-step reduction until a value is finally obtained. 
Type-level computation, on the other hand, is controlled by the program:
each step of the computation is induced by a cast. If a type-level 
program requires $n$ steps of computation to reach normal form, 
then it will require $n$ casts to compute a (type-level) value.

\paragraph{Cast Operators in $n$ steps}
For the brevity purpose, we use syntactic sugar to denote $n$ consecutive cast operators.
Suppose we have $[[e]] : [[t]]$ and sequences of
reduction $[[t1]] [[-->]] [[t2]] [[-->]] \dots [[-->]] [[tn]] [[-->]] [[t]]$
and $[[t]] [[-->]] [[T1]] [[-->]] [[T2]] [[-->]] \dots [[-->]] [[Tn]]$. 
The $n$-step \cast operators can be defined:
\begin{flalign*}
    &[[foldn]] [ [[t1]], \dots, [[tn]] ] [[e]] & [[:=]] & [[castup]] [ [[t1]] ]
([[castup]] [ [[t2]] ] (\dots ( [[castup]] [ [[tn]] ] [[e]] ) \dots )) \\
    &[[unfoldn]] [[e]] & [[:=]] & \underbrace{[[castdown]] ([[castdown]] (\dots
( [[castdown]]}_n [[e]]) \dots ))
\end{flalign*}

By rules \ruleref{T\_CastUp} and \ruleref{T\_CastDown}, we have the following
typing results:
\[\begin{array}{lll}
    &[[foldn]] [ [[t1]], \dots, [[tn]] ] [[e]] & : [[t1]] \\
    &[[unfoldn]] [[e]] & : [[Tn]]
\end{array}\]

\paragraph{Benefits of Type in Type}
In the context of \cc, if a term $[[x]]$ has the type $[[t1]]$, and
$[[t2]]$ is a type, i.e. $[[x]]:[[t1]]:[[star]]$ and
$[[t2]]:[[square]]$, we call the type $[[Pi x:t1.t2]]$ a
\emph{dependent product}. \ecore follows \cc to use the same
$[[Pi]]$-notation to represent dependent function types.

However, a higher-kinded polymorphic function type such as $[[Pi
    x:square.x->x]]$ is not allowed in \cc, because $[[square]]$ is
the highest sort, and it cannot be typed. In contrast $[[Pi]]$-notation in
\ecore is more expressive and does not have such limitation because of
the axiom $[[star]]:[[star]]$. In the surface language, we
interchangeably use the arrow form $[[(x:t1)->t2]]$ of the product for
clarity. By convention, we also use the syntactic sugar $[[t1 -->
    t2]]$ to represent the product if $[[x]]$ does not occur free in
$[[t2]]$.

\paragraph{Syntactic Equality}
Finally, the definition of type equality in \ecore differs from
\cc. Without \cc's conversion rule, the type of a term cannot be
converted freely against $\beta$-equality, unless using \cast
operators. Thus, types of expressions are equal only if they are
syntactically equal, i.e. satisfy the $\alpha$-equality.

\begin{figure}
    \ottdefnctx{}\ottinterrule
    \ottdefnexpr{}
    \caption{Typing rules of \ecore}
    \label{fig:ecore:typing}
\end{figure}

\subsection{Meta-theory}\label{sec:ecore:meta}
We now discuss the meta-theory of \ecore. We focus on two properties: the
decidability of type checking and the type-safety of the language. First, we
show that type checking \ecore is decidable without requiring strong
normalization. Second, the
language is type safe, proven by standard subject reduction and progress
lemmas.

\paragraph{Decidability of Type Checking}
For the decidability, we need to show there exists a type checking algorithm,
which never loops forever and returns a unique type for a well-formed
expression $[[e]]$. This is done by induction on the length of $[[e]]$ and
ranging over typing rules. Most expression typing rules, which have only typing
judgements in premises, are already decidable by the induction hypothesis. Thus, it
is straightforward to follow the syntax-directed judgement to derive a unique
type checking result.

The critical case is for rules \ruleref{T\_CastUp} and \ruleref{T\_CastDown}.
Both rules contain a premise that needs to judge if two types $[[t1]]$ and
$[[t2]]$ follow the one-step reduction, i.e. if $[[t1 --> t2]]$ holds. We need
to show such $[[t2]]$ is \emph{unique} with respect to the one-step reduction,
or equivalently, reducing $[[t1]]$ by one step will get only a sole result
$[[t2]]$. Otherwise, assume $[[e]]:[[t1]]$ and there exists $[[t2']]$ such that
$[[t1 --> t2]]$ and $[[t1 --> t2']]$. Then the type of $[[castdown e]]$ can be
either $[[t2]]$ or $[[t2']]$ by rule \ruleref{T\_CastDown}, which
would not be
decidable. The property is proven by the following lemma:

\begin{lem}[Decidability of One-step Reduction]\label{lem:ecore:unique}
	The one-step reduction $[[-->]]$ is called decidable if 
given $[[e]]$ there is a unique $[[e']]$ such that $[[e --> e']]$ or no such $[[e']]$.
\end{lem}

\begin{proof}
	By induction on the structure of $[[e]]$.
\end{proof}

With this result, we show a decidable algorithm to check whether one-step
relation $[[t1 --> t2]]$ holds. An intuitive algorithm is to reduce the type
$[[t1]]$ by one step to obtain $[[t1']]$ (which is unique by Lemma
\ref{lem:ecore:unique}), and compare if $[[t1']]$ and $[[t2]]$ are
syntactically equal. Thus, checking if $[[t1 --> t2]]$ is decidable and rules
\ruleref{T\_CastUp} and \ruleref{T\_CastDown} are therefore decidable. We can
conclude the decidability of type checking:

\begin{thm}[Decidability of Type Checking \ecore]\label{lem:ecore:decide}
	There is an algorithm which given $[[G]], [[e]]$ computes the unique
$[[t]]$ such that $[[G |- e:t]]$ or reports there is no such $[[t]]$.
\end{thm}

\begin{proof}
	By induction on the structure of $[[e]]$.
\end{proof}

We emphasize that when proving the decidability of type checking, we do not rely on the
on strong normalization. Intuitively, because explicit type conversion rules use one-step
reduction, which already has a decidable checking algorithm according to Lemma
\ref{lem:ecore:unique}. We do not need to further require the normalization of
terms. This is different from the proof for \cc which requires the
language to be strongly
normalizing \cite{pts:normalize}. Because \cc's conversion rule needs to
examine the $\beta$-equivalence of terms, which is decidable only if every term
has a normal form.

\paragraph{Type-safety}
The proof of the type-safety (or type soundness) of \ecore is fairly standard by subject
reduction (or preservation) and progress lemmas. The subject reduction proof
relies on the substitution lemma. We give the proof sketch of related lemmas as
follows:

\begin{lem}[Substitution]\label{lem:ecore:subst}
	If $[[G1, x:T, G2 |- e1:t]]$ and $[[G1 |- e2:T]]$, then $[[G1, G2 [x |-> e2]
|- e1[x |-> e2]  : t[x |-> e2] ]]$.
\end{lem}

\begin{proof}
    By induction on the derivation of $[[G1, x:T, G2 |- e1:t]]$.
\end{proof}

\begin{thm}[Subject Reduction]\label{lem:ecore:reduct}
If $[[G |- e:T]]$ and $[[e]] [[->>]] e'$ then $[[G |- e':T]]$.
\end{thm}

\begin{proof}
    (\emph{Sketch}) We prove the case for one-step reduction, i.e. $[[e -->
e']]$. The lemma can follow by induction on the number of one-step reductions
of $[[e]] [[->>]] [[e']]$.
    The proof is by induction with respect to the definition of one-step
reduction $[[-->]]$.
\end{proof}

\begin{thm}[Progress]\label{lem:ecore:prog}
If $[[empty |- e:T]]$ then either $[[e]]$ is a value $v$ or there exists $[[e]]'$
such that $[[e --> e']]$.
\end{thm}

\begin{proof}
    By induction on the derivation of $[[empty |- e:T]]$.
\end{proof}

\paragraph{Simplified form of $[[foldn]]$}
Because of the decidability of one-step reduction, we can simplify the form of 
$n$-step $[[castup]]$ operator. From Lemma \ref{lem:ecore:unique}, 
we immediately have the following corollary for $n$-step reduction:

\begin{lem}[Decidability of $n$-step Reduction]\label{lem:ecore:uniquen}
    The $n$-step reduction $[[-->>]]$ is called decidable if 
    given $[[e]]$ there is a unique $[[e']]$ such that $[[e]] [[-->>]] [[e']]$ or no such $[[e']]$.
\end{lem}

\begin{proof}
    Immediate from Lemma \ref{lem:ecore:unique}, by induction on the number of
reduction steps.
\end{proof}

Suppose $[[e]] : [[t]]$ and there exists a reduction sequence
$[[t1]] [[-->]] [[t2]] [[-->]] \dots [[-->]] [[tn]] [[-->]] [[t]]$,
the following typing of $[[foldn]]$ holds:
\[
    [[foldn]] [ [[t1]], \dots, [[tn]] ] [[e]] : [[t1]].
\]
Note that $[[t1]] [[-->>]] [[t]]$ are decidable by Lemma
\ref{lem:ecore:uniquen}. Thus, the intermediate types in $[[t1]] [[-->>]] [[t]]$,
i.e. $[[t2]], \dots, [[tn]]$, can be uniquely determined. So we can leave
them out in the $[[foldn]]$ operator with the following form:
\[
    [[foldn [t1] e]] : [[t1]].
\]
