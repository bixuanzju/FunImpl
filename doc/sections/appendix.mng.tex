\section{Specification of core language}

\subsection{Syntax}
\gram{\otte\ottinterrule
        \otts\ottinterrule
        \ottG\ottinterrule
        \ottv}

\subsection{Operational semantics and expression typing}
\ottdefnstep{}
\ottusedrule{\ottdruleSXXMu{}}
\ottdefnexpr{}
\ottusedrule{\ottdruleTXXMu{}}

\section{Proofs of core language}
\subsection{Properties}
\begin{lem}[Generation lemma]\label{lem:gen}
$\quad$
\begin{enumerate}
	\item If $[[G |- x:T]]$, then there exist an expression $[[t]]$ and a sort $s$ such that $[[t =a T]]$, $[[G |- t:s]]$ and $[[x:t elt G]]$.
	\item If $[[G |- e1 e2:T]]$, then there exist expressions $[[t1]]$ and $[[t2]]$ such that $[[G |- e1 : (Pi x:t1.t2)]]$, $[[G |- e2:t2]]$ and $[[T =a t1[x |-> e2] ]]$.
	\item If $[[G |- (\x:t1.e):T]]$, then there exist a sort $s$ and an expression $[[t2]]$ such that $[[T =a Pi x:t1.t2]]$ where $[[G |- (Pi x:t1.t2):s]]$ and $[[G,x:t1 |- e:t2]]$.
	\item If $[[G |- (Pi x:t1.t2):T]]$, then there exist sorts $[[s1]]$ and $[[s2]]$ such that $[[T == s2]]$, $[[G |- t1:s1]]$ and $[[G, x:t1 |- t2:s2]]$.
	\item If $[[G |- (mu x:t.e):T]]$, then there exists a sort $[[s]]$ such that $[[G |- t:s]]$, $[[T =a t]]$ and $[[G, x:t|-e:t]]$.
	\item IF $[[G |- (castup [t1] e):T]]$, then there exist an expression $[[t2]]$ and a sort $s$ such that $[[G |- e:t2]]$, $[[G |- t1:s]]$, $[[t1 --> t2]]$ and $[[T =a t1]]$.
	\item If $[[G |- (castdown e):T]]$, then there exist expressions $[[t1]],[[t2]]$ and a sort $s$ such that $[[G |- e:t1]]$, $[[G |- t2:s]]$, $[[t1 --> t2]]$ and $[[T =a t2]]$.
\end{enumerate}
\end{lem}

\begin{lem}[Substitution lemma]
	If $[[G1]], x:[[t1]], [[G2 |- e1:t2]]$ and $[[G1 |- e2:t1]]$, then $[[G1]], [[G2]] [x [[|->]] [[e2]] ] [[|-]] [[e1[x |-> e2] ]] : [[t2[x |-> e2] ]]$.
\end{lem}

\begin{dfn}[Well-formed context]
	A \textbf{well-formed} context $[[G]]$ is defined by the following rules:
	
	\textnormal{\ottdefnctx{}}
\end{dfn}

\begin{lem}[Consistency of well-formed context]\label{lem:wfc}
	Given a well-formed initial context $[[G]]$, it remains well-formed through type checking.
\end{lem}

\begin{proof}
	Suppose $[[G]]$ is the initial context which is well-formed. To safely extend $[[G]]$ with a variable $x:[[t]]$, one should have $[[G |- t:s]]$ due to rule \ruleref{Env\_Var}. Note that when applying typing rules of $[[G |- e:t]]$, rule \ruleref{T\_Pi}, \ruleref{T\_Mu} and \ruleref{T\_Lam} will extend the context. We show that these rules cover the condition $[[G |- t:s]]$ with respect to $x:[[t]]$ as follows:
	\begin{description}
		\item[Case \ruleref{T\_Pi}:] \ottusedrule{\ottdruleTXXPi{}} For $x:[[t1]]$, $[[G |- t1:s]]$ is directly the premise of the rule.
		\item[Case \ruleref{T\_Mu}:] \ottusedrule{\ottdruleTXXMu{}} For $x:[[t]]$, $[[G |- t:s]]$ is directly the premise of the rule.
		\item[Case \ruleref{T\_Lam}:] \ottusedrule{\ottdruleTXXLam{}} For $x:[[t1]]$, note that the premise $[[G |- (Pi x:t1.t2):s]]$ can be derived from rule \ruleref{T\_Pi}, which has the pre-condition $[[G |- t1:s]]$.
	\end{description}
\end{proof}

\begin{lem}[Valid context optimization]\label{lem:wfcopt}
	With a well-formed initial context $[[G]]$, the \ruleref{T\_Var} and \ruleref{T\_Weak} can be replaced by the following rule: \ottusedrule{\ottdruleTSXXVar{}}
\end{lem}

\begin{proof}
	By Lemma \ref{lem:wfc}, the context $[[G]]$ remains well-formed if it is initially well-formed. Thus, it is not necessary to use \ruleref{T\_Var} and \ruleref{T\_Weak} to check the well-formedness of $[[G]]$. By Lemma \ref{lem:gen}, in order to check the type of a variable $x$, it is necessary and sufficient to check if $[[x:t elt G]]$, which is simply rule \ruleref{TS\_Var}.
\end{proof}

\subsection{Decidability of type checking}
\begin{lem}[Uniqueness of one-step reduction]\label{lem:unired}
	The relation $[[-->]]$, i.e. one-step reduction, is \textbf{unique} in the sense that given $e$ there is at most one $e'$ such that $[[e --> e']]$.
\end{lem}

\begin{proof}
	By induction on the structure of $e$:
	\begin{description}
		\item[Case $e=v$:] $e$ has one of the following forms:
		\begin{inparaenum}[(1)]
			\item $[[\x:t.e]]$,
			\item $[[Pi x:t1.t2]]$,
			\item $[[castup [t] e]]$,
		\end{inparaenum}
		which cannot match any rules of $[[-->]]$. Thus there is no $e'$ such that $[[e-->e']]$.
		\item[Case $e=[[(\x:t.e1) e2]]$:] There is a unique $e'=[[ e1[x|->e2] ]]$ by rule \ruleref{S\_Beta}.
		\item[Case $e=[[castdown (castup [t] e)]]$:] There is a unique $e'=e$ by rule \ruleref{S\_CastDownUp}.
		\item[Case $e=[[mu x:t.e]]$:] There is a unique $e'=[[e[x|->mu x:t.e] ]]$ by rule \ruleref{S\_Mu}.
		\item[Case $e=[[e1 e2]]$ with $[[e1 --> e1']]$:] $[[e1]]$ cannot be a $\lambda$-term $[[\x:t.e]]$ which is a value that contradicts $[[e1]]$ can be reduced to $[[e1']]$. By the induction hypothesis, $[[e1']]$ is unique reduction of $[[e1]]$. Thus by rule \ruleref{S\_App}, $e'=[[e1' e2]]$ is the unique reduction for $e$.
		\item[Case $e=[[castdown e1]]$ with $[[e1 --> e1']]$:] $[[e1]]$ cannot have the form $[[castup [t] e]]$ which is a value that contradicts $[[e1]]$ can be reduced to $[[e1']]$. By the induction hypothesis, $[[e1']]$ is unique reduction of $[[e1]]$. Thus by rule \ruleref{S\_CastDown}, $e'=[[castdown e1']]$ is the unique reduction for $e$.
	\end{description}
\end{proof}

\begin{lem}[Decidability of type checking]
	There is a decidable algorithm which given $[[G]], [[e]]$ computes the unique $[[t]]$ such that $[[G |- e:t]]$ or reports there is no such $[[t]]$.
\end{lem}

\begin{proof}
	By induction on the derivation of $e$:
	\begin{description}
		\item[Case $e=x$:] By Lemma \ref{lem:wfcopt}, we only need to consider context $[[G]]$ that is well-formed. By rule \ruleref{TS\_Var}, if $[[x:t elt G]]$, $t$ is the unique type of $x$.
		\item[Case $e=[[e1 e2]]$, or $[[\x:t1.e1]]$, or $[[Pi x:t1.t2]]$, or $[[mu x:t.e1]]$:] Trivial according to Lemma \ref{lem:gen} by applying rule \ruleref{T\_App}, \ruleref{T\_Lam}, \ruleref{T\_Pi}, or \ruleref{T\_Mu} respectively.
		\item[Case $e=[[castup [t1] e1]]$:] From rule \ruleref{T\_CastUp}, by induction hypothesis, we can derive the type of $[[e1]]$ as $[[t2]]$, and check whether $[[t1]]$ is legal, i.e. its sorts is either $[[star]]$ or $[[square]]$. If $[[t1]]$ is legal, by Lemma \ref{lem:unired}, there is at most one $[[t1']]$ such that $[[t1 --> t1']]$. If such $[[t1']]$ does not exist, then we report the type checking is failed. Otherwise, we examine if $[[t1']]$ is syntactically equal to $[[t2]]$, i.e. to check the $\alpha$-equality $[[t1' =a t2]]$. If the equality holds, we obtain the unique type of $[[e]]$ which is $[[t1]]$. Otherwise, we report $[[e]]$ fails to type check.
		\item[Case $e=[[castdown e1]]$:] From rule \ruleref{T\_CastDown}, by induction hypothesis, we can derive the type of $[[e1]]$ as $[[t1]]$. By Lemma \ref{lem:unired}, there is at most one $[[t2]]$ such that $[[t1 --> t2]]$. If such $[[t2]]$ exists and its sorts is either $[[star]]$ or $[[square]]$, we have found the unique type of $[[e]]$ is $[[t2]]$. Otherwise, we report $[[e]]$ fails to type check.
	\end{description}
\end{proof}

\subsection{Soundness}
\begin{lem}[Subject reduction]
If $[[G |- e:t]]$ and $e [[->>]] e'$ then $[[G |- e':t]]$.
\end{lem}

\begin{lem}[Progress]
If $[[empty |- e:t]]$ then either $e$ is a value $v$ or there exists $e'$ such that $e [[->>]] e'$.
\end{lem}

