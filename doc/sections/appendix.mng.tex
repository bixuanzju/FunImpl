\section{Full Specification of Core Language}

\subsection{Syntax}
\gram{\otte\ottinterrule
        \ottG\ottinterrule
        \ottv}
\\[2.0mm]
Syntactic Sugar\\
\resizebox{\columnwidth}{!}{$\ottcoresugar$} % defined in otthelper.mng.tex

\subsection{Operational Semantics}
\ottdefnstep{}
\ottusedrule{\ottdruleSXXMu{}}

\subsection{Typing}
\ottdefnctx{}\ottinterrule
\ottdefnexpr{}
\ottusedrule{\ottdruleTXXMu{}}

\section{Proofs about Core Language}
\subsection{Properties}
We follow the naming of lemmas and proofs of properties 
for Pure Type System from \cite{handbook}. Some lemmas have other well-known names, like
Lemma \ref{lem:appendix:thin} is often called \emph{Weakening} and 
Lemma \ref{lem:appendix:gen} is often called \emph{Inversion}.

\begin{comment}
\begin{lem}[Free Variable]\label{lem:appendix:free}
    If $[[G |- e:t]]$, then $\FV(e) \subseteq \dom([[G]])$ and $\FV([[t]])
\subseteq \dom([[G]])$.
\end{lem}

\begin{proof}
    By induction on the derivation of $[[G |- e:t]]$. We only treat cases
\ruleref{T\_Mu}, \ruleref{T\_CastUp} and \ruleref{T\_CastDown} (since proofs of
other cases are the same as \cc \cite{handbook}):
    \begin{description}
        \item[Case \ruleref{T\_Mu}:] From premises of $[[G |- (mu x:t.e1) :
t]]$, by the induction hypothesis, we have $\FV(e_1) \subseteq \dom([[G]]) \cup
\{[[x]]\}$ and $\FV(\tau) \subseteq \dom([[G]])$. Thus the result follows by
$\FV([[mu x:t.e1]])=\FV(e_1) \setminus \{[[x]]\} \subseteq \dom([[G]])$ and
$\FV(\tau) \subseteq \dom([[G]])$.
        \item[Case \ruleref{T\_CastUp}:] Since $\FV([[castup [t]
e1]])=\FV([[e1]])$, the result follows directly by the induction hypothesis.
        \item[Case \ruleref{T\_CastDown}:] Since $\FV([[castdown
e1]])=\FV([[e1]])$, the result follows directly by the induction hypothesis.
    \end{description}
\end{proof}
\end{comment}

\begin{lem}[Thinning]\label{lem:appendix:thin}
    Let $[[G]]$ and $[[G']]$ be legal contexts such that $[[G]] \subseteq
[[G']]$. If $[[G |- e : t]]$ then $[[G' |- e : t]]$.
\end{lem}

\begin{proof}
    By trivial induction on the derivation of $[[G |- e : t]]$.
\end{proof}

\begin{lem}[Substitution]\label{lem:appendix:subst}
	If $[[G1, x:T, G2 |- e1:t]]$ and $[[G1 |- e2:T]]$, then $[[G1, G2 [x |-> e2]
|- e1[x |-> e2]  : t[x |-> e2] ]]$.
\end{lem}

\begin{proof}
    By induction on the derivation of $[[G1, x:T, G2 |- e1:t]]$. We use the notation $[[e* == e
[x |-> e2] ]]$ to denote the substitution for short. Then the result can be written as \[ [[G1, G2* |- e1*  : t* ]]\].
We only treat cases \ruleref{T\_Mu}, \ruleref{T\_CastUp} and
\ruleref{T\_CastDown} since other cases can be easily followed by the proof for PTS in \cite{handbook}.
Consider the last step of derivation of the following
cases:
    \begin{description}
        \item[Case \ruleref{T\_Mu}:] $\inferrule{[[G1, x:T, G2, y:t |- e1:t]] \\
[[G1, x:T, G2 |- t:s]]}{[[G1, x:T, G2 |- (mu y:t.e1): t]]}$ 
        
        By the induction hypothesis, we have $[[G1, G2*, y:t* |- e1* : t*]]$ and $[[G1,
G2* |- t* : star]]$. Then by the derivation rule, $[[G1, G2* |- (mu
y:t*.e1*):t*]]$. Thus we can conclude $[[G1, G2* |- (mu y:t.e1)*:t*]]$.
        \item[Case \ruleref{T\_CastUp}:] $\inferrule{[[G1, x:T, G2 |- e1:t2]]
\\ [[G1, x:T, G2 |- t1:s]] \\ [[t1 --> t2]]}{[[G1, x:T, G2 |- (castup [t1]
e1):t1]]}$ 
        
        By the induction hypothesis, we have $[[G1, G2* |- e1*:t2*]]$, $[[G1, G2*
|- t1*:star]]$ and $[[t1 --> t2]]$. By the definition of substitution, we can
obtain $[[t1* --> t2*]]$ by $[[t1 --> t2]]$. Then by the derivation rule, $[[G1,
G2* |- (castup [t1*] e1*):t1*]]$. Thus we can conclude $[[G1, G2* |- (castup [t1]
e1)*:t1*]]$.
        \item[Case \ruleref{T\_CastDown}:] $\inferrule{[[G1, x:T, G2 |- e1:t1]]
\\ [[G1, x:T, G2 |- t2:s]] \\ [[t1 --> t2]]}{[[G1, x:T, G2 |- (castdown
e1):t2]]}$ 
        
        By the induction hypothesis, we have $[[G1, G2* |- e1*:t1*]]$, $[[G1, G2*
|- t2*:star]]$ and $[[t1 --> t2]]$ thus $[[t1* --> t2*]]$. Then by the
derivation rule, $[[G1, G2* |- (castdown e1*):t2*]]$. Thus we can conclude $[[G1, G2* |-
(castdown e1)*:t2*]]$.
    \end{description}
\end{proof}

\begin{lem}[Generation]\label{lem:appendix:gen}
If the alpha equivalence is witnessed by notation $[[=a]]$, we have the following results:
\begin{enumerate}[(1)]
	\item If $[[G |- x:T]]$, then there exist an expression $[[t]]$ such that $[[t
=a T]]$, $[[G |- t:s]]$ and $[[x:t elt G]]$.
	\item If $[[G |- e1 e2:T]]$, then there exist expressions $[[t1]]$ and
$[[t2]]$ such that $[[G |- e1 : (Pi x:t2.t1)]]$, $[[G |- e2:t2]]$ and $[[T =a
t1[x |-> e2] ]]$.
	\item If $[[G |- (\x:t1.e):T]]$, then there exist an expression $[[t2]]$ such
that $[[T =a Pi x:t1.t2]]$ where $[[G |- (Pi x:t1.t2):s]]$ and $[[G,x:t1 |-
e:t2]]$.
    \item If $[[G |- (Pi x:t1.t2):T]]$, then $[[T == s]]$, $[[G |- t1:s]]$ and
$[[G, x:t1 |- t2:s]]$.
	\item If $[[G |- (mu x:t.e):T]]$, then $[[G |- t:s]]$, $[[T =a t]]$ and $[[G,
x:t|-e:t]]$.
	\item If $[[G |- (castup [t1] e):T]]$, then there exist an expression $[[t2]]$
such that $[[G |- e:t2]]$, $[[G |- t1:s]]$, $[[t1 --> t2]]$ and $[[T =a t1]]$.
	\item If $[[G |- (castdown e):T]]$, then there exist expressions
$[[t1]],[[t2]]$ such that $[[G |- e:t1]]$, $[[G |- t2:s]]$, $[[t1 --> t2]]$ and
$[[T =a t2]]$.
\end{enumerate}
\end{lem}

\begin{proof}
    Consider a derivation of $[[G |- e:T]]$ for one of cases in the lemma. We
follow the process of derivation until expression $[[e]]$ is introduced the
first time. The last step of derivation can be done by
    \begin{itemize}
        \item rule \ruleref{T\_Var} for case 1;
        \item rule \ruleref{T\_App} for case 2;
        \item rule \ruleref{T\_Lam} for case 3;
        \item rule \ruleref{T\_Pi} for case 4;
        \item rule \ruleref{T\_Mu} for case 5;
        \item rule \ruleref{T\_CastUp} for case 6;
        \item rule \ruleref{T\_CastDown} for case 7.
    \end{itemize}
    In each case, assume the conclusion of the rule is $[[G' |- e : t']]$ where
$[[G']] \subseteq [[G]]$ and $[[t' =a T]]$. Then by inspection of used
derivation rules and Lemma \ref{lem:appendix:thin}, it can be shown that the
statement of the lemma holds and is the only possible case.
\end{proof}

\begin{lem}[Correctness of Types]\label{lem:appendix:corrtyp}
    If $[[G |- e:t]]$ then $[[t == s]]$ or $[[G |- t : s]]$.
\end{lem}

\begin{proof}
    Trivial induction on the derivation of $[[G |- e:t]]$ using Lemma
\ref{lem:appendix:gen}.
\end{proof}

\subsection{Decidability of Type Checking}
\begin{lem}[Decidability of One-step Reduction]\label{lem:appendix:unired}
	The one-step reduction $[[-->]]$ is called decidable if 
given $[[e]]$ there is a unique $[[e']]$ such that $[[e --> e']]$ or there is no such $[[e']]$.
\end{lem}

\begin{proof}
	By induction on the structure of $[[e]]$:
	\begin{description}
        \item[Case $[[e=x]]$:] $[[e]]$ is a variable which does not match any rules of $[[-->]]$. 
        Thus there is no $[[e]]'$ such that $[[e-->e']]$.
		\item[Case $[[e=v]]$:] $[[e]]$ is a value that has one of the following forms:
		\begin{inparaenum}[(1)]
		    \item $[[star]]$,
			\item $[[\x:t.e]]$,
			\item $[[Pi x:t1.t2]]$,
			\item $[[castup [t] e]]$.
		\end{inparaenum}
		Thus, it does not match any rules of $[[-->]]$. Then there is no $[[e]]'$ such that $[[e-->e']]$.
		\item[Case $[[e]]=[[(\x:t.e1) e2]]$:] Since the first term $[[\x:t.e1]]$ is a value, rule \ruleref{S\_App} does not apply to this case. Thus, only rule \ruleref{S\_Beta} can be applied and there is a unique $[[e']]=[[ e1[x|->e2] ]]$.
		\item[Case $[[e]]=[[castdown (castup [t] e1)]]$:] Since the inner term $[[castup [t] e1]]$ is a value, rule \ruleref{S\_CastDown} does not apply to this case. Thus, only rule \ruleref{S\_CastDownUp} can be applied and there is a unique $[[e']]=[[e1]]$.
		\item[Case $[[e]]=[[mu x:t.e1]]$:] Only rule \ruleref{S\_Mu} can be applied. Thus, there is a unique $[[e]]'=[[e1[x|->mu x:t.e1] ]]$.
		\item[Case $[[e]]=[[e1 e2]]$ and $[[e1]]$ is not a $\lambda$-term:] If
$[[e1]]=v$ and is not a $\lambda$-term, there is no rule to reduce $[[e]]$. 
Then there is no $[[e1']]$ such that $[[e1 --> e1']]$, which does not satisfy the premise of 
rule \ruleref{S\_App}. Thus, there is no $[[e]]'$ such that $[[e-->e']]$.

		Otherwise, if $[[e1]]$ is not a value, there exists some $[[e1']]$ such that $[[e1 --> e1']]$. By the
induction hypothesis, $[[e1']]$ is the unique reduction of $[[e1]]$. Thus, by rule
\ruleref{S\_App}, $[[e]]'=[[e1' e2]]$ is the unique reduction of $[[e]]$.
		\item[Case $[[e]]=[[castdown e1]]$ and $[[e1]]$ is not a $[[castup]]$-term:] If
$[[e1]]=v$ and is not a $[[castup]]$-term, there is no rule to reduce $[[e]]$. 
Then there is no $[[e1']]$ such that $[[e1 --> e1']]$, which does not satisfy the premise of 
rule \ruleref{S\_CastDown}. Thus, there is no $[[e]]'$ such that $[[e-->e']]$.

        Otherwise, if $[[e1]]$ is not a value, there exists some $[[e1']]$ such that $[[e1 --> e1']]$. By the
induction hypothesis, $[[e1']]$ is the unique reduction of $[[e1]]$. Thus, by rule
\ruleref{S\_CastDown}, $[[e]]'=[[castdown e1']]$ is the unique reduction of $[[e]]$.
	\end{description}
\end{proof}

\begin{thm}[Decidability of Type Checking]
	There is an algorithm which given $[[G]], [[e]]$ computes the unique
$[[t]]$ such that $[[G |- e:t]]$ or reports there is no such $[[t]]$.
\end{thm}

\begin{proof}
	By induction on the structure of $[[e]]$:
	\begin{description}
	    \item[Case $[[e=star]]$:] Trivial by applying \ruleref{T\_Ax} and $[[t ==
star]]$.
		\item[Case $[[e=x]]$:] Trivial by rule \ruleref{T\_Var}. If $[[x:t elt G]]$, then $[[t]]$ is the
unique type of $[[x]]$ such that $[[G |- x : t]]$. Otherwise, if $[[x]] \not \in \dom([[G]])$, there is no such $[[t]]$.
		\item[Case $[[e]]=[[e1 e2]]$:] By rule \ruleref{T\_App} and induction
hypothesis, there exist unique $[[t1]]$ and $[[t2]]$ such that $[[G
|- e1 : (Pi x:t1.t2)]]$, $[[G |- e2:t1]]$. Thus, $[[t2[x |-> e2] ]]$ is the unique type of $[[e]]$ such that $[[G |- e : t2[x |-> e2] ]]$.
		\item[Case $[[e=\x:t1.e1]]$:] By rule \ruleref{T\_Lam} and induction
hypothesis, there exist unique $[[t2]]$ such that $[[G |- (Pi
x:t1.t2):s]]$ and $[[G,x:t1 |- e:t2]]$. Thus, $[[Pi x:t1.t2 ]]$ is the unique type of $[[e]]$ such that $[[G |- e : Pi x:t1.t2  ]]$.
		\item[Case $[[e=Pi x:t1.t2]]$:] By rule \ruleref{T\_Pi} and induction
hypothesis, we have $[[G |- t1:s]]$ and $[[G, x:t1 |- t2:s]]$. Thus, $[[s]]$ is the unique type of $[[e]]$ such that $[[G |- e : s  ]]$.
		\item[Case $[[e=mu x:t.e1]]$:] By rule \ruleref{T\_Mu} and induction
hypothesis, we have $[[G |- t:s]]$ and $[[G, x:t|-e:t]]$. Thus, $[[t]]$ is the unique type of $[[e]]$ such that $[[G |- e : t]]$.
		\item[Case $[[e]]=[[castup [t1] e1]]$:] From the premises of rule
\ruleref{T\_CastUp}, by the induction hypothesis, we can derive the type of
$[[e1]]$ as $[[t2]]$ by $[[G |- e1:t2]]$, and check whether $[[t1]]$ is legal by $[[G |- t1:star]]$. 
For a legal $[[t1]]$, by Lemma \ref{lem:appendix:unired}, there is
a unique $[[t1']]$ such that $[[t1 --> t1']]$ or there is no such $[[t1']]$. 
If such $[[t1']]$ does not exist, then we report type checking fails. 

Otherwise, we examine if $[[t1']]$ is syntactically equal to $[[t2]]$, 
i.e., $[[t1' =a t2]]$. If the equality
holds, we conclude the unique type of $[[e]]$ is $[[t1]]$, i.e., $[[G |- e:t1]]$. Otherwise, we
report $[[e]]$ fails to type check.
		\item[Case $[[e]]=[[castdown e1]]$:] From the premises of rule
\ruleref{T\_CastDown}, by the induction hypothesis, we can derive the type of
$[[e1]]$ as $[[t1]]$ by $[[G |- e1:t1]]$. By Lemma \ref{lem:appendix:unired}, there is a unique
$[[t2]]$ such that $[[t1 --> t2]]$ or such $[[t2]]$ does not exist. 

If such $[[t2]]$ exists and its sorts is
$[[star]]$, we find the unique type of $[[e]]$ is $[[t2]]$ and can conclude $[[G |- e:t2]]$. Otherwise, we
report $[[e]]$ fails to type check.
	\end{description}
\end{proof}

\subsection{Type-safety}
\begin{dfn}[Multi-step reduction]
    The relation $[[->>]]$ is the transitive and reflexive closure of
$[[-->]]$.
\end{dfn}

\begin{dfn}[$n$-step reduction]
    The $n$-step reduction is denoted by $[[e0]] [[-->>]] [[en]]$, if
    there exists a sequence of one-step reductions $[[e0]] [[-->]]
    [[e1]] [[-->]] [[e2]] [[-->]] \dots [[-->]] [[en]]$, where $n$ is
    a positive integer and $[[ei]]\,(i=0,1,\dots,n)$ are valid
    expressions.
\end{dfn}

\begin{thm}[Subject Reduction]
If $[[G |- e:T]]$ and $[[e]] [[->>]] e'$ then $[[G |- e':T]]$.
\end{thm}

\begin{proof}
    We prove the case for one-step reduction, i.e., $[[e --> e']]$. The theorem
follows by induction on the number of one-step reductions of $[[e]] [[->>]]
[[e']]$.
    The proof is by induction with respect to the definition of one-step
reduction $[[-->]]$ as follows:
    \begin{description}
        \item[Case $\ottdruleSXXBeta{}$:] $\quad$ \\
        Suppose $[[G |- (\x:t1.e1)e2 :T]]$ and $[[G |- e1 [x |-> e2] :T']]$. By
Lemma \ref{lem:appendix:gen}(2), there exist expressions $[[t1']]$ and $[[t2]]$
such that 
        \begin{align}
            &[[G |- (\x:t1.e1):(Pi x:t1'.t2)]] \label{equ:lam} \\
            &[[G |- e2:t1']] \nonumber \\
            &[[T =a t2 [x |-> e2] ]] \nonumber
        \end{align}
        By Lemma \ref{lem:appendix:gen}(3), the judgement (\ref{equ:lam})
implies that there exists an expression $[[t2']]$ such that
        \begin{align}
            &[[Pi x:t1'.t2 =a Pi x:t1.t2']] \label{equ:lameq}\\
            &[[G, x:t1 |- e1:t2']] \nonumber
        \end{align}
        Hence, by (\ref{equ:lameq}) we have $[[t1 =a t1']]$ and $[[t2 =a
t2']]$. Then we can obtain $[[G, x:t1 |- e1:t2]]$ and $[[G |- e2:t1]]$. By
Lemma \ref{lem:appendix:subst}, we have $[[G |- e1[x |-> e2] : t2[x |-> e2]
]]$. Therefore, we conclude with $[[T' =a t2[x |-> e2] ]] [[=a]] [[T]]$.
        
        \item[Case $\ottdruleSXXApp{}$:] $\quad$ \\
        Suppose $[[G |- e1 e2 :T]]$ and $[[G |- e1' e2 :T']]$. By Lemma
\ref{lem:appendix:gen}(2), there exist expressions $[[t1]]$ and $[[t2]]$ such
that 
        \begin{align*}
            &[[G |- e1:(Pi x:t1.t2)]] \\
            &[[G |- e2:t1]]\\
            &[[T =a t2 [x |-> e2] ]]
        \end{align*}
        By the induction hypothesis, we have $[[G |- e1':(Pi x:t1.t2)]]$. By rule
\ruleref{T\_App}, we obtain $[[G |- e1' e2 : t2[x |-> e2] ]]$. Therefore, $[[T'
=a t2[x |-> e2] ]] [[=a]] [[T]]$.
        
        \item[Case $\ottdruleSXXCastDown{}$:] $\quad$ \\
        Suppose $[[G |- castdown e :T]]$ and $[[G |- castdown e' :T']]$. By
Lemma \ref{lem:appendix:gen}(7), there exist expressions $[[t1]], [[t2]]$ such
that 
        \begin{align*}
            &[[G |- e:t1]] \qquad [[G |- t2:s]] \\
            &[[t1 --> t2]] \qquad [[T =a t2 ]]
        \end{align*}
        By the induction hypothesis, we have $[[G |- e':t1]]$. By rule
\ruleref{T\_CastDown}, we obtain $[[G |- castdown e' : t2 ]]$. Therefore, $[[T'
=a t2]] [[=a]] [[T]]$.
        
        \item[Case $\ottdruleSXXCastDownUp{}$:] $\quad$ \\
        Suppose $[[G |- castdown (castup [t1] e) :T]]$ and $[[G |- e :T']]$. By
Lemma \ref{lem:appendix:gen}(7), there exist expressions $[[t1']], [[t2]]$ such
that 
        \begin{align}
            &[[G |- (castup [t1] e):t1']] \label{equ:fold} \\
            &[[t1' --> t2]] \label{equ:foldeq1} \\
            &[[T =a t2 ]] \label{equ:foldeq4}
        \end{align}
        By Lemma \ref{lem:appendix:gen}(6), the judgement (\ref{equ:fold})
implies that there exists an expression $[[t2']]$ such that
        \begin{align}
            &[[G |- e:t2']] \label{equ:foldr} \\
            &[[t1 --> t2']] \label{equ:foldeq2} \\
            &[[t1' =a t1]] \label{equ:foldeq3}
        \end{align}
        By (\ref{equ:foldeq1}, \ref{equ:foldeq2}, \ref{equ:foldeq3}) and Lemma
\ref{lem:appendix:unired} we obtain $[[t2 =a t2']]$. From (\ref{equ:foldr}) we
have $[[T' =a t2' ]]$. Therefore, by (\ref{equ:foldeq4}), $[[T' =a t2' ]]
[[=a]] [[t2 =a T]]$.
        
        \item[Case $\ottdruleSXXMu{}$:] $\quad$ \\
        Suppose $[[G |- (mu x:t.e) :T]]$ and $[[G |- e[x |-> mu x:t.e] :T']]$.
By Lemma \ref{lem:appendix:gen}(5), we have $[[T =a t]]$ and $[[G, x:t |-
e:t]]$. Then we obtain $[[G |- (mu x:t.e) : t]]$. Thus by Lemma
\ref{lem:appendix:subst}, we have $[[G |- e[x |-> mu x:t.e] : t[x |-> mu x:t.e]
]]$.
        
        Note that $[[x]]:[[t]]$, i.e., the type of $[[x]]$ is $[[t]]$, then
$[[x]] \notin \FV([[t]])$ holds implicitly. Hence, by the definition of
substitution, we obtain $[[t[x |-> mu x:t.e] == t]]$. Therefore, $[[T' =a t[x
|-> mu x:t.e] ]] [[==]] [[t =a T]]$.
    \end{description}
\end{proof}

\begin{thm}[Progress]
If $[[empty |- e:T]]$ then either $[[e]]$ is a value $v$ or there exists $[[e]]'$
such that $[[e --> e']]$.
\end{thm}

\begin{proof}
    By induction on the derivation of $[[empty |- e:T]]$ as follows:
    \begin{description}
        \item[Case $[[e=x]]$:] Impossible, because the context is empty.
        \item[Case $[[e=v]]$:] Trivial, since $[[e]]$ is already a value that
has one of the following forms:
		\begin{inparaenum}[(1)]
		    \item $[[star]]$,
			\item $[[\x:t.e]]$,
			\item $[[Pi x:t1.t2]]$,
			\item $[[castup [t] e]]$.
		\end{inparaenum}
		\item[Case $[[e]]=[[e1 e2]]$:] By Lemma \ref{lem:appendix:gen}(2), there
exist expressions $[[t1]]$ and $[[t2]]$ such that $[[empty |- e1:(Pi x:t1.t2)]]$ and
$[[empty |-e2:t1]]$. Consider whether $[[e1]]$ is a value:
    		\begin{itemize}
    		    \item If $[[e1]]=v$, by Lemma \ref{lem:appendix:gen}(3), it must be a
$\lambda$-term such that $[[e1 == \x:t1.e1']]$ for some $[[e1']]$ satisfying
$[[empty |- e1':t2]]$. Then by rule \ruleref{S\_Beta}, we have $[[(\x:t1.e1') e2 -->
e1' [x |-> e2] ]]$. Thus, there exists $[[e' == e1' [x |-> e2] ]]$ such that
$[[e --> e']]$.
    		    \item Otherwise, by the induction hypothesis, there exists $[[e1']]$ such
that $[[e1 --> e1']]$. Then by rule \ruleref{S\_App}, we have $[[e1 e2 --> e1'
e2]]$. Thus, there exists $[[e' == e1' e2]]$ such that $[[e --> e']]$.
    		\end{itemize}
		\item[Case $[[e]]=[[castdown e1]]$:] By Lemma \ref{lem:appendix:gen}(7),
there exist expressions $[[t1]]$ and $[[t2]]$ such that $[[empty |- e1:t1]]$ and
$[[t1 --> t2]]$. Consider whether $[[e1]]$ is a value:
		     \begin{itemize}
    		    \item If $[[e1]]=v$, by Lemma \ref{lem:appendix:gen}(6), it must be a
$[[castup]]$-term such that $[[e1 == castup [t1] e1']]$ for some $[[e1']]$
satisfying $[[empty |- e1':t2]]$. Then by rule \ruleref{S\_CastDownUp}, we can obtain
$[[castdown (castup [t1] e1') --> e1']]$. Thus, there exists $[[e' == e1']]$
such that $[[e --> e']]$.
    		    \item Otherwise, by the induction hypothesis, there exists $[[e1']]$ such
that $[[e1 --> e1']]$. Then by rule \ruleref{S\_CastDown}, we have $[[castdown
e1 --> castdown e1']]$. Thus, there exists $[[e' == castdown e1']]$ such that
$[[e --> e']]$.
    		\end{itemize}
		\item[Case $[[e]]=[[mu x:t.e1]]$:] By rule \ruleref{S\_Mu}, there always
exists $[[e' == e1[x |-> mu x:t.e1] ]]$.
    \end{description}
\end{proof}

\section{Full Specification of Surface Language}
\subsection{Syntax}
See Figure \ref{fig:appendix:syntax}.
\begin{figure*}
\centering
\gram{\ottpgm\ottinterrule
\ottdecl\ottinterrule
\ottu\ottinterrule
\ottp\ottinterrule
\ottE\ottinterrule
\ottGs}
\begin{align*}
&\text{Syntactic Sugar} \\
&\ottsurfsugar % defined in otthelper.mng.tex
\end{align*}
\caption{Syntax of the surface language}
\label{fig:appendix:syntax}
\end{figure*}

\subsection{Expression Typing}
See Figure \ref{fig:appendix:typing}.

\subsection{Translation to the Core}
See Figure \ref{fig:appendix:translate}.

\section{Proofs about Surface Language}
\subsection{Type-safety of Translation}

\begin{thm}[Type-safety of Expression Translation]
Given a surface language expression $[[E]]$ and context $[[Gs]]$, 
if $[[Gs |- E:A ~> e]]$, $[[Gs |- A:star ~> t]]$ and $[[|- Gs ~> G]]$, then
$[[G |- e:t]]$.
\end{thm}

\begin{proof}
    By induction on the derivation of $[[Gs |- E : A ~> e]]$. Suppose there is
a core language context $[[G]]$ such that $[[|- Gs ~> G]]$.
    \begin{description}
        \renewcommand{\hlmath}[1]{#1}
        \item[Case $\ottdruleTRXXAx{}$:] $\quad$ \\ Trivial. $[[e]] = [[t]] = [[star]]$ and
$[[G |- star:star]]$ holds by rule \ruleref{T\_Ax}.
        \item[Case $\ottdruleTRXXVar{}$:] $\quad$ \\ Trivial. By rule \ruleref{T\_Var}, we
have $[[|- Gs ~> G]]$, then $[[x]]:[[t]] [[elt]] [[G]]$ where $[[Gs |-
A:star~>t]]$.
        \item[Case \resizebox{.9\columnwidth}{!}{$\ottdruleTRXXApp{}$}:] $\quad$ \\ Suppose
            \[\begin{array}{l}
            [[Gs |- E1 E2 : A1[x |-> E2] ~> e1 e2]] \\
            [[Gs |- A1[x |-> E2] : star ~> t1 [x |-> e2] ]].
            \end{array} \]
            By induction
            hypothesis, we have 
            $
            [[G |- e1 : (Pi x:t2.t1)]],
            [[G |- e2:t2]],
            $
            where
            \[\begin{array}{l}
             [[Gs |- E1 : (Pi x:A2.A1) ~> e1]] \\
              [[Gs |- (Pi x:A2.A1) : star ~> (Pi x:t2.t1)]] \\
              [[Gs |- E2 : A2 ~> e2]] \\
              [[Gs |- A2 : star ~> t2]].
            \end{array}\] Thus by rule \ruleref{T\_App}, we can conclude $[[G |- e1 e2 : t1 [x |-> e2] ]]$.
        \item[Case $\ottdruleTRXXLam{}$:] $\quad$ \\ Suppose
            \[\begin{array}{l}
            [[Gs |- (\x:A1.E):(Pi x:A1.A2) ~> \x:t1.e]] \\ 
            [[Gs |- Pi x:A1.A2 : star ~> Pi x:t1.t2]].
            \end{array} \]
            By the induction hypothesis, we have 
            $
            [[G, x : t1 |- e:t2]],
            [[G |- Pi x:t1.t2 : star]]
            $
            where 
            \[
            \begin{array}{ll}
            [[Gs, x : A1 |- E : A2 ~> e]] & \\
            [[Gs |- A1 : star ~> t1]] & [[Gs |- A2 : star ~> t2]] \\
            [[Gs |- (Pi x:A1.A2) : s ~> Pi x:t1.t2]] &
            \end{array}
            \]
            Thus by rule \ruleref{T\_Lam}, we can conclude $[[G |- (\x:t1.e):(Pi x:t1.t2)]]$.
        \item[Case $\ottdruleTRXXPi{}$:] $\quad$ \\ Suppose 
                \[ [[Gs |- (Pi x:A1.A2):r ~> Pi x:t1.t2]]. \] 
            By the induction hypothesis, we have 
            $
                [[G |- t1 : star]], [[G, x : t1 |- t2 : star]]
            $
            where
            $
                [[Gs |- A1 : s ~> t1]], [[Gs, x: A1 |- A2 : r ~> t2]]
            $
            Thus by rule \ruleref{T\_Pi} we can conclude $[[G |- (Pi x:t1.t2) : star]]$.
        \item[Case $\ottdruleTRXXMu{}$:] $\quad$ \\ Suppose 
                \[\begin{array}{l}
                    [[Gs |- (mu x:A . E):A ~> mu x:t.e]] \\
                    [[Gs |- A : star ~> t]]. 
                \end{array}\]
            By the induction hypothesis, we have 
                \[ [[G, x : t |- e : t]],\text{ where }[[Gs, x:A |- E:A ~> e]]. \] 
            Thus by rule \ruleref{T\_Mu}, we can conclude $[[G |- (mu x:t.e) : t]]$.
        \item[Case \resizebox{.9\columnwidth}{!}{$\ottdruleTRXXCase{}$}:] $\quad$ \\ Suppose 
            \[\begin{array}{l}
                [[Gs |- case E1 of << p => E2>> : B ~> (unfoldnp e1) T <<e2>>]] \\
                [[Gs |- B : star ~> T]].
            \end{array}\]
            By the induction hypothesis, we have 
            \[\begin{array}{ll}
                [[Gs |- E1 : D@<<U>>n ~> e1]] &
                [[Gs |- D@<<U>>n : star ~> t1]] \\
                [[G |- e1 : t1]] &
                [[<< Gs |- p => E2 : D@<<U>>n -> B ~> e2 >>]]            
            \end{array}\]
            By rule \ruleref{TRpat\_Alt}, we have
            \begin{align*}
                [[p]] &[[==]] [[K <<x:A[<< u |-> U >>]>>]] \\
                [[<<e2>>]] &[[==]] [[<<\ <<x:t'>> .e>>]]
            \end{align*}
            where
            \[\begin{array}{ll}
                [[<<Gs |- E2 : B ~> e>>]] &
                [[<<G |- e : T>>]] \\
                [[<<Gs |- U : star ~> uu'>>]] &
                [[<<Gs |- A[<< u |-> U >>]:star ~> t[<<uu |-> uu'>>]>>]] \\
                [[t']] [[==]] [[ t[<<uu |-> uu'>>] ]]
            \end{array}\]
            By rule \ruleref{TRdecl\_Data}, we have $[[D]]  [[ == ]] \ottdeclD$. Thus,
            \[ [[t1]] [[==]] [[D]] [[<<uu'>>]]^n,\text{ where }[[<<G |- uu' : ro>>]].\] 
            Note that by operational semantics of \name, the following reduction sequence follows for $[[t1]]$:
            \begin{align*}
                [[D]] [[<<uu'>>]]^n~
                &[[-->]]~ \mathscale[0.7]{[[(\ <<u:ro>>n . (bb:star) -> << ((<<x : t[D |-> X][X |-> D]>>) -> bb) >> -> bb) ]][[<<uu'>>]]^n}\\
                &[[-->>]]~ [[(bb:star) -> << (<<x:t'>>) -> bb >> -> bb]]
            \end{align*}
            Then by
            rule \ruleref{T\_CastDown} and the definition of $n$-step cast operator, the
            type of $[[unfoldnp e1]]$ is \[ [[(bb:star) -> << (<<x:t'>>) -> bb >> -> bb]].\] Note
            that by rule \ruleref{T\_Lam}, $[[G |- e2 : (<<x:t'>>) -> T]]$. Therefore, by rule
            \ruleref{T\_App}, we can conclude $[[G |- (unfoldnp e1) T <<e2>> : T]]$.
    \end{description}
\end{proof}

\begin{figure*}
\renewcommand{\hlmath}[1]{}
\renewcommand{\ottdrulename}[1]{\textsc{\replace{#1}{TR}{TS}}}
\renewcommand{\ottcom}[1]{\text{\replace{#1}{translation}{typing}}}
\ottdefnctxtrans{}\ottinterrule
\ottdefnpgmtrans{}\ottinterrule
\ottdefndecltrans{}\ottinterrule % defined in otthelper.mng.tex
\ottdefnpattrans{}\ottinterrule
\ottdefnexprtrans{}
\caption{Typing rules of the surface language}
\label{fig:appendix:typing}
\end{figure*}

\begin{figure*}
\ottdefnctxtrans{}\ottinterrule
\ottdefnpgmtrans{}\ottinterrule
\ottdefndecltrans{}
\[\hlmath{\ottdecltrans}\]\ottinterrule % defined in otthelper.mng.tex
\ottdefnpattrans{}\ottinterrule
\ottdefnexprtrans{}
\caption{Translation rules of the surface language}
\label{fig:appendix:translate}
\end{figure*}

