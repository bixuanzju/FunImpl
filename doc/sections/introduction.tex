\section{Introduction}

Modern statically typed functional languages (such as ML, Haskell,
Scala or OCaml) have increasingly expressive type systems. Often these
large source languages are translated into a much smaller typed core
language. The choice of the core language is essential to ensure that
all the features of the source language can be encoded. For a simple
polymorphic functional language it is possible, for example, to pick a
variant of System $F$~\cite{systemfw,Reynolds:1974} as a core
language. However, the desire for more expressive type system features
puts pressure on the core languages, often requiring them to be
extended to support new features.  For example, if the source language
supports \emph{higher-kinded types} or \emph{type-level functions}
then System $F$ is not expressive enough and can no longer be used as
the core language. Instead another core language that does provide
support for higher-kinded types, such as
System~$F_{\omega}$~\cite{systemfw}, needs to be used. Of course the
drive to add more and more advanced type-level features means that
eventually the core language needs to be extended again. Indeed modern
functional languages like Haskell use specially crafted core
languages, such as System $F_{C}$~\cite{fc}, that provide support for all
modern features of Haskell. Although \emph{extensions} of System
$F_{C}$~\cite{fc:pro,Eisenberg:2014} satisfy the current needs of
modern Haskell, it is very likely to be extended again in the
future~\cite{fc:kind}. Moreover System $F_{C}$ has grown to be a relatively
large and complex language, with multiple syntactic levels, and dozens
of language constructs.

\begin{comment}
However System~$F_{\omega}$ is
significantly more complex than System F and thus harder to
maintain. If later a new feature, such as \emph{kind polymorphism}, is
desired the core language may need to be changed again to account for
the new feature, introducing at the same time new sources of
complexity. Indeed the core language for modern versions of 
functional languages are quite complex, having multiple syntactic 
sorts (such as terms, types and kinds), as well as dozens of 
language constructs~\cite{}\bruno{$F_{C}$}. 
\end{comment}

The more expressive type (and kind) systems become, the more types become similar
to the terms. Therefore a natural idea is to unify terms and
types. There are obvious benefits in this approach: only one syntactic
level (terms) is needed; and there are much less language constructs,
making the core language easier to reason, implement and maintain. At the same
time the core language becomes more expressive, giving us for free
many useful language features. Moreover, due to the inherent
expressiveness, extensions are less likely to be required.
\emph{Pure type systems} (PTS)~\cite{handbook} build
on such observations and show how a whole family of type systems
(including System $F$ and System $F_{\omega}$) can be implemented
using just a single syntactic form. With the added expressiveness it
is even possible to have type-level programs expressed using the same
syntax as terms, as well as dependently typed programs~\cite{coc}.

However having the same syntax for types and terms can also be
problematic. Usually type systems based on PTS have a conversion rule
to support type-level computation.  In such type systems ensuring the
\emph{decidability} of type checking requires type-level computation
to terminate. When the syntax of types and terms is the same, the
decidability of type-checking is usually dependent on the strong
normalization of the calculus. An example is the proof of decidability
of type-checking for the \emph{calculus of constructions}~\cite{coc}
(and other normalizing PTS), which depends on strong normalization
~\cite{pts:normalize}.  Modern dependently
typed languages such as Idris~\cite{idris} and Agda~\cite{agda}, which also
built on a unified syntax for types and terms, require strong
normalization as well: all recursive programs must pass a termination
checker.  An unfortunate consequence of coupling
decidability of type-checking and strong normalization is that adding
(unrestricted) general recursion to such calculi is difficult. There
is a clear tension between decidability of type-checking and allowing
general recursion in calculi with unified syntax.

This paper proposes \name: a variant of the calculus of constructions
that allows the same syntax for types and terms, supports type-level
computation, and preserves decidability of type-checking under the
presence of general recursion. In \name, each type-level computation
step is \emph{explicit}.  The key idea, which is inspired by the traditional
treatment of \emph{iso-recursive types}~\cite{tapl}, is to introduce each beta
reduction or expansion at the type-level by a \emph{type-safe
  cast}. The
casts allow control over the type-level computation. For example, if
a type-level program requires two beta-reductions to reach normal
form, then two casts are needed in the program. If a non-terminating
program is used at the type-level, it is not possible to cause
non-termination in the type-checker, because that would require a
program with an infinite number of casts. Therefore, since single beta-steps are
trivially terminating, decidability of type-checking is possible even
in the presence of non-terminating programs at the type-level. 
At the same time term-level programs using general recursion 
work as in any conventional functional languages, and can even 
be non-terminating. 

Our motivation to develop \name is to use it as a simpler alternative
to existing core languages for languages such as Haskell. 
The paper shows how many of programming language features of
Haskell, including some of the latest extensions, can be encoded in
\name via a surface language. In particular the surface language
supports \emph{algebraic datatypes}, \emph{higher-kinded types},
\emph{nested datatypes}~\cite{nesteddt}, \emph{kind polymorphism}~\cite{fc:pro} and
\emph{datatype promotion}~\cite{fc:pro}.  This result is interesting because
\name is a minimal calculus with only 8 language constructs and a
single syntactic sort. In contrast the latest versions of System
$F_{C}$ (Haskell's core language) have multiple syntactic sorts and
dozens of language constructs.  
%Even if support for equality and
%coercions, which constitutes a significant part of System $F_{C}$,
%would be removed the resulting language would still be significantly
%larger and more complex than \name.

It is worth noting that \name does sacrifice the convinience of use of type-level computation
to gain the ability of doing arbitrary general recursion at the term
level.  For this reason, we believe \name is particularly well-suited
as a core for Haskell-like languages, where the core language (System
$F_{C}$) also comes with a similarly weak notion of type-equality.  In
both System $F_{C}$ and \name, type-equality in \name is purely
syntactic (modulo alpha-conversion).

A non-goal of the current work (although a worthy avenue for future
work) is to use \name as a core language for modern dependently typed
languages like Agda or Idris. In contrast to \name, those languages
use a much more powerful notion equality based on $\beta$-equality.  Moreover, an
additional concern, which does not exist in traditional functional
languages like Haskell, is how to ensure \emph{logical consistency}:
that is ensuring the soundness of proofs written as programs. Both
\name and System $F_{C}$ not logically consistent.  
Various researchers~\cite{zombie:popl14,zombie:thesis}\bruno{more
  references?}  have been investigating how to combine logical
consistency, general recursion and dependent types. However, this is
usually done by having the type system carefully control the total and
partial parts of computation, which adds complexity to the system. In
\name, logical consistency is traded by the simplicity of the system.

\begin{comment}
In particular
the treatment of type-level computation in \name shares similar ideas
with Haskell. Although Haskell's surface language provides a rich set
of mechanisms to do type-level computation~\cite{}, the core language
lacks fundamental mechanisms todo type-level computation. Type
equality in System $F_{C}$ is, like in \name, purely syntactic (modulo
alpha-conversion).
\end{comment}

\begin{comment}
 and there is no type-level
abstraction. In other words in Haskell, mechanisms such as type
classes and type families

Although it may seem that forcing each step of computation 
at the type-level to be explicit will prevent convinient use of 
type-level computation.

Point about the treatment of type-level computation in Haskell. Haskell's
core language has type applications, but no type-level lambda. Equality 
is syntactic modulo alpha-conversion. This design choice was rooted in the 
desire to support Hindley-Milner type-inference... 
\end{comment}

In summary, the contributions of this work are:

\begin{itemize}
\item {\bf Explicit type casts:} An alternative to the conversion rule
  that allows decoupling decidability of type-checking from strong
  normalization in PTS-like systems.
%Using explicit type casts, decidability of
%  type-checking in \name does not depend on the strong normalization
%  of the calculus.

\item {\bf A generalization of iso-recursive types:} \name introduces 
  a single recursion operator that serves two purposes:
  general recursion on terms; and recursive types. The combination of
  explicit type casts and recursion generalizes iso-recursive types.

\item {\bf Formalization of \name:} a core
  language based on Calculus of Constructions (CoC) that collapses
  terms, types and kinds into the same hierarchy and supports general
  recursion. \name is type-safe and the type system is decidable.

\item {\bf An expressive surface language}, built on top of \name,
  that supports datatypes, pattern matching and various advanced
  language extensions of Haskell. The type-safety of the type-directed
  translation to \name is proved.

\item {\bf A prototype implementation:} \name has an
  implementation, which is publicly available.
\end{itemize}

\begin{comment}
\begin{enumerate}[a)]
\item Motivations:

\begin{itemize}

\item Because of the reluctance to introduce dependent
  types\footnote{This might be changed in the near future. See
    \url{https://ghc.haskell.org/trac/ghc/wiki/DependentHaskell/Phase1}.},
  the current intermediate language of Haskell, namely System $F_C$
  \cite{fc}, separates expressions as terms, types and kinds, which
  brings complexity to the implementation as well as further
  extensions \cite{fc:pro,fc:kind}.

\item Popular full-spectrum dependently typed languages, like Agda,
  Coq, Idris, have to ensure the termination of functions for the
  decidability of proofs. No general recursion and the limitation of
  enforcing termination checking make such languages impractical for
  general-purpose programming.

\item We would like to introduce a simple and compiler-friendly
  dependently typed core language with only one hierarchy, which
  supports general recursion at the same time.

\end{itemize}

\item Contribution:

\begin{itemize}

\item A core language based on Calculus of Constructions (CoC) that
  collapses terms, types and kinds into the same hierarchy.

\item General recursion by introducing recursive types for both terms
  and types by the same $\mu$ primitive.

\item Decidable type checking and managed type-level computation by
  replacing implicit conversion rule of CoC with generalized
  \textsf{fold}/\textsf{unfold} semantics.

\item First-class equality by coercion, which is used for encoding
  GADTs or newtypes without runtime overhead.

\item Surface language that supports datatypes, pattern matching and
  other language extensions for Haskell, and can be encoded into the
  core language.

\end{itemize}


\end{enumerate}
\end{comment}
