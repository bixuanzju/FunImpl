% Hack to use mathpartir for ott
\newcommand{\ottlinebreak}{}
\renewcommand{\ottdrule}[4][]{{\inferrule{#2 }{#3}\quad\ottdrulename{#4}}}
\renewcommand{\ottrulehead}[3]{ $#1$  & $#2$ &  & \multicolumn{2}{l}{#3}}
\renewcommand{\ottprodline}[6]{& $#1$ & $#2$ & $#5$ & $#6$}
\renewcommand{\ottgrammartabular}[1]{\begin{tabular}{lclllll}#1\end{tabular}}
\renewcommand{\ottinterrule}{\\[2.0mm]}
\newcommand{\gram}[1]{\ottgrammartabular{#1\ottafterlastrule}}
\newcommand{\ruleref}[1]{\ottdrulename{#1}}

\newcommand{\castup}{$[[castup]]$\xspace}
\newcommand{\castdown}{$[[castdown]]$\xspace}
\newcommand{\cast}{cast\xspace}

\newcommand{\lam}[3]{[[\]] #1:#2.\,#3}
\newcommand{\pai}[3]{[[Pi]] #1:#2.\,#3}
\newcommand{\miu}[3]{[[mu]] #1:#2.\,#3}

\newcommand{\FV}{\mathsf{FV}}
\newcommand{\dom}{\mathsf{dom}}

\newcommand*{\mathscale}[2][4]{\scalebox{#1}{$#2$}}%

\def\replace#1#2#3{%
 \def\tmp##1#2{##1#3\tmp}%
   \tmp#1\stopreplace#2\stopreplace}
\def\stopreplace#1\stopreplace{}

\newcommand{\ottcoresugar}{%%
\begin{array}{llll} %
    [[t1->t2]] & \triangleq & [[Pi x:t1.t2]],~\text{where}~x \not \in %
        \FV([[t2]]) \\ %
    [[let x : t = e2 in e1]] & [[:=]] & [[ e1 [x |-> e2] ]] \\ %
    [[foldn [t1] e]] & [[:=]] & [[castup]] [ [[t1]] ] ([[castup]] [ %
        [[t2]] ] (\dots ( [[castup [tn] e]] ) \dots )) \\ %
    [[unfoldn e]] & [[:=]] & \underbrace{[[castdown]] ([[castdown]] %
        (\dots ( [[castdown]]}_n [[e]]) \dots ))
\end{array} %
}

\newcommand{\ottsurfsugar}{%%
\begin{array}{lllll} %
    &&&& \text{Syntactic Sugar} \\ %
    [[A1->A2]] & \triangleq & [[Pi x:A1.A2]] & x \not \in \FV([[A2]]) & %
        \quad\text{Function Type} \\ %
    [[(x:A1)->A2]] & \triangleq & [[Pi x:A1.A2]] &  & %
        \quad\text{Dependent Function Type} \\ %
    [[letrec x : A = E2 in E1]] & \triangleq & [[let x : A = mu x : A . E2 in E1]] &  & %
        \quad\text{Recursive Let Binding}\\ %
    [[data R <<u:k>>n = K { <<N:A>> }]] & \triangleq & %
        [[data R <<u:k>>n = K <<x:A>>]]; &\forall i & \quad\text{Record} \\ %
        && [[let]]~[[N]]_i : [[ (<<u:k>>n) -> R@<<u>>n -> A]]_i = & %
         &  \\ %
        && [[\ <<u:k>>n . \ y : R@<<u>>n . case y of K <<x:A>> => x]]_i~[[in]] && \\ %
\end{array} %
}

\newcommand{\ottdeclD}{%%
[[mu X : (<<u:ro>>n) -> star . \ <<u:ro>>n . (bb:star) -> << ((<<x : t[D |-> X]>>) -> bb) >> -> bb]] %
}

\newcommand{\ottdeclK}{%%
[[\ <<u:ro>>n . \<<x:t>> . foldnp [D@<<u>>n] (\bb:star.\<<cc : (<<x:t>>) -> bb >> . cci <<x>>)]] %
}

\newcommand{\ottdecltrans}{%%
\begin{array}{ll} %
    [[e]] [[:=]] & [[ let ]]~[[D]] : [[(<<u:ro>>n) -> star]] = \ottdeclD~[[ in ]] \\ %
    & [[ let ]]~[[Ki]] : [[ (<<u:ro>>n) -> (<<x:t>>) -> D@<<u>>n ]] = \ottdeclK~[[ in ]] %
\end{array} %
}

\newcommand{\ottdruleTRXXCaseT}[1]{\ottdrule[#1]{%
{\,\overline{   \Sigma  \labeledjudge{p}  \ottnt{p}   \Rightarrow   \ottnt{E_{{\mathrm{2}}}}  :  T_1   \rightarrow   S  \hlmath{ \rightsquigarrow   \ottnt{e_{{\mathrm{2}}}} }   }\,}%
\\ { \Sigma  \labeledjudge{s}  \ottnt{E_{{\mathrm{1}}}}  :  T_1  \hlmath{ \rightsquigarrow   \ottnt{e_{{\mathrm{1}}}} } }%
\\ { \Sigma  \labeledjudge{s}  S  :  \star  \hlmath{ \rightsquigarrow   \sigma } }%
}{
 \Sigma  \labeledjudge{s}  \kw{case} \, \ottnt{E_{{\mathrm{1}}}} \, \kw{of} \, \,\overline{  \ottnt{p}  \Rightarrow  \ottnt{E_{{\mathrm{2}}}}  }\,  :  S  \hlmath{ \rightsquigarrow   \ottsym{(}  \mathsf{cast}_{\downarrow}^{n+1} \, \ottnt{e_{{\mathrm{1}}}}  \ottsym{)} \, \sigma \, \,\overline{  \ottnt{e_{{\mathrm{2}}}}  }\, } }{%
{\ottdrulename{TR\_Case}}{}%
}}
