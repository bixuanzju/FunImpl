\section{The Explicit Calculus of Constructions}
\label{sec:formal}

\bruno{Linus: can you write up this section? I think this section should be your priority.
First bring in all results and formalization: syntax; semantics; proofs ... then write text}

This section formalizes the syntax and semantics of the explicit calculus 
of constructions. This section also shows that how in the explicit 
calculus of constructions decidability of the type system does not 
depend on strong normalization.

\begin{itemize}
\item Give an overview of the core language and its syntax.
\item Show the typing rules and operational semantics.
\item The original formalization is suggested to rewrite using \textsf{ott}\footnote{\url{http://www.cl.cam.ac.uk/~pes20/ott/}} which is a standard in academia. For example, the formalization of GHC \url{https://github.com/ghc/ghc/tree/master/docs/core-spec}.
\item Give formal proof of the soundness of the core language.
\item Subject reduction and progress theorems will be proved.
\end{itemize}

\newcommand{\gram}[1]{\ottgrammartabular{#1\ottafterlastrule}}

\begin{figure}[ht]
    \gram{\otte\ottinterrule
          \otts\ottinterrule
          \ottG\ottinterrule
          \ottv}
    \caption{Syntax}
\end{figure}

\begin{figure}[ht]
    \ottdefnstep{}
    \caption{Dynamic semantics}
\end{figure}

\begin{figure}[ht]
    \ottdefnexpr{}
    \caption{Typing rules}
\end{figure}
