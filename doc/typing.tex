The extended typing rules are shown in Figure~\ref{fig:mutype}. Compared
with $\lambda C$, the original \emph{Conv} rule is replaced by the new
\emph{Beta} rule where the latter only performs one step of reduction defined in Figure~\ref{fig:mueval}.

\begin{figure}[ht]
  \centering
  \small
  \begin{tabular}{lcl}
    (Ax) & \ruleI{}{\ctxz{\star:\square}} \\

    (Var) & \ruleI{\ctx{T:s}}{\ctxw{x:T}{x:T}}
          & $x \not \in \mathrm{dom}(\Gamma)$ \\

    (Weak) & \ruleII{\ctx{E:T_{2}}}{\ctx{T_{1}:s}}{\ctxw{x:T_{1}}{E:T_{2}}}
           & $x \not \in \mathrm{dom}(\Gamma)$ \\

    (App) & \ruleII{\ctx{E_{1}:(\pai{x}{T_{2}}{T_{1}})}}{\ctx{E_{2}:T_{2}}}{\ctx{E_{1}E_{2}:T_{1}[x:=E_{2}]}} \\

    (Lam) & \ruleII{\ctxw{x:T_{1}}{E:T_{2}}}{\ctx{(\pai{x}{T_{1}}{T_{2}}):t}}
                   {\ctx{(\lam{x}{T_{1}}{E}}):(\pai{x}{T_{1}}{T_{2}})}
          & $t \in \{\star, \square\}$ \\

    (Pi) & \ruleII{\ctx{T_{1}:s}}{\ctxw{x:T_{1}}{T_{2}:t}}{\ctx{(\pai{x}{T_{1}}{T_{2}}):t}}
         & $(s,t) \in \mathcal{R}$ \\

    (Mu) & \ruleI{\ctxw{x:s}{T:s}}{\ctx{(\miu{x}{T}):s}} \\

    (Fold) & \ruleII{\ctx{E:(T[x:=\miu{x}{T}])}}{\ctx{\miu{x}{T}:s}}%
                    {\ctx{(\fold{\miu{x}{T}}{E}):\miu{x}{T}}} \\

    (Unfold) & \ruleII{\ctx{E:\miu{x}{T}}}{\ctx{T[x:=\miu{x}{T}]:s}}%
                      {\ctx{(\unfold{E}):T[x:=\miu{x}{T}]}} \\

    (Beta) & \ruleIII{\ctx{E:T_{1}}}{\ctx{T_{2}:s}}{T_{1} \tolong T_{2}}{\ctx{(\bet{E}):T_{2}}}
  \end{tabular}
  \caption{Typing rules for $\lambda C_\mu$}\label{fig:mutype}
\end{figure}

%%% Local Variables:
%%% mode: latex
%%% TeX-master: "pts"
%%% End:
